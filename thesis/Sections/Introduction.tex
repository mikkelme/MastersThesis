\chapter{Introduction}

\section{Motivation}
Friction is the force that prevents the relative motion of objects in contact.
In our everyday life, we recognize it as the inherent resistance to sliding
motion. Some surfaces appear slippery and some appear rough, and we know intuitively
that sliding down a snow-covered hill is much more exciting than its grassy
counterpart. Without friction, it would not be possible to walk across a flat
surface, lean against the wall without falling over or secure an object by the
use of nails or screws~\cite[p. 5]{gnecco_meyer_2015}. It is probably safe to
say that the concept of friction is integrated into our everyday life to such an
extent that most people take it for granted. However, the efforts to control
friction date back to the early civilization (3500 B.C.) with the use of the
wheel and lubricants to reduce friction in translational
motion~\cite{bhushan_2013}. Today, friction is considered a part of the wider
field \textit{tribology} derived from the Greek word \textit{tribos} meaning
``rubbing''. It includes the science of friction, wear and
lubrication~\cite{bhushan_2013}. The most compelling motivation to study
tribology is ultimately to gain full control of friction and wear for various
technical applications. Especially, the reduction of friction is of great interest since this can be utilized to improve energy efficiency in mechanical systems with moving parts. Hence, it has been reported that
tribological problems have a significant potential for both economic and
environmental improvements~\cite{kim_nano-scale_2009}:
\begin{quote}
    ``On global scale, these savings would amount to 1.4\% of the GDP annually
    and 8.7\% of the total energy consumption in the long
    term.''~\cite{holmberg_influence_2017}. 
\end{quote}
On the other hand, the reduction of friction is not the only sensible
application for tribological studies. Controlling frictional properties, besides
minimization, might be of interest in the development of a grasping robot where
finetuned object handling is required. While achieving a certain ``constant''
friction response is readily obtained through appropriate material choices, we
are yet to unlock the full capabilities to alter friction dynamically on the go.
One example from nature inspiring us to think along these lines is the gecko
feet. More precisely, the Tokay gecko has received a lot of attention in
scientific studies aiming to unravel the underlying mechanism of its
``togglable'' adhesion properties. Although the gecko can produce large adhesive
forces, it retains the ability to remove its feet from an attachment surface
at will~\cite{Gekko}. This makes the gecko able to achieve a high adhesion on
the feet when climbing a vertical surface while lifting them for the next step
remains relatively effortless. For a grasping robot, we might consider an analog
frictional concept of a surface material that can change from slippery to rough
on demand depending on specific tasks; slippery and smooth when interacting with
people and rough and firmly gripping when moving heavy objects.


In recent years an increasing amount of interest has gone into the studies of
the microscopic origins of friction, due to the increased possibilities in
surface preparation and the development of nanoscale experimental methods.
Nano-friction is also of great concern for the field of nano-machining where the
frictional properties between the tool and the workpiece dictate machining
characteristics~\cite{kim_nano-scale_2009}. With concurrent progress in
computational capacity and development of Molecular Dynamics (\acrshort{MD}),
numerical investigations serve as an invaluable tool for getting insight into
the nanoscale mechanics associated with friction. This simulation-based approach
can be considered as a ``numerical experiment'' enabling us to create and probe
a variety of high-complexity systems which are still out of reach for modern
experimental methods.

In materials science such \acrshort{MD}-based numerical studies have been used
to explore the concept of so-called \textit{metamaterials} where the material
compositions are designed meticulously to enhance certain physical
properties~\mbox{\cite{PhysRevLett.121.255304, PhysRevResearch.2.042006,
graphene/hBN, Mao, Yang, Forte}}. This is often achieved either by intertwining
different material types or removing certain regions completely. In recent
papers by Hanakata et al.~\cite{PhysRevLett.121.255304,
PhysRevResearch.2.042006}, numerical studies have showcased that the mechanical
properties of a graphene sheet, yield stress and yield strain, can be altered
through the introduction of so-called \textit{Kirigami}-inspired cuts into the
sheet. Kirigami is a variation of origami where the paper is cut additionally to
being folded. While these methods originate as an art form, aiming to produce
various artistic objects, they have proven to be applicable in a wide range of
fields such as optics, physics, biology, chemistry and
engineering~\cite{chen_kirigamiorigami_2020}. Various forms of stimuli enable
direct 2D to 3D transformations through the folding, bending, and twisting of
microstructures. While original human designs have contributed to specific
scientific applications in the past, the future of this field is highly driven
by the question of how to generate new designs optimized for certain physical
properties. However, the complexity of such systems and the associated design
space makes for seemingly intractable\footnote{In computer science we define an
\textit{intractable} problem as a problem with no \textit{efficient} algorithm
to solve it nor any analytical solutions. The only way to solve such problems is
the \textit{brute-force} approach, simply trying all possible combinations, which
is often beyond the capabilities of computational resources.} problems ruling
out analytic solutions.

Earlier design approaches such as bioinspiration, looking at gecko
feet for instance, and Edisonian, based on trial and error, generally rely on
prior knowledge and an experienced designer~\cite{Mao}. While the Edisonian approach is certainly more feasible through numerical studies than real-world
experiments, the number of combinations in the design space rather quickly
becomes too large for a systematic search, even when considering the computation
time on modern-day hardware. However, this computational time constraint can be
relaxed by the use of machine learning (\acrshort{ML}) which has been proven 
successful in the establishment of a mapping from the design space to physical
properties of interest. This gives rise to two new styles of design approaches:
One, by utilizing the prediction from a trained network we can skip the
\acrshort{MD} simulations altogether resulting in an \textit{accelerated search}
of designs. This can be further improved by guiding the search according to
the most promising candidates. For instance, as done with the \textit{genetic
algorithm} based on mutation and crossing. Another
more sophisticated approach is through generative methods such as
\textit{Generative Adversarial Networks} (\acrshort{GAN}) or diffusion models.
The latter is being used in state-of-the-art AI systems such as OpenAI's
DALL$\sq$E2~\cite{DALLE} or Midjourney~\cite{Midjourney}. By working with a
so-called \textit{encoder-decoder} network structure, one can build a model that
reverses the prediction process. This is often referred to as \textit{inverse
design}, where the model predicts a design based on physical target
properties. In the papers by Hanakata et al.~\cite{PhysRevLett.121.255304,
PhysRevResearch.2.042006} both the accelerated search and the inverse design approach was proven successful to create novel
metamaterial Kirigami designs with the graphene sheet. 

Hanakata et al.\ attribute the variation in mechanical properties to the
non-linear effects arising from the out-of-plane buckling of the sheet. Since it
is generally accepted that the surface roughness is of great importance for
frictional properties it can be hypothesized that Kirigami-induced out-of-plane
buckling can also be exploited for the design of frictional metamaterials. For
certain designs, we might hope to find a relationship between the stretching of
the sheet and frictional properties. If significant, this could give rise to an
adjustable friction beyond the point of manufacturing. For instance,
the grasping robot might apply such a material as artificial skin for which
stretching or relaxing of the surface could result in a changeable friction
strength.

In addition, the Kirigami graphene properties can be explored through a
potential coupling between the strain and the normal load, through a
nanomachine design, with the aim of altering the friction coefficient. This
invites the idea of non-linear friction coefficients which might in principle also
take on negative values. This would constitute a rarely found property which is mainly observed for the unloading phase of adhesive
surfaces~\cite{deng_adhesion-dependent_2012} or in the loading phase of particular
heterojunction materials~\cite{Liu_2020, Mandelli_2019}.

To the best of our knowledge, Kirigami has not yet been implemented to alter the
frictional properties of a nanoscale system. However, in a recent paper by
Liefferink et al.~\cite{LIEFFERINK2021101475} it is reported that macroscale
Kirigami can be used to dynamically control the macroscale roughness of a
surface through stretching. They reported that the roughness change led to a
changeable frictional coefficient by more than one order of magnitude. This
supports the idea that Kirigami designs can be used to alter friction, but we
believe that taking this concept to the nanoscale would involve a different set
of governing mechanisms and thus contribute to new insight in this field.



\section{Goals}\label{sec:goals} 
In this thesis, I investigate the prospects of
altering the frictional properties of a graphene sheet through the application
of Kirigami-inspired cuts and stretching of the sheet. With the use of Molecular
Dynamics (\acrshort{MD}) simulations, I evaluate the frictional properties of
various Kirigami designs under different physical conditions. Based on the
\acrshort{MD} results, I investigate the possibility to use machine learning
for the prediction of frictional properties and subsequently using the model for
an accelerated search of new designs. The main goals of the thesis can be
summarized as follows.
\begin{enumerate} 
    \item Design an \acrshort{MD} simulation procedure to evaluate the
    frictional properties of a Kirigami graphene sheet under specified physical
    conditions.
    \item Develop a numerical tool to generate various Kirigami designs,
    both by seeking inspiration from macroscale designs and by the use of a
    random-walk-based algorithm.
    \item Investigate the frictional behavior under varying strain and load for
    different Kirigami designs.
    \item Develop and train a machine learning model to predict the
    \acrshort{MD} simulation results and perform an accelerated search of new
    designs with the goal of optimizing certain frictional properties.
\end{enumerate}



\section{Contributions}
By working toward the goals outlined above (\cref{sec:goals}), I have
discovered a non-linear relationship between the kinetic friction and the strain
for certain Kirigami patterns. This phenomenon was found to be associated with
the out-of-plane buckling of the Kirigami sheet but with no clear relationship
to the contact area or the tension in the sheet. I found that this method does
not provide any mechanism for a reduction in friction, in comparison to a
non-cut sheet. However, the straining of certain Kirigami sheets allows for a
non-monotonic increase in friction. The relationship to normal load was proven
negligible in this context and I have demonstrated that a coupled system of load
and strain (through sheet tension) can exhibit a negative friction coefficient
in certain load ranges. Moreover, I have created a dataset of roughly 10,000
data points for assessing the employment of machine learning and accelerated
search of Kirigami designs. I have found, that this approach might be useful,
but that it requires an extended dataset in order to produce reliable results for a
search of new designs.

During my investigations, I have built three numerical tools, in addition to the usual scripts for data analysis, which are available on
Github~\cite{Github}. The tools are summarized in the following. 
\begin{itemize}
    \item I have written a LAMMPS-based~\cite{LAMMPS} tool for simulating and
    measuring the frictional properties of a graphene sheet sliding on a
    substrate. The code is generally made flexible with regard to the choice of
    sheet configuration, system size, simulation parameters and \acrshort{MD}
    potentials, which makes it applicable for further studies on this topic.
    I have also built an automized procedure to carry out multiple simulations
    under varying parameters by submitting jobs to a computational cluster via
    an ssh connection. This was done by writing minor additions to the Python package developed by E. M. Nordhagen~\cite{lammps_simulator}.
    \item I have generated a Python-based tool for generating Kirigami patterns
    and exporting these in a compatible format with the simulation software created. The
    generation of molecular structures is done with the use of
    ASE~\cite{ase-paper}. Our software includes two classes of patterns inspired
    by macroscale designs and a random walk algorithm which allows for a variety
    of different designs through user-defined biases and constraints. Given our
    system size of choice, the first two pattern generators are capable of
    generating on the order of \num{e8} unique designs while the random walk
    generator allows for significantly more. 
    \item I have built a machine-learning tool based on
    Pytorch~\cite{NEURIPS2019_9015} which includes setting up the data loaders,
    a convolutional network architecture, a loss function, and general
    algorithms for training and validating the results. Additionally, I have
    written several scripts for performing grid searches and analyzing the model
    predictions in the context of the frictional properties of graphene. 
\end{itemize}
All numerical implementations have been originally
developed for this thesis except for the libraries mentioned above along with common Python libraries such as Numpy and
Matplotlib.

\section{Thesis structure}
The thesis is divided into two parts. In~\cref{part:theory} I introduce the
relevant theoretical background, and in~\cref{part:simulations} I present the
numerical implementations and the results of this thesis. \cref{part:theory} contains a description of the theoretical background related
to Friction (\cref{chap:friction}), Molecular Dynamics (\cref{chap:MD}) and
Machine Learning (\cref{chap:ML}). In~\cref{sec:research_questions} we
formulate our research questions in the light of the friction theory.  In~\cref{part:simulations}, I begin by presenting the system in~\cref{chap:system} which includes a definition of the main parts of the system and the numerical procedures related to the \acrshort{MD} simulation. Here I also present the generation of Kirigami designs. In~\cref{chap:pilot_study}, I carry out a pilot
study where I evaluate the simulation results for
various physical conditions and compare a non-cut sheet to two different
Kirigami designs. In~\cref{chap:dataset_study}, I further explore the Kirigami patterns through the creation of a
dataset and the employment of machine learning and an accelerated search for new designs. In~\cref{chap:negative_coef}, I use
the results from the pilot study to demonstrate the possibility to achieve a
negative friction coefficient for a system with coupled load and strain. Finally, in~\cref{chap:summary}, I summarize the results and provide an outlook for further studies. Additional figures are shown
in~\cref{sec:sheet_stretch}, \cref{sec:dataset_conf} and~\cref{sec:data_stretch_profiles}. 

