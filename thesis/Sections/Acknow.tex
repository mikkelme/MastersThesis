\chapter*{Acknowledgments}
The task of writing a master's thesis is a demanding and extensive project which I could not have done without the support of many good people around me. First
of all, I want to thank my supervisors Henrik Andersen Sveinsson and Anders
Malthe-Sørenssen for the assistance in this thesis work. I am especially
grateful for the weekly meetings with Henrik and the inspiring discussions
had as we unraveled the discoveries related to the topic of this thesis. I
remember that I initially asked for an estimate of how much time he had
available for supervision and the answer was something along the lines of
``There are no limits really, just send me an email and we figure it out''. This
attitude captures the main experience I have had working with Henrik and I am
profoundly grateful for the time and effort he has devoted to this project. I
hope that he did not regret this initial statement too much because I have
certainly been taken advantage of it. I also want to thank Even Marius Nordhagen
for technical support regarding the use of the computational cluster. In that
context, I also want to acknowledge the Center for Computing in Science
Education (CCSE) for making these resources available. 

I would like to express my gratitude to all the parties involved in making it
possible for me to write my thesis from Italy. I am particularly grateful for
the flexibility shown by my supervisors and for the support of Anders
Kvellestad, who allowed me to work remotely as a group teacher. I would also
like to thank Scuola Normale Superiore for providing me with access to their
library.

I realize that it is a commonly used cliché to express gratitude for the support
of loved ones. However, I want to highlight the exceptional role played by my
fiancé, Ida, who deserves the main credit for enabling me to maintain a healthy
state of mind. She has provided me with a solid foundation for a fulfilling life
that enables me to pursue secondary objectives, such as an academic career. I
look forward to spending the rest of my life with you. \\
\\
In this thesis, I have used the formal pronoun ``we'' mainly as a customary
habit related to the formalities of scientific writing in a team. Nonetheless, I
have realized that this usage is more fitting as I have not been working alone
on this project. I have received support all the way from colleagues and friends
at the University of Oslo, my family residing in Denmark, and my life partner who slept beside me every night here in Italy. They are the ``good people
around me'' who have made this thesis possible.





