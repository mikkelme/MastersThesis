\chapter{Strain simulation frames}\label{sec:sheet_stretch}
\cref{fig:tetrahedron_strain} and~\cref{fig:honeycomb_strain} show the
sheet-substrate system during a strain simulation. The used Kirigami patterns
are the Tetrahedron $(7,5,1)$ and Honeycomb $(2,2,1,5)$ respectively which are
used in the pilot study \cref{chap:pilot_study}. The simulations correspond to
the investigation of contact area during strain as seen
in~\cref{fig:contact_vs_stretch}. We strain the sheet until rupture and use a
zero normal load. The remaining parameters follow the default values
from~\cref{tab:final_param}. Notice also that the flip book animation in the right page corner corresponds to the Honeycomb simulation in~\cref{fig:honeycomb_strain}. Each page represents a frame in the strain simulation although it reverses and plays backward halfway in. By going through the pages quickly one can get a feeling for the motion in the simulation. This works better in print by releasing the pages held with the thumb.
\newpage


\vspace*{0cm}
\begin{figure}[H]
    \centering
    \begin{subfigure}[b]{0.49\textwidth}
        \centering
        \includegraphics[width=\textwidth]{figures/baseline/contact_vs_stretch/popup/pop_stretch0000.png}
        \caption{Strain = 0.00}
        % \label{fig:}
    \end{subfigure}
    \hfill
    \begin{subfigure}[b]{0.49\textwidth}
        \centering
        \includegraphics[width=\textwidth]{figures/baseline/contact_vs_stretch/popup/pop_stretch0006.png}
        \caption{Strain = 0.06}
        % \label{fig:}
    \end{subfigure}
    \begin{subfigure}[b]{0.49\textwidth}
        \centering
        \includegraphics[width=\textwidth]{figures/baseline/contact_vs_stretch/popup/pop_stretch0008.png}
        \caption{Strain = 0.08}
        % \label{fig:}
    \end{subfigure}
    \hfill
    \begin{subfigure}[b]{0.49\textwidth}
        \centering
        \includegraphics[width=\textwidth]{figures/baseline/contact_vs_stretch/popup/pop_stretch0010.png}
        \caption{Strain = 0.10}
        % \label{fig:}
    \end{subfigure}
    \hfill
    \begin{subfigure}[b]{0.49\textwidth}
        \centering
        \includegraphics[width=\textwidth]{figures/baseline/contact_vs_stretch/popup/pop_stretch0012.png}
        \caption{Strain = 0.12}
        % \label{fig:}
    \end{subfigure}
    \hfill
    \begin{subfigure}[b]{0.49\textwidth}
        \centering
        \includegraphics[width=\textwidth]{figures/baseline/contact_vs_stretch/popup/pop_stretch0014.png}
        \caption{Strain = 0.14}
        % \label{fig:}
    \end{subfigure}
    \begin{subfigure}[b]{0.49\textwidth}
        \centering
        \includegraphics[width=\textwidth]{figures/baseline/contact_vs_stretch/popup/pop_stretch0016.png}
        \caption{Strain = 0.16}
        % \label{fig:}
    \end{subfigure}
    \hfill
    \begin{subfigure}[b]{0.49\textwidth}
        \centering
        \includegraphics[width=\textwidth]{figures/baseline/contact_vs_stretch/popup/pop_stretch0018.png}
        \caption{Strain = 0.18}
        % \label{fig:}
    \end{subfigure}
    \begin{subfigure}[b]{0.49\textwidth}
        \centering
        \includegraphics[width=\textwidth]{figures/baseline/contact_vs_stretch/popup/pop_stretch0020.png}
        \caption{Strain = 0.20}
        % \label{fig:}
    \end{subfigure}
    \begin{subfigure}[b]{0.49\textwidth}
        \centering
        \includegraphics[width=\textwidth]{figures/baseline/contact_vs_stretch/popup/pop_stretch0022.png}
        \caption{Strain = 0.22}
        \label{fig:tetrahedron_strain_j}
    \end{subfigure}
    \hfill
       \caption{Straining of the Tetrahedron $(7,5,1)$ pattern against the substrate. The top part of each frame (a) to (j) shows a top-down view into the x-y plane, with the y-direction on the horizontal axis and the x-direction on the vertical axis. The bottom part of each frame shows a side view of the system, with the y-direction on the horizontal axis and the z-direction on the vertical axis. White-colored atoms indicate graphene sheet atoms in contact with the substrate while the yellow-colored atoms are not in contact. The substrate is shown in blue. }
       \label{fig:tetrahedron_strain}
  \end{figure}
  \pagebreak

% Honeycomb (contact)
\vspace*{0cm}
\begin{figure}[H]
    \centering
    \begin{subfigure}[b]{0.49\textwidth}
        \centering
        \includegraphics[width=\textwidth]{figures/baseline/contact_vs_stretch/honeycomb/hon_stretch0000.png}
        \caption{Strain = 0.00}
        % \label{fig:}
    \end{subfigure}
    \hfill
    \begin{subfigure}[b]{0.49\textwidth}
        \centering
        \includegraphics[width=\textwidth]{figures/baseline/contact_vs_stretch/honeycomb/hon_stretch0014.png}
        \caption{Strain = 0.14}
        % \label{fig:}
    \end{subfigure}
    \begin{subfigure}[b]{0.49\textwidth}
        \centering
        \includegraphics[width=\textwidth]{figures/baseline/contact_vs_stretch/honeycomb/hon_stretch0028.png}
        \caption{Strain = 0.28}
        % \label{fig:}
    \end{subfigure}
    \hfill
    \begin{subfigure}[b]{0.49\textwidth}
        \centering
        \includegraphics[width=\textwidth]{figures/baseline/contact_vs_stretch/honeycomb/hon_stretch0042.png}
        \caption{Strain = 0.42}
        % \label{fig:}
    \end{subfigure}
    \hfill
    \begin{subfigure}[b]{0.49\textwidth}
        \centering
        \includegraphics[width=\textwidth]{figures/baseline/contact_vs_stretch/honeycomb/hon_stretch0056.png}
        \caption{Strain = 0.56}
        % \label{fig:}
    \end{subfigure}
    \hfill
    \begin{subfigure}[b]{0.49\textwidth}
        \centering
        \includegraphics[width=\textwidth]{figures/baseline/contact_vs_stretch/honeycomb/hon_stretch0070.png}
        \caption{Strain = 0.70}
        % \label{fig:}
    \end{subfigure}
    \begin{subfigure}[b]{0.49\textwidth}
        \centering
        \includegraphics[width=\textwidth]{figures/baseline/contact_vs_stretch/honeycomb/hon_stretch0084.png}
        \caption{Strain = 0.84}
        % \label{fig:}
    \end{subfigure}
    \hfill
    \begin{subfigure}[b]{0.49\textwidth}
        \centering
        \includegraphics[width=\textwidth]{figures/baseline/contact_vs_stretch/honeycomb/hon_stretch0098.png}
        \caption{Strain = 0.98}
        % \label{fig:}
    \end{subfigure}
    \begin{subfigure}[b]{0.49\textwidth}
        \centering
        \includegraphics[width=\textwidth]{figures/baseline/contact_vs_stretch/honeycomb/hon_stretch0112.png}
        \caption{Strain = 1.12}
        % \label{fig:}
    \end{subfigure}
    \begin{subfigure}[b]{0.49\textwidth}
        \centering
        \includegraphics[width=\textwidth]{figures/baseline/contact_vs_stretch/honeycomb/hon_stretch0126.png}
        \caption{Strain = 1.26}
        % \label{fig:}
    \end{subfigure}
    \hfill
       \caption{Straining of the Honeycomb $(2,2,1,5)$ pattern against the substrate. The top part of each frame (a) to (j) shows a top-down view into the x-y plane, with the y-direction on the horizontal axis and the x-direction on the vertical axis. The bottom part of each frame shows a side view of the system, with the y-direction on the horizontal axis and the z-direction on the vertical axis. White-colored atoms indicate graphene sheet atoms in contact with the substrate while the yellow-colored atoms are not in contact. The substrate is shown in blue. }
       \label{fig:honeycomb_strain}
  \end{figure}
  \pagebreak



  

