\chapter{Friction}\label{chap:friction} 
Since we aim for controlling frictional properties, we will review the relevant theoretical understanding of friction in this chapter. We limit ourselves to the tribological subcategory, wear-less dry friction, meaning that we consider friction in the absence of any lubricant and wear between the contacting surfaces. We will direct the review towards our system of interest consisting of a nanoscale graphene sheet sliding on a substrate. This will serve as a basis for a formal definition of our research questions at the end of this chapter.


\section{Friction across scales}
Tribological systems span a wide range of time and length scales, from
geological stratum layers involved in earthquakes~\cite{kim_nano-scale_2009} to
atomistic processes, such as the gliding motion of nanoclusters or nanomotors~\cite{Manini_2016}. This vast difference in scale leads to different dominant
frictional mechanisms. At the macroscale, the experimental systems are typically subjected to relatively high loads and sliding speeds, resulting in significant contact stress and wear. This makes for a macroscale friction that is often reduced into a few variables such as load, material type, sliding speed and surface roughness. On the other hand, the micro-/nanoscale regime is usually studied in the opposite domain operating under a relatively small load and sliding speed with negligible wear~\cite{kim_nano-scale_2009}~\cite[p. 5]{bhushan_2013}. This reveals a change in the dominant mechanism at play with an emphasis on the importance of surface properties. The work of Bhushan and Kulkarni~\cite{BHUSHAN199649} showed that the friction coefficient decreased with scale even though the materials used were unchanged. This reveals an intrinsic relationship between friction and scale as the contact condition is altered. The phenomenological descriptions of macroscale friction cannot yet be derived
from the fundamental atomic principles, and bridging the gap between different
length scales in tribological systems remains an open challenge~\cite{Manini_2016}. Hence, the following sections will be organized into
macroscale (\cref{sec:macroscale}), microscale (\cref{sec:microscale}) and
nanoscale (\cref{sec:nanoscale}) representing the theoretical understanding
governing each scale regime. Realizing that the field of friction across all
scales is a vastly broad topic, we will only introduce the most essential findings for each scale while keeping a main focus on features associated with our system at the nanoscale.


\section{Macroscale}\label{sec:macroscale}
Our working definition of the \textit{macroscale} is everything on the scale of millimeters and above~\cite{HUNG2015215}. This represents the scale of visible objects and includes items from our everyday life to big geological systems. 


\subsection{Amontons’ law}
% Based on~\cite{gnecco_meyer_2015}
% and~\cite{gao_frictional_2004}
In order to start and keep a solid block moving against a solid surface we must
overcome certain frictional forces $F_{\text{fric}}$~\cite{gnecco_meyer_2015}.
The static friction force $F_s$ corresponds to the minimum tangential force
required to initiate the sliding while the kinetic friction force $F_k$
corresponds to the tangential force needed to sustain such a sliding at a steady
speed. The work of Leonardo da Vinci (1452–-1519), Guillaume Amontons (1663--1705)
and Charles de Coulomb (1736--1806) all contributed to the empirical law,
commonly known as \textit{Amontons’ law}, which serves as a common base for macroscale
friction. Amontons’ law states that the frictional forces are entirely
independent of contact area and sliding velocity. Instead, it relies only on
the normal force $F_N$, acting perpendicular to the surface, and the material-specific friction coefficient $\mu$ as
\begin{align}
  F_{\text{fric}} = \mu F_N.
  \label{eq:amonton}
\end{align}
Notice that the term \textit{normal force} is often used interchangeably with \textit{load} and \textit{normal load} although the load and normal load refer to the applied force that pushes the object into the surface, whereas the normal force is the reaction force acting from the surface on the object. In equilibrium, these forces are equal in magnitude but opposite in direction. We will mainly consider a system that is in equilibrium with respect to the loading direction, and thus we will refer to the magnitude of the forces and will not distinguish between these terms either. On the same note, we point out that the friction force is different from a conventional force which in the Newtonian definition acts on a body from the outside and makes it accelerate~\cite{gao_frictional_2004}. Rather than being an independent external force the friction force is an internal \textit{reaction} force opposing the externally applied ``sliding'' force. 

The friction coefficient $\mu$ is typically different for the cases of static
($\mu_s$) and kinetic ($\mu_k$) friction, usually both with values lower than
one and $\mu_s \ge \mu_k$ in all cases~\cite[p. 6]{gnecco_meyer_2015}. The friction coefficient is taken to be a constant defined by either~\cite{gao_frictional_2004} \\
\vspace{0.1cm}
\begin{subequations}
\noindent\begin{minipage}{.2\linewidth}
  \hfill
\end{minipage}
\begin{minipage}[b]{0.2\linewidth}
  \begin{align}
    \mu_1 = \frac{F_{\text{fric}}}{F_N},
    \label{eq:mu_def1}
  \end{align}
\end{minipage}
\begin{minipage}[b]{0.2\linewidth}
  \begin{align*}
    \text{or}
  \end{align*}
\end{minipage}
\begin{minipage}[b]{0.2\linewidth}
  \begin{align}
    \mu_2 = \frac{dF_{\text{fric}}}{dF_N}.
    \label{eq:mu_def2}
  \end{align}
\end{minipage}
\begin{minipage}{.2\linewidth}
\end{minipage}
\label{eq:mu_def}
\end{subequations}
\vspace{0.1cm}
\\
\noindent The first definition~\cref{eq:mu_def1} requires zero friction at zero
load, i.e.\ $F_{\text{fric}} = 0$ at $F_N = 0$, while the second definition~\cref{eq:mu_def2} allows for a finite friction force at zero load as the
coefficient is defined by the slope of the friction-load curve. The
consequences of these definitions are illustrated in~\cref{fig:fric_coef_example}, for selected friction-load-curves in~\cref{fig:fric_coef_example_a} and corresponding friction coefficients in~\cref{fig:fric_coef_example_b} and~\cref{fig:fric_coef_example_c}. For adhesive
contacts, the friction force will not be zero under zero load~\cite{gao_frictional_2004} (red curve: Linear
+ shift) which can be mitigated by adding an extra constant to Amontons’ law (\cref{eq:amonton}). Using~\cref{eq:mu_def1} for adhesive contacts would make the friction coefficient diverge for decreasing load as illustrated in~\cref{fig:fric_coef_example_b}. Thus, we find the second
definition~\cref{eq:mu_def2} more robust and versatile. This also allows for a better interpretation of the friction coefficient in the case where
friction depends non-linearly on load as seen with the purple curve in~\cref{fig:fric_coef_example}. 


\begin{figure}[H]
  \centering
  \begin{subfigure}[t]{0.32\textwidth}
      \centering
      \includegraphics[width=\textwidth]{figures/theory/fric_coef_example_a.pdf}
      \caption{}
      \label{fig:fric_coef_example_a}
    \end{subfigure}
    \hfill
    \begin{subfigure}[t]{0.32\textwidth}
      \centering
      \includegraphics[width=\textwidth]{figures/theory/fric_coef_example_b.pdf}
      \caption{}
      \label{fig:fric_coef_example_b}
    \end{subfigure}
    \hfill
    \begin{subfigure}[t]{0.32\textwidth}
      \centering
      \includegraphics[width=\textwidth]{figures/theory/fric_coef_example_c.pdf}
      \caption{}
      \label{fig:fric_coef_example_c}
  \end{subfigure}
  \hfill
  \caption{Illustration of the consequences for the two definitions of the friction coefficient in~\cref{eq:mu_def}. (a) Three examples of friction-load curves consisting of a typical linear curve (blue), a linear curve with a shift representing an adhesive contact (red), and a non-linear curve (purple). The corresponding friction coefficients $\mu_1$ and $\mu_2$ are shown for the first definition~\cref{eq:mu_def1} in (b) and the second definition~\cref{eq:mu_def1} in (c).}
  \label{fig:fric_coef_example}
\end{figure}


Amontons’ law represents the frictional behavior relatively accurately for many surfaces in contact, involving both dry and lubricated, ductile and brittle and rough and smooth surfaces (as long as they are not adhesive) under a variety of conditions~\cite{gao_frictional_2004}. But it has its limitations. For instance, at low velocities, Amontons' model breaks down due to thermal effects, and for high velocities due to inertial effects~\cite[pp.\ 5--6]{gnecco_meyer_2015}. Additionally, static friction depends on the so-called contact history, with increasing static friction as the logarithm of time in stationary contact~\cite{dieterich_1972}.

In cases where Amontons' law breaks down, we might still use the conceptual
definition of the friction coefficient as defined by~\cref{eq:mu_def2}.
Especially, in the context of achieving negative friction coefficients (for certain load ranges), we would refer to this definition, since~\cref{eq:mu_def1}
would imply a truly unphysical situation of the friction force acting in the
same direction as the sliding motion. This would accelerate the object
indefinitely\footnote{You would most likely have a good shot at the Nobel Prize
with that paper.}.

Due to the empirical foundation of Amontons’ law, it does not provide any
physical insight into the underlying mechanisms of friction. However, as we will
later discuss in more detail, we can understand the overall phenomena of
friction through statistical mechanics by the concept of \textit{equipartition
of energy}~\cite{Manini_2016}. A system in equilibrium has its kinetic energy
uniformly distributed among all its degrees of freedom. When a macroscale object
is sliding in a given direction it is clearly not in equilibrium since one of
its degrees of freedom carries considerably more kinetic energy. Thus, the
system will tend to transfer kinetic energy to the remaining
degrees of freedom in the form of heat dissipation to the surroundings. This will make the object slow down if not continuously driven forward by an external energy source. Hence, we can understand the overall concept of friction simply
as the tendency towards energy equipartitioning among many
interacting degrees of freedom~\cite{Manini_2016}. From this point of view, it is
clear that friction is an inevitable part of contact physics, but even though
friction cannot be removed altogether, we are still capable of manipulating it
in useful ways. 

The attentive reader might point out that we have already moved the discussion
into the microscopic regime as \textit{statistical mechanics} generally
aim to explain macroscale behavior by microscopic interactions. This 
highlights the necessity to consider smaller scales in order to achieve a more fundamental understanding of friction.
\\
\\
We note that more advanced models for macroscale friction exist. For instance, the earthquake-like (EQ) model, also known as the \textit{spring-and-block} model or the \textit{multi-contact} model~\cite{Manini_2016}, developed by Burridge and Knopoff~\cite{Burridge_1967}. This has been used in many studies of earthquake friction~\cite{PhysRevLett.88.096102} and similar schemes have since been used to model the failure of fiber bundles and faults~\cite{newman_failure_1991, Smalley_1985}. Also, \textit{rate and state} models have been used for such macroscale modeling~\cite{SELVADURAI2023229689}. However, these extensions are beyond the scope of this thesis as we will mainly focus on the nanoscale description. 


\section{Microscopic scale}\label{sec:microscale}
Going from a macro- to a microscale perspective, at a length scale on the order of
\SI{e-6}{m}, it was realised that most surfaces are in fact rough~\cite{mo_friction_2009}. The contact between two surfaces consists of numerous
smaller contact points, so-called \textit{asperities}, which form junctions due to contact pressure and adhesion as visualized in~\cref{fig:asperity_contact}~\cite{kim_nano-scale_2009}. In the macroscale perspective of Amonton's law, we refer to time- and space-averaged values, i.e.\ the apparent contact area and the average
sliding speed~\cite{gao_frictional_2004}. However, microscopically we find the
real contact area to be much smaller than the apparent area~\cite{kim_nano-scale_2009}, and the shearing motion of local microjunctions to happen at large fluctuations rather than as one synchronized movement throughout the surface. 

It is generally accepted that friction is caused by two mechanisms: Mechanical
friction and chemical friction~\cite{kim_nano-scale_2009}. Mechanical
friction is the ``plowing'' of the surface by hard particles or said asperities
with an energy loss attributed to deformations of the asperities. While plastic
deformations, corresponding to wear, gives rise to an obvious attribution for
the energy loss, elastic deformations are also sufficient in explaining energy
loss due to phonon excitations. The assumption of plastic deformations
has been criticized as this is theorized only to be present at the beginning of
a surface contact while it is negligible for prolonged or repeated contacts~\cite{CARBONE20082555}. That is, when machine parts slide against each other for
millions of cycles, the plastic deformation would only take place at the beginning for which the system then reaches a steady state with only elastic deformations.
The chemical friction arises from adhesion between microscopic contacting
surfaces, with an energy loss attributed to the breaking and forming of chemical bonds between the interacting surfaces. 


\subsection{Asperity theories} 
Asperity theories have their foundations in the adhesion model proposed by Bowden and Tabor~\cite{bowden2001friction} which is based on the fundamental reasoning that friction is governed by the adhesion between two surfaces~\cite{Kim_2012}. Adhesion is proportional to the real contact area defined by asperity junctions, and interfacial shear strength $\vec{\tau}$ between such contacting junctions. For an asperity contact area $A_{\text{asp}}$ we get a true contact area $\sum A_{\text{asp}}$ leading to 
\begin{align*}
  F_\text{fric} = \vec{\tau} \sum A_{\text{asp}}.
\end{align*}
Note that this is still compatible with Amontons’ law in~\cref{eq:amonton} by having a linear relationship between the real contact area and the
applied load. By increasing the normal load it is hypothesized that the real contact area will increase as the asperity tips are deformed (plastically or elastically) into broader contact points as visualized qualitatively in~\cref{fig:asperity_contact}.

\begin{figure}[H]
  \centering
  \begin{subfigure}[b]{0.49\textwidth}
      \centering
      \includegraphics[width=\textwidth]{figures/theory/asperities_top.png}
      \caption{Low load.}
      \label{fig:asp_left}
  \end{subfigure}
  \hfill
  \begin{subfigure}[b]{0.49\textwidth}
      \centering
      \includegraphics[width=\textwidth]{figures/theory/asperities_bottom.png}
      \caption{High load.}
      \label{fig:asp_right}
  \end{subfigure}
  \hfill
     \caption{Qualitatively illustration of the microscopic asperity deformation
     under increasing load from frame (a) to (b). While this figure seemingly portrays plastic deformation the concept of increased contact area with increased load applies to elastic deformation as well. Reproduced from~\cite{wiki:asperities}.}
     \label{fig:asperity_contact}
\end{figure}

Many studies have focused on single asperity contacts to reveal the relationship
between the contact area and load~\cite{Szlufarska_2008, PhysRevLett.56.930,
perry_scanning_2004}. By assuming perfectly smooth asperities, with radii of
curvature from micrometers all the way down to nanometers, continuum mechanics
can be used to predict the deformation of asperities as load is applied. A model
for non-adhesive contact between homogenous, isotropic, linear elastic spheres
was first developed by Hertz~\cite{HertzOnTC}, which predicted $A_{\text{asp}}
\propto F_N^{2/3}$. Later adhesion effects were included in a number of
subsequent models, including Maugis-Dugdale theory~\cite{MAUGIS1992243}, which
also predicts a sublinear relationship between $A_{\text{asp}}$ and $F_N$. Thus,
the common feature of all single-asperity theories is that $A_{\text{asp}}$ is a
sublinear function of $F_N$, leading to a similar sublinear relationship for
$F_\text{fric}(F_N)$. This fails to align with the macroscale observations
modeled by Amontons’ law (\cref{eq:amonton}).

Concurrently with single-asperity studies, roughness contact theories are being developed~\cite{PhysRevLett.100.055504, Persson, GW, BUSH197587} to bridge the gap between single asperities and macroscopic contacts~\cite{mo_friction_2009}. A variety of multi-asperity theories has attempted to combine single asperity
mechanics by statistical modeling of the asperity height and spatial
distributions~\cite{CARBONE20082555}. This has led to partial success in the establishment of a linear relationship between $A_{\text{asp}}$ and $F_N$. Unfortunately, these results are restricted in terms of the magnitude of the load and contact area, where multi-asperity
contact models based on the original ideas of Greenwood and Williamson~\cite{GW}
only predicts linearity at vanishing loads, or Persson~\cite{Persson} which predicts linearity for more reasonable loads up to 10--15\% of the macroscale contact area. However, as the load is further increased all multi-asperity models
predict the contact area to fall into the sublinear dependency of normal force
as seen for single asperity theories as well~\cite{CARBONE20082555}.


\section{Nanoscale --- Atomic scale}\label{sec:nanoscale}
Going from a micro- to a nanoscale, on the order of \SI{e-9}{m}, it has been
predicted that continuum mechanics will start to break down~\cite{luan_breakdown_2005} due to the discreteness of individual atoms. In a
numerical \acrshort{MD} study by Mo et al.~\cite{mo_friction_2009}, considering
asperity radii of 5--30 nm, it has been shown that the asperity area
$A_{\text{asp}}$, defined by the circumference of the contact zone, is
sublinear with $F_N$. This is accommodated by the observation that not all atoms
within the circumference make chemical contact with the substrate. By modeling
the real contact area $A_{\text{real}} = NA_{\text{atom}}$, where $N$ is the
number of atoms within the range of chemical interaction and $A_{\text{atom}}$
the associated surface area for a contacting atom, they found a consistent linear relationship between friction and the real contact area. Without adhesive
forces, this leads to a similar linear relationship $F_{\text{fric}} \propto F_N$,
while adding van der Waals adhesion to the simulation gave a sublinear
relationship matching microscale single asperity theory, even though the
$F_{\text{fric}} \propto A_{\text{real}}$ was maintained. This result emphasizes
that the predictions of continuum mechanisms might still apply at the nanoscale
and that the contact area can be expected to play an important role in
nanoscale asperity contacts. It is simply the definition of the contact area that
changes when transitioning from the microscale to the nanoscale.


While the study by Mo et al.~\cite{mo_friction_2009} considers a single
asperity on a nanoscale, some models take this even further to what we will
denote as the atomic scale. This final leap is motivated by the fact that our
system of interest, an atomically flat graphene sheet imposed on a flat silicon
substrate, lacks the presence of nanoscale asperities in its initial uncut
undeformed state. In the lack of noteworthy structural asperities, friction can
instead be modeled as a consequence of the ``rough'' potential laid out by
the atomic landscape. A series of so-called \textit{reduced-order} models build on a
simplified system of atomic-scale contacts based on three essential parts: 1) A
periodic potential modeling the substrate as a rigid crystalline surface. 2) An
interacting particle, or collection of particles, placed in the potential. 3) A
moving body, moving at a steady speed, connected to the particles through a
harmonic spring. In figure~\cref{fig:PT_FK_FKT} three of the most common 1D
models are displayed which we will address in the following sections. The
time-honored Prandtl-Tomlinson (\acrshort{PT}) model describes a point-like tip sliding
over a space-periodic fixed crystalline surface with a harmonic coupling to the
moving body. This is analog to that of an experimental cantilever used for
Atomic Force Microscopy which we will introduce in more detail in~\cref{sec:SPM}. Further extensions were added in the Frenkel-Kontorova
(\acrshort{FK}) model by substituting the tip with a chain of harmonically coupled
particles dragged from the end, and finally combined in the
Frenkel-Kontorova-Tomlinson (\acrshort{FKT}) with the addition of a more
rigorous harmonic coupling between the moving body and each of the atoms in the
chain. While these models cannot provide the same level of detail as atomistic
simulations, such as \acrshort{MD}, they enable investigation of atomic friction
under most conditions, some of which are inaccessible to \acrshort{MD}~\cite{Yalin_2011}. This makes these models an appropriate tool for investigating
individual parameters and mechanisms governing friction.

\begin{figure}[H]
  \centering
  \includegraphics[width=0.5\linewidth]{figures/theory/PT_FK_FKT_edit.png}
  \caption{Illustration of the key features of the Prandtl-Tomlinson (\acrshort{PT}), Frenkel-Kontorova (\acrshort{FK}) and Frenkel-Kontorova-Tomlinson (\acrshort{FKT}) reduced-order models respectively. Reproduced from~\cite{Yalin_2011} with modifications of the displayed notation.}
  \label{fig:PT_FK_FKT}
\end{figure}


\subsection{Prandtl–Tomlinson} 
The Prandtl–Tomlinson model (\acrshort{PT}) considers a 1D simplification of
the frictional system as a single ball-tip sliding along the rigid substrate as
shown in~\cref{fig:PT_FK_FKT}. The tip is coupled harmonically to a moving support, moving at a constant speed, which drives the tip forward. The interaction between
the tip and the substrate is modeled by a sinusoidal corrugation potential
mimicking the periodicity found in a crystalline substrate. We will consider the
Prandtl–Tomlinson model with added thermal activation as proposed by Gnecco et
al.~\cite{PhysRevLett.84.1172}. For the theoretical foundation of this section,
we generally refer to~\cite{Yalin_2011}. The potential energy for the tip at position $x$ at time $t$ is given as
\begin{align}
  V(x,t) = \frac{1}{2}K(vt - x)^2 - \frac{1}{2}U_0 \cos \left(\frac{2\pi x}{a} \right).
  \label{eq:V_PT}
\end{align}
The first term describes the harmonic coupling with spring constant $K$, between the tip at position $x$ and the moving body at position $vt$, given by its constant speed $v$. The second term describes the corrugation potential with amplitude $U_0$ and period $a$ representing the lattice spacing of the substrate. The dynamics of the tip can be described by the Langevin equations 
\begin{align}
  m \ddot{x}+m \mu \dot{x}=-\frac{\partial V(x, t)}{\partial x}+R(t),
  \label{eq:Langevin_PT}
\end{align}
where $m$ is the mass of the tip, $\mu$ the viscous friction and $R(t)$ the thermal activation term. The equation is solved for tip position $x$ and the friction force is retrieved as the force acting on the moving body
\begin{align*}
  F_{\text{fric}} = K(vt - x).
\end{align*}
The governing equation~\cref{eq:Langevin_PT} belongs to a family of stochastic differential equations composed of both deterministic dynamics and stochastic processes. In this case, the deterministic term is the viscous friction, $m\mu\dot{x}$, resisting the movement of the tip. The stochastic term is a random force field modeling thermal noise according to the Fluctuation-dissipation relation. Thus, there is no single path but rather multiple paths the tip can take. While the Langevin equation is one of the most common ways to handle thermal activation other methods exist to solve this problem such as Monte Carlo sampling methods. We omit the numerical scheme for the solving of the Langevin equations here and refer instead to a more in-depth discussion of the Langevin equation regarding the use in \acrshort{MD} simulations in~\cref{sec:langevin}. 


\subsubsection{Thermal activation}
The solving of the Langevin equation, as opposed to Newton's equation of motion, introduces thermal effects to the system. Generally, when the energy barrier comes close to $k_B T$ (\SI{0.026}{eV} at room temperature) thermal effects can not be neglected~\cite{Yalin_2011}. In the case of a single asperity contact the energy barrier is on the order \SI{1}{eV} which makes thermal activation significant. Due to the moving body traveling at a constant speed, the potential energy will increase steadily. Without any temperature, $T = 0$, the slip will only occur when the energy barrier between the current potential well $i$ and the adjacent $j$ is zero $\Delta V_{i\to j} = 0$. However, in the presence of temperature, we get thermal activation, meaning that the tip can slip to the next potential well sooner at $\Delta V_{i\to j} > 0$. Provided that the sliding speed is slow enough the transition rate $\kappa$ for a slip from the current to the next well is given by
\begin{align}
  \kappa = f_0 e^{-\Delta V / k_B T},
  \label{eq:PT_kappa}
\end{align}
with $\Delta V$ being the energy barrier and $f_0$ the attempt rate. The attempt rate following Kramer’s rate theory~\cite{RevModPhys.62.251} is related to the mass and damping of the system and can be thought of as the frequency at which the tip ``attempts'' to overcome the barrier. Notice that~\cref{eq:PT_kappa} resembles a microstate probability in the canonical ensemble with $f_0$ in place of the inverse partition function $Z^{-1}$ which provides an additional interpretation of $f_0$. The probability $p_i$ that the tip occupies the current well $i$ relative to the adjacent well $j$, as illustrated in~\cref{fig:PT_slip}, is governed by 
\begin{align}
  \frac{dp_i}{dt} = -\kappa_{i\to j}p_i + \kappa_{j\to i}p_j.
  \label{eq:dpdt_PT}
\end{align}
This probability is related to temperature, speed and mass~\cite{Yalin_2011}.

\begin{figure}[H]
  \centering
  \includegraphics[width=0.5\linewidth]{figures/theory/PT_slip.png}
  \caption{An illustration of slip between two adjacent energy minima. $p_i$ is the probability of the tip residing in the current potential well, $i$, where the energy barrier is $\Delta V_{i \rightarrow j}$. $p_j$ is the probability of the tip residing at the next minima, $j$, where $\Delta V_{j \rightarrow i}$ is the corresponding energy barrier. Figure and caption reproduced from~\cite{Yalin_2011}.}
  \label{fig:PT_slip}
\end{figure}


Generally, there exist two temperature regimes in the Prandtl–Tomlinson model: The \textit{thermal activation} regime at low temperatures and the \textit{thermal drift} at high temperatures as shown in~\cref{fig:PT_temp}. At lower temperatures, the system is subject to standard thermal activation with a much lower energy barrier for slipping forward than backward $\Delta V_{j \to i} \gg \Delta V_{i \to j}$. This results in a higher transition rate for forward slips, $\kappa_{j \to i} \ll \kappa_{i \to j}$, which effectively inhibits any backward slips and simplifies~\cref{eq:dpdt_PT} to
\begin{align*}
  \frac{dp_i}{dt} = -\kappa_{i\to j}p_i.
\end{align*}
This leads to the relationship between friction, temperature and speed following Sang et al.’s prediction~\cite{Sang_2001}
\begin{align}
  F_{\text{fric}} = F_c-\left|\beta k_B T \ln \left(\frac{v_c}{v}\right)\right|^{2 / 3}, \qquad v_c = \frac{2f_0\beta k_B T}{3 C_{\text{eff}} \sqrt{F_c}},
  \label{eq:F_thermal_ac}
\end{align}
where $F_c$ is the maximum friction at $T = 0$, $v_c$ a critical velocity, $f_0$
is the attempt rate, $c_{\text{eff}}$ the effective stiffness, and $\beta$ a
parameter determined by the shape of the corrugation well. \cref{eq:F_thermal_ac} characterizes the decrease in friction with temperature
in the thermal activation regime, shown in~\cref{fig:PT_temp_a} at low temperature. This corresponds with the assumption of only forward slips, as seen in the force trace in~\cref{fig:PT_temp_a}. When the temperature is high enough for the system to be consistently close to thermal equilibrium, it enters the regime of thermal drift~\cite{PhysRevE.71.065101}. This regime transition can be understood through a comparison between two time scales: The time it takes for the moving body to travel one lattice spacing
$t_v = a/v$ and the average time for a slip to occur due to thermal activation
$\tau = 1/\kappa = f_0^{-1}\exp(\Delta V / k_BT)$. If $t_v \gg \tau$ the system falls into the thermal drift regime, where slips happen both in the forward and backward direction as shown in the force trace in~\cref{fig:PT_temp_b}. For the thermal drift regime, the friction follows the prediction by Krylov et
al.~\cite{Krylow_2007, PhysRevE.71.065101, Jinesh_2008}
\begin{align}
  F_{\text{fric}} \propto \frac{v}{T}e^{1/T}.
  \label{eq:PT_thermal_drift}
\end{align}
Notice that the friction dependence on sliding speed also changes from~\cref{eq:F_thermal_ac} to~\cref{eq:PT_thermal_drift} as it transitions from the thermal activation to the thermal drift regime. 


\subsubsection{Sliding speed}
In the thermal activation regime (low temperature) and at low sliding speeds, the
friction relation follows~\cref{eq:F_thermal_ac} which means that friction
increases logarithmically with speed. For higher speeds, above the critical
velocity $v > v_c$, if only thermal effects are considered,
\cref{eq:F_thermal_ac} predicts that friction will eventually saturate and come
to a plateau at $F_{\text{fric}} = F_C$. This is illustrated in
\cref{fig:PT_speed} with this prediction being represented by the dotted line.
However, as given away by the figure, for higher speeds the model will enter an
\textit{athermal} regime where the thermal effects are negligible compared to
other contributions~\cite{PhysRevLett.89.224301}. In the athermal regime, the
damping term $m\mu \dot{x}$ will dominate yielding $F_{\text{fric}}\propto v$.
The athermal regime is often observed in reduced-models if the system is
overdamped or at high speeds. This concept is related to \acrshort{MD}
simulations as well where the accessible speeds often fall into the athermal
regime~\cite{Li_2011}. It is unclear how this affects real physical systems for
which there exist more dissipation channels than just a single viscous
term~\cite{Dong_2013}. For the thermal drift regime, at higher temperatures, friction increase linearly with sliding speed $F_{\text{fric}} \propto v$ as given by~\cref{eq:PT_thermal_drift}.


\begin{figure}[!htb]
  \centering
  \begin{subfigure}[t]{0.49\textwidth}
      \centering
      \includegraphics[width=\textwidth]{figures/theory/PT_temp.png}
      \caption{}
      \label{fig:PT_temp_a}
  \end{subfigure}
  \hfill
  \begin{subfigure}[t]{0.49\textwidth}
      \centering
      \includegraphics[width=\textwidth]{figures/theory/PT_temp_force.png}
      \caption{}
      \label{fig:PT_temp_b}
  \end{subfigure}
  \hfill
  \hfill
     \caption{Illustration of the temperature difference between the thermal activation regime and the thermal drift regime. (a) The mean friction as a function of temperature showcasing the regime transition. The figure corresponds to the numerical results of Dong et al.~\cite{Yalin_2011} of a Prandtl–Tomlinson model with model parameters: $m=\SI{e-12}{kg}$, $U_0={0.6}{eV}$, $v=\SI{4e3}{nm/s}$, $\mu=\SI{2}{\sqrt{K/m}}$, $a=\SI{0.288}{nm}$. (b) The force traces of a system in the thermal activation regime (top) and thermal drift regime (bottom) with several characteristic forward and backward slips highlighted by dashed lines. The forward slips are identified as a sudden decrease in friction while the backward slips are identified as a sudden increase in friction. Reproduced from~\cite{Yalin_2011}.}
     \label{fig:PT_temp}
\end{figure}


\begin{figure}[!htb]
  \centering
  \includegraphics[width=0.5\linewidth]{figures/theory/PT_speed.png}
  \caption{The friction dependence on sliding speed for the simulated Prandtl–Tomlinson model by Dong et al.~\cite{Yalin_2011} in the thermal activation temperature regime, revealing two different sliding speed regimes. In the thermal regime, friction increases logarithmically with sliding speed following~\cref{eq:F_thermal_ac}, and in the athermal regime, friction is governed by damping leading to a proportional relationship to sliding speed $F_{\text{fric}}\propto v$. The friction plateau ($F_c = \SI{0.39}{nN}$) predicted by thermal activation~\cref{eq:F_thermal_ac} is shown as a dotted line. Other models parameters: $m=\SI{e-12}{kg}$, $U_0={0.6}{eV}$, $T = \SI{300}{K}$, $\mu=\SI{2}{\sqrt{K/m}}$, $a=\SI{0.288}{nm}$. Reproduced from~\cite{Yalin_2011}.}
  \label{fig:PT_speed}
\end{figure}


\subsubsection{Tip mass}
The mass of the tip affects the dynamics due to a change of inertia, which changes the attempt rate $f_0$. Smaller inertia leads to a larger attempt rate and vice versa. Effectively, this will affect the transition point for the temperature and speed regimes described previously. A smaller inertia, yielding a larger attempt rate, will cause an earlier transition, i.e.\ at a lower temperature, to the thermal drift regime. Additionally, this will also result in a later transition to the athermal regime, i.e.\ at a higher speed.


\subsubsection{Friction regimes: Smooth sliding, single slip, and multiple slip}
Stick-slip motion is a crucial instability mechanism associated with high energy dissipation and high friction. Thus, controlling the transition between smooth sliding and stick-slip is considered key to controlling friction. We can divide the frictional stick-slip behavior into three regimes: 1) Smooth sliding, where the tip slides smoothly on the substrate. 2) Single slip, where the tip stick at one potential well before jumping one lattice spacing to the next. 3) Multiple slip, where the tip jumps more than one lattice spacing in a slip event. The underlying mechanisms behind these regimes can be understood through static and dynamic contributions. 

To understand the static mechanism we consider a quasistatic process for which temperature, speed and damping can be neglected. For a quasistatic process, we require $\partial V/\partial x = 0$. This simplifies~\cref{eq:V_PT} to 
\begin{align}
  \frac{\pi U_0}{a} \sin\left(\frac{2\pi x}{a}\right) = K(vt - x).
  \label{eq:static_V}
\end{align}
The friction regime is determined by the number of solutions $x$ to~\cref{eq:static_V}. Only one solution corresponds to
smooth sliding, two solutions to a single slip and so on. It turns out that the
regimes can be defined by the parameter $\eta = 2\pi^2U_0/a^2K$~\cite{Johnson_1998, Medyanik_2006} yielding transitions at $\eta = 1, 4.6, 7.79, 10.95, \hdots$, such that $\eta \le 1$
corresponds too smooth sliding, $1<\eta \le 4.6$ to a single slip and so on. These static derivations lay out the fundamental probabilities for being in one of the stick-slip regimes. Notice that increasing the spring constant $K$ (stiff spring) will decrease the probability of stick-slip behavior. Similarly, $\eta$ can be altered by a change in the potential corrugation $U_0$ through an increased load~\cite{Vanossi_2013}.

Considering the dynamics on top, one finds that damping, speed and temperature will affect this probability. High damping, equivalent to a high transfer
of kinetic energy to heat, will result in less energy available for the slip events. This will make multiple slip events less likely. By a similar argument, we find that increasing the speed will contribute to more kinetic energy which will increase the likelihood of multiple slip events. Finally, the temperature will contribute to earlier slips, due to thermal activation, such that
less potential energy can be accumulated and it will result in fewer multiple slip events. 

% The effects of damping, speed and temperature are illustrated for the force traces in~\cref{fig:PT_slip_var}


% \begin{figure}[!htb]
%   \centering
%   \begin{subfigure}[t]{0.32\textwidth}
%       \centering
%       \includegraphics[width=\textwidth]{figures/theory/PT_slip_damping.png}
%       \caption{}
%   \end{subfigure}
%   \hfill
%   \begin{subfigure}[t]{0.32\textwidth}
%       \centering
%       \includegraphics[width=\textwidth]{figures/theory/PT_slip_speed.png}
%       \label{fig:PT_slip_vel}
%       \caption{}
%   \end{subfigure}
%   \hfill
%   \begin{subfigure}[t]{0.32\textwidth}
%       \centering
%       \includegraphics[width=\textwidth]{figures/theory/PT_slip_temp.png}
%       \caption{}
%   \end{subfigure}
%   \hfill
%      \caption{\hl{Temporary} figure from~\cite{Yalin_2011}. \hl{Consider removing since the interpretation of smooth sliding might get a bit tricky.}}
%      \label{fig:PT_slip_var}
% \end{figure}


\subsection{Frenkel-Kontorova}
The Frenkel-Kontorova (\acrshort{FK}) model~\cite{Frenkel_1938} extends the Prandtl–Tomlinson model by considering a chain of atoms in contrast to just a single particle (tip). This extension is useful for understanding the importance of the alignment between the atoms and the substrate, the so-called \textit{commensurability}. Our review of the Frenkel-Kontorova is based on~\cite{Manini_2016, Vanossi_2013}.

The standard Frenkel-Kontorova model consists of a 1D chain of $N$ classical particles of equal mass, representing atoms, interacting via harmonic forces and moving in a sinusoidal potential as sketched in~\cref{fig:FK_model}~\cite{Manini_2016}. The Hamiltonian is 
\begin{align}
  H = \sum_{i=1}^N \left[\frac{p_i^2}{2m} + \frac{1}{2}K(x_{i+1} - x_i - a_c)^2 + \frac{1}{2}U_0 \cos{\left(\frac{2\pi x_i}{a_b}\right)}\right],
  \label{eq:H_FK}
\end{align}
where the atoms are labelled sequently $i = 1, \hdots, N$. The first term $p_i^2/2m$ represents the kinetic energy with momentum $p_i$
and mass $m$. Often the effects of inertia are neglected, referred to as the static Frenkel-Kontorova model, while the inclusion in~\cref{eq:H_FK} is known as the dynamic Frenkel-Kontorova model~\cite{FK2D}. The next term describes the harmonic interaction with elastic
constant $K$, nearest neighbor distance $\Delta x = x_{i+1} - x_i$ and 
corresponding nearest neighbor equilibrium distance $a_c$. The final term represents the periodic corrugation potential, with amplitude $U_0$ and period $a_b$. By comparison to the potential used in the Prandtl–Tomlinson model~\cref{eq:V_PT}, we find the difference to be the introduction of a harmonic coupling between particles in the chain. Notice also, that we have not yet specified the motion of the connected moving body. Different boundary choices can be made where both free ends and periodic conditions give similar results. The choice of fixed ends however makes the chain incapable of sliding.

\begin{figure}[!htb]
  \centering
  \includegraphics[width=0.6\linewidth]{figures/theory/FK_model.png}
  \caption{An illustration of the Frenkel-Kontorova model with two competing lengths: The interparticle distance $a_c$ and the substrate periodicity $a_b$. Figure reproduced from~\cite{Vanossi_2013} with permission from the American Physical Society.}
  \label{fig:FK_model}
\end{figure}

To probe static friction one can apply an external adiabatically increasing force, i.e.\ without loss or gain of heat, until sliding occurs. This corresponds to the static Frenkel-Kontorova model, and it turns out that the sliding properties are entirely governed by its topological excitations referred to as so-called \textit{kinks} and \textit{antikinks}.

\subsubsection{Commensurability} We can subdivide the frictional behavior in terms of commensurability, that is, how well the spacing of the atoms matches the periodicity of the substrate potential. We describe this by the length ratio $\theta = a_b / a_c = N / M$ where $M$ denotes the number of minima in the potential within the length of the chain. A rational number for $\theta$ means that we can achieve a perfect alignment between the atoms in the chain and the potential minima, without stretching the chain, corresponding to a \textit{commensurate} case. If $\theta$ is irrational the chain and substrate cannot fully align without some stretching of the chain, and we denote this as being \textit{incommensurate}.

We begin with the simplest commensurate case of $\theta = 1$ where the spacing
of the atoms matches perfectly with the substrate potential periodicity, i.e.\
$a_c = a_b$, $N = M$. The ground state (\acrshort{GS}) is the configuration
where each atom is aligned with one of the substrate minima. By adding an extra
atom to the chain we would effectively shift some of the atoms out of this
ideal state, giving rise to a kink excitation. This leads to the case where two
atoms will have to ``share'' the same potential corrugation as sketched in
\cref{fig:incommensurable_example}. On the other hand, removing an atom from
the chain results in an antikink excitation where one potential corrugation will
be left ``atomless''. In order to reach a local minimum the kink (antikink) will
expand in space over a finite length such that the chain undertakes a local
compression (expansion). Notice that for low ratios of $\theta$, there are fewer atoms than minima, and the chain will not be able to fill each corrugation well. In this case, the kink excitations can instead be thought of as whether the atoms are forced to be in a different potential well than otherwise dictated by the spring forces in-between. 

\begin{figure}[!htb]
  \centering
  \includegraphics[width=0.55\linewidth]{figures/theory/incommensurable_example.png}
  \caption{Qualitative example of an incommensurable case where the atoms sit slightly closer together than otherwise dictated by the substrate periodicity. This results in a single kink which is here seen as the presence of two atoms within the same potential corrugation well. Reproduced from~\cite{BRAUN19981}.}
  \label{fig:incommensurable_example}
\end{figure}

When applying a tangential force to the chain it is much
easier for an excitation to move along the chain than it is for the non-excited
atoms since the activation energy for a kink/antikink
displacement is systematically smaller (often much smaller) than the potential
barrier $U_0$. Thus, the motion of kinks (antikinks), i.e.\ the displacement of
extra atoms (atom vacancies), is representing the fundamental mechanism for
mass transport. These displacements are responsible for the mobility,
diffusivity and conductivity within this model. In the zero temperature commensurable case with an adiabatical increase in force, all atoms would be put into an accelerating motion as soon as the potential barrier energy is present. However, similar to our discussion on the Prandtl-Tomlinson model, thermal activations will excite the system at an earlier stage resulting in kink-antikink pairs traveling down the chain. For a non-periodic chain of finite length, these often occur at the end of the chain running in opposite directions. This cascade of kink-antikink excitations is shown in~\cref{fig:kink_antikink}. Notice, that for the 2D case, where an island (flake) is deposited on a surface, we generally also expect the sliding to be initiated by kink-antikink pairs at the boundaries. 


\begin{figure}[!htb]
  \centering
  \includegraphics[width=0.45\linewidth]{figures/theory/kink_antikink.png}
  \caption{Atomic trajectories vs.\ time at the depinning transition for a periodic nonzero temperature Frenkel-Kontorova chain with $\theta = 1$. The onset of motion is marked by the creation of one kink-antikink pair. The kink and antikink move in opposite directions. They collide after passing through the boundaries, and soon a second kini-antikink pair is created in the tail of the primary kink. This process repeats with exponential avalanche-like growth of the kink-antikink concentration, leading to the total sliding state. Adapted from~\cite{PhysRevLett.79.3692}, figure from~\cite{Vanossi_2013} with permission from the American Physical Society.}
  \label{fig:kink_antikink}
\end{figure}


For the case of incommensurability, i.e.\ $\theta = a_b/a_c$ being irrational,
the \acrshort{GS} is characterized by a sort of ``staircase''  deformation. That
is, the chain will exhibit regular periods of regions with approximate
commensurability separated by regularly spaced kinks or antikinks. 

The incommensurable Frenkel-Kontorova model contains a critical elastic constant $K_c$, such that for $K > K_c$ the static friction $F_s$ drops to zero, making the chain able to initiate a slide at no energy cost, while the low-velocity kinetic friction is dramatically reduced. This can be explained by the
fact that the displacement occurring in the incommensurable case will yield just
as many atoms climbing up a corrugation as atoms climbing down. For a big (infinite) chain this will exactly balance the forces making it
non-resistant to sliding. Generally, incommensurability guarantees that the
total energy (at $T=0$) is independent of the relative position to the
potential. However, when sliding freely, a single atom will eventually occupy a
maximum of the potential, and thus when increasing the potential magnitude $U_0$ or
softening the chain stiffness, lowering $K$, the possibility to occupy such a
maximum disappears. This marks the so-called \textit{Aubry transition},
at the critical elastic constant $K = K_c(U_0, \theta)$, where the chain goes
from a free sliding to a \textit{pinned} state with nonzero static friction.
$K_c$ is a discontinuous function of the ratio $\theta$, due to the reliance on
irrational numbers for incommensurability. The minimal
value $K_c \simeq 1.0291926 $ in units $[2 U_0 (\pi / a_b)^2]$ is achieved for
the golden-mean ratio $\theta = (1+\sqrt{5}/2)$. The Aubry transition can be investigated as a first-order phase transition for which power laws can be defined for the order parameter, but this is beyond the scope of this thesis.


The phenomenon of non-pinned configurations is named \textit{superlubricity} in
tribological context. Despite the misleading name, this refers to the case where
the static friction is zero while the kinetic friction is nonzero, but reduced.
For the case of a 2D sheet, it is possible to alter the commensurability, not
only by changing the lattice spacing through material choice but also by
changing the orientation of the sheet relative to the substrate. Dienwiebel et
al.~\cite{DIENWIEBEL2005197} have observed experimentally that the kinetic friction, for a
graphene flake sliding over a graphite surface (multiple layers of graphene),
exhibits extremely low friction at certain orientations as shown in
\cref{fig:graphene_rot}. As the orientation is changed they observed two spikes
of considerable friction while the remaining valleys correspond to effectively
zero friction in consideration of the measurement uncertainty. This phenomenon
relates to the transition between frictional slip regimes, as introduced through the
Prandtl–Tomlinson model, since the change in orientation affects the effective
substrate potential. Merely from the static consideration, we found that
lowering the potential amplitude $U_0$ will decrease the parameter $\eta =
2\pi^2U_0/a^2K$ shifting away from the regime of multiple slip towards smooth
sliding associated with low friction. Such transitions will also be affected by
the shape of the potential and corresponding 2D effects of the sliding
path~\cite{Yalin_2011}.

\begin{figure}[!htb]
  \centering
  \includegraphics[width=0.55\linewidth]{figures/theory/graphene_rot.png}
  \caption{Average friction force versus rotation angle $\Phi$ of the
  graphite sample around an axis normal to the sample surface.
  Two narrow peaks of high friction are observed at 0$^{\circ}$ and 61$^{\circ}$,
  respectively. Between these peaks, a wide angular range with
  ultra-low friction, close to the detection limit of the instrument, is
  found. The first peak has a maximum friction force of $306 \pm \SI{40}{pN}$, and the second peak has a maximum of $203 \pm \SI{20}{pN}$.
  The curve through the data points shows results from a
  Tomlinson model for a symmetric 96-atom graphite flake
  sliding over the graphite surface (for details about the calculation see~\cite{PhysRevB.70.165418}). Figure and caption adapted from~\cite{DIENWIEBEL2005197}, reproduced from~\cite{Vanossi_2013} with permission from the American Physical Society.}
  \label{fig:graphene_rot}
\end{figure}


\subsubsection{Velocity resosnance} % Velocity resosnance 
While many of the same arguments used for the Prandtl–Tomlinson model regarding velocity dependence for friction can be made for the Frenkel-Kontorova model as well, the addition of multiple atoms introduces the possibility of resonance. In the Frenkel-Kontorova model, the kinetic friction is primarily attributed to resonance between the sliding-induced vibrations and phonon modes in the chain~\cite{FK2D}. The specific dynamics are found to be highly model and dimension specific, and even for the 1D case, this is rather complex. However, we make a simplified analysis of the 1D rigid chain in order to showcase the reasoning behind the phenomenon.

When all atoms are sliding rigidly with center of mass (\acrshort{CM}) velocity $v_{{\text{CM}}}$ the atoms will pass the potential maxima with the so-called \textit{washboard frequency} $\Omega = 2\pi v_{{\text{CM}}} / a_b$. For a weak coupling between the chain and the potential we can use the zero potential case as an approximation for which the known dispersion relation for the 1D harmonic chain is given~\cite[p. 92]{Kittel2004}
\begin{align*}
  \omega_k = \sqrt{\frac{4 K}{m}} \left|\sin{\left(\frac{k}{2}\right)}\right|,
\end{align*}
where $\omega_k$ is the phonon frequency and $k = 2\pi i / N$ the wavenumber with $i\in [N/2, N/2)$. Resonance will occur when the washboard frequency $\Omega$ is close to the frequency of the phonon modes $\omega_q$ in the chain with wavenumber $q = 2\pi a_c / a_b = 2\pi \theta^{-1}$ or its harmonics $nq$ for $n = 1, 2, 3, \hdots$~\cite{van_den_Ende_2012}. Thus, we can approximate the resonance \acrshort{CM} speed as
\begin{align*}
    n \Omega &\sim \omega_{nq} \\
    n \frac{2\pi v_{\text{CM}}}{a_b} &\sim \sqrt{\frac{4K}{m}} \left| \sin{\left(\frac{2n \pi \theta^{-1}}{2}\right)}\right| \\
    v_{\text{CM}} &\sim \frac{\sin{(n\pi \theta^{-1})}}{n \pi} \sqrt{\frac{Ka_b^2}{m}}.
\end{align*}
When the chain slides with a velocity around resonance speed, the washboard
frequency can excite acoustic phonons which will dissipate to other phonon modes
as well. At zero temperature, the energy will transform back and forth between
internal degrees of freedom and \acrshort{CM} movement of the chain. Without any dissipation mechanism, this is theorized to speed up the translational decay~\cite{FK2D}. However, as soon as we add a dissipation channel through the substrate, energy will dissipate from the chain to the substrate's degrees of freedom. This suggests that certain sliding speeds will exhibit relatively high kinetic friction while
others will be subject to relatively low kinetic friction. Simulations of
concentric nanotubes in relative motion (telescopic sliding) support this idea
as it has revealed the occurrence of certain velocities at which the friction is
enhanced, corresponding to the washboard frequency of the
system~\cite{Zhang_2007, Zhang_2009}. The friction response was observed to be highly non-linear as the resonance velocities were approached. 

The analysis of the phonon dynamics is highly simplified here, and a numerical study of the 2D Frenkel-Kontorova model by Norell et al.~\cite{FK2D} showed that the behavior was highly dependent on model parameter choices, but that the friction generally increased with velocity and temperature. This temperature dependence differs qualitatively from that of the Prandtl–Tomlinson model.


\subsection{Frenkel-Kontorova-Tomlinson}
A final extension of the reduced-order models worth mentioning here is the
Frenkel-Kontorova-Tomlinson (\acrshort{FKT}) model~\cite{weiss_dry_1997}, which
introduces a harmonic coupling between the moving body and each of the atoms in the sliding chain, effectively combining the Prandtl–Tomlinson and Frenkel-Kontorova models
(see~\cref{fig:PT_FK_FKT}). This introduces more degrees of freedom to the model
which is based on the intention of achieving a more realistic modeling of the connection between the moving body and the chain. Dong et al.~\cite{Yalin_2011} carried out a numerical analysis
using the 1D Frenkel-Kontorova-Tomlinson model to investigate the effect of
chain length. They observed that the friction generally increased linearly with the number
of atoms in the chain on a long range, but certain lattice mismatches resulted in
local non-linear relationships as shown in~\cref{fig:FKT_contact}. Similarly, by
extending the Frenkel-Kontorova-Tomlinson model to 2D they were able to achieve
a similar sensitivity to commensurability as observed experimentally
by~\cite{DIENWIEBEL2005197} (see~\cref{fig:graphene_rot}). This numerical result is shown in~\cref{fig:FKT_2D_rot}. Besides a demonstration of the
commensurability effect in 2D they also observed increasing friction with an
increasing flake size. Combined, the 1D and 2D results support the idea of
increasing friction with contact size although it might showcase non-linear
behavior depending on commensurability.


\subsection{Shortcomings of the reduced-models}
It should be noted that the reduced-models presented in the previous sections provide a simplified description of the friction behavior. One major limitation is that all models assume a rigid substrate with a constant sinusoidal potential shape. In reality, the potential shape might be more complex and also dynamically changing as the substrate reacts to the sliding motion, resulting in a more intricate system. Moreover, the energy dissipation is simplified through a viscous term $-m\mu \dot{x}$ in the Langevin equation~\cref{eq:Langevin_PT}, which neglects the complexity associated with electron and phonon dissipation. For example, considering phonon dissipation, there exist many vibration modes ($3N$), and thus many dissipation channels for the tip~\cite{Yalin_2011}. Lastly, it should be mentioned that the moving body is assumed to move rigidly with constant speed, whereas, in reality, it may exhibit a more complicated dynamic behavior.


\begin{figure}[!htb]
  \centering
  \begin{subfigure}[t]{0.49\textwidth}
      \centering
      \includegraphics[width=\textwidth]{figures/theory/FKT_contact.png}
      \caption{}
      \label{fig:FKT_contact}
    \end{subfigure}
    \hfill
    \begin{subfigure}[t]{0.49\textwidth}
      \centering
      \includegraphics[width=\textwidth]{figures/theory/FKT_2D_rot.png}
      \caption{}
      \label{fig:FKT_2D_rot}
    \end{subfigure}
    \hfill
     \caption{Friction in the Frenkel-Kontorova-Tomlinson model for varying size and commensurability corresponding to the numerical result by Dong et al.~\cite{Yalin_2011}. The spring constant is $K_t$ for the interatomic coupling and $K$ for the coupling to the moving body. (a) The 1D case with an increasing number of atoms in the chain and different mismatch length ratios $\theta = a_b / a_c$. The figure notation corresponds to $a_b = a$ and $a_c =b$ yielding $b/a = \theta^{-1}$. The model parameters are $K = \SI{5}{N/m}$ and $K_t = \SI{50}{N/m}$. (b) The 2D case with varying angles (misfit angle) between the flake and the substrate. The model parameters are $K = \SI{10}{N/m}$ and $K_t = \SI{50}{N/m}$. Reproduced from~\cite{Yalin_2011}.}
     \label{fig:FKT_size}
\end{figure}



\subsection{Experimental procedures}
Experimentally, the study of nanoscale friction is challenging due to the low
forces on the scale of nano-newtons along with the difficulties of mapping the
nanoscale topography of the sample. In contrast to numerical simulations, which provide full
transparency regarding atomic-scale structures, sampling of forces, velocities
and temperature, the experimental results are limited by the state-of-the-art
experimental methods. To facilitate the comparison of numerical and experimental results, we will address a few of the most relevant experimental methods.


\subsubsection{Scanning Probe Microscopy}\label{sec:SPM} 
Scanning probe microscopy (\acrshort{SPM}) includes a variety of experimental
methods which are used to examine surfaces with atomic resolution~\cite[pp.
6--27]{BHUSHAN20051507}. This was originally developed for surface topography
imaging, but today it plays a crucial role in nanoscale science as it is used
for probe-sampling regarding tribological, electronic, magnetic, biological and
chemical character. The family of methods involving the measurement of forces is
generally referred to as \textit{scanning force microscopy} (\acrshort{SFM})
or for friction purposes \textit{friction force microscopy} (\acrshort{FFM}).

One such method arose from the \textit{atomic force microscope} \acrshort{AFM}, which consists of a sharp micro-fabricated tip attached to a cantilever force sensor, usually with a sensitivity below \SI{1}{nN} all the way down to pN. The force is measured by recording the bending of
the cantilever, either as a change in electrical conduction or more commonly, by monitoring a light beam reflected from the back of the cantilever into a photodetector~\cite[p. 183]{gnecco_meyer_2015} as shown in~\cref{fig:AFM}. By adjusting the tip-sample height to keep a constant normal force while scanning across the surface, the \acrshort{AFM} can be used to produce a
surface topography map.  However, when scanning perpendicularly to the cantilever axis, the frictional force can be measured as the torsion of the cantilever. By utilizing a photodetector with four quadrants (as depicted in~\cref{fig:AFM}), the normal force and friction force can be simultaneously measured as the probes scan across the surface. \acrshort{AFM} can also be utilized to drag a nanoflake across the substrate, as demonstrated by Dienwiebel et al.~\cite{DIENWIEBEL2005197}, who attached a graphene flake to a \acrshort{AFM} tip and dragged it across graphite. However, it should be noted that this method concentrates the normal loading to a single point on the flake, rather than achieving an evenly distributed load.


\subsubsection{Surface Force Apparatus}
Another method worth mentioning is the Surface Force Apparatus (\acrshort{SFA}), which consists of two curved, molecularly smooth surfaces brought into contact~\cite[p. 188]{gnecco_meyer_2015}. The material of choice is usually mica since it can be easily cleaved into atomically flat surfaces over macroscopic areas. The sample is then placed between the two surfaces as a lubricant film, and the friction properties can be studied by applying a tangential force to the surfaces. This method provides a uniform load distribution on the surface as opposed to the setup of dragging a nanoflake by an \acrshort{AFM} tip.

\begin{figure}[!htb]
  \centering
  \includegraphics[width=0.5\linewidth]{figures/theory/AFM.png}
  \caption{Schematic diagram of a beam-deflection atomic force microscope. Figure and caption reproduced from~\cite[p. 184]{gnecco_meyer_2015}.}
  \label{fig:AFM}
\end{figure}


\section{Summary of previous results}\label{sec:prev_results}
Several studies have investigated the frictional behavior of graphene by varying
different parameters such as normal force, sliding velocity, temperature,
commensurability and graphene thickness~\cite{penkov_tribology_2014}. In
general, we find three types of relevant systems being studied: 1) An
\acrshort{FFM}-type setup where the graphene, either resting on a substrate or
suspended, is probed by an \acrshort{AFM} tip scanning across the surface. 2) A
\acrshort{SFA} setup with the graphene ``sandwiched'' in between two substrate
layers moving relative to each other using the graphene as a solid lubricant. 3)
A graphene flake sliding on a substrate, either being dragged by an
\acrshort{AFM} tip or by a more complex arrangement in numerical simulations.
Considering that even the sharpest \acrshort{AFM} tip will effectively put
multiple atoms in contact with the sample, all methods are relatable to the
study of nanoscale surface contact. However, the \acrshort{FFM}-type is more
closely related to asperity theory since it is expected to deform under
increasing load, while the latter two are more aligned with the
Frenkel-Kontorova type models and our specific system of interest. Nonetheless,
we will consider results across all three types of systems. The most relevant studies considered are summarized in~\cref{tab:friction_ref} for convenience. 


\begin{table}[!htb]
  % \begin{center}
  \centering
  \caption{A summary of the most relevant studies considered for the review of previous results in~\cref{sec:prev_results}. The table provides a distinction between the different systems being studied: \acrshort{FFM}, \acrshort{SFA} or flake on a substrate, as well as whether they were carried out numerically (num.) or experimentally (exp.).}
  \label{tab:friction_ref}
  \begin{tabular}{ |M{1cm}|M{1cm}|M{1.5cm}|X{2.8cm}|X{4cm}|X{4cm}| } \hline
  System & Type & Year & Researcher & Materials & Keywords \\ \hline
  \parbox[t]{2mm}{\multirow{10}{*}{\rotatebox[origin=c]{90}{\acrshort{FFM}}}} & \multirow{3}{*}{Exp.} & 2007~\cite{zhao_thermally_2007} & Zhao et al.\ & Si\textsubscript{3}N\textsubscript{4} tip on graphite. & Temperature dependence. \\ \cline{3-6} 
  & & 2015~\cite{Paolicelli_2015} & G. Paolicelli et al.\ & Si tip, graphene on SiO2 and Ni(111) substrate.  &  Load, environment, layer thickness. \\ \cline{2-6} 
  & Both & 2019~\cite{zhang_tuning_2019} & Zhang et al.\ & Monolayer graphene.  & Straining of the sheet. \\ \cline{2-6} 
  & \multirow{3}{*}{Num.} & 2015~\cite{Yoon2015MolecularDS} & Yoon et al.\ & Si tip, graphene on SiO\textsubscript{2}. & Stick-slip: tip size, scan angle, layer thickness, substrate flexibility. \\ \cline{3-6} 
  & & 2016~\cite{li_evolving_2016} & Li et al.\ & Si tip, graphene on amorph-Si substrate. & Layer thickness, friction strengthening, stick-slip. \\ \cline{1-6} 
  \parbox[t]{2mm}{\multirow{3}{*}{\rotatebox[origin=c]{90}{\acrshort{SFA}}}} & \multirow{3}{*}{Num.} & 2011~\cite{Wijn_2011} & Wijn et al.\ & Graphene flakes between graphite.  & Commensurability, rotational dynamics, superlubricity, temperature.  \\ \cline{3-6} 
  & & 2012~\cite{Kim_2012} & H.\ J.\ Kim and D.\ E.\ Kim. & Carbon sheet and nanotubes.  & Corrugated nano-structured surfaces.  \\ \cline{1-6} 
  \parbox[t]{2mm}{\multirow{18}{*}{\rotatebox[origin=c]{90}{Flake}}} & \multirow{5}{*}{Exp.} & 2005~\cite{DIENWIEBEL2005197} & Dienwiebel et al.\ & Graphene on graphite & Commensurability, superlubricity, load.  \\ \cline{3-6} 
  &  & 2013~\cite{feng_superlubric_2013}  & Feng et al.\ & Graphene on graphite. &  Commensurability, superlubricity, temperature.  \\ \cline{2-6} 
  & \multirow{13}{*}{Num.} & 2009~\cite{bonelli_atomistic_2009} & Bonelli et al.\ & Graphene on graphite.  & Tight-binding, commensurability, load, flake size. \\ \cline{3-6} 
  &  & 2012~\cite{Reguzzoni_2012} & Reguzzoni et al.\ & Graphene on graphite. & Layer thickness. \\ \cline{3-6} 
  &  & 2014~\cite{liu_high-speed_2014} & Liu et al.\ & Graphene on graphite. & High speed, superlubricity, rotational dynamics, sheet strain. \\ \cline{3-6} 
  &  & 2018~\cite{zhu_study_2018} & P. Zhu and Li & Graphene on gold. & Stick-slip, commensurability, flake size and shape. \\ \cline{3-6} 
  &  & 2019~\cite{ma12091425} & Zhang et al.\  & Graphene on diamond. & Temperature, commensurability, friction coefficient.  \\ \cline{1-6} 
\end{tabular}
% \end{center}
\end{table}


% Stick-slip
One of the earliest tribological simulations of graphene was carried out by
Bonelli et al.~\cite{bonelli_atomistic_2009} in 2009 using a tight-binding\footnote{The tight-binding method involves computing the electronic structure of the system, but it uses a semi-empirical approach to reduce the computational cost of the calculations. Thus, this method lies between traditional \acrshort{MD} and more expensive ab initio methods~\cite{colombo_tight-binding_2005}.} method (excluding thermal excitations) to simulate a graphene flake on an
infinite graphene sheet~\cite{penkov_tribology_2014}. They implemented a
Frenkel-Kontorova-Tomlinson-like setup where each atom in the flake is coupled horizontally
to a rigid support by elastic springs. They recovered the stick-slip behavior,
which is also observed in \acrshort{FFM} setups both experimentally~\cite{zhao_thermally_2007, zhang_tuning_2019} and numerically~\cite{li_evolving_2016, zhu_study_2018}. Moreover, they found an agreement with
the qualitative observation that soft springs allow for a clean stick-slip
motion while hard springs ($\sim \SI{40}{N/m}$) inhibited it. This also aligns with the predictions of the Prandtl–Tomlinson and Frenkel-Kontorova models. In \acrshort{AFM} and \acrshort{SFA} experiments,
the stick-slip motion tends to transition into smooth sliding when the speed
exceeds $\sim \SI{1}{\mu/s}$ while in \acrshort{MD} modeling the same transition
is observed in the $\sim \SI{1}{m/s}$ region~\cite{Manini_2016}. More precisely
Liu et al.~\cite{liu_high-speed_2014} finds this transition in \acrshort{MD}
simulations at \SI{15}{m/s}. This 6-order-of-magnitude discrepancy has been
largely discussed in connection to simplifying assumptions in \acrshort{MD}
simulations. On the other hand, the Prandtl–Tomlinson model qualitatively disagrees
as it predicts smooth sliding for low speeds only.
However, in an extension of the Prandtl–Tomlinson for the study of nanoscale rolling friction by Sircar and Patra~\cite{Sircar_2020}, they found smooth sliding for high speeds as well.


% Commensurability
Bonelli et al.~\cite{bonelli_atomistic_2009} also found that commensurability,
through orientation of the flake and the direction of sliding, had a great
impact on the frictional behavior which generally aligns with the predictions of
the Frenkel-Kontorova models. They confirmed qualitatively the
observation of superlubricity for certain incommensurable orientations which has
been reported in experiments by Dienwiebel et al.\cite{DIENWIEBEL2005197} and
further supported by experimental measurements of interaction energies by Feng
et al.~\cite{feng_superlubric_2013}. The importance of commensurability is also
reported for \acrshort{MD} simulations~\cite{ma12091425, zhu_study_2018,
Wijn_2011}. Bonelli et al.\ found the friction force and coefficient to be one
order of magnitude higher than that of the experimental results which they
attribute to the details of the numerical modeling. Generally, the experimental
coefficients between graphite and most materials lie in the range of 0.08--0.18~\cite{DIENWIEBEL2005197}. While Dienwiebel et al.~\cite{DIENWIEBEL2005197}
reported a wide range of frictional forces from $28 \pm \SI{16}{pN}$ to $453 \pm \SI{16}{pN}$ with loads $\sim [-10, 20]$ nN, the change in friction with applied load was
as low as 0.05--0.4\% for the incommensurable orientations. When using the slope definition for the frictional coefficient \cref{eq:mu_def2}, this corresponds to a coefficient in the range of 0.0005--0.004. Bonelli et al.\ attribute the low dependency to a lacking change in contact area as the flake is loaded. 

% Flake size
Furthermore, Bonelli et al.~\cite{bonelli_atomistic_2009} found friction to
decrease with increasing flake size which is also reported in \acrshort{MD}
simulations for graphene on gold~\cite{zhu_study_2018}. Bonelli et al.\ mainly
attribute this to boundary effects, but also note that the coupling to the
support in their simulations made for decreased rotational freedom as flake size was increased. Thus,
they hypothesized that the decreased freedom led to the graphene taking a more
forced path which is associated with a decreased stick-slip behavior. However, the
general observation disagrees with the Frenkel-Kontorova
models which predict the reverse; an increase in friction with increasing size. 

An additional numerical study of monolayer islands of Kr on Cu by
Reguzzoni and Righi~\cite{PhysRevB.85.201412} supports the importance of
commensurability regarding size effects. They report that the effective
commensurability increases drastically below a critical flake radius on the order
of \SI{10}{\text{Å}}. In a numerical study by Varini et al.~\cite{Varini_2015},
based on Kr islands adsorbed on Pb(111), this is further elaborated as they
found that finite size effects are especially important for static friction due
to a pinning barrier arising from the edge, preventing otherwise superlubricity
due to incommensurability. They reported a relationship $F_s \sim A^{\gamma_s}$
not only sublinear, $\gamma_s < 1$, but also sublinear with respect to the
island perimeter, $P \propto A^{1/2}$, by having $\gamma_s = 0.25$ for a
hexagonal edge and $\gamma_s = 0.37$ when circular, indicating that only a
subset of the edge is responsible for the pinning effect. This aligns with the
general change in friction found by Zhu and Li~\cite{zhu_study_2018} for different flake
geometries (square, triangle, circle). Additionally, Varini et al.\ found the
edge pinning effect to decrease with increasing temperature as the edge energy
barriers are reduced. Bringing all this together, the main picture forming is
that flake size, which can be related to contact area, is affecting friction
through a commensurability mechanism. If the flake is constrained in some way we
might not observe the same dependency. While flake size nor contact area is
easily measured in experimental \acrshort{FFM}, Mo et
al.~\cite{mo_friction_2009} found in an \acrshort{MD} simulation that friction
is proportional to contact area for an indenting sphere on a nanoscale.


% Evolution effects and graphene layers
Evolution effects, or so-called friction strengthening, are also found. This means that friction increases during the initial stick-slip cycles, which is observed
experimentally by Zhang et al.~\cite{zhang_tuning_2019} and numerically by Li
et al.~\cite{li_evolving_2016}. However, this is only found when having the
graphene sheet resting on a substrate~\cite{zhang_tuning_2019}, as opposed to a
suspended sheet. It is also found to diminish with an increasing number of
graphene layers stacked (graphite)~\cite{li_evolving_2016}. Multiple studies report a general decrease in friction with an increasing number of layers~\cite{li_evolving_2016, Yoon2015MolecularDS, Paolicelli_2015, Filleter_2009, Lee_2010}, but the opposite trend has also been reported~\cite{Reguzzoni_2012}.


% Deformations
A few numerical studies have investigated friction under mechanical deformations.
Zhang et al.~\cite{zhang_tuning_2019} found that straining a suspended graphene
sheet will lower the kinetic friction. They attribute this to a modulation of
flexibility which consequently changes the local pinning capability of the
contact interface. Liu et al.~\cite{liu_high-speed_2014} carried out an
\acrshort{MD} simulation of high-speed ballistic nanofriction ($\SI{400}{m/s}$)
of graphene on graphite. They found that a biaxial stretching of the graphite 
substrate could be used to suppress frictional scattering and achieve persistent
superlubricity. Another surface manipulating study was performed by H.\ J.\ Kim
and D.\ E.\ Kim~\cite{Kim_2012} who investigated the effects of corrugated
nano-structured surfaces. The study revealed that the corrugation of the surfaces, involving an altering of the contact areas and structural stiffness, could result in both increased or
slightly decreased friction under certain load ranges. Altogether, these studies
highlight the importance of surface structure and mechanical conditions. 


% Normal load
The friction dependency of normal load turns out to be a complex matter and has
proven to be a highly system-dependent feature. As already mentioned, asperity
theory mainly points to a sublinear relationship between friction and load,
while the reduced-models point to a more intricate relationship through
the change of the effective substrate potential which leads to an altering of
the commensurability and the phonon dynamics. Experimentally rather different trends have been observed, although the majority agree on increasing friction with
increasing load~\cite[p. 200]{gnecco_meyer_2015}. For the graphene flake,
Dienwiebel et al.~\cite{DIENWIEBEL2005197} found a seemingly non-dependent
relationship while a \acrshort{FFM} study by G. Paolicelli et al.~\cite{Paolicelli_2015} yielded a sublinear relationship matching the predictions of
Maugis-Dugdale theory $(F_{\text{fric}} \propto (F_N - F_{N,0})^{2/3})$. This discrepancy
might be attributed to the difference in system type; a spherical tip indenting the graphene sheet as opposed to the atomic flatness of the graphene-graphite
interface, which does not make for a changing contact area under load. However,
numerical studies using a graphene-graphite interface still find both sublinear~\cite{bonelli_atomistic_2009} and linear~\cite{ma12091425, zhang_tuning_2019}
load dependencies. 

In an experimental \acrshort{FFM} study by Deng et al.~\cite{deng_adhesion-dependent_2012} it was discovered that
the friction force kept increasing after unloading the probe tip from the
graphite surface. This has been argued to be a general phenomenon
related to hysteresis in the adhesive interaction between two sliding bodies~\cite{thormann_negative_2013}. Following the slope definition for the friction
coefficient, these results correspond to a negative friction coefficient. More recently, a negative friction coefficient has also been observed for the loading phase by Liu et al.~\cite{Liu_2020} in an experimental study of the interface between graphite and muscovite mica heterojunction. With supporting numerical modeling this is attributed to ``synergetic and nontrivial redistribution of water molecules at the interface''. Similar results are also reported numerically by Mandelli et al.~\cite{Mandelli_2019} for graphite in contact with hexagonal boron nitride heterojunctions which is attributed to ``load-induced suppression of the moiré superstructure out-of-plane distortions leading to a less dissipative interfacial dynamics''. Thus, the concept of a negative friction coefficient has been proven for the unloading phase of adhesive contacts and in the loading phase for a few specific systems. 

% Speed
The dependency of velocity is generally found to increase logarithmically with
velocity in experimental \acrshort{AFM} studies~\cite[p. 201]{gnecco_meyer_2015}
which match the low-velocity regime of the Prandtl–Tomlinson type models. At
higher velocities, thermally activated processes are less important and friction
becomes independent of velocity according to the friction saturation of the Prandtl–Tomlinson model~\cref{eq:F_thermal_ac} when ignoring the athermal regime. Saturation of the
velocity dependency has been observed numerically for Si tips interacting with
diamond, graphite and amorphous carbon surfaces respectively with scan
velocities above \SI{1}{\mu/s}~\cite{zworner1998velocity}. However, when
considering the effects of damping the Prandtl–Tomlinson model predicts an
athermal regime with viscous friction, i.e.\ friction being proportional to
sliding velocity. Guerra et al.~\cite{Guerra_2010}, studying gold clusters on
graphite using \acrshort{MD} simulations, found a viscous friction response in
both low and high speed domains. In addition, thermal effects reversed as they found
friction to decrease with increasing temperature at low speed (diffusive regime)
but found friction to increase with temperature at high speed (ballistic
regime). This crossover from the diffusive to the ballistic regime occurred between 1 and \SI{10}{m/s}. 


% Temperature
Regarding temperature, the general experimental trend is decreasing friction with increasing temperature as found by Zhao et al.~\cite{zhao_thermally_2007} in a series of \acrshort{AFM} graphene on graphite experiments yielding $F_{\text{fric}} \propto \exp{(1/T)}$. This agrees with the dominant term in the thermal drift regime of the Prandtl–Tomlinson model even though the exact temperature range does not agree. Moreover, Wijn et al.~\cite{Wijn_2011} found that friction commensurability can be lost at higher temperatures (above \SI{200}{K}) where they found a power law behavior $F_{\text{fric}} \propto T^{-1.13 \pm0.04}$. Numerically, Zhang et al.~\cite{ma12091425} found that friction increased with temperature, using a sliding speed of \SI{10}{m/s}. Considering the findings of Guerra et al.~\cite{Guerra_2010} this qualitative different behavior can be attributed to the transition from low speed diffusive friction to high speed ballistic friction in the \acrshort{MD} simulations.
\\
\\
From the review of previous results, we find several gaps and discrepancies in
the description of friction provided by the reduced-models, \acrshort{MD}
simulations and experimental methods respectively. Some of the discrepancies can
be attributed to the fact that different physical mechanisms are included in the 
numerical modeling. The reduced-models provide a simplistic description, while the \acrshort{MD} simulations are expected to capture a more complex behavior. We might also point to differences in the studied systems as an
important factor to consider. This includes the physical conditions such as
sliding speed and temperature, but also higher-level features related to the
mechanical properties of the system. For instance, the \acrshort{FFM}-based
results consider an asperity-like system where the tip is expected to deform
under loading, which gives rise to a change in the contact area. This feature is
lacking for the flake on a substrate, and thus we might question the role of
the contact area in these systems. More precisely, when inflicting an
out-of-plane buckling through Kirigami cuts and stretching, the contact area is
expected to decrease as well for which asperity theory predicts a decrease in friction. However, as the system undergoes deformation, it may also lead to a change in commensurability, which can result in significant modifications to the friction due to its effect on stick-slip behavior. Based on the results obtained from a non-cut sheet under tension, there are indications that strain alone can lead to a reduction in friction, even without taking into account the contact area. Similarly, the findings from a corrugated nanosurface suggest that surface stiffness may also be a significant factor in determining friction for our nanoscale Kirigami system.


\section{Research questions}\label{sec:research_questions}
Based on the review of friction presented in~\cref{chap:friction}, it is evident that the behavior of friction is influenced by various factors, such as the specific system under investigation, the numerical modeling approach, and the physical conditions related to the environment and the probing of friction. In our study, we aim to investigate the frictional behavior
of a Kirigami sheet under the effects of strain. Previous studies have
demonstrated that strained Kirigami sheets are prone to out-of-plane buckling~\cite{PhysRevLett.121.255304, PhysRevResearch.2.042006} which is indicative of a possible transition between two distinct systems: An atomically flat interface and an asperity system. These systems are usually only studied
separately, and therefore, our primary objective is to investigate the possible frictional effects linked to strain-induced system transformations. In particular, we want to investigate the significance of the contact area and evaluate the hypothesis that reducing the contact area will lead to a decrease in friction. Additionally, we seek to examine the relationship between the friction-load curve and this phenomenon. For the sake of contributing new insight to the field of nanoscale
friction, we are interested in non-linear dependencies between
friction and strain for various Kirigami designs. Drawing on this
perspective, we aim to investigate the prospects of achieving a negative friction coefficient for a system of coupled load and strain. In order to contextualize our findings within the theoretical framework, we will take into account the results from prior studies.


To gain a more comprehensive understanding of the potential applications of Kirigami design, we aim to develop a dataset based on \acrshort{MD} simulations that capture the frictional effects on Kirigami designs when subjected to strain and load. We intend to employ machine learning techniques to discern any meaningful trends in the data that may be used to inform future research endeavors. Specifically, we seek to leverage the machine learning model to facilitate an accelerated search for optimizing specific frictional properties. Our focus will be on evaluating the prospects of reducing or increasing the friction force, as well as reducing or increasing the friction coefficient for a coupled system of load and strain. Our main research questions can be summarized as follows.


\begin{enumerate}
  \item  How can we design an \acrshort{MD} simulation that provides a reliable foundation for an investigation of the frictional behavior for a Kirigami graphene sheet sliding on a substrate? How do physical conditions such as temperature and sliding speed control friction?
  \item How can we design an ensemble of Kirigami patterns for the investigation of its frictional properties with the scope of getting out-of-plane buckling and also randomized design features?   
  \item Can we control friction for a Kirigami sheet through Kirigami pattern design and straining of the sheet?
  \begin{enumerate}
    \item Does friction dependent on a changing contact area?
    \item How does the friction-load curve relate to strained Kirigami sheets?
    \item Are the effects of strain and pattern design significant when considered independently?
    \item Is the frictional behavior consistent with the Prandtl–Tomlinson, Frenkel-Kontorova and Frenkel-Kontorova-Tomlinson models? 
  \end{enumerate}
  \item Is it possible to utilize machine learning to identify general trends in the relationship between friction and kirigami patterns, strain and load?
  \item Can we use a trained machine learning model to predict new designs through an accelerated search?
  \item What are the prospects of achieving a negative friction coefficient for a system of coupled load and strain through Kirigami design?
\end{enumerate}