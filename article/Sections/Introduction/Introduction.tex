\chapter*{Introduction}
\addcontentsline{toc}{chapter}{Introduction} 
Introduction. A citation to avoid error for now: \cite{li_evolving_2016}.

\begin{itemize}
    \item Nanotribology
    \item Quantitative Structure-Property Relationship
    \item Forward simulation using ML
    \item Inverse designs
\end{itemize}


Practically, systems achieving low values of dry sliding friction are of great technological interest to significantly reduce dissipation and wear in mechanical devices functioning at various scales. (Current trends in the physics of nanoscale friction)

These experiments have demonstrated that the relationship between friction and surface roughness is not always simple or obvious. (Introduction to Tribology, p. 527).


“In other words, it’s not just the material itself” that determines how it slides, but also its boundary condition — including whether it is loose and wrinkled or flat and stretched tight, he says. (https://news.mit.edu/2016/sliding-flexible-graphene-surfaces-1123).<-- Talking about quality of contact for friciton.