\chapter*{Introduction}
\addcontentsline{toc}{chapter}{Introduction} 

\section{Some headline for the introtext}

\section{Introduction to friction}
friction a part of the (wider) field tribology.

\begin{itemize}
    \item Where is friction important (motivation)
    \item The economical interest in tribology (quote)
    \item The missing knowdelegde about friction.
    \item What possibilities do we have if we could control friction (friction coefficient).
\end{itemize}


\section{Introduction to MD simulations and machine learning approaches}

\section{Defining the goal of the thesis and restrictions}
Make bullet point objectives for the thesis and state which is completed, which is perhaps not conslusive and which I did not answer at all / do at all. Perhaps also make a list of problems/questions to answer (also state which one I actually answer here).


\section{Contributions}

\section{Thesis structure }

\newpage
Introduction. A citation to avoid error for now: \cite{li_evolving_2016}.

\begin{itemize}
    \item Nanotribology
    \item Quantitative Structure-Property Relationship
    \item Forward simulation using ML
    \item Inverse designs
\end{itemize}


Practically, systems achieving low values of dry sliding friction are of great technological interest to significantly reduce dissipation and wear in mechanical devices functioning at various scales. (Current trends in the physics of nanoscale friction)

These experiments have demonstrated that the relationship between friction and surface roughness is not always simple or obvious. (Introduction to Tribology, p. 527).


“In other words, it’s not just the material itself” that determines how it slides, but also its boundary condition — including whether it is loose and wrinkled or flat and stretched tight, he says. (https://news.mit.edu/2016/sliding-flexible-graphene-surfaces-1123).<-- Talking about quality of contact for friciton.