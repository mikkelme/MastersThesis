\chapter{Summary}\label{chap:summary}
In this thesis, we have studied the nanoscale friction of a Kirigami graphene
sheet under the influence of strain using molecular dynamics (\acrshort{MD})
simulations. We have developed a numerical tool for generating diverse Kirigami
designs which we have utilized to create a dataset for the frictional behavior
depending on Kirigami pattern, strain, and loading. Our findings suggest that
the frictional behavior of a Kirigami sheet is highly dependent on the geometry
of the pattern and the strain conditions. We observed that the out-of-plane
buckling can be associated with a non-linear friction-strain relationship which
can be utilized to demonstrate a negative friction coefficient in a system with
coupled load and strain. Moreover, we have investigated the possibility to use
machine learning on this dataset and we have attempted an accelerated search for
the optimization of various friction properties. Our findings imply that machine
learning can be feasible for this approach, but additional data is required to
establish a more reliable foundation for the prediction on new Kirigami
patterns. In this chapter, we will provide a summary of our findings and draw
conclusions based on the results obtained. Finally, we will suggest some topics
for further research.


\section{Summary and conclusions}

% 1: Design MD simulations
\subsection{Designing an MD simulation}
We have designed an \acrshort{MD} simulation for the examination of friction for
a graphene sheet sliding on a silicon substrate. The key system features were
the introduction of the pull blocks, defined as the end regions of the sheet
with respect to the sliding direction, which was utilized for applying normal
load and sliding the sheet. The pull blocks were made partly rigid and used to
employ a thermostat as well. Through an analysis of the friction forces retrieved from sliding simulations, we have established a standardized metric for kinetic friction. In particular, we measured the force exerted by the substrate on the full sheet, including the pull blocks, with respect to the sliding direction. We then determined the kinetic friction as the force mean value of the last half of the simulation. The uncertainties were estimated based on the fluctuations in the running mean. We found that the assessment of static friction was ambiguous for our simulation and did not pursue this further. From the analysis of the
force traces, friction force vs.\ time, we identify the friction behavior in our
simulation domain as being in the smooth sliding regime. This is attributed to the choice of a relatively high sliding speed (\SI{20}{m/s}) and infinitely stiff springs for the tethering of the sheet. This was supported by a demonstration of a transition to the stick-slip regime through the use of softer springs and a decrease in sliding speed. By conducting a more systematic investigation of the
effects of temperature, sliding speed, spring constant and timestep, we identified a set of default values based on numerical stability and computational cost. During this process, we aimed to select the variables that would maintain relatively stable measures for friction with moderate perturbations around these default values. We found that friction increased with temperature which is in disagreement with the Prandtl–Tomlinson model and most experimental results. However, this agrees with the predictions from the Frenkel-Kontorova models and other \acrshort{MD} studies which attribute this to ballistic sliding. In the absence of clear indications from the investigation regarding an appropriate temperature, we opted for the standard choice of room temperature, \SI{300}{K}. Furthermore, we found friction to increase with velocity as
expected from other studies and the Prandtl–Tomlinson model, with some signs of phonon resonance at certain sliding speeds as well which aligns with the predictions from the Frenkel-Kontorova models. We chose a rather high velocity of \SI{20}{m/s} mainly for the consideration of
computational costs. For the spring constant, we found decreasing friction with
increasing stiffness of the springs. This is associated with the transition from
a stick-slip-influenced regime toward smooth sliding and can be attributed to an underlying change in commensurability. This is predicted by the Frenkel-Kontorova models and supported by both numerical and excperimental results. The choice of an
infinitely stiff spring was made from an assessment of the variation in friction with perturbations in the spring constant value. Finally, we
confirmed that a timestep of \SI{1}{fs} provides reasonable numerical stability.
However, based on fluctuations with timestep we find that the uncertainty in the
simulations might be higher than first estimated. For the non-strained Kirigami sheet, these fluctuations were on the order of $\pm \SI{0.017}{nN}$ for the evaluation of the kinetic friction.


% 2: Design Kirigami tool
\subsection{Generating Kirigami patterns}
In order to investigate the effects of Kirigami design we have created a
numerical tool for generating various patterns. By defining an indexing system
for the hexagonal lattice structure we were able to define the Kirigami designs
as a 2D binary matrix for numerical implementation. We have selected two
macroscale designs, which we denote the \textit{Tetrahedron} and
\textit{Honeycomb} patterns, based on their ability to exhibit out-of-plane
buckling when subjected to strain. By digitalizing the designs to match the
hexagonal graphene lattice, we found that the characteristic design features can
be translated to the nanoscale, as we observed similar out-of-plane buckling in
\acrshort{MD} simulations. Through our numerical tool we were able to create an
ensemble of perturbed unique variations which yielded approximately \num{1.35e5}
and \num{2.025e6} unique configurations for the Tetrahedron and Honeycomb
patterns respectively. When considering the possibility to translate the
periodic patterns on the sheet, the number of possible patterns can be increased
by approximately a factor of 100. To introduce some random design features, we
have developed a tool for generating Kirigami patterns based on random walks.
The tool includes mechanisms such as bias, avoidance of existing cuts,
preference for maintaining a direction, and procedures for repairing the sheet
for simulation purposes. In general, we found that the capabilities of the
numerical tools for generating Kirigami designs exceeded our computational
resources with regard to performing \acrshort{MD} simulation under different
loads and strains for each of the designs. Our \acrshort{MD}-based dataset only
utilized a subset of configurations with 9660 data points based on 216 Kirigami
configurations (Tetrahedron: 68, Honeycomb: 45, Random walk: 100, Pilot study:
3). Hence, we argue that the Kirigami generative tool can be valuable for
further studies on an extended dataset.


% 3: Control friction using Kirigami
\subsection{Friction control using Kirigami design and strain}
We have investigated the frictional behavior of the Tetrahedron and Honeycomb
patterns in comparison to a non-cut sheet under various strains and loads.
Initially, we observed that straining the Kirigami sheets in a vacuum resulted
in an out-of-plane buckling. When adding the substrate to the simulation this
translated into a decreasing contact area with strain. We found the Honeycomb
sheet to exhibit the most significant buckling with a corresponding reduction of
relative contact to approximately 43\%, whereas the non-cut sheet did not
produce any significant buckling in comparison. For the Kirigami sheets, we
found that friction initially increased with strain, which made for increasing
friction with decreasing contact area. As the strain continued to increase the
friction-strain curve exhibited highly non-linear trends with strong negative
slopes as well (see~\cref{fig:multi_stretch}). During the full strain, the
contact area was decreasing monotonically except for a slight increase just
before rupturing. These results contradict the asperity theory hypothesis of
decreasing friction with decreasing contact area, which is also supported by the
predictions of the 2D Frenkel-Kontorova models suggesting increasing friction
with (contacting) system size. Thus, we conclude that the contact area is not a
governing mechanism for the friction-strain relationship observed. From the
investigation of the friction-load relationship, we found that both friction and
contact area increased slightly with load, but we were not able to find
significant evidence of any linear relationship between friction and contact
area as predicted by asperity theory. The non-cut sheet did not show any
dependency on the strain as both the friction and the contact area remained
constant with strain. Thus our findings suggest that a changing contact area and
the strain-induced friction effects might be associated with an underlying
mechanism related to the buckling of the sheet. When analyzing the independent
effect of the non-strained Kirigami sheets, we found a slight increase in
friction between the non-cut sheet and the Kirigami sheets. However, this
increase was one order of magnitude lower than the friction changes induced by
the strain in combination. Therefore, we can conclude that the observed friction
behavior cannot be attributed solely to the effects of the non-strained Kirigami
sheet or the tension induced by the strain in a non-cut sheet. When considering the friction dependency with load, we generally found a weak
dependency corresponding to a friction coefficient on the order of
\num{e-4}--\num{e-5} even though we could not confirm any clear relationship.
This is best attributed to the superlubric state of the graphene sheet as seen in
other studies as well. The slope of the friction-load curve was not considerably
affected by the straining of the Kirigami sheet which led us to the conclusion
that strain-induced effects are dominant in comparison to any load-related
effects. 


When considering our findings in light of previous related results we find it
plausible that the governing mechanism for the observed friction effects is
related to commensurability as predicted by the Frenkel-Kontorova models. Since we have observed extremely low friction in
our system, we argue that the non-deformed sheet corresponds to an
incommensurable configuration. This is supported by the fact that we were not
able to lower the friction below the original starting point. Additionally, this
also aligns with the observation that the introduction of the non-strained
Kirigami designs increased friction slightly. Since the Kirigami designs correspond to the removal of atoms from the sheet, it can be hypothesized to relax the incommensurability to some extent. When the Kirigami sheet buckles during stretching, it allows for a
considerable rearrangement of the atoms in contact with the substrate. Hence, it
may transition in and out of commensurable configurations, which could explain
the non-linear trend for the friction-strain curve. One way to test this hypothesis is to alter the simulation conditions such that the non-strained sheet starts in a commensurable phase. This might be achieved by a softening of the springs for tethering and a lowering of the sliding speed since this was found to yield stick-slip behavior which can be associated with a commensurable phase.  Another way is to reorient the sheet or change the sliding direction as reported in both numerical and experimental results. Then we might find a transition from a commensurable to an
incommensurable case during the initial straining which would result in a
lowering in friction with respect to the starting point. 



% 4: Capture trends with ML
\subsection{Capturing trends with machine learning}
By utilizing the numerical tool for generating Kirigami designs we have
created a \acrshort{MD}-based dataset for the frictional behavior depending on
Kirigami design, load and strain. The dataset reveals some general correlations
with mean friction, such as a positive correlation to strain (0.77) and porosity
(0.60), and a negative correlation to contact area ($-0.67$). These results
align with the finding in the pilot study, suggesting that the change in
friction is associated with cuts in the sheet (porosity) and a changing contract area indicating out-of-plane buckling. 

By defining the friction property metrics: $\min F_{\text{fric}}$, $\max
F_{\text{fric}}$, $\max \Delta F_{\text{fric}}$ and max drop (maximum decrease
in friction with strain), we investigated the top design candidates within our
dataset. From these results, we found no indication of the possibility to
reduce friction with the Kirigami approach since the non-cut sheet provided the
overall lowest friction. Furthermore, among the top candidates, we found that a
flat friction-strain profile is mainly associated with little decrease in the
contact area and vice versa. These observations are consistent with the results
of the pilot study and support the hypothesis that commensurability plays a key
role in governing the behavior of the system. In terms of the maximum
properties, we observed an improvement compared to the values obtained in the
pilot study, with the Honeycomb patterns exhibiting the highest scores. This
indicates that the dataset contains some relevant information for optimizing these properties since it includes examples of design improvements.

For the machine learning investigation, we have implemented a VGGNet-inspired
convolutional neural network with a deep ``stairlike'' architecture:
C32-C64-C128-C256-C512-C1024-D1024-D512-D256-D128-D64-D32, for convolutional
layers $C$ with the number denoting channels and fully connected (dense) layers
$D$ with the number denoting nodes. The final model contains \num{1.3e7}
parameters and was trained using the ADAM optimizer for a cyclic learning rate
and momentum scheme. We trained the network for a total of 1000 epochs while
saving the best model during training based on the validation score. The model
validation performance gives a mean friction $R^2$ score of $\sim 98\%$ and a
rupture accuracy of $\sim 96 \%$. However, we got lower scores for a selected
subset of the Tetrahedon ($R^2 \sim 88.7 \%$) and Honeycomb ($R^2 \sim 96.6$)
pattern based on the top 10 max drop property scores respectively. The scores
obtained were lower, even though the selected configurations were partly
included in the training data and the hyperparameter selection favored the
performance on this selected set. Thus we conclude that these selected
configurations, associated with a highly non-linear friction-strain curve,
represent a bigger challenge for machine learning prediction. One interpretation
is that these involve the most complex dynamics and perhaps that this is not
readily distinguished from the behavior of the other configurations which
constitute the majority of the dataset. By evaluating the ability of the model
to rank the dataset based on property scores, we found that it was able to
effectively represent the top three scores for the maximum categories. However,
the ranking for the minimum friction property was lacking, which we attribute to
the requirement of higher prediction precision that the model did not meet To
obtain a more accurate evaluation of the model's performance, we generated a
test set using \acrshort{MD} simulations for some of the random walk based
suggestions obtained from the accelerated search. The results showed a
significantly worse performance compared to the validation set, with a two-order
of magnitude higher loss and a negative $R^2$ score for the mean friction
property. The negative $R^2$ score suggests that the model's predictions were
worse than simply predicting the mean value of the true data. However, by
reevaluating the hypertuning and choosing solely based on validation loss, we
still found poor results on the test set. This suggests that the inadequate
performance is not solely due to a biased hypertuning process,  but rather
because our original dataset did not cover a sufficiently diverse range of
Kirigami configurations. The validation scores indicate that the use of machine
learning is a feasible approach for addressing this problem, as we were able to
identify some general trends in the data. Nonetheless, from the test scores, it
is evident that further improvements to the dataset are necessary in order to
develop a reliable model.


% 5: Accelerated search
\subsection{Accelerated search for Kirigami patterns}
Using the machine learning model we performed two types of accelerated search. One
by evaluating the property scores of an extended dataset and another with the
use of the genetic algorithm approach. For the extended dataset search, we used
the developed pattern generators to generate \num{1.35e6} Tetrahedron,
\num{2.025e7} Honeycomb and \num{e4} Random walk patterns. The search results
for the minimum friction property indicate a preference for low cut density. This aligns with the overall observation that the dataset does not
provide any suggestions for a further reduction in friction.

The search for the maximum properties resulted in some minor score increases, but
the suggested candidates were mainly overlapping with the original dataset. By investigating the sensitivity to translations of the Tetrahedron and Honeycomb patterns, we observed significant variations in the model's predictions with only minor translations. This can be attributed to a physical dependency since these translations affect the edge of the sheet. However, given the poor model performance on the test set, we believe it is more likely that this variation is due to an insufficiency in the model caused by the limitations of the dataset.

In our investigation of the genetic algorithm approach, we used a starting population that was based on the results from the extended dataset accelerated search, as well as some randomly generated initializations with different porosity values. However, this approach did not provide any noteworthy
indication for new design structures worth more investigation. In general, the
initialization of the population itself proved to be a more promising strategy
than the genetic algorithm. We acknowledge that further effort could potentially yield useful results with the genetic algorithm approach. However, we believe that the current lack of promising results can be attributed to the uncertainty of the model which where the reason for not pursuing this any further.

By considering the Grad-CAM explanation method, we observed that the model predictions were often substantially reliant on the top and bottom edges of the Kirigami configurations. This was unexpected since these edges are not true edges but are connected to the pull blocks used in the simulation. Despite the model's uncertain predictions, we speculate that this may be due to the thermostat effects from the pull blocks. Therefore, we note this as a feature worth investigating in the simulations.


% 6: Negative friction coefficient
\subsection{Negative friction coefficient}
Based on our initial investigations of the Kirigami sheet, we have discovered a
highly non-linear friction-strain relationship. By proposing a linear coupling
between load and strain with ratio $R$, we found that these results suggest the
possibility to utilize the negative slope on the friction-strain curve to
achieve a negative friction coefficient. Based on the decrease in friction from the top to bottom of
these curves, using the Tetrahedron $(7,5,1)$ and Honeycomb $(2,2,1,5)$ pattern
from the pilot study, we estimate that the average coefficient within this range
will be $-R\SI{12.75}{nN}$ for the Tetrahedron pattern and
$-R\cdot\SI{2.72}{nN}$ for the Honeycomb pattern.

To investigate this hypothesis, we conducted a simulation with a coupling
between load and sheet tension, mimicking a nanomachine attached to the sheet,
using the Kirigami configurations from the Pilot study. We observed that the non-linear behavior in the friction-strain curve also translated into a non-linear friction-load relationship for the coupled system. Additionally, we found that the Honeycomb pattern exhibited a non-linear strain-tension curve which resulted in an almost discontinuous increase in friction for the initial
increase in load. We attribute this feature to an unfolding process visually
confirmed from the simulation frames. For the coupled system with a
load-to-tension ratio of 6, we found regions in the friction-load curve with
significant negative slopes. By considering the maximum and minimum points for
such regions we estimated the average friction coefficient to be $-0.31$ in the
load range $F_N = [4.65, 6.55]$ nN for the Tetrahedron pattern and $-0.38$ in the
range $F_N = [0.71, 4.31]$ nN for the Honeycomb pattern. These results can be scaled by adjusting the load-to-tension ratio.

Based on our investigations, we have found that the combination of Kirigami cuts and strain has significant potential for controlling friction. Specifically, we have demonstrated that by enforcing a coupling between load and strain through tension, it is possible to achieve a negative friction coefficient. Therefore, we believe that this approach could be promising for developing novel friction-control strategies.


\section{Outlook}
In this thesis, we have demonstrated that certain Kirigami designs exhibit non-linear friction behavior with strain. This discovery was made through an exploration of different designs, which invites further investigation into the underlying mechanisms of this phenomenon. To this end, it would be valuable to choose only one or two selected designs, such as the Tetrahedron and Honeycomb patterns, and study the effects of various physical parameters on the friction-strain curve.

First of all, we suggest an investigation of how the friction-strain curve
depends on temperature, sliding speed, spring constant, and loads for an
increased range $F_N > 100 nN$. This is especially interesting in the context of
physical conditions leading to a stick-slip behavior since our study takes a
basis in smooth sliding friction. Moreover, it would be valuable to verify that
the choices for relaxation time, pauses, interatomic potentials and substrate
material are not critical for the qualitative observations found for the
friction-strain relationship. Especially the Adaptive Intermolecular Reactive
Empirical Bond Order (AIREBO) potential for the modeling of the graphene sheet
might be of interest. In this context, it might also be useful to investigate
the effects of excluding adhesion from the simulations. 

In order to investigate the hypothesis of commensurability as a governing
mechanism we suggest an investigation of the friction-strain curve for different
scan angles. If commensurability is an important factor, we hypothesize that the friction-strain curve will exhibit different qualitative shapes at varying scan
angles. Additionally, it may be interesting to investigate the friction-strain
relationship under a uniform load to gain insight into how the loading
distribution affects the out-of-plane buckling and associated commensurability
effect.

Another topic worth exploring is the impact of scale and edge effects. This includes an investigation of scaling the ratio of the sheet area to the sheet edge length. However, the machine learning predictions also suggest a study of Kirigami-induced edge effects as we translate the patterns on the sheet. With this regard, we would also suggest a more detailed study of the effect of the thermostat in the pull blocks which is suggested to have a possible importance when judging from the Grad-CAM analysis. 

Regarding the machine learning approach, our findings indicate that there is a
significant need to expand the dataset. In order to get more insight into this
issue one could use unsupervised clustering techniques like the t-Distributed
Stochastic Neighbor Embedding (t-SNE) to visualize the distribution of Kirigami
configurations in the dataset. Another valuable approach is the active learning
method similar to that used by Hanakata et al.~\cite{PhysRevLett.121.255304}.
That is, we extend the dataset using the top candidates of a machine
learning-driven accelerated search and repeat the process of training the model,
searching for new candidates and extending the dataset. This provides a
direction for the extension of the dataset which could lead to a more efficient
approach to address the dataset problem. We note that one can also create a dataset based on a fixed Kirigami design and vary the physical parameters to support the investigations mentioned above. For both variations, we believe that the results could benefit from a consideration of more advanced model architectures and machine learning techniques. For instance, we suggest increasing the receptive field in the convolutional part of the model, by the use of bigger strides or dilated convolution. If we can develop a reliable machine learning model, it would invite further studies of inverse design methods such as \acrshort{GAN} or diffusion models. 

