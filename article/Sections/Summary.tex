\chapter{Summary}\label{chap:summary}

In this thesis we have studied nanoscale friction of a Kirigami graphene sheet
under the influence of load and strain using \acrshort{MD} simulations. We have
developed a numerical framework for generating various Kirigami designs which
was used to create a dataset of the frictional behavior depending on the
Kirigami pattern, strain and loading. Our findings suggest that the frictional
behavior of a Kirigami sheet is highly dependent on the geometry of the pattern
and the strain conditions. We observed that the out-of-plane buckling can be
associated with a non-linear friction-strain curve which can be utilized to
demonstrate a negative friction coefficeint in a system with coupled load and
strain. Moreover, we have investigated the possibility to use machine learning
on this dataset and attempted an accelerated search. Our result suggest that
machine learning can be feasible for this approach, but more data is needed to
provide a more reliable foundation for a search of new Kirigami patterns. In
this chapter we will summarize the findings in more detail and draw some conclusions. At the end we will provide some topics for further research.




% In this thesis we have studied nanoscale friction of a Kirigami graphene sheet
% under the influence of load and strain. To this end, we have created an
% \acrshort{MD} simulation which enabled us to study the frictional behavior of
% a graphene sheet sliding on a silicon substrate. In addition, we have created
% a numerical framework for creating Kirigami design patterns and introducing
% these into the friction simulations. This was used to study the effects of the
% out-of-plane buckling in comparison to a non-cut sheet under the influence of
% strain. Morevoer, we have created a dataset of various Kirigami designs at
% different load and strain with the aim of getting insight into the friction
% behavior of various design patterns. We have investigated the possibility to
% use machine learning on this dataset and attempted an accelerated search.
% Finally we considered the prospects of achieving a negative friction
% coefficient for a system with coupled load and stretch. In this chapter we
% will summarize the findings and draw some conclusions. We will also provide
% some topics for further research.





\section{Summary and conclusions}

% 1: Design MD simulations
\subsection{Designing an MD simulation}
We have designed an \acrshort{MD} simulation for the examination of friction for
a graphene sheet sliding on a silicon substrate. The key system features were
the introduction of the pull blocks, defined as the end regions of the sheet
with respect to the sliding direction, which was utilized for applying normal
load and sliding the sheet. The pull blocks were made partly rigid and used to
employ a thermostat as well. By an analysis of the friction forces retrieved from sliding simulations we defined a standardized metric for kinetic friction. We measured the force acting from the substrate on the full sheet (including the pull blocks) with respect to the sliding direction and defined kinetic friction by the mean value of the last half of the simulation. The uncertainties were estimated on the basis of the fluctuations in the running mean.  We found that the assessment of static friction was ambiguous for our simulation and did not pursue this further. From the analysis of the
force traces, friction force vs.\ time, we identify the friction behavior in our
simulation domain as being in the smooth sliding regime mainly due to the choice of sliding speed (\SI{20}{m/s}) and infintely stiff springs. This was further supported by a demonstration of a transition to stick-slip behavior with softer springs and a lowering of sliding speed. By conducting a more systematic investigation of the
effects of temperature, sliding speed, spring constant and timestep, we settled
on the default values based on numerical stability and computational cost. We
found that friction increased with temperature which we attribute to being in the ballistic sliding sliding regimem. We used the room temperature \SI{300}{K} as a standard choice. Furthermore, we found friction to increase with velocity as expected, with some signs of phonon resonance at certain sliding speeds as well. We chose a rather high velocity of \SI{20}{m/s} mainly for the consideration of computational costs. For the spring constant, we found decreasing friction with increasing stiffness of the springs which is associated with the transition from a stick-slip-influenced regime toward smooth sliding. The choice of an infintely stiff spring was made from a stability assessment. Finally, we confirmed that a timestep of \SI{1}{fs} provides reasonable numerical stability. However, based on fluctuations with timestep we find that the uncertainty in the simulations might be higher than first estimated.


% 2: Design Kirigami framework
\subsection{Generetig Kirigami patterns}
In order to invstigate the effects of Kirigami design we have created a
numerical framework for generating various patterns. By defining an indexing
system for the hexagonal lattice structure we were able to define the Kirigami
designs as a 2D binary matrix for numerical implementation. We digitalized two
different macroscale designs, which we named the \textit{Tetrahedron} and
\textit{Honeycomb} pattern, that successfully produced out-of-plane
buckling when stretched. Through our numerical framework we were able to create an ensemble of perturbed unique variations which yielded approximately 135k and 2025k for the Tetrahedron and Honeycomb patterns respectively. When considering the possibility to translate the patterns we find the ability to increase the number by roughly a factor 100. In addition we created a framework for generating random walk based Kirigami patterns. This was regulated by introducing features such as bias, avoidance of existing cuts,
preference to keeping a direction and procedures to repairing the sheet for
simulation purposes. In general, the capabilities of the numerical framework for
generating Kirigami designs exceeded our computational resources with regard to performing \acrshort{MD} simulation under different load and strain for each of the designs. Thus our \acrshort{MD}-based dataset only utilized a subset of configurations with 9660 data points based on 216 Kirigami configurations (Tetrahedron: 68, Honeycomb: 45, Random walk: 100, Pilot study: 3). Thus our  Kirigami generative framework can be valuable for further studies on an extended dataset.


% 3: Control friction using Kirigami
\subsection{Control friction using Kirigami}
We have investigated the frictional behavior of the Tetrahedron and Honeycomb
patterns in comparison to a non-cut sheet under various strains and loads.
Initially, we observed that straining the Kirigami sheets in vacuum resulted in an out-of-plane buckling. When adding the substrate to the simulation
this translated into a decreasing contact area with strain. We found the
Honeycomb sheet to exhibit the most significant buckling with a corresponding
reduction of relative contact area to approximately 43\%. The non-cut sheet did
not produce any significant buckling in comparison. We found that friction generally increased with strain which contradicts the asperity theory hypothesis of decreasing friction with
decreasing contact area. Moreover, the friction-strain curve exhibited highly
non-linear trends with strong negative slopes (see~\cref{fig:multi_stretch}), while the non-cut sheet did not show any significant dependency on the strain We also found that the non-stretched Kirigami patterns did affect friction to some degree, but this was one order of magnitude lower than the effects associated with the strain in combination. This led us to the conclusion that the changing contact area cannot be regarded as a dominant mechanism for friction in the Kirigami sheet system nor the independent consideration of sheet configuration or tension in the sheet. When considering the dependency with load we generally found a weak dependency which can be associated with a friction coefficient on the order of
\num{e-4}--\num{e-5} even though we could not confirm any clear relationship.
This is best attributed to a superlubric state of the graphene sheet as seen in other studies as well. The slope of the friction-load curves was not considerably affected by the straining of the Kirigami sheet which led us to the conclusion that the strain-induced effects are dominant in comparison to any load-related effects. By proposing a linear coupling between load and strain with ratio $R$ we find that these results suggest the possibility to find negative friction coefficients in certain load ranges following $-R\SI{12.75}{nN}$ for the Tetrahedon and $-R\cdot\SI{2.72}{nN}$ for the Honeycomb pattern.



% increased. This is shown in \cref{fig:multi_stretch}.  This led us to the conclusion that the contact area cannot be attributed a dominant mechanism for friction throughout the straining of the studied Kirigami sheets. In general we found a non-existing relationship between friction and load considering the uncertainties in the simulation. This is best attributed to the superlubric state of the graphene sheet on the substrate. The slope of the friction-load curves were not significantly affected by the straining of the Kirigami sheet and thus we conclude that the load effect on friction is neglible cimpared to the strain effects. 

% However,
% this disagreed with the asperity theory hypothesis of a decreasing friciton with
% decreasing contact area. We found that the strain-induced buckling was initially
% (at low relative strain) assocated with an increase in friction. Moreover, the
% friction-strain curve produced a no-linear behaviour which was not compatible
% with the approximately monotonic decreasing contact area as strain were


%  This is even when considering that the
% non-cut sheet had a yield strain of $0.35$ while the Tetrahedron had a lower
% yield strain of $0.21$ and the Honeycomb a considerable larger one at $1.27$
% based on a stretch in vacuum. The out-of-plane buckling resulted in a
% signifciant reduction of the contact area as the sheet were stretched, towards a
% minimum of $X$ for the Tetrahedron and $y$ for the Honeycomb pattern.


% 4: Capture trends with ML
\subsection{Capture trends with machine learning}
The dataset reveals some general correlations with mean friction,
such as a positive correlation to strain (0.77)
and porosity (0.60), and a negative correlation to contact area (-0.67). These
results align with the finding in the pilot study, suggesting that the change in friction is associated with cuts in the sheet (porosity) and a changing contact area. 

By defining the friction property metrics: $\min F_{\text{fric}}$, $\max
F_{\text{fric}}$, $\max \Delta F_{\text{fric}}$ and max drop (maximum decrease in friction with strain), we investigated the top candidates within our dataset. From these results, we found no incentive for the possibility to reduce friction with the Kirigami approach since the non-cut sheet provided the lowest overall friction. Regarding the maximum
properties, we found an improvement from the original pilot study values and
with the Honeycomb pattern producing the highest scores. This suggests that the data contains some relevant information for optimization with respect to
these properties. Among the top candidates, we found that a flat friction-strain profile is mainly associated with little decrease in the
contact area and vice versa which again aligns with the pilot study findings.  

For the machine learning investigation, we have implemented a VGGNet-16-inspired
convolutional neural network with a deep ``stairlike'' architecture:
C32-C64-C128-C256-C512-C1024-D1024-D512-D256-D128-D64-D32, for convolutional
layers $C$ with the number denoting channels and fully connected (dense) layers
$D$ with the number denoting nodes. The final model contains \num{1.3e7} parameters and was trained using the ADAM optimizer for a cyclic learning rate and momentum scheme. We trained the network for a 1000 epochs while saving the best model during training based on the validation score. The model validation performance gives a mean friction $R^2$
score of $\sim 98\%$ and a rupture accuracy of $\sim 96 \%$. However, we got
lower scores for a selected subset of the Tetrahedon ($R^2 \sim 88.7 \%$) and
Honeycomb ($R^2 \sim 96.6$) pattern based on the top 10 max drop scores
respectively. These scores were lower despite the fact that the selected set was
partly included in the training data as well in addition to the fact that the hyperparameter selection favored the performance on this selected set as well. Thus we conclude that these selected configurations, associated with a highly non-linear friction-strain curve, represent a bigger challenge for the machine learning prediction. One interpretation is that these involve
the most complex dynamics and perhaps that this is not readily distinguished
from the behavior of the other configurations which constitute the majority of the data set. By evaluating the ability of the model to rank the dataset according to the property scores we found in general a good representation of the top 3 scores for the maximum categories, while the minimum friction property ranking was lacking. We attribute this latter observation to a higher need for precision which the model did not possess.

In order to provide a more true evaluation of the model performance we created a
test set based on \acrshort{MD} simulations for an extended Random walk search.
This test revealed a significantly worse performance than seen for the
validation set with a two-order of magnitude higher loss and a negative friction
mean $R^2$ score which corresponds to the prediction being worse than simply
guessing on the true data mean. However, by reconsidering the choice of
architecture complexity hypertuning with respect to the validation loss, we
found similar poor results on the test set. This indicates, that the test score
is not simply a product of a biased hypertuning process but instead points to
the fact that our original dataset did not cover a wide enough configuration
distribution to accurately capture the full physical complexity of the Kirigami
friction behavior. Based on the validation scores we conclude that the use of
machine learning is feasible, but that we need to improve the dataset in order
to reach a reliable model. 


% 5: Accelerated search
\subsection{Accelerated search}
Using the \acrshort{ML} model we performed two types of accelerated search. One
by evaluating the property scores of an extended dataset and another with the
use of the genetic algorithm approach. For the extended dataset search, we used
the developed pattern generators to generate $\SI{135}{k} \times 10$
Tetrahedon, $\SI{2025}{k} \times 10$ Honeycomb and \SI{10}{k} Random walk
patterns. For the minimum friction property, the search suggests a favoring of a
low cut density (low porosity) which aligns with the overall idea that the
dataset does not provide an incentive for further friction reduction with the non-cut sheet resulting in the lowest friction. The
maximum properties resulted in some minor score increases but the suggested
candidates were mainly overlapping with the original dataset. By investigating the model prediction sensitivity to the translation of the Tetrahedron and Honeycomb patterns we found that the model predictions varied drastically with small translations. This can be attributed to a physical dependency since the edge of the sheet is affected by this
translation. However, due to the poor model performance on the test set, we find it more likely to be a model insufficiency related to the lacking dataset.

For the genetic algorithm approach, we investigated the optimization for the max
drop property using starting population based on the result from the
extended dataset accelerated search, and some random noise initializations with
different porosity values. This approach did not provide any noteworthy
incentive for new design structures worth more investigation. In general, the
initialization of the population itself proved to be a more promising strategy
than the genetic algroithm. However, this is highly affected by the uncertainty
of the model predictions, and thus we did not pursue this any further. By
considering the Grad-CAM explanation method we found that the model predictions
occasionally payed considerable attention to the top and bottom edge of the
configurations. This is surprising since these are not true edges but are
connected to the pull blocks in the simulation. Despite the uncertainties in the
predictions, we argue that this might be attributed to thermostat effects from
the pull blocks and thus we note this as a feature worth investigating in the simulations.


% 6: Negative friction coefficient
\subsection{Negative friction coefficient}
By enforcing a coupling between load and sheet tension, mimicking a nanomachine
attached to the sheet, we investigated the load curves arising from the loading
of the Tetrahedron $(7,5,1)$ and Honeycomb $(2,2,1,5)$ pattern from the pilot
study. The non-linear trend observed for increasing strain carried over to the
coupled system as well producing a highly non-linear friction-load curve. The
Honeycomb pattern exhibited additionally a non-linear strain-tension curve which
resulted in an almost discontinuous increase in friction for the initial
increase in load. We attribute this feature to an unfolding process visually
confirmed from the simulation frames. For the coupled system with a
load-to-tension ratio of 6 we found regions in the friction-load curve with
significant negative slopes. By considering the maximum and minimum points for
such regions we estimated the average friction coefficient to be $-0.31$ in the
range $F_N = [4.65, 6.55]$ nN for the Tetrahedron pattern and $-0.38$ in the
range $F_N = [0.71, 4.31]$ nN for the Honeycomb pattern. .


Our findings suggest that the combined use of Kirigami cuts and strain has considerable potential for friction control, especially under the influence of a coupled system of load and strain.
\section{Outlook}
In this thesis, we have successfully shown that certain Kirigami designs exhibit non-linear behavior with strain. This discovery was done through a focus on the exploration of the effects of different designs. This invites further investigation of the underlying mechanisms behind this phenomenon. By considering one or two design, based on Tetrahedron and Honeycomb patterns, it would be valuable to investigate the effects on the friction-strain curve under various physical parameters. 

First of all, we suggest an investigation of how the friction-strain curve
depends on temperature, sliding speed, spring constant, and loads for an
increased range $F_N > 100 nN$. This is especially interesting in the context of
physical conditions leading to a stick-slip behavior since our study takes a
basis in smooth sliding friction. Moreover, it would be valuable to verify that
the choices for relaxation time, pauses, interatomic potentials and substrate
material are not critical for the qualitative observations. Especially the
Adaptive Intermolecular Reactive Empirical Bond Order (AIREBO) potential for the
modeling of the graphene sheet might be of interest. In this context, it might
also be useful to investigate the effects of excluding adhesion in the
simulations. In order to gain further insight into the role of commensurability
one could vary the scan angle as well. Since we suspect that our simulation
corresponds to a incommensurable superlubric state, certain scan angles is
hypothesized to yield higher friction. If the friction-strain curve is based on a commensurability effect this might yield qualitative different results and perhaps also allowing for a lowering of friction with strain. We might also consider investigating the friction-strain relationship under a uniform load to get insight into how the loading distribution effects the out-of-plane buckling and associated frictional effects.

Another topic worth studying is the relation to scale and edge effects. This includes an investigation of scaling effects, considering the ratio of the sheet to the edge, but also a translation of the sheet patterns to study the presence of any Kirigami-induced edge effects. The latter is motivated by the findings from the machine learning predictions. With this regard, we would also suggest a more detailed study of the effect of the thermostat in the pull blocks which is suggested to have a possibly importance when judging from the Grad-CAM analysis. 

For the machine learning approach, our findings suggest that the dataset is extended considerably. This can be done with respect to the aim of exploring Kirigami configurations as done in this thesis, but it can also be done for a single configuration under variation of some of the simulation parameters to support some of the above mentioned investigations. In that context, one could consider more advanced model architectures and machine learning techniques. If successful this would invite further studies of inverse design methods such as GAN or diffusion models. 

% \begin{itemize}
%   \item Investigate if the contact area is effecting the friciton non-linear by turning of friction force for atoms corresopnding to those that lift of from the sheet during the out-of-plane buckling. 
% \end{itemize}

% Things to include here
% \begin{itemize}
%   \item Could be valuable to spend more time on the validation of the MD simulations. How does material choice and potential effects the results. How realistic is the simmulations?
%   \item Are there any interesting approaches for compressed kirigami structures?
%   \item How does these results scale? I imagaind that the nanomachine systems should be applied in small units to avoid scaling problems, but in general I could spend way more time on the scaling investigation.
%   \item Since the normal force is applied at the pull blocks the normal force distribution changes from the sides more towards and even distribution as the sheet is put under tension (stretched). If we imagined a sheet for which the center part was either a different material or had some kind of pre-placed asperity on it, could we then exploit this force distirbution to get exotic properties as well? By studying this we might get a clearer understanding of what is the cause of my results. 
%   \item Possibility to study hysteresis effects. Maybe the frictional behvaiours change significantly through repeated cycles of stretch and relax. 
%   \item Try making substrate rigid to investigate the effect of this. 
%   \item Let the structure relax a bit longer perhaps, just to be sure that this does not affect the results. 
%   \item Extend load range
%   \item Try turning of adhesive forces to investigate the effects from adhesive forces. 
%   \item One way to investigate the importance of contact area is to turn of the friction force for of the atoms lifting away from the sheet during stretch. If this was purely a non-linear contact area dependency one would find the same results. If not we might point to the stress or other variables related to out-of-plane buckling.
%   \item Is the stretching effect caused by a commensurability effect. It does change where the pressure is applied. To test this one should look at the curves for other sliding direction. However, this would probably also predict similar differences in friction at a zero stretch state between different configuration which I do not think we get. However, since the stretched configurations put atom outside the hexagonal grid stucture it is per definition not achieveable to get those states for a non-strecthed sheet which support the idea of a commensurability effect.
%   \item How does it effect the results that we had a part on the pull blocks. 
%   \item Is there any effects from going over the same substrate multiple times. This is probably mainly interesting in the context of discussing velocity relationship where the substrate has less time to relax.
%   \item Interesting test could be to do the stretch curves with normal force applied on the whole sheet. The fact that the stretch provides tension might result in a better contact in the center... But the contact often goes down so maybe this was a silly idea.
%   \item Recap the suggestion to study translational of the patterns.
%   \item Could be interesting to optimize solely for the pop-up effect, i.e.\ through the biggest change in contact area. Maybe this is the key, since the succesfull pattern is choosen in that way. The correlation supports this, and the fact that the correlation is only about 60\% can be explained by the fact that the cases without considerable pop-up might not make any difference or perhaps that only some kinds of pop-up triggers the effect. 
%   \item Top and bottom edges showed to be of importance for the model prediction which leads to the idea of the thermostat in the pull blocks being important. This could be investigated in a further study. 
% \end{itemize}



% Thinks to do better in the simulation procedure:
% The graphene is first relaxed using conjugate gradient energy minimization and the relaxed within an NVT (fixed number of particles N , volume V and temperature T ) ensemble for 50 ps with non-periodic boundary conditions in all directions at a fixed temperature T = 4.2 K. The graphene is then stretched by moving the right and left edges at a constant strain rate of ∼ 0.005/ps which is slower than our previous work. \cite{PhysRevResearch.2.042006}

% Also to tru AIREBO which is also used in the above.