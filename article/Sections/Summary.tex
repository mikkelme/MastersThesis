\chapter{Summary}\label{chap:summary}
In this thesis, we have studied the nanoscale friction of a Kirigami graphene
sheet under the influence of strain using molecular dynamics (\acrshort{MD})
simulations. We have developed a numerical tool for generating diverse Kirigami
designs which we have utilized to create a dataset for the frictional behavior
depending on Kirigami pattern, strain, and loading. Our findings suggest that
the frictional behavior of a Kirigami sheet is highly dependent on the geometry
of the pattern and the strain conditions. We observed that the out-of-plane
buckling can be associated with a non-linear friction-strain relationship which
can be utilized to demonstrate a negative friction coefficient in a system with
coupled load and strain. Moreover, we have investigated the possibility to use
machine learning on this dataset and attempted an accelerated search. Our
findings imply that machine learning can be feasible for this approach, but
additional data is required to establish a more reliable foundation for a search
for new Kirigami patterns. In this chapter, we will provide a summary of our
findings and draw conclusions based on the results obtained. Finally, we will
suggest some topics for further research.


\section{Summary and conclusions}

% 1: Design MD simulations
\subsection{Designing an MD simulation}
We have designed an \acrshort{MD} simulation for the examination of friction for
a graphene sheet sliding on a silicon substrate. The key system features were
the introduction of the pull blocks, defined as the end regions of the sheet
with respect to the sliding direction, which was utilized for applying normal
load and sliding the sheet. The pull blocks were made partly rigid and used to
employ a thermostat as well. Through an analysis of the friction forces retrieved from sliding simulations, we have established a standardized metric for kinetic friction. In particular, we measured the force exerted by the substrate on the full sheet, including the pull blocks, with respect to the sliding direction. We then determined the kinetic friction as the mean value of the last half of the simulation. The uncertainties were estimated based on the fluctuations in the running mean. We found that the assessment of static friction was ambiguous for our simulation and did not pursue this further. From the analysis of the
force traces, friction force vs.\ time, we identify the friction behavior in our
simulation domain as being in the smooth sliding regime. This is attributed to the choice of a relatively high sliding speed (\SI{20}{m/s}) and infinitely stiff springs for the tethering of the sheet. This was supported by a demonstration of a transition to the stick-slip regime through the use of softer springs and a decrease in sliding speed. By conducting a more systematic investigation of the
effects of temperature, sliding speed, spring constant and timestep, we identified a set of default values based on numerical stability and computational cost. During this process, we aimed to select the variables that would maintain relatively stable measured friction with moderate perturbations around these default values. We found that friction increased with temperature which is in disagreement with the Prandtl–Tomlinson model and most experimental results. However, this agrees with the predictions from the Frenkel-Kontorova models and other \acrshort{MD} studies which attribute this to ballistic sliding. In the absence of clear indications from the investigation regarding an appropriate temperature, we opted for the standard choice of room temperature, \SI{300}{K}. Furthermore, we found friction to increase with velocity as
expected from previous result, with some signs of phonon resonance at certain sliding speeds as well which aligns with predictions from the Frenkel-Kontorova models.
%
%
% Working here
%
%


.
We chose a rather high velocity of \SI{20}{m/s} mainly for the consideration of
computational costs. For the spring constant, we found decreasing friction with
increasing stiffness of the springs. This is associated with the transition from
a stick-slip-influenced regime toward smooth sliding and can be attributed to an underlying change in commensurability. The choice of an
infinitely stiff spring was made from an assessment of the variation in friction with perturbations in the spring constant value. Finally, we
confirmed that a timestep of \SI{1}{fs} provides reasonable numerical stability.
However, based on fluctuations with timestep we find that the uncertainty in the
simulations might be higher than first estimated. For the non-strained Kirigami sheet, these fluctuations were on the order of $\pm \SI{0.017}{nN}$ for the evaluation of the kinetic friction.


% 2: Design Kirigami framework
\subsection{Generating Kirigami patterns}
In order to investigate the effects of Kirigami design we have created a
numerical framework for generating various patterns. By defining an indexing
system for the hexagonal lattice structure we were able to define the Kirigami
designs as a 2D binary matrix for numerical implementation. We have selected two
macroscale designs, which we denote the \textit{Tetrahedron} and
\textit{Honeycomb} patterns, based on their ability to exhibit
out-of-plane buckling when subjected to strain. By digitalizing the designs to
match the hexagonal graphene lattice, we found that the characteristic design
features can be translated to the nanoscale, as we observed similar out-of-plane
buckling in \acrshort{MD} simulations. Through our numerical framework we were
able to create an ensemble of perturbed unique variations which yielded
approximately \num{135e3} and \num{2.025e6} unique configurations for the
Tetrahedron and Honeycomb patterns respectively. When considering the
possibility to translate the periodic patterns on the graphene sheet the number of possible patterns can be increased by approximately a factor of 100. To introduce some random design features, we have developed a framework for generating Kirigami patterns based on random walks. The framework includes mechanisms such as bias, avoidance of existing cuts, preference for maintaining a direction, and procedures for repairing the sheet for simulation purposes. In general, we found that the
capabilities of this numerical framework for generating Kirigami designs exceeded
our computational resources with regard to performing \acrshort{MD} simulation
under different loads and strains for each of the designs. In general, we found that the computational capacity of our numerical framework for generating Kirigami designs surpassed our available resources for conducting  \acrshort{MD} simulations under various load and strain for each design. Our \acrshort{MD}-based dataset only utilized a subset of configurations with 9660 data points based on 216 Kirigami configurations (Tetrahedron: 68, Honeycomb:
45, Random walk: 100, Pilot study: 3). Hence, we argue that the Kirigami generative framework can be valuable for further studies on an extended dataset.


% 3: Control friction using Kirigami
\subsection{Control friction using Kirigami}

% \item Is the frictional behavior consistent with the \hl{PT}, FK and FKT models? 

We have investigated the frictional behavior of the Tetrahedron and Honeycomb
patterns in comparison to a non-cut sheet under various strains and loads.
Initially, we observed that straining the Kirigami sheets in a vacuum resulted
in an out-of-plane buckling. When adding the substrate to the simulation this
translated into a decreasing contact area with strain. We found the Honeycomb
sheet to exhibit the most significant buckling with a corresponding reduction of
relative contact to approximately 43\%, whereas the non-cut sheet did not
produce any significant buckling in comparison. For the Kirigami sheets, we
found that friction initially increased with strain, which made for increasing
friction with decreasing contact area. As the strain continued to increase the
friction-strain curve exhibited highly non-linear trends with strong negative
slopes as well (see~\cref{fig:multi_stretch}). During the full strain, the
contact area was decreasing monotonically except for a slight increase just
before rupturing. These results contradict the asperity theory hypothesis of
decreasing friction with decreasing contact area, which is also supported by the predictions of the 2D Frenkel-Kontorova models suggesting increasing friction with (contacting) system size. Thus, we conclude that the contact area is not a governing mechanism for the friction-strain relationship observed. From the investigation of the friction-load relationship, we found that both friction and contact area increased slightly with load, but we were not able to find significant evidence of any linear relationship between friction and contact area as predicted by asperity theory. The non-cut sheet
did not show any dependency on the strain as both the friction and
the contact area remained constant with strain. Thus our findings suggest that a
changing contact area and the strain-induced friction effects might be
associated with an underlying mechanism connected to the buckling of the
sheet. We did find a small independent effect from the introduction of Kirigami
cuts before applying strain. We observed a small increase in friction between
the non-cut sheet and the Kirigami sheet, but this was one order of magnitude
lower than the effects associated with the strain in combination. On the other hand, the non-cut sheet did not exhibit any significant effects from strain and we conclude that the independent effects from Kirigami or strain-induced tension cannot be regarded as a dominant mechanism on its own. When
considering the friction dependency with load, we generally found a weak
dependency corresponding to a friction coefficient on the order of
\num{e-4}--\num{e-5} even though we could not confirm any clear relationship.
This is best attributed to a superlubric state of the graphene sheet as seen in
other studies as well. The slope of the friction-load curve was not considerably
affected by the straining of the Kirigami sheet which led us to the conclusion
that strain-induced effects are dominant in comparison to any load-related
effects. 
% In summary, we have found clear evidence that the friction in the
% Kirigami sheet system is controlled by strain. 

When considering our findings in light of previous related results we find it
plausible that the governing mechanism for the observed friction effects is
related to commensurability as predicted by the Frenkel-Kontorova models. Since we have observed extremely low friction in
our system, we argue that the non-deformed sheet corresponds to an
incommensurable configuration. This is supported by the fact that we were not
able to lower the friction from the original starting point. Additionally, this
also aligns with the observation that the introduction of the non-strained
Kirigami designs increased friction slightly. Since some atoms were removed from
the lattice this can be hypothesized to relax the incommensurability to some
degree. When the Kirigami sheet buckles during stretching, it allows for a
considerable rearrangement of the atoms in contact with the substrate. Hence, it
may transition in and out of commensurable configurations, which could explain
the non-linear trend for the friction-strain curve. One way to test this
hypothesis is to reorientate the sheet in order to reach a commensurable start
configuration. Then we might find a transition from a commensurable to an
incommensurable case during the initial straining which would result in a
lowering in friction with respect to the starting point. 



% 4: Capture trends with ML
\subsection{Capturing trends in friction data with machine learning}
By utilizing the numerical framework for generating Kirigami design we have
created a \acrshort{MD}-based dataset for the frictional behavior depending on
Kirigami designs, load and strain. The dataset reveals some general correlations
with mean friction, such as a positive correlation to strain (0.77) and porosity
(0.60), and a negative correlation to contact area ($-0.67$). These results
align with the finding in the pilot study, suggesting that the change in
friction is associated with cuts in the sheet (porosity) and a changing contact
area. 

By defining the friction property metrics: $\min F_{\text{fric}}$, $\max
F_{\text{fric}}$, $\max \Delta F_{\text{fric}}$ and max drop (maximum decrease
in friction with strain), we investigated the top design candidates within our
dataset. From these results, we found no indication for the possibility to
reduce friction with the Kirigami approach since the non-cut sheet provided the
lowest overall friction. Furthermore, among the top candidates, we found that a
flat friction-strain profile is mainly associated with little decrease in the
contact area and vice versa. These observations are consistent with the results
of the pilot study and support the hypothesis that commensurability plays a key
role in governing the behavior of the system. In terms of the maximum
properties, we observed an improvement compared to the values obtained in the
pilot study, with the Honeycomb pattern exhibiting the highest scores. This
indicates that the data may be useful for optimizing these properties. 

For the machine learning investigation, we have implemented a VGGNet-16-inspired
convolutional neural network with a deep ``stairlike'' architecture:
C32-C64-C128-C256-C512-C1024-D1024-D512-D256-D128-D64-D32, for convolutional
layers $C$ with the number denoting channels and fully connected (dense) layers
$D$ with the number denoting nodes. The final model contains \num{1.3e7}
parameters and was trained using the ADAM optimizer for a cyclic learning rate
and momentum scheme. We trained the network for a total 1000 epochs while saving
the best model during training based on the validation score. The model
validation performance gives a mean friction $R^2$ score of $\sim 98\%$ and a
rupture accuracy of $\sim 96 \%$. However, we got lower scores for a selected
subset of the Tetrahedon ($R^2 \sim 88.7 \%$) and Honeycomb ($R^2 \sim 96.6$)
pattern based on the top 10 max drop scores respectively. The scores obtained
were lower, even though the selected configurations were partly included in the
training data and the hyperparameter selection favored the performance on this
selected set. Thus we conclude that these selected configurations, associated
with a highly non-linear friction-strain curve, represent a bigger challenge for
the machine learning prediction. One interpretation is that these involve the
most complex dynamics and perhaps that this is not readily distinguished from
the behavior of the other configurations which constitute the majority of the
data set. By evaluating the ability of the model to rank the dataset based on
property scores, we found that it was able to effectively represent the top
three scores for the maximum categories. However, the ranking for the minimum
friction property was lacking. We attribute this discrepancy to the fact that
this property requires a higher level of precision, which the model was not able
to achieve. To obtain a more accurate evaluation of the model's performance, we
generated a test set using \acrshort{MD} simulations for the random walk based
suggestions obtained from the accelerated search. The results showed a
significantly worse performance compared to the validation set, with a two-order
of magnitude higher loss and a negative $R^2$ score for the mean friction
property. The negative $R^2$ score suggests that the model's predictions were
worse than simply predicting the mean value of the true data.

However, by reevaluating the hypertuning and choosing solely on the basis of
validation loss, we still found poor results on the test set. This suggests that
the inadequate performance is not solely due to biased hypertuning but rather
because our original dataset did not sufficiently cover a diverse range of
Kirigami configurations to accurately capture the full complexity of its
frictional behavior. The validation scores indicate that the use of machine
learning is a feasible approach for addressing this problem, as we were able to
identify some general trends in the data. Nonetheless, from the test scores, it
is evident that further improvements to the dataset are necessary in order to
develop a reliable model.


% 5: Accelerated search
\subsection{Accelerated search}
Using the \acrshort{ML} model we performed two types of accelerated search. One
by evaluating the property scores of an extended dataset and another with the
use of the genetic algorithm approach. For the extended dataset search, we used
the developed pattern generators to generate \num{1.35e6} Tetrahedron,
\num{2.025e7} Honeycomb and \SI{10}{k} Random walk patterns. The search results
for the minimum friction property indicate a preference for low cut density or
low porosity. This aligns with the overall observation that the dataset does not
provide any suggestions to further reduction in friction, as the non-cut sheet
has the lowest friction. 

The search for the maximum properties resulted in some minor score increases but
the suggested candidates were mainly overlapping with the original dataset. By investigating the sensitivity to translations of the Tetrahedron and Honeycomb patterns, we observed significant variations in the model's predictions with only minor translations. This can be attributed to a physical dependency since these translations affect the edge of the sheet. However, given the poor model performance on the test set, we believe it is more likely that this variation is due to an insufficiency in the model caused by the limitations of the dataset.

In our investigation of the genetic algorithm approach, we used a starting population that was based on the results from the extended dataset accelerated search, as well as some randomly generated initializations with different porosity values. However, this approach did not provide any noteworthy
indication for new design structures worth more investigation. In general, the
initialization of the population itself proved to be a more promising strategy
than the genetic algorithm. We acknowledge that further effort could potentially yield useful results with the genetic algorithm approach. However, we believe that the current lack of promising results can be attributed to the uncertainty of the model which where the reason for not pursuing this any further.

By considering the Grad-CAM explanation method, we observed that the model predictions were often substantially reliant on the top and bottom edges of the Kirigami configurations. This was unexpected since these edges are not true edges but are connected to the pull blocks used in the simulation. Despite the model's uncertain predictions, we speculate that this may be due to the thermostat effects from the pull blocks. Therefore, we note this as a feature worth investigating in the simulations.



% 6: Negative friction coefficient
\subsection{Negative friction coefficient}
Based on our initial investigations of the Kirigami sheet, we have discovered a
highly non-linear friction-strain relationship. By proposing a linear coupling
between load and strain with ratio $R$, we find that these results suggest the
possibility to utilize the negative slope on the friction-strain curve to
achieve a negative friction coefficient. Based on the drop from top to bottom of
these curves, using the Tetrahedron $(7,5,1)$ and Honeycomb $(2,2,1,5)$ pattern
from the pilot study, we estimate that the average coefficient within this range
will be $-R\SI{12.75}{nN}$ for the Tetrahedron pattern and
$-R\cdot\SI{2.72}{nN}$  for the Honeycomb pattern.

To investigate this hypothesis, we conducted a simulation with a coupling
between load and sheet tension, mimicking a nanomachine attached to the sheet,
using the Kirigami configurations from the Pilot study. We observed that the non-linear behavior in the friction-strain curve also translated into a non-linear friction-load relationship for the coupled system. Additionally, we found that the Honeycomb pattern exhibited a non-linear strain-tension curve which resulted in an almost discontinuous increase in friction for the initial
increase in load. We attribute this feature to an unfolding process visually
confirmed from the simulation frames. For the coupled system with a
load-to-tension ratio of 6 we found regions in the friction-load curve with
significant negative slopes. By considering the maximum and minimum points for
such regions we estimated the average friction coefficient to be $-0.31$ in the
load range $F_N = [4.65, 6.55]$ nN for the Tetrahedron pattern and $-0.38$ in the
range $F_N = [0.71, 4.31]$ nN for the Honeycomb pattern. These results can be scaled by adjusting the load-to-tension ratio.

Based on our investigations, we have found that the combination of Kirigami cuts and strain has significant potential for controlling friction. Specifically, we have demonstrated that by enforcing a coupling between load and strain through tension, it is possible to achieve a negative friction coefficient. Therefore, we believe that this approach could be promising for developing novel friction-control strategies.




\section{Outlook}
In this thesis, we have demonstrated that certain Kirigami designs exhibit non-linear friction behavior with strain. This discovery was made through an exploration of different designs, which invites further investigation into the underlying mechanisms of this phenomenon. To this end, it would be valuable to one or two selected designs, such as the Tetrahedron and Honeycomb patterns, and study the effects of various physical parameters on the friction-strain curve.

First of all, we suggest an investigation of how the friction-strain curve
depends on temperature, sliding speed, spring constant, and loads for an
increased range $F_N > 100 nN$. This is especially interesting in the context of
physical conditions leading to a stick-slip behavior since our study takes a
basis in smooth sliding friction. Moreover, it would be valuable to verify that
the choices for relaxation time, pauses, interatomic potentials and substrate
material are not critical for the qualitative observations. Especially the
Adaptive Intermolecular Reactive Empirical Bond Order (AIREBO) potential for the
modeling of the graphene sheet might be of interest. In this context, it might
also be useful to investigate the effects of excluding adhesion from the
simulations. 

In order to investigate the hypothesis of commensurability as a governing mechanism we suggest an investigation of the friction-strain curve for different scan angles. 
If commensurability is an important factor, we hypothesize that the friction-strain curve will exhibit different qualitative shapes at varying scan angles Additionally, it may be interesting to investigate the friction-strain relationship under a uniform load to gain insight into how the loading distribution affects the out-of-plane buckling and associated commensurability effect.

Another topic worth studying is the relation to scale and edge effects. This includes an investigation of scaling effects, considering the ratio of the sheet area to the edge length, but also a translation of the sheet patterns to study the presence of any Kirigami-induced edge effects. The latter is motivated by the findings from the machine learning predictions. With this regard, we would also suggest a more detailed study of the effect of the thermostat in the pull blocks which is suggested to have a possible importance when judging from the Grad-CAM analysis. 

Regarding the machine learning approach, our findings indicate that there is a
significant need to expand the dataset. In order to get more insight into this
issue one could use unsupervised clustering techniques like the t-Distributed
Stochastic Neighbor Embedding (t-SNE) to visualize the distribution of Kirigami
configurations in the dataset. Another valuable approach is the active learning
method similar to that used by Hanakata et al.~\cite{PhysRevLett.121.255304}.
That is, we extend the dataset using the top candidates of a machine
learning-driven accelerated search and repeat the process of training the model,
searching for new candidates and extending the dataset. This provides a
direction for the extension of the dataset which could lead to a more efficient
approach to address the complex space of Kirigami designs. We note that one can also create a dataset of a fixed kirigami designs and vary the physical parameters to support the investigations mentioned above. For both variations we believe that the results could benefit from a consideration of more advanced model architectures and machine learning techniques. If successful this would invite further studies of inverse design methods such as \acrshort{GAN} or diffusion models. 



% \begin{itemize}
%   \item Are there any interesting approaches for compressed kirigami structures?
%   \item Since the normal force is applied at the pull blocks the normal force distribution changes from the sides more towards and even distribution as the sheet is put under tension (stretched). If we imagined a sheet for which the center part was either a different material or had some kind of pre-placed asperity on it, could we then exploit this force distirbution to get exotic properties as well? By studying this we might get a clearer understanding of what is the cause of my results. 
%   \item Possibility to study hysteresis effects. Maybe the frictional behvaiours change significantly through repeated cycles of stretch and relax. 
%   \item Try making substrate rigid to investigate the effect of this. 
%   \item Is there any effects from going over the same substrate multiple times. This is probably mainly interesting in the context of discussing velocity relationship where the substrate has less time to relax.
%   \item Could be interesting to optimize solely for the pop-up effect, i.e.\ through the biggest change in contact area. Maybe this is the key, since the succesfull pattern is choosen in that way. The correlation supports this, and the fact that the correlation is only about 60\% can be explained by the fact that the cases without considerable pop-up might not make any difference or perhaps that only some kinds of pop-up triggers the effect. 
% \end{itemize}



% Thinks to do better in the simulation procedure:
% The graphene is first relaxed using conjugate gradient energy minimization and the relaxed within an NVT (fixed number of particles N, volume V and temperature T ) ensemble for 50 ps with non-periodic boundary conditions in all directions at a fixed temperature T = 4.2 K. The graphene is then stretched by moving the right and left edges at a constant strain rate of ∼ 0.005/ps which is slower than our previous work. \cite{PhysRevResearch.2.042006}

% Also to use AIREBO which is also used in the above.