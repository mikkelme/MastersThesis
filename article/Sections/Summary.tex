% \newpage
% \chapter*{Summary}
% \addcontentsline{toc}{chapter}{Summary} 
\chapter{Summary}\label{chap:summary}

\section{Summary and conclusion}

\section{Outlook / Perspective}

\begin{itemize}
  \item What did we not cover?
  \item What kind of further investigations does this study invite?
\end{itemize}


Things to include here
\begin{itemize}
  \item Could be valuable to spend more time on the validation of the MD simulations. How does material choice and potential effects the results. How realistic is the simmulations?
  \item Are there any interesting approaches for compressed kirigami structures?
  \item How does these results scale? I imagaind that the nanomachine systems should be applied in small units to avoid scaling problems, but in general I could spend way more time on the scaling investigation.
  \item Since the normal force is applied at the pull blocks the normal force distribution changes from the sides more towards and even distirbution as the sheet is put under tension (stretched). If we imagined a sheet for which the center part was either a different material or had some kind of pre-placed asperity on it, could we then exploit this force distirbution to get exotic properties as well? By studying this we might get a clearer understanding of what is the cause of my results. 
  \item Possibility to study hysteresis effects. Maybe the frictional behvaiours change significantly through repeated cycles of stretch and relax. 
  \item Try making substrate rigid to investigate the effect of this. 
  \item Let the structure relax a bit longer perhaps, just to be sure that this does not affect the results. 
  \item Extend load range
  \item Try turning of adhesive forces to investigate the effects from adhesive forces. 
  \item One way to investigate the importance of contact area is to turn of the friction force for of the atoms lifting away from the sheet during stretch. If this was purely a non-linear contact area dependency one would find the same results. If not we might point to the stress or other variables related to out-of-plane buckling.
  \item Is the stretching effect caused by a commensurability effect. It does change where the pressure is applied. To test this one should look at the curves for other sliding direction. However, this would probably also predict similar differences in friction at a zero stretch state between different configuration which I do not think we get. However, since the stretched configurations put atom outside the hexagonal grid stucture it is per definition not achieveable to get those states for a non-strecthed sheet which support the idea of a commensurability effect.
  \item How does it effect the results that we had a part on the pull blocks. 
  \item Is there any effects from going over the same substrate multiple times. This is probably mainly interesting in the context of discussing velocity relationship where the substrate has less time to relax.
  \item Interesting test could be to do the stretch curves with normal force applied on the whole sheet. The fact that the stretch provides tension might result in a better contact in the center... But the contact often goes down so maybe this was a silly idea.
  \item Recap the suggestion to study translational of the patterns.
  \item Could be interesting to optimize solely for the pop-up effect, i.e.\ through the biggest change in contact area. Maybe this is the key, since the succesfull pattern is choosen in that way. The correlation supports this, and the fact that the correlation is only about 60\% can be explained by the fact that the cases without considerable pop-up might not make any difference or perhaps that only some kinds of pop-up triggers the effect. 
  \item Top and bottom edges showed to be of importance for the model prediction which leads to the idea of the thermostat in the pull blocks being important. This could be investigated in a further study. 
\end{itemize}


% Thinks to do better in the simulation procedure:
% The graphene is first relaxed using conjugate gradient energy minimization and the relaxed within an NVT (fixed number of particles N , volume V and temperature T ) ensemble for 50 ps with non-periodic boundary conditions in all directions at a fixed temperature T = 4.2 K. The graphene is then stretched by moving the right and left edges at a constant strain rate of ∼ 0.005/ps which is slower than our previous work. \cite{PhysRevResearch.2.042006}

% Also to tru AIREBO which is also used in the above.