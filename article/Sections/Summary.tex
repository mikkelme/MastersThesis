\newpage
\chapter*{Summary}
\addcontentsline{toc}{chapter}{Summary} 

\section{Summary and conclusion}

\section{Outlook / Perspective}

\begin{itemize}
  \item What did we not cover?
  \item What kind of further investigations does this study invite?
\end{itemize}


Things to include here
\begin{itemize}
  \item Could be valuable to spend more time on the validation of the MD simulations. How does material choice and potential effects the results. How realistic is the simmulations?
  \item Are there any interesting approaches for compressed kirigami structures?
  \item How does these results scale? I imagaind that the nanomachine systems should be applied in small units to avoid scaling problems, but in general I could spend way more time on the scaling investigation.
  \item Since the normal force is applied at the pull blocks the normal force distribution changes from the sides more towards and even distirbution as the sheet is put under tension (stretched). If we imagined a sheet for which the center part was either a different material or had some kind of pre-placed asperity on it, could we then exploit this force distirbution to get exotic properties as well? By studying this we might get a clearer understanding of what is the cause of my results. 
\end{itemize}
