\newpage
\chapter*{Summary}
\addcontentsline{toc}{chapter}{Summary} 

\section{Summary and conclusion}

\section{Outlook / Perspective}

\begin{itemize}
  \item What did we not cover?
  \item What kind of further investigations does this study invite?
\end{itemize}


Things to include here
\begin{itemize}
  \item Could be valuable to spend more time on the validation of the MD simulations. How does material choice and potential effects the results. How realistic is the simmulations?
  \item Are there any interesting approaches for compressed kirigami structures?
  \item How does these results scale? I imagaind that the nanomachine systems should be applied in small units to avoid scaling problems, but in general I could spend way more time on the scaling investigation.
  \item Since the normal force is applied at the pull blocks the normal force distribution changes from the sides more towards and even distirbution as the sheet is put under tension (stretched). If we imagined a sheet for which the center part was either a different material or had some kind of pre-placed asperity on it, could we then exploit this force distirbution to get exotic properties as well? By studying this we might get a clearer understanding of what is the cause of my results. 
  \item Possibility to study hysteresis effects. Maybe the frictional behvaiours change significantly through repeated cycles of stretch and relax. 
  \item Try making substrate rigid to investigate the effect of this. 
  \item Let the structure relax a bit longer perhaps, just to be sure that this does not affect the results. 
  \item Extend load range
  \item Try turning of adhesive forces to investigate the effects from adhesive forces. 
\end{itemize}


% Thinks to do better in the simulation procedure:
% The graphene is first relaxed using conjugate gradient energy minimization and the relaxed within an NVT (fixed number of particles N , volume V and temperature T ) ensemble for 50 ps with non-periodic boundary conditions in all directions at a fixed temperature T = 4.2 K. The graphene is then stretched by moving the right and left edges at a constant strain rate of ∼ 0.005/ps which is slower than our previous work. \cite{PhysRevResearch.2.042006}

% Also to tru AIREBO which is also used in the above.