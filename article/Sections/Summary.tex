\chapter{Summary}\label{chap:summary}

The work presented in this thesis covers several topics (find a better opening
line?). We have created an \acrshort{MD} simulation which enabled us to study
the frictional behavior of a graphene sheet sliding on a Si substrate. In
addition, we have created a numerical framework for creating Kirigami design
patterns and introducing these into the friction simulations. This was used to
study the effects of the out-of-plane buckling induced by a selected pair of
Kirigami designs in relation to a non-cut sheet under the influence of strain.
Further, we have created a dataset of various Kirigami designs for the scope of
investigating the possibilities with Kirigami design. We have investigated the
possibility to use machine learning on this dataset and attempted an accelerated
search. Finally we look into the prospects of achieving a negative friction
coefficient for a system with coupled load and stretch. In this chapter we will
summarize the findings and draw some final conclusions. We will also provide
some topics for further research.

\section{Summary and conclusions}

% 1: Design MD simulations
\subsection{Design MD simulations}
We have designed an \acrshort{MD} simulation for the examination of friction for a graphene sliding on a substrate. Some of the key features for the numerical procedure were that we managed the sheet through pull blocks in the ends. We could then apply load and stretch the sheet without acting directly on the inner parts. Say something about parameter dependencies from the Pilot study. 


We find our simulation settings to yield a smooth sliding behavior. For softer spring and lower velocity we find a transition to stick-slip behvaior. 

% 2: Design Kirigami framework
\subsection{Design Kirigami framework}
We have desgined a numerical framework for creating Kirigami designs. By
defining an indexing system for the hexagonal lattice structure we were able to
define the Kirigami designs as 2D matrix for numerical implementation. We
digitalized two different macroscale designs, which we named the
\textit{Tetrahedron} and \textit{Honeycomb} pattern respectively, that
successfully produced out-of-plane buckling when stretched. Through a numerical
framework we could create an ensemble of perturbed variations which gave
approximately 135k configurations for the Tetrahedron pattern and 2025k patterns
for the size of the sheet used in our study. When considering the possibility to
translate the patterns this gave roughly a factor 100 more of unique
pertubations. We also created a framework for creating Kirigami designs through
a random walk. This was further controlled by introducing features such as bias,
avoidance of existing cuts, preference to keeping a direction and procedures to
repair the sheet for simulation purposes. The capabilities of the numerical framework for generating Kirigami designs was far larger than the capabilities for producing \acrshort{MD} designs within the time constraint of this thesis. Thus we believe that this contains the possibility to benefit more extended studies and for the creation of a larger dataset. 



% 3: Control friction using Kirigami
\subsection{Control friction using Kirigami}
We have investigated the friction behavior of the non-cut sheet and a selected
Tetrahedron and Honeycomb pattern under various stretch and load. The non-cut
sheet did not exhibit significant out-of-plane buckling as opposed to the
Tetrahedron and Honeycomb pattern. This is even when considering that the
non-cut sheet had a yield strain of $0.35$ while the Tetrahedron had a lower
yield strain of $0.21$ and the Honeycomb a considerable larger one at $1.27$
based on a stretch in vacuum. The out-of-plane buckling resulted in a
signifciant reduction of the contact area as the sheet were stretched, towards a
minimum of $X$ for the Tetrahedron and $y$ for the Honeycomb pattern. However,
this disagreed with the asperity theory hypothesis of a decreasing friciton with
decreasing contact area. We found that the strain-induced buckling was initially
(at low relative strain) assocated with an increase in friction. Moreover, the
friction-strain curve produced a no-linear behaviour which was not compatible
with the approximately monotonic decreasing contact area as strain were
increased. This is shown in \cref{fig:multi_stretch}.  This led us to the conclusion that the contact area cannot be attributed a dominant mechanism for friction throughout the straining of the studied Kirigami sheets. In general we found a non-existing relationship between friction and load considering the uncertainties in the simulation. This is best attributed to the superlubric state of the graphene sheet on the substrate. The slope of the friction-load curves were not significantly affected by the straining of the Kirigami sheet and thus we conclude that the load effect on friction is neglible cimpared to the strain effects. 


% 4: Capture trends with ML
\subsection{Capture trends with ML}
With the use of \acrshort{MD} simulations, we have generated an extended dataset
of 9660 data points based on 216 Kirigami configurations (Tetrahedron: 68,
Honeycomb: 45, Random walk: 100, Pilot study: 3) under various strains and normal loads. The dataset reveals some general correlations with mean friction,
such as a positive correlation to strain (0.77)
and porosity (0.60), and a negative correlation to contact area (-0.67). These
results align with the findings from the pilot study suggesting that these
features are relevant, but not necessarily the cause, of the observed phenomena.
By defining the friction property metrics: $\min F_{\text{fric}}$,  $\max
F_{\text{fric}}$, $\max \Delta F_{\text{fric}}$ and max drop (maximum decrease in friciton with strain), we investigated
the top candidates within our dataset. From these results, we found no incentive of the possibility to reduce friction with the Kirigami approach since
the non-cut sheet provided the lowest overall friction. Regarding the maximum
properties, we found an improvement from the original pilot study values and
with the Honeycomb pattern producing the highest scores. This suggests that the data contains some relevant information for optimization with respect to
these properties. Among the top candidates, we found that a flat friction-strain profile is mainly associated with little decrease in the
contact area and vice versa. 

For the machine learning investigation, we have implemented a VGGNet-16-inspired
convolutional neural network with a deep ``stairlike'' architecture:
C32-C64-C128-C256-C512-C1024-D1024-D512-D256-D128-D64-D32, for convolutional
layers $C$ with the number denoting channels and fully connected (dense) layers
$D$ with the number denoting nodes. The final model contains \num{1.3e7} and was
trained using the ADAM optimizer for a cyclic learning rate and momentum scheme
for 1000 epochs while saving the best model during training based on the
validation score. The model validation performance gives a mean friction $R^2$
score of $\sim 98\%$ and a rupture accuracy of $\sim 96 \%$. However, we got
lower scores for a selected subset of the Tetrahedon ($R^2 \sim 88.7 \%$) and
Honeycomb ($R^2 \sim 96.6$) pattern based on the top 10 max drop scores
respectively. These scores were lower despite the fact that the selected set was
partly included in the training data as well and the fact that the hyperparameter selection favored the performance on this selected set. Thus we conclude that these
selected configurations, associated with a highly non-linear friction-strain
curve, represent a bigger challennge for machine learning prediction. One interpretation is that these involve
the most complex dynamics and perhaps that this is not readily distinguished
from the behavior of the other configurations which constitutes the majoriry of the data set. By evaluating the ability for the model to rank the dataset according to the property scores we found in general a good representation of the top 3 scores for the maximum categories, while the minimum friction property ranking was lacking. We attribute this latter observation to a higher need for precision in order to rank the lowest friction values properly which the model did not possess.

In order provide a more true evaluation of the model performance we created a test set based on \acrshort{MD} simulations for an extended Random walk search. This test revealed a significantly worse performance than seen for the validation set with a two-order of magnitude higher loss and a negative friction mean $R^2$ score
which corresponds to the prediction being worse than simply guessing on a
constant value based on the true data mean. However, by considering one of the early hypertuning choices, regarding architecture complexity, we evaluated the model when prioritizing mainly for the lowest validation loss. This gave similar performance on the test set which indicates that it is not simply a product of a biased hypertuning process, since we based our choices on the selected confgiruation set (which overlapped with the training data). Instead, it points to the fact that our original dataset did not cover a wide enough configuration distribution to accurately capture the full physical complexity of the Kirigami friction behavior.  


% 5: Accelerated search
\subsection{Accelerated search}
Using the \acrshort{ML} model we performed two types of accelerated search. One
by evaluating the property scores of an extended dataset and another with the
use of the genetic algorithm approach. For the extended dataset search we used
the developed pattern generators to generate $\SI{135}{k} \times 10$
Tetrahedon, $\SI{2025}{k} \times 10$ Honeycomb and \SI{10}{k} Random walk
patterns. For the minimum friction property, the search suggests a favoring of a
low cut density (low porosity) which aligns with the overall idea that the
dataset does not provide an incentive for further friction reduction. The
maximum properties resulted in some minor score increases but the suggested
candidates were overlapping with the original dataset. By investigating the
sensitivity to translation of the Tetrahedron and Honeycomb patterns we found that the model predictions varied drastically with pattern translation. This can be attributed to a physical dependency since the edge of the sheet is effected by this
translation. However, due to the poor model performance on the test set, we find it more likely to be a model insufficienty arising from a lacking training dataset. 

For the genetic algorithm approach, we investigated the optimization for the max
drop property with respect to starting population based on the result from the
extended dataset accelerated search, and some random noise initializations with
different porosity values. This approach did not provide any noteworthy
incentive for new design structures worth more investigation. In general, the
initialization of the population itself proved to be a more promising strategy
tha the genetic algroithm. However, this is highly affected by the uncertainty
of the model predictions, and thus we did not pursue this any further. By
considering the Grad-CAM explanation method we found that the model predictions
sometimes seem to pay considerable attention to the top and bottom edge of the
configurations. This is surprising since these are not true edges but are
connected to the pull blocks in the simulation. Despite the uncertainties in the
predictions, we argue that this might be attributed to thermostat effects from
the pull blocks and thus we note this as a feature worth more studying. 



% 6: Negative friction coefficient
\subsection{Negative friction coefficient}
By enforcing a coupling between load and stretch, mimicking a nanomachine attached to the sheet, we investigated the load curves arising from loading of the Tetrahedron $(7,5,1)$ and Honeycomb $(2,2,1,5)$ pattern from the pilot study. The non-linear trend observed for increasing strain carried over to the coupled system as well producing a highly non-linear friction-load curve. This demonstrates a negative friction coefficient \hl{say something about the values}.



\section{Outlook / Perspective}
% Now we can lock in on a single Kirigami sheet design for further investigation

Having successfully demonstrated a non-linear effect on friction with increasing
strain of the sheet our results invite a series of further studies to investigate this relation. First of all, it would be valuable to investigate how the friction-strain curve depend on temperature, sliding speed, spring constant, and on load for an increased range $F_N > 100 nN$. This is especially interesting in the context up conditions leading to a stick-slip behvaior as our results were carried in the smooth sliding. Moreover, it would be important to verify that the choices for relaxation time and pauses are not critical for the qualitative observations as well as trying a different
interatomic potential for the graphene and perhaps an entirely different
substrate material. Especially the Adaptive Intermolecular Reactive Empirical
Bond Order (AIREBO) potential for the modeling of the graphene sheet might be of interest. The effects from excluding adhesion (the \acrshort{LJ} interaction) can also be useful for the investigation of the observed phenomena. 


In order to get a better understanding of the underlying mechanism for the friction-strain relationship we might investigate commensurability effects further by varying the scan angle. we might also consider investigating the friction-strain relationship under a uniform load to get insight into whether the loading distribution is of importance. Another topic worth studying is the relation to scale. Thus it would be interesting to study size effects but also further look into edge effects by translating the pattern. With this regard, we would also suggest a more detailed study of the effect from the thermostat in the pull blocks which is suggested to have a possibly importance by judging from the machine learning model Grad-CAM analysis. 

For machine learning, we can either try to extend the data set to resolve the issue of the model not being generalized enough. We can also create a dataset for a single kirigami design and include some of the mentioned physical variables above and attempt to use machine learning for unraveling these relations. In that context we would advice for as more detailed investigation of machine learning techniques. If succesfull this would invite a study of inverse design methods such as GAN or diffusion models. 

\begin{itemize}
  \item How is this behavior effected by scaling?
  \item How does the distirbution of normal load effect the Kirigami friction behavior?
  \item Things to vary: load range, scan directions, adhesive forces, longer relaxation time, different potential (AIREBO)
  \item Investigate if the contact area is effecting the friciton non-linear by turning of friction force for atoms corresopnding to those that lift of from the sheet during the out-of-plane buckling. 
  \item Investigations of commensurability effects.
  \item Study dependency of translation of the patterns as suggested by the ML results. 
  \item Investigate effects from pull blocks...
  \item Investigate effects from the thermostat since the top and bottom edges was shown interest by the model prediction.
\end{itemize}

% Things to include here
% \begin{itemize}
%   \item Could be valuable to spend more time on the validation of the MD simulations. How does material choice and potential effects the results. How realistic is the simmulations?
%   \item Are there any interesting approaches for compressed kirigami structures?
%   \item How does these results scale? I imagaind that the nanomachine systems should be applied in small units to avoid scaling problems, but in general I could spend way more time on the scaling investigation.
%   \item Since the normal force is applied at the pull blocks the normal force distribution changes from the sides more towards and even distribution as the sheet is put under tension (stretched). If we imagined a sheet for which the center part was either a different material or had some kind of pre-placed asperity on it, could we then exploit this force distirbution to get exotic properties as well? By studying this we might get a clearer understanding of what is the cause of my results. 
%   \item Possibility to study hysteresis effects. Maybe the frictional behvaiours change significantly through repeated cycles of stretch and relax. 
%   \item Try making substrate rigid to investigate the effect of this. 
%   \item Let the structure relax a bit longer perhaps, just to be sure that this does not affect the results. 
%   \item Extend load range
%   \item Try turning of adhesive forces to investigate the effects from adhesive forces. 
%   \item One way to investigate the importance of contact area is to turn of the friction force for of the atoms lifting away from the sheet during stretch. If this was purely a non-linear contact area dependency one would find the same results. If not we might point to the stress or other variables related to out-of-plane buckling.
%   \item Is the stretching effect caused by a commensurability effect. It does change where the pressure is applied. To test this one should look at the curves for other sliding direction. However, this would probably also predict similar differences in friction at a zero stretch state between different configuration which I do not think we get. However, since the stretched configurations put atom outside the hexagonal grid stucture it is per definition not achieveable to get those states for a non-strecthed sheet which support the idea of a commensurability effect.
%   \item How does it effect the results that we had a part on the pull blocks. 
%   \item Is there any effects from going over the same substrate multiple times. This is probably mainly interesting in the context of discussing velocity relationship where the substrate has less time to relax.
%   \item Interesting test could be to do the stretch curves with normal force applied on the whole sheet. The fact that the stretch provides tension might result in a better contact in the center... But the contact often goes down so maybe this was a silly idea.
%   \item Recap the suggestion to study translational of the patterns.
%   \item Could be interesting to optimize solely for the pop-up effect, i.e.\ through the biggest change in contact area. Maybe this is the key, since the succesfull pattern is choosen in that way. The correlation supports this, and the fact that the correlation is only about 60\% can be explained by the fact that the cases without considerable pop-up might not make any difference or perhaps that only some kinds of pop-up triggers the effect. 
%   \item Top and bottom edges showed to be of importance for the model prediction which leads to the idea of the thermostat in the pull blocks being important. This could be investigated in a further study. 
% \end{itemize}



% Thinks to do better in the simulation procedure:
% The graphene is first relaxed using conjugate gradient energy minimization and the relaxed within an NVT (fixed number of particles N , volume V and temperature T ) ensemble for 50 ps with non-periodic boundary conditions in all directions at a fixed temperature T = 4.2 K. The graphene is then stretched by moving the right and left edges at a constant strain rate of ∼ 0.005/ps which is slower than our previous work. \cite{PhysRevResearch.2.042006}

% Also to tru AIREBO which is also used in the above.