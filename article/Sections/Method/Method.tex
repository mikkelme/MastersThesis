\newpage
\chapter*{Method}
\addcontentsline{toc}{chapter}{Method} 

Big lines
\begin{itemize}
    \item Make indexing system/ description of the sheet 
    \item Collect data 
    \subitem pop-up pattern
    \subitem RN walk 
    \subitem RN straight cuts?
    \subitem RN single atoms removes
    \subitem Rules for patterns
    \item Train mahcine learning algorithm to predict properties
    \subitem Static/Dynamic friction coefficient from atom matrix. 
\end{itemize}    


Possible subjects
\begin{itemize}
    \item Indexing the graphene sheet
    \item Creating a pop-up pattern
    \item Potentials and materials
    \item Creating substrate
    \subitem quenching
    \item Creating data sets
    \subitem random walk?
\end{itemize}    




\subsubsection*{Choosing material and potentials}

Looking at https://aip.scitation.org/doi/pdf/10.1063/1.481208.

The main material of study is the graphene sheet. Graphene is simply a single layer of graphite. For the friction study we need a substrate and a tip which pushes down into the sheet. For the tip and substrate we have considered both diamond and silicon. Here we look at tersoff, REBO and Airebo as possible potentials candiates for intramolecular potentials. For the intermolecular potential we can use a typical 12-6 Lennard-Jones (LJ) potential. Could also choose exp-6 potential which is slightly more complex I think. The repulsive wall is known to be quite hard. Above article is talking about a LJ switch to overcome the hard repulsive wall.  


The LJ potential is taking from https://pubs.rsc.org/en/content/articlehtml/2015/nr/c4nr07445a refering to https://journals.aps.org/prb/pdf/10.1103/PhysRevB.81.155408.


\subsubsection*{Work in progress simulation setup}
Silicon substrate (crystalline or amorphous) with a single graphene sheet resting on top. A Si tip apex described as a rigid body connected to a moving support (with no atomic interaction) via a harmonic spring to drag the tip apex across the sheet. \par
Step 1: Load the tip with a normal force such that the tip begin to interact with the sheet. Step 2: Drag the tip in the horizontal direction and measure either static or dynamic friction. 

- Which way to drag? Different angles (zigzag direction, armchair direction or something inbetween). The optimial cut-pattern for friction properties will depend on the "scan" angle (see https://pubs.rsc.org/en/content/articlehtml/2015/nr/c4nr07445a). 