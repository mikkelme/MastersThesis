\newpage
\chapter*{Method}
\addcontentsline{toc}{chapter}{Method} 

\section{Free floating bullet points to remember}
\begin{itemize}
  \item Describe two different approaches:
  \subitem Nanomachine setup (sheet as the inner layer of nanomachine influencing the stretch)
  \subitem Graphene skin setup (sheet on the outside probed with tip, stretched and fixed on object)
\end{itemize}


\section{Setting up the system}
\begin{itemize}
  \item Substrate material (crystalline or amorphous)
  \item Intra- and intermolecular potentials
  \item Ensembles: NVE, NVT
  \item Choice of dt, relax time etc.  
\end{itemize}
\section{Measuring properties}
\begin{itemize}
  \item Out-of-plane buckling
  \item Contact area
  \item Friction (static, dynamic)
\end{itemize}


\section{Making cuts in graphene}
\begin{itemize}
  \item Indexing the sheet
  \item  Manual patterns as a starting point(Pop-up pattern and half octans) 
  \item Cut rules and problems with dangling fringes
  \item Different variations of manual patterns 
  \item Random walks 
\end{itemize}

\section{Simulation procedures}
\begin{itemize}
  \item Relaxing
  \item Stretching 
  \item Friction 
  \item Different combinations of stretch and applied normal force
\end{itemize}

\section{Working title: tweeking simulation settings} % Test effects of the following bullet points
\begin{itemize}
  \item Substrate structure
  \item Drag speed
  \item Spring stifness
  \item ...
\end{itemize}

\section{Sampling data}
\begin{itemize}
  \item Different drag angles
\end{itemize}

\section{Machine learning}
\begin{itemize}
  \item Input: atom position matrix
  \item Target properties: friction coefficient (low/high), maybe load curve for nonlinear relations
  \item Output: Cut pattern, stretch amount (\%)
  \item Architecture and network types
  \item Loss function and evaluation
\end{itemize}



\newpage
Big lines
\begin{itemize}
    \item Make indexing system/ description of the sheet 
    \item Collect data 
    \subitem pop-up pattern
    \subitem RN walk 
    \subitem RN straight cuts?
    \subitem RN single atoms removes
    \subitem Rules for patterns
    \item Train mahcine learning algorithm to predict properties
    \subitem Static/Dynamic friction coefficient from atom matrix. 
\end{itemize}    


Possible subjects
\begin{itemize}
    \item Indexing the graphene sheet
    \item Creating a pop-up pattern
    \item Potentials and materials
    \item Creating substrate
    \subitem quenching
    \item Creating data sets
    \subitem random walk?
\end{itemize}    


\subsection*{Things to remember}
\begin{itemize}
    \item Word: Nanotribology
\end{itemize}

\subsubsection*{Choosing material and potentials}

Looking at https://aip.scitation.org/doi/pdf/10.1063/1.481208.

The main material of study is the graphene sheet. Graphene is simply a single layer of graphite. For the friction study we need a substrate and a tip which pushes down into the sheet. For the tip and substrate we have considered both diamond and silicon. Here we look at tersoff, REBO and Airebo as possible potentials candiates for intramolecular potentials. For the intermolecular potential we can use a typical 12-6 Lennard-Jones (LJ) potential. Could also choose exp-6 potential which is slightly more complex I think. The repulsive wall is known to be quite hard. Above article is talking about a LJ switch to overcome the hard repulsive wall.  


The LJ potential is taking from https://pubs.rsc.org/en/content/articlehtml/2015/nr/c4nr07445a refering to https://journals.aps.org/prb/pdf/10.1103/PhysRevB.81.155408.


\subsubsection*{Work in progress simulation setup}
Silicon substrate (crystalline or amorphous) with a single graphene sheet resting on top. A Si tip apex described as a rigid body connected to a moving support (with no atomic interaction) via a harmonic spring to drag the tip apex across the sheet. \par
Step 1: Load the tip with a normal force such that the tip begin to interact with the sheet. Step 2: Drag the tip in the horizontal direction and measure either static or dynamic friction. 

- Which way to drag? Different angles (zigzag direction, armchair direction or something inbetween). The optimial cut-pattern for friction properties will depend on the "scan" angle (see https://pubs.rsc.org/en/content/articlehtml/2015/nr/c4nr07445a). 


\subsubsection*{Find right timestep}
From article (Nanoscrathing of multi-layer graphene): The equations of particles motion were solved using the Verlet algorithm, and the simulation time step is 1 fs, which is adequate for system relaxation by examining the stability through the root mean square deviations of the atoms.