\chapter*{Abstract} 
% https://www.nature.com/documents/nature-summary-paragraph.pdf


\subsubsection*{Basic introduction}
Various numerical models and experimental results propose different governing mechanisms for friction at the nanoscale.


\subsubsection*{More detailed background}

We consider a graphene sheet modified with Kirigami-inspired cuts and under the influence of strain. Prior research has demonstrated that this system exhibits out-of-plane buckling, which could result in a decrease in contact area when sliding on a substrate. According to asperity theory, this decrease in contact area is expected to lead to a reduction of friction.


\subsubsection*{General problem}
However, to the best of our knowledge, no previous studies have investigated the friction behavior of a nanoscale Kirigami graphene sheet under strain. 


\subsubsection*{Summarize main result: ``here we show''}
Here we show that specific Kirigami designs yield a non-linear dependency between kinetic friction and the strain of the sheet.


\subsubsection*{General context}
Using molecular dynamics simulation, we have found a non-monotonic increase in friction with strain. We found that the friction-strain relationship does not show any clear dependency on contact area which contradicts asperity theory. Our findings suggest that the effect is associated with the out-of-plane buckling of the graphene sheet and we attribute this to a commensurability effect. By mimicking a load-strain coupling through tension, we were able to utilize this effect to demonstrate a negative friction coefficient on the order of $-0.3$ for loads in the range of a few nN. In addition, we have attempted to use machine learning to capture the relationship between Kirigami designs, load, and strain, with the objective of performing an accelerated search for new designs. While this approach produced some promising results, we conclude that further improvements to the dataset are necessary in order to develop a reliable model.

\subsubsection*{Broader perspective}
We anticipate our findings to be a starting point for further investigations of the underlying mechanism for the frictional behavior of a Kirigami sheet. For instance, the commensurability hypothesis could be examined by varying the sliding angle in simulations. We propose to use an active learning strategy to extend the dataset for the use of machine learning to assist these investigations. If successful, further studies can be done on the method of inverse design. In summary, our findings suggest that the application of nanoscale Kirigami can be promising for developing novel friction-control strategies.

\newpage

\section*{353 words}
Various numerical models and experimental results propose different governing
mechanisms for friction at the nanoscale. We consider a graphene sheet modified
with Kirigami-inspired cuts and under the influence of strain. Prior research
has demonstrated that this system exhibits out-of-plane buckling, which could
result in a decrease in contact area when sliding on a substrate. According to
asperity theory, this decrease in contact area is expected to lead to a
reduction of friction. However, to the best of our knowledge, no previous
studies have investigated the friction behavior of a nanoscale Kirigami graphene
sheet under strain. Here we show that specific Kirigami designs yield a
non-linear dependency between kinetic friction and the strain of the sheet.
Using molecular dynamics simulation, we have found a non-monotonic increase in
friction with strain. We found that the friction-strain relationship does not
show any clear dependency on contact area which contradicts asperity theory. Our
findings suggest that the effect is associated with the out-of-plane buckling of
the graphene sheet and we attribute this to a commensurability effect. By
mimicking a load-strain coupling through tension, we were able to utilize this
effect to demonstrate a negative friction coefficient on the order of $-0.3$ for
loads in the range of a few nN. In addition, we have attempted to use machine
learning to capture the relationship between Kirigami designs, load, and strain,
with the objective of performing an accelerated search for new designs. While
this approach produced some promising results, we conclude that further
improvements to the dataset are necessary in order to develop a reliable model.
We anticipate our findings to be a starting point for further investigations of
the underlying mechanism for the frictional behavior of a Kirigami sheet. For
instance, the commensurability hypothesis could be examined by varying the
sliding angle in simulations. We propose to use an active learning strategy to
extend the dataset for the use of machine learning to assist these
investigations. If successful, further studies can be done on the method of
inverse design. In summary, our findings suggest that the application of
nanoscale Kirigami can be promising for developing novel friction-control
strategies.
