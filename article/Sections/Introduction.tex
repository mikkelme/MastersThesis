\chapter{Introduction}


\section{Motivation}
Friction is the force that prevents the relative motion of objects in contact. Even though the everyday person might not be familiar with the
term \textit{friction} we recognize it as the inherent resistance to sliding
motion. Some surfaces appear slippery and some rough, and we know intuitively
that sliding down a snow-covered hill is much more exciting than its grassy
counterpart. Without friction, it would not be possible to walk across a flat
surface, lean against the wall without falling over or secure an object by the
use of nails or screws [p. 5] \cite{gnecco_meyer_2015}. It is probably safe to
say that the concept of friction is integrated into our everyday life to such an
extent that most people take it for granted. However, the efforts to control
friction date back to the early civilization (3500 B.C.) with the use of the
wheel and lubricants to reduce friction in translational motion
\cite{bhushan_2013}. Today, friction is considered a part of the wider field
\textit{tribology} derived from the Greek word \textit{Tribos} meaning
``rubbing'' and includes the science of friction, wear and lubrication
\cite{bhushan_2013}. The most compelling motivation to study tribology is
ultimately to gain full control of friction and wear for various technical
applications. Especially, reducing friction is of great interest as this has
tremendous advantages for energy efficiency. It has been reported that
tribological problems have a significant potential for economic and
environmental improvements \cite{kim_nano-scale_2009}:
\begin{quote}
    ``On global scale, these savings would amount to 1.4\% of the GDP annually
    and 8.7\% of the total energy consumption in the long term.''
    \cite{holmberg_influence_2017}. 
\end{quote}
On the other hand, the reduction of friction is not the only sensible
application for tribological studies. Controlling frictional properties, besides
minimization, might be of interest in the development of a grasping robot where
finetuned object handling is required. While achieving a certain ``constant''
friction response is readily obtained through appropriate material choices, we are yet to unlock the full capabilities to alter friction
dynamically on the go. One example from nature inspiring us to think along
these lines are the gecko feet. More precisely, the Tokay gecko has received a
lot of attention in scientific studies aiming to unravel the underlying
mechanism of its ``togglable'' adhesion properties. Although geckos can
produce large adhesive forces, they retain the ability to remove their feet from
an attachment surface at will \cite{Gekko}. This makes the gecko able to achieve a high adhesion on the feet when climbing a vertical surface while lifting them for the next step remains relatively effortless. For a grasping robot, we might
consider an analog frictional concept of a surface material that can change from
slippery to rough on demand depending on specific tasks; Slippery and smooth when interacting with people and rough and firmly gripping when moving heavy objects.


In recent years an increasing amount of interest has gone into the studies of
the microscopic origin of friction, due to the increased possibilities in
surface preparation and the development of nanoscale experimental methods.
Nano-friction is also of great concern for the field of nano-machining where the
frictional properties between the tool and the workpiece dictate machining
characteristics \cite{kim_nano-scale_2009}. With concurrent progress in
computational capacity and development of Molecular Dynamics (\acrshort{MD}),
numerical investigations serve as an invaluable tool for getting insight into
the nanoscale mechanics associated with friction. This simulation-based approach
can be considered as a ``numerical experiment'' enabling us to create and probe
a variety of high-complexity systems which are still out of reach for modern
experimental methods.

In materials science such \acrshort{MD}-based numerical studies have been used
to explore the concept of so-called \textit{metamaterials} where material
compositions are designed meticulously to enhance certain physical properties
\cite{PhysRevLett.121.255304, PhysRevResearch.2.042006, graphene/hBN, Mao, Yang,
Forte}. This is often achieved either by intertwining different material types
or removing certain regions completely. In recent papers by Hanakata et al.\
\cite{PhysRevLett.121.255304, PhysRevResearch.2.042006}
numerical studies have showcased that the mechanical properties of a graphene
sheet, yield stress and yield strain, can be altered through the introduction of
so-called \textit{kirigami} inspired cuts into the sheet. Kirigami is a
variation of origami where the paper is cut additionally to being folded. While
these methods originate as an art form, aiming to produce various artistic
objects, they have proven to be applicable in a wide range of fields such as
optics, physics, biology, chemistry and engineering
\cite{chen_kirigamiorigami_2020}. Various forms of stimuli enable direct 2D to
3D transformations through folding, bending, and twisting of microstructures.
While original human designs have contributed to specific scientific
applications in the past, the future of this field is highly driven by the
question of how to generate new designs optimized for certain physical
properties. However, the complexity of such systems and the associated design
space make for seemingly intractable problems ruling out analytic solutions.


Earlier architecture design approaches such as bioinspiration, looking at gecko
feet for instance, and Edisonian, based on trial and error, generally rely on
prior knowledge and an experienced designer \cite{Mao}. While the Edisonian
approach is certainly more feasible through numerical studies than real-world
experiments, the number of combinations in the design space rather quickly
becomes too large for a systematic search, even when considering the computation
time on modern-day hardware. However, this computational time constraint can be
relaxed by the use of machine learning (\acrshort{ML}) which has proven
successful in the establishment of a mapping from the design space to physical
properties of interest. This gives rise to two new styles of design approaches:
One, by utilizing the prediction from a trained network we can skip the
\acrshort{MD} simulations altogether resulting in an \textit{accelerated search}
of designs. This can be further improved by guiding the search accordingly to
the most promising candidates, for instance, as done with the \textit{genetic
algorithm} based on mutation and crossing of the best candidates so far. Another
more sophisticated approach is through generative methods such as
\textit{Generative Adversarial Networks} (GAN) or diffusion models with the
latter being used in state-of-the-art AI systems such as OpenAI's DALL$\sq$E2 or
Midjourney \hl{SOURCE?}. By working with a so-called \textit{encoder-decoder}
network structure, one can build a model that reverses the prediction process.
That is, the model predicts a design from a set of physical target properties.
In the papers by Hanakata et al.\ both the \textit{accelerated search} and the
\textit{inverse design} approach was proven successful to create novel
metamaterial kirigami designs with the graphene sheet. 

Hanakata et al.\ attribute the variety in yield properties to the non-linear
effects arising from the out-of-plane buckling of the sheet. Since it is
generally accepted that the surface roughness is of great importance for
frictional properties it can be hypothesized that Kirigami-induced out-of-plane buckling can also be exploited for the design of frictional metamaterials. For
certain designs, we might hope to find a relationship between the stretching of the
sheet and frictional properties. If significant, this could give rise to an adjustable friction behavior beyond the point of manufacturing. For
instance, the grasping robot might apply such a material as artificial skin for
which stretching or relaxing of the surface could result in a changeable friction strength.

In addition, the Kirigami graphene properties can be explored through a
potential coupling between the stretch and the normal load, through a
nanomachine design, with the aim of altering the friction coefficient. This
invites the idea of non-linear friction coefficients which might in theory also
take on negative values. The latter would constitute a rarely found property
which is mainly found for the unloading phase of adhesive surfaces
\cite{deng_adhesion-dependent_2012} or for the loading phase of particular heterojunction materials \cite{Liu_2020, Mandelli_2019}

To the best of our knowledge, Kirigami has not yet been implemented to alter the
frictional properties of a nanoscale system. However, in a recent paper by
Liefferink et al.\ \cite{LIEFFERINK2021101475} it is reported that macroscale
kirigami can be used to dynamically control the macroscale roughness of a
surface through stretching. They reported that the roughness change led to a
changeable frictional coefficient by more than one order of magnitude. This
supports the idea that Kirigami designs can be used to alter friction, but we
believe that taking this concept to the nanoscale regime would involve a different set of underlying mechanisms and thus contribute to new insight in
this field.



%%%%%%%%%%%%%%%%%%%%%%%%%%%%%%%%%%%%%%%%%%%%%%%%%%%%%%%%%%%%%%%%%%%%%%%%%%%%%%
%%%%%%%%%%%%%%%%%%%%%%%%%%%%%%%%%%%%%%%%%%%%%%%%%%%%%%%%%%%%%%%%%%%%%%%%%%%%%%




% % The usual approach is to perform “active learning” where the model is
% trained incrementally with data proposed by the ML [8, 11], or by training the
% model with a significant amount of data to predict top candidates [12]. For
% both approaches, ML (the “forward solver”) must be applied to the entire
% library. Even when the computational cost of the ML approach is much lower
% than the ground truth data generator (physics-based simulations or
% experimental data), in a highly complex system with many degrees of freedom,
% it is not practical to use ML to calculate the properties of all candidates to
% find the best candidates. \cite{PhysRevResearch.2.042006}


% Graphite is undoubtedly the most common solid lubricant. Mainly used as flaky
% powder, it is e.g. applied where liquid lubricants cannot be used, especially
% in high-temperature applications, or as a friction-reducing additive in oils and
% polymers and as brushes in electrical motors. It is not surprising that
% research on the tribological properties of graphite has a long history and
% it is well established that the friction coefficient for many materials
% against graphite in ambient conditions is in the range of 0.08–0.18
% \cite{DIENWIEBEL2005197}


\section{Goals} % Approach 
In this thesis, we investigate the possibility to alter the frictional
properties of a graphene sheet through the application of Kirigami-inspired cuts and stretching of the sheet. With the use of molecular dynamics (\acrshort{MD}) simulations, we evaluate the frictional properties of various Kirigami designs under different physical conditions. With the use of machine learning (\acrshort{ML}), we perform an accelerated search of designs with the aim of exploring new designs. The main goals of this thesis can be summarized as follows.
\begin{enumerate} 
    \item Design a robust \acrshort{MD} simulation procedure to evaluate the frictional properties of a Kirigami graphene sheet under specified physical conditions.
    \item Develop a numerical framework for creating various Kirigami designs, both by seeking inspiration from macroscale designs and by the use of stochastically based algorithms.
    \item Investigate the friction behavior under varying load and stretch for a selected subset of Kirigami designs.
    \item Develop and train a \acrshort{ML} model to predict the \acrshort{MD} simulation result and perform an accelerated search of new designs for the scope of optimizing certain frictional properties.
\end{enumerate}




% In the study by Hanakata et al.\ \cite{PhysRevResearch.2.042006} they used a
% machine learning (ML) approach to overcome the complexity of the nonlinear
% effects arising from the out-of-plane buckling which made them successfully
% map the cutting patterns to the mechanical properties of yield and stress. The
% dataset used for the ML training was generated by molecular dynamics (MD)
% simulations for a limited set of cut configurations. By training a neural
% network the MD simulations could effectively be skipped altogether making for
% an accelerated search through new cut configurations for certain mechanical
% properties. By setting up a MD simulation that quantifies the frictional
% properties of the graphene sheet we aim to make an analog study regarding the
% search for certain frictional properties. 

% We will take this one step further by creating a GAN network that utilizes the
% latter network for creating an inverse design framework. That is, a network
% that takes frictional properties as input and returns the corresponding cut
% configuration. By having such a tool we can execute a targeted search for
% exotic frictional properties. Particularly, we are interested in nonlinear and
% possibly even negative friction coefficients. Friction is essentially observed
% to increase with increasing load on the frictional surface, and we often
% describe this as having a positive friction coefficient. However, if we are
% able to couple the stretching of the sheet with friction we might be able to
% break this barrier for the coefficient. By imagining some nanomachine that
% translates downward pressure into either compression or expansion of the
% altered graphene, we could have a coupling between downward pressure and
% stretch of the sheet. In that case, a friction force depending on stretch
% could effectively be made to decrease with an increasing load which would
% correspond to a negative friction coefficient following this definition
% (formulate such that we do not imply free acceleration from anything).

% One of the features of inverse design, separating it from the general class of
% ML approaches are that we do not depend on trusting the ML predictions. While
% a standard neural network might be extremely efficient on a certain prediction
% task we have usually no information on how these predictions are based. We say
% that the internal workings of the network are a black box beyond our capacity
% for interpretation. However, for the inverse design problem, we are prompted
% with a few promising design proposals which can immediately be tested in the
% MD simulations which we will regard as the most reliable predictor in this
% setting. Hence, if arriving at a successful design in alignment with our
% search prompt, we can disregard any uncertainty in the network. In that case,
% the remaining gap to bridge is that of the MD simulation and real-life
% implementations. 


% The good thing about the process of inverse design is that the uncertainty and
% missing information from the network (black box) is not important if we are
% able to locate a working design. Thus we can test the suggested design and
% remove the doubt of whether this is a good design or not. Thus we do not have
% to trust the network prediction at all. However, if the predictions are not
% accurate enough we will most likely never get any useful designs from the ML
% process, but at least we will be informed whether the designs are good or bad
% in the end with respect to the simulations. However, the question is rather if
% we can trust the simulation results. In the end, we should test the designs in
% real life to be completely sure and thus the certainty of the quality of any
% proposed designs is determined by the simulation quality. 


% On the topic of theoretical/physical understanding benefits of using ML for design proposals. Since a successful ML search or inverse design procedure will not immediately grant any physical insight this does not necessarily exclude physical understanding as a possible benefit. When solving design problems scientists often seek inspiration from nature, i.e.\ biological solutions build from the greatest and probably most complex parameter optimization search ever recorded, evolution. Let it be that of Gekko feet or XXX. By studying the working mechanism of biological systems we can unravel exotic mechanisms which can be exploited for artificial/human build contraptions. Similarly, we can consider the output of a successful ML framework similar to that of evolution. We are not immediately granted a physical explanation for its working, but we now have a case study for which we can seek to understand this behavior. 

% Defining the goal of the thesis and restrictions Make bullet point objectives
% for the thesis and state which is completed, which is perhaps not conclusive
% and which I did not answer at all. Perhaps also make a list of
% problems/questions to answer (also state which one I actually answer here).


% Generally we want to contribute to the understanding of nanoscale friction while finding kirigami designs associated with exotic friction properties. This also serves as a proof of concept for future work in this direction, perhaps with some more clearly defined applications in mind. 




\section{Contributions}
\hl{What did I actually achieve}
\hl{Include Githib link} 

\section{Thesis structure}

In \cref{part:theory}: Background Theory, we cover the theoretical background related to Friction (\cref{chap:friction}), Molecular Dynamics (\cref{chap:MD}) and Machine Learning (\cref{chap:ML}). 

In \cref{chap:friction}: Friction, we introduce the most relevant theoretical concepts of friction through a division by scale: Macroscale (\cref{sec:macroscale}), Microscale (\cref{sec:microscale}) and nanoscale (\cref{sec:nanoscale}). We emphasize the nanoscale since this is of the most importance for our study. This is followed by a summary of relevant experimental and numerical results \cref{sec:prev_results} and a more formal specification of our research questions (\cref{sec:research_questions}). 

In \cref{chap:MD}: Molecular Dynamics, we introduce the main concepts related to the simulations used in this thesis. The main parts involve a description of the potentials used (\cref{sec:potentials}), the numerical solutions (\cref{sec:integration}) and the modeling of temperature (\cref{sec:thermostat})

In \cref{chap:ML}: Machine Learning, we introduce the basics of machine learning through a general presentation of the neural network \cref{sec:NN} followed by the convolutional network (\cref{sec:CNN}) which we will use in our study. Additionally, we discuss a strategy for choosing model hypertuning (\cref{sec:hypertuning}) and a simple approach for model prediction explanations (\label{sec:explanation}). Finally, we introduce a version of the genetic algorithm applicable for accelerated search based on a machine learning model (\cref{sec:GA}).
\\
\\
In \cref{part:simulations}: Simulations, we define our numerical procedure and present and discuss the main findings of this thesis. 
% our definition of the system  (\cref{chap:system}), an initial pilot study of a subset of Kirigami designs (\cref{sec:pilot_study})


In \cref{chap:system}: System, we ...

In \cref{sec:pilot_study}: Pilot study, we ...

In \cref{chap:dataset_study}: XXX ...

In \cref{chap:negative_coef}: XXX, ...

The thesis is summarized in \cref{chap:summary}
\\
\\
Additional figures are shown in \cref{sec:sheet_stretch}, \cref{sec:data_stretch_profiles} and \cref{sec:dataset_conf}. \hl{get appendix with only letter A., B. and C}.



% These experiments have demonstrated that the relationship between friction and
% surface roughness is not always simple or obvious. (Introduction to Tribology,
% p. 527).


% “In other words, it’s not just the material itself” that determines how it
% slides, but also its boundary condition — including whether it is loose and
% wrinkled or flat and stretched tight, he says.
% (https://news.mit.edu/2016/sliding-flexible-graphene-surfaces-1123).<--
% Talking about the quality of contact for friction.