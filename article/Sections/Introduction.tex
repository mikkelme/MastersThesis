\chapter*{Introduction}
\addcontentsline{toc}{chapter}{Introduction} 

\section{Some headline for the introtext}

\section{Introduction to friction and motivation}
friction a part of the (wider) field tribology.

\begin{itemize}
    \item Where is friction important (motivation)
    \item The economical interest in tribology (quote)
    \item The missing knowdelegde about friction.
    \item What possibilities do we have if we could control friction (friction coefficient).
\end{itemize}


\section{Introduction to MD simulations and machine learning approaches}


The good thing about the process of inverse design is that the uncertainty and missing information from the network (black box) is not important if we are able to locate a working design. Thus we can test the suggested design and remove the doubt of whether this is a good design or not. Thus we do not have to trust the network prediction at all. However, if the predictions is not accurate enough we will most liekely never get any usefull designs from the ML process, but at least we will be informed whether the designs are good or bad in the end with respect to the simulations. However the question is rather if we can trust the simulation results. In the ned we should test the designs in real life to be completely sure and thus the certainity of the quality of any proposed designs is determined by the simulation quality. 


\section{Defining the goal of the thesis and restrictions}
Make bullet point objectives for the thesis and state which is completed, which is perhaps not conslusive and which I did not answer at all / do at all. Perhaps also make a list of problems/questions to answer (also state which one I actually answer here).


\section{Contributions}

\section{Thesis structure }

\newpage
Introduction. A citation to avoid error for now: \cite{li_evolving_2016}.

\begin{itemize}
    \item Nanotribology
    \item Quantitative Structure-Property Relationship
    \item Forward simulation using ML
    \item Inverse designs
\end{itemize}


Practically, systems achieving low values of dry sliding friction are of great technological interest to significantly reduce dissipation and wear in mechanical devices functioning at various scales. (Current trends in the physics of nanoscale friction)

These experiments have demonstrated that the relationship between friction and surface roughness is not always simple or obvious. (Introduction to Tribology, p. 527).


“In other words, it’s not just the material itself” that determines how it slides, but also its boundary condition — including whether it is loose and wrinkled or flat and stretched tight, he says. (https://news.mit.edu/2016/sliding-flexible-graphene-surfaces-1123).<-- Talking about quality of contact for friciton.