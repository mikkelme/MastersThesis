\chapter*{Introduction}
\addcontentsline{toc}{chapter}{Introduction} 

\section{Motivation}


Friction is a fundamental force that takes part in almost all interactions with
physical matter. Even though the everyday person might not be familiar with the
term ``friction'' we would undoubtedly notice its disappearing. Without
friction, it would not be possible to walk across a flat surface, lean against
the wall or secure an object by the use of nails or screws. Similarly, we expect
a moving object to eventually come to a stop if not supplied with new energy,
and we know intuitively that sliding down a snow covered hill is much more
exciting than its grassy counterpart. It is probably safe to say that the
concept of friction is well integrated in our everyday life to such an extent
that most people take it for granted. However, the efforts to control friction
dates back to the early civilization (3500 B.C.) with the use of the wheel and
lubricants to reduce friction in translational motion \cite{bhushan_2013}. Friction is a part of the wider field tribology derived from the Greek word \textit{Tribos} meaning rubbing and includes the science of friction, wear and lubrication \cite{bhushan_2013}.

The most important motivation to study tribology is ultimately to gain full control of frictional and wear for vairous technical applications. Especially, reducing friction is of great interest as this has tremendous advantages regarding
energy effeciency. It has been reported that that monetary value of tribological problems has significant potential for economic and environmental improvements \cite{kim_nano-scale_2009}:

\begin{quote}
    ``On global scale, these savings would amount to 1.4\% of the GDP annually and 8.7\% of the total energy consumption in the long term.'' \cite{holmberg_influence_2017}. 
\end{quote}

The reduction of friction is not the only sensible application as a controlled increase in friction might be of interest in the development of grasping robots or perhaps breaking system (get some sourced examples maybe...). 


In this thesis we investigate the 

Friction is classically/most of the time/by default observed to increase with increasing load on the frictional surface. We knoe this intuitively as the fact that a heavy box is more exhausting to push across the floor than a lighter one. 


With the current tribological knowdeledge the friction is governed positive friction coefficients. That is, the


Today's friction application is ale governed by the 




% \section{Introduction to friction and motivation}
% friction a part of the (wider) field tribology.

% \begin{itemize}
%     \item Where is friction important (motivation)
%     \item The economical interest in tribology (quote)
%     \item The missing knowdelegde about friction.
%     \item What possibilities do we have if we could control friction (friction coefficient).
% \end{itemize}


\section{Introduction to MD simulations and machine learning approaches}


The good thing about the process of inverse design is that the uncertainty and missing information from the network (black box) is not important if we are able to locate a working design. Thus we can test the suggested design and remove the doubt of whether this is a good design or not. Thus we do not have to trust the network prediction at all. However, if the predictions is not accurate enough we will most liekely never get any usefull designs from the ML process, but at least we will be informed whether the designs are good or bad in the end with respect to the simulations. However the question is rather if we can trust the simulation results. In the ned we should test the designs in real life to be completely sure and thus the certainity of the quality of any proposed designs is determined by the simulation quality. 


\section{Defining the goal of the thesis and restrictions}
Make bullet point objectives for the thesis and state which is completed, which is perhaps not conslusive and which I did not answer at all / do at all. Perhaps also make a list of problems/questions to answer (also state which one I actually answer here).


\section{Contributions}

\section{Thesis structure }



% \begin{itemize}
%     \item Nanotribology
%     \item Quantitative Structure-Property Relationship
%     \item Forward simulation using ML
%     \item Inverse designs
% \end{itemize}


% Practically, systems achieving low values of dry sliding friction are of great technological interest to significantly reduce dissipation and wear in mechanical devices functioning at various scales. (Current trends in the physics of nanoscale friction)

% These experiments have demonstrated that the relationship between friction and surface roughness is not always simple or obvious. (Introduction to Tribology, p. 527).


% “In other words, it’s not just the material itself” that determines how it slides, but also its boundary condition — including whether it is loose and wrinkled or flat and stretched tight, he says. (https://news.mit.edu/2016/sliding-flexible-graphene-surfaces-1123).<-- Talking about quality of contact for friciton.