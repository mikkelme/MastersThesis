\newpage
\chapter{Background Theory and Method}
% \addcontentsline{toc}{chapter}{Background Theory and Method} 

Small introtext to motivate this chapter. What am I going to go over here.


\section{Tribology - friction}


Friction is a part of the wider field tribology which includes the study of
friction, wear and lubrication between two surfaces in relative motion \cite[p.
1]{gnecco_meyer_2015}. In this thesis we will only concern ourselves with the
first, wearless dry friction. Tribological systems take place across a broad
range of time and length scales, ranging from geological stratum layers involved
in earthquakes \cite{kim_nano-scale_2009} to microscopic atomistic processes, as
in the gliding motion of nanocluster of a nanomotor \cite{Manini_2016}. This
vast difference in scale gives rises to different frictional mechanism being
dominating at different scales. On a macro scale the system is usally inflicted
to relatively high large loads and speeds leading to high contact stresses and
wear. The micro-/nanoscale regime occupies the opposite domain operating under
relatviely small loads and speeds with negligible wear
\cite{kim_nano-scale_2009} \cite[p. 5]{bhushan_2013}. 

While macroscale friction is often reduced into a few variables such as normal load, material type, speed
and surface roughness it is clear that micro-/nanoscale friction cannot be
generalized under such a simple representation. On the micro-/nanoscale the tribological propteries is dominated by surface properties which will introduce additional sensitivity to more variables such as temperature, humidity and even sliding history. The works of Bhushan and Kulkarni (1996)\cite{BHUSHAN199649} showed that the friction coefficient decreased with scale, revealing the change in the mechanism of friction as the contact condition was altered, even though the materials used was unchanged.

The phenomenological descriptions of macroscale friction cannot yet be derived from the fundamental atomic principles, and bridging the gap between different length scales in tribological systems remains an open challenge \cite{Manini_2016}. Hence, the following sections will be organized into macro-, micro- and nanoscale representing the different theoretical understanding governing each scale. While our study of the graphene sheet is based on a nanoscale perspective the hypothesizing about application possibilities will eventually link back to the macroscale perspective. Thus spanning all three major scales in a brief becomes usefull for a more complete interpreation of the findings of this thesis. 


% Tribological systems pose a wide range of dimensional scale. At the largest
% scale, geological stratum layers that are involved in earthquakes may be
% considered as a tribological system. The movement of the stratum occurs when
% the frictional forces between the layers are overcome by internal pressure
% inside the earth. At the smallest scale, relative motion of atoms at the
% interface of two materials would be a good example that involves frictional
% interaction. Owing to the vast difference in scale, the dominant mechanisms of
% friction and wear in macro-scale systems may be different from those of
% micro/nano-scale systems. In macro-scale, the tribological systems experience
% relatively large contact stresses and speeds. On the other hand,
% micro/nano-scale systems operate under relatively low loads and speeds.
% Particularly, the inertial effects that are prominent in macro-scale may be
% insignificant at the micro/nano-scale. Rather, surface forces often dictate
% the tribological interactions at small scales. \cite{kim_nano-scale_2009}.




% The differences between the conventional or macrotribology and
% micro/nanotribology are contrasted in Figure 1.3.1. In macrotribology, tests
% are conducted on components with relatively large mass under heavily loaded
% conditions. In these tests, wear is inevitable and the bulk prop- erties of
% mating components dominate the tribological performance. In
% micro/nanotribology, measurements are made on components, at least one of the
% mating components, with relatively small mass under lightly loaded conditions.
% In this situation, negligible wear occurs and the surface properties dominate
% the tribological performance. \cite{bhushan_2013}[p. 5]



% Quotes: Sliding friction that takes place between two surfaces in the absence
% of lubricant is termed "dry" friction even if the process occurs in an ambient
% environment. (Nanotribology and Nanomechanics, p. 329)




% We were astonished to discover that molecules that could flex or slide even
% just a little in response to the oscillatory motion of the microbalance were
% linked to low friction levels at the macro-scale. Put another way,
% exceptionally low friction at the atomic scale was not a prerequisite for the
% substantial reduction in macroscopic friction.
% (\url{https://physicsworld.com/a/friction-at-the-nano-scale/})



% Sliding friction that takes place between two surfaces in the absence of
% lubricant is termed ``dry''  friction even if the process occurs in an ambient
% environment. (Nanotriology and Nanomechanics, p. 329)



% It is generally accepted that friction is caused by more than one mechanism in
% a given sliding system. Generally, frictional force arises due to two
% fundamentally different causes, namely one that is mechanical in nature and
% the other being chemical in its origin. In the case of mechanical cause of
% friction, plowing of the surface by hard particles or asperities is mainly
% responsible for generating the frictional force.2,4,5-7 As for the chemical
% mechanism of friction, adhesion between surfaces of the two solids in contact
% is the cause of friction.2,4,5,8 Another point to note is that tribological
% phenomena are heavily dependent on system parameters of the operating machine
% such as speed, temperature, load, and environment. As such, the dominating.
% \cite{kim_nano-scale_2009}.







\subsection{Macroscale}

Our working definition of the \textit{macroscale} is everything on the scale of visible everyday objects, which is usually denoted to milimeters $10^{-3}$ m and above. Most importantly, we want to make a distinction to the microscale, where the prefix indicates the size of micrometers $m^{-6}$, and hence we essentially assign everything larger than micro into the term macroscale\footnote{The width of a human hair is on the length scale $10^{-5}$ to $10^{-4}$ m which constitute a reasonable boundary between macro- and microscale which fit well with a lower bound of our perception capabilities.}.

\subsubsection{Amontons’ law}
% Based on \cite{gnecco_meyer_2015}
 
 In order to start and keep a solid block moving against a solid
 surface we most overcome certain frictional forces $F_{\text{fric}}$ \cite{gnecco_meyer_2015}. The static friction force $F_s$ corresponds to the minimum tangential force required to
 iniziate the sliding while the kintec friciton force $F_k$ corresponds to the
 tangential force needed to sustain such a sliding at steady speed. The work of Leonardo da Vinci (1452–1519), Guillaume Amontons (1663-705) and
 Charles de Coulomb (1736-1806) all contributed to the empirical law, commonly known Amontons’ law, which is a common baseline for an introduction to friction. Amontons’ law states the fricitonal forces is entirely independent of contact area and sliding velocity (at ordinary sliding velocities). Instead, it relies only on the normal force\footnote{This is often used interchangeably with the term \textit{load} or \textit{normal load}.} $F_N$, acting perpendicular to the surface, and the material specific friction coefficient $\mu$ as
\begin{align*}
  F_{\text{fric}} = \mu F_N.
\end{align*}
The friction coeffcient is typically different for the cases of static ($\mu_s$)
and kinetic ($\mu_k$) friction, usually with values lower than one and $\mu_s \ge
\mu_k$ in all cases \cite[p. 6]{gnecco_meyer_2015}. 

Allthough Amontons’ law has been succesfull in the modelling of macroscale friction it has its limitations. For instance, it was later discovered that the staitc friction is not independent of time. It depends on the so-called contact history with increasing friction as the logarithm of time of stationary contact
\cite{dieterich_1972}. For the kinetic friction the independency of sliding
velocity dissapears at low velocities as thermal effects becomes important and
for high velocities due to intertial effetcs. \cite[pp. 5-6]{gnecco_meyer_2015}. 

While Amontons’ law lacks any attempt to explain the phenomena of friction statistical mechanics provides an overall explanation by the concept of \textit{equipartition of energy} \cite{Manini_2016}. A system in equilibrium has its kinetic energy uniformly distributed among all its degrees of freedom. When a macroscale object is sliding in a given direction it is
clearly not in equilibrium since one of its degrees of freedom carriers
considerable more kinetic energy. Thus, the system will have a tendency to
transfer the kinetic energy to the remaining degrees of freedom as heat. This heat will dissipate to the sourroundings and the object will slow down as a final result. Hence, friction is really just the tendency of going toward equilibrium energy equipartitioning among many interacting degrees of freedom. \cite{Manini_2016}

When including \textit{statistical mechanics} we have already taken a peek into lower scales as this aim to explain macroscale behaviour by microscopic interactions. In fact, this marks the limiting view of most macroscale theories:  They lack fundamental explanations. 

% Amontons’ law disguises the importance of a more complicated contact interface. 

% This includes taking the microsopic roughness
% into account together with surface chemistry. This more complex perspective
% introduces new (...() as real contact area, contact stresses, surface adhesion
% which makes frictional properties dependent on sliding speed, temperature and
% environment in general \cite{kim_nano-scale_2009}. 



% The conclusion is that the friction coefficient is not an intrinsic physical
% property \cite{Szlufarska_2008}.

% The basic difficulty of friction is intrinsic, involving the dissipative
% dynamics of large systems, often across ill-characterized interfaces, and
% generally violent and nonlinear \cite{Manini_2016}


% The severity of the task is also related to the experimental difficulty to
% probe systems with many degrees of freedom under a forced spatial confinement,
% that leaves very limited access to probing the buried sliding interface.
% Thanks to remarkable developments in nanotechnology, new inroads are being
% pursued and new discoveries are being made. \cite{Manini_2016}




\subsection{Microscopic scale}

%
%
% Working here 
%
%


% \subsection{Friction on a microscopic scale - Nanotribology}
% \cite{kim_nano-scale_2009}.



% Da Vinci-Amontons law – friction independent of area – is not confirmed at the
% microscopic scale. In most nanoscale investigations the friction of a single
% con- tact is found to increase linearly with the contact area [27–29]. In
% contrast, structurally mismatched atomically flat and hard crystalline or
% amorphous surfaces are expected to produce a sublinear increase of friction
% with contact area. The frequent finding of friction proportional to area even
% in some of these cases can be understood as a consequence of softness, either
% if the interface, or of surface contaminants leading to effectively pseudo-
% commensurate interfaces [30, 31] (Current trends in the physics of nanoscale
% friction)

Going beyond a macroscopic perspective to the microscopic scale we realise that
most surfaces is in fact rough (Bowden, F. P. \& Tabor, D. The Friction and Lubrication of Solids (Clarendon, 1950).). The contact between two surfaces consist of
numerous smaller contact point, so-called asperities, for which the friction
between two such opposing surfaces involves interlocking of those asperities.
Additionally, the adhesion between surfaces is taken to be a dominiant factor,
for which the chemical state of the surfaces has to be considered.

% It is generally accepted that friction is caused by two mechanism: mechanical
% friction and chemical friction. The mechanical friction is the plowing of the
% surface by hard particles or asperities. The chemical mechanism is adhesion
% between contacting surfaces. \cite{kim_nano-scale_2009}.



% Quite some years after the FKT model was presented, Bowden and Tabor (1954) proposed the adhesion model that is illustrated in Fig 3 as the major mechanism of friction. They suggested that small junctions of asperities are formed due to contact pressure and adhesion. The combined area of all the junctions essentially represents the real contact area. According to the model, frictional force, Fr, is proportional to the real contact area \cite{kim_nano-scale_2009}

% In the model, the adhering asperities are formed as normal load is applied and friction is caused during relative motion by loss of energy due to plastic deformation of the asperities. \cite{kim_nano-scale_2009}



% Concurrently with single-asperity studies, roughness contact theories are being developed8–10,16 to bridge the gap between the mechanics of single asperities and that of macroscopic contacts.\cite{mo_friction_2009}
\subsubsection{Surface roughness - Asperity theories}
% Sources in general: \cite{mo_friction_2009}, \cite{kim_nano-scale_2009} \\

Asperity theories is based on the observation that microscopic rough surfaces
with contacting asperities, each with a contact area of $A_{\text{asp}}$, will
have a true contact area $\sum A_{\text{asp}}$ which is found to be much smaller
than the apperent macroscopic area $A_{\text{macro}}$. The friction force was
shown to be proportional (extra source on this) to this true contact area as 
\begin{align*}
  F_\text{fric} = \vec{\tau} \sum A_{\text{asp}},
\end{align*}
where $\vec{\tau}$ is an effective shear strength of the contacting bodies. This
is still compatible with Amontons’ law for the cases where the real contact area
dependt linearly on the applied normal force $F_N$.


\begin{figure}[H]
  \centering
  \includegraphics[width=0.5\linewidth]{figures/theory/asperities.png}
  \caption{Example figure shwoing asperity deformation as the load is incread
  from top to bottom. Source:
  \url{https://en.wikipedia.org/wiki/Asperity\_(materials\_science)} \hl{Put in
  reference list? Find better picture?}}
  \label{fig:asperity_contact}
\end{figure}


% The common feature of all the single-asperity theories is that Aasp is a sublinear function of L. \cite{mo_friction_2009}

% Our results confirm the conclusions of other authors that single-asperity theories break down at the nanoscale1,5. \cite{mo_friction_2009}

% Other authors proposed an empirical model in which mechanics of a nanoscale non-adhesive contact is controlled by load, that is, $F_f = \mu L$ and the contact area is undefined and unnecessary5,29 \cite{mo_friction_2009}

% \cite{mo_friction_2009} agues that the break-down of single-asperity theories of friction is due to the asperity (circumfereance defined) area is not proportional to the real one. By obtaining the real area (contacting bond) he arrives at the macroscale relationship... Quote: As shown in Table 1, friction force is now proportional to contact area at all length scales as long as the contact area is correctly defined at each length scale. When adhesion is added they arrive at the sublinear trend again. 


Thus, many studies have focused on the individual single asperity contact to
reveal the relationship between the contact area and $F_N$ (13-15 from
\cite{mo_friction_2009}). By assuming perfectly smooth asperities with radii of
curvature from nanometers to micrometres continuum mechanics can be used to
predict the deformation of asperities as normal force is applied. A model for
non-adhesive contact between homogenous, isotropic, linear elastic spheres was
first developed by Hertz (17 \cite{mo_friction_2009}), which predicted
$A_{\text{asp}} \propto F_N^{2/3}$. Later adhesion effects were included in a
number of subsequent models, including Maugis-Dugdale theory (18 from
\cite{mo_friction_2009}), which also predicts a sublinear relatinship between
$A_{\text{asp}}$ and $f_N$, leading to a similar sublinear relationship for
$F_\text{fric}(F_N)$. (Adress that this deviates with the macroscale). 

\hl{What about multi asperity theory?}

% Our model predicts that as the adhesion between the contacting surfaces is reduced, a transition takes place from nonlinear to linear dependence of friction force on load. \cite{mo_friction_2009}

However, even though the successes of continuum mechanics there is no reasion to
believe that it will be capable of reproducing tribological behaviour at the
nanometre length scale where the discreteness of atoms often has a direct effect
on physical properties \cite{Szlufarska_2008}.



This approach enables the bottom-up derivation of the linear scaling laws of macroscopic friction with size, and their transition to the sublinear ones for incommensurate nanosized contacts. We can now understand that such transition takes place when the contact roughness becomes large compared to the range of interfacial interactions [162] \cite{Manini_2016}.
% However, practical single- and multiplecontact conditions are characterized by
% complex interaction profiles plus nontrivial internal dynamics. As a result,
% the interplay of thermal drifts, contact ageing, contact-contact interactions,
% and macroscopic elastic deformations introduce significant complications, and
% make the depinning transition from static to kinetic friction an active field
% of research. \cite{Manini_2017}[p. 2]. 


% Rate and state models?

\subsection{Nanotribology - Atomic scale}

% Experimental research to examine the frictional characteristics at the
% atomic-scale has been conducted for the past two decades. It is well known
% that frictional behavior cannot be generalized by a few factors such as normal
% load, surface roughness, speed, and material type of the tribological system.
% Other conditions such as temperature, humidity, and even sliding history can
% affect the tribological phenomena significantly. Particularly at nano-scale,
% the tribological behavior tends to be more sensitive to the state of outermost
% layer of the surface region. Thus, contamination layer, adsorbed gas,
% capillary junctions, and oxide layer become more important at small scales.
% This is because at nano-scale the contact forces are often too low for the
% asperities to penetrate the surface layers and the magnitude of the surface
% force may be comparable to the frictional force. {kim_nano-scale_2009}.


% Together with the current experimental possibility to perform well-defined
% measurements on well-characterized materials at the fundamental microscopic
% level of investigation of the sliding contacts, advances in the computer
% modeling of interatomic interactions in materials science and complex systems
% encompass molecular-dynamics (MD) simulations of medium to large scale for the
% exploration of the tribo-dynamics with atomic resolution [4, 5].


On the smallest possible scale, atomic scale, the surfaces does not have
structural asperities. Instead, atomic level friction is being modelled as a
consequencse of the rough potential of the atomic landscape and real contact
area defined in terms of atomic bonding. 


\subsubsection{Tomlinson model}
% \cite{kim_nano-scale_2009}

One of the first atomic scale models. Here we have no asperities but it is based on the assumptions that the atomic surface is not completely smooth. Since atoms was modelled as spheres the surface topography would not be completely flat. The idea is shown in figure \ref{fig:tomlinson_model} where the moving body is connected the atom with springs.
\begin{figure}[H]
  \centering
  \includegraphics[width=0.5\linewidth]{figures/theory/tomlinson_model.png}
  \caption{\hl{Temporary} figure from \cite{kim_nano-scale_2009}}
  \label{fig:tomlinson_model}
\end{figure}


This model gives an explanation to the stick-slip behaviour. This layed the foundation for the Frenkel-Kontorova model so maybe go straight to that?

\subsubsection{Frenkel-Kontorova (FK)}

% Based on \cite{Manini_2016}. One model for the description of atomic scale
% fricion is the Frnkel-Kontorova-Tomlinson (FKT) model, named after its main
% contributers, which considers ...


The FK model builds upon the time-honered PT model describing a
point-like tip sliding over a space-periodic crystalline surface. The FK model
extents this description to hat of two 1D crystaline surfaces sliding on top of
eahc other. The standard FK model consists of a 1D chain of $N$ classical
particles, representing atoms, interacting via a hamornic forces and movig in a
sinusoidal potential as sketched in figure \ref{fig:FK_model}. The hamiltonian
is 
\begin{align}
  H = \sum_{i=1}^N \left[\frac{p_i^2}{2m} + \frac{1}{2}K(x_{i+1} - x_i - a_c)^2 + \frac{1}{2}U_0 \cos{\left(\frac{2\pi x_i}{a_b}\right)}\right],
  \label{eq:H_FK}
\end{align}
where the first term $p_i^2/2m$ represents the kinetic energy with momentum $p$
and mass $m$. The next term describes the harmonic interaction with elastic
constant $K$, nearest neighbour distance $\Delta x = x_{i+1} - x_i$ with a
correspnding equilibrium distance $a_c$. The final term represents the periodic
substrate potential with amplitude $U_0$ and period $a_b$. 

% What about Boundary conditions???

\begin{figure}[H]
  \centering
  \includegraphics[width=0.8\linewidth]{figures/theory/FK_model.png}
  \caption{\hl{Temporary} figure from \cite{Manini_2016}}
  \label{fig:FK_model}
\end{figure}

By applying an external force to the chain one can use the FK model to
investigate static friction. The sliding properties are entirely governed by its
topological excitations referred to as so-called kinks and anti-kinks

\paragraph*{Commensurability} We can describe the interactions of the model in a
commensurate or incommensurate case. Let us begin with the commensurate case
where the spacing of the atoms matches perfectly the substrate potential
periodicity, i.e. $a_c = a_b$. The ground state (GS) is the configuration where
each atom fits in one of the $M$ minema in the substrate potential such that the
coverage, $\Theta = N / M = a_b / a_c$, equals 1. By adding an atom we would
excite the system with a so-called kink excitation (two atoms share the same
potential well) while subtracting an atom results in an antikink excitation (one
potential well becomes atomless). In order to reach a local minimum the kink
(antikink) will expand in space over a finite length such that the chain is
imposed a local compression (expansion). When applying a tangential force to the
chain it is much easier for a kink to move along the chain than it is for atoms
since the activation energy $\epsilon_{PN}$ for a kink displacement is
systematically smaller (often much smaller) than the potential barrier $U_0$.
Thus the motion of kinks (anti kinks), i.e. the displacement of extra atoms or
vacancies, is represententing the fundamental mechanism for mass transport. In
the ideal zero temperature case with addibatically (right term?) increase in
force all atoms would be put into an accelerating motion as soon as the energy
$NU_0$ is present. However, in reality any thermal excitation would excite the
system with the creation of kink-antikink pairs that would travel down the
chain. For a chain of finite length these often accour at the end of the chain
running in opposite direction. As the kinks travels by the atoms is advanced by
one atom spacing $a_b$ along the force direction. This cascade of kink-antikink
exications is shown in figure \ref{fig:kink_antikink}


\begin{figure}[H]
  \centering
  \includegraphics[width=0.8\linewidth]{figures/theory/kink_antikink.png}
  \caption{\hl{Temporary} figure from \cite{Manini_2016}}
  \label{fig:kink_antikink}
\end{figure}

For the 2D case where an island is deposited on a surface, in our case the
graphene sheet on the Si substrate, we generally expect the sliding to be
initated by kink-antikink pairs at the boundary. 

For the case of incommensurability, i.e. $\theta = a_b/a_c$ is irrational, the
GS is characterized by a sort of ´´staircase'' deformation. That is, the chain
will exhibit regular periods of regions where the chain is slightly compressed
(expanded) to match the substrate potential, seperated by kinks (antikinks),
where the increased stress is eventually realised through a localized expansion
(compression) as illustrated in figure \ref{fig:incommensurable_example}.


\begin{figure}[H]
  \centering
  \includegraphics[width=0.5\linewidth]{figures/theory/incommensurable_example.png}
  \caption{\hl{Temporary} figure from
  url{http://www.iop.kiev.ua/~obraun/myreprints/surveyfk.pdf} p. 14.
  Incommensurable case ($\theta = ?$) where atoms sits slightly closer than
  otherwise dictated by the substrate potential for which this regularly result
  in a kink here seen as the presence of two atoms closeæy together in on of the
  potential wells.}
  \label{fig:incommensurable_example}
\end{figure}

The incommensurable FK model contains a critical elastic constant $K_c$, such
that for $K > K_c$ (infinite chain, right?) the static friction $F_s$ drop to
zero (article says low-velocity kintetic friction as well which I do not get for
no temperature?), making the chain able to slide at no energy cost. This can be
explained by the fact that the displacement accouring in the incommensurable
case will yield one atom climbing up a corrugation (introduce this word earlier)
for every atom clibing down on. In turns out that this will exactly balance the
energy making it non-resistant to sliding. Generally, incommensurability
guarantees that the total energy (for $T=0$) is independent of the relative
position to the potential. However, when sliding freely a single atom will
eventually occupy a maximum of the potential. When increasing the potential
magnitude $U_0$ of softning the chain stifness, lowering $K$, (the probability
for a particle to occupy that position drops from a finite value to exactly
zero, how can this work for zero temp?) this is no longer possible, which is
described by the Aubry transition. The phase transistion is marked by the
critical elastic constant $K = K_c$ where the chain goes from free sliding to a
pinned state with a nonzero static friction. $K_c$ is a discontinuous function
of the ratio $\theta$, due to the reliance on irrational numbers for
incommensurability (?, get plot of this?). The minimal value $K_c \simeq
1.0291926 $ in units $[2 U_0 (\pi / a_b)^2]$ is achieved for the golden-mean
ratio $\theta = (1+\sqrt{5}/2)$ (compare to my model?).  The pinning is provided despite translational invariance by the inaccessibility to move past the energy barrier which act as dynamical constraint. The Aubry transistion can be invistigated as ?-order phase transistion for which power laws can be defined for the order parameter (beyond the scope of this thesis).

The phenonema of non pinned configurations is named \textit{superlubricity} in trobological context. Despite the misleadning name this referes to the case where the static friction is zero while the kinetic friction is only reduced but nonzero. 

Experimentally superlubricity has been studied for a graphene flake (single layer) sliding over a graphite surface (multiple layers) where the friction force was extremely weak (50 pN) for orientations between flake and substrate yielding incommensurability.

When considering the nonzero temperature case for the FK model one would have to add a thermostat such as the Langevin thermostat (see theory section ?). Thermal fluctuations can then overcome pinning effects even in fully commensurate cases. This also give rise to a damping force proportional to sliding velcoty as a mean of modelling the dissapitve force accouring.

By applying a finite driving force it is known that a pinned configuration will go through several first-order dynamical pahse transistions as the system transfers from a pinned to a sliding state. 



% At face value, the transition from a static strained configuration to full
% sliding is conceptually as simple as overcoming an energy barrier. However,
% practical single- and multiple- contact conditions are characterized by
% complex interaction profiles plus nontrivial internal dynamics. As a result,
% the interplay of thermal drifts, contact ageing, contact-contact in-
% teractions, and macroscopic elastic deformations introduce significant
% complications, and make the depinning transition from static to kinetic
% friction an active field of research. The depinning dynamics affects in
% particular the transition between stick-slip and smooth slid- ing for sliding
% friction. (Current trends in the physics of nanoscale friction)


% In Atomic Force Microscopy (AFM) experiments, when the tip scans over the
% monolayers at low speeds, friction force is reported to increase with the
% logarithm of the velocity, similar to that observed when the tip scans across
% crystalline surfaces. This velocity dependence is interpreted in terms of
% thermally activated depinning of interlocking barriers involving interfacial
% atoms. (Current trends in the physics of nanoscale friction)




% \begin{enumerate}
%   \item Properties depending on $\Theta = N/M = a_b/a_c$
%   \item Elastic constant $K$
%   \item Kinks anti kinks travelling
%   \item Importance of commensurability between lattice and potential 
%   \item Aubry transition
%   \item Pinning and unpinning, stick slip
%   \item superlubricity (no stick slip, but still kinetic friction)
%   \item Adding a Langevin thermostat on top of this introduces temperature. 
%   \item Phase transistion?
% \end{enumerate}







\subsubsection{FK extension: Frenkel-Kontorova-Tomlinson (FKT)}

Several extensions has been provided for the FK model with modifications of the
interactions or system dimensionality. Anharmonicity of the chain 


However, Weiss and Elmer (1995) proposed that the model had a deficiency. They suggested that in the FK model, there was no connection between the atoms and the sliding body. Therefore, Frenkel-Kontorova-Tomlinson (FKT) model that combines the FK model with the Tomlinson model was proposed. \cite{kim_nano-scale_2009}


% The Frenkel-Kontorova-Tomlinson (FKT) model [61, 62] introduces an harmonic
% coupling of the sliding atomic chain to a driving support, thus making it
% possible to investigate stick-slip features in a 1D extended simplified
% contact. The FKT framework provided the ideal platform to investigate the
% tribological consequences of combined interface incommensurability,
% finite-size effects, mechanical stiffness of the contacting materials, and
% normal-load variations \cite{Manini_2016}.


% Important generalizations involving increased dimensionality compared to the
% regular FK model bear significant implications for tribological properties
% such as critical exponents, size-scaling of the friction force, depinning
% mechanisms, and others. \cite{Manini_2016}

% Maybe check this out: An interesting example of such a transient is the
% depinning of an atomic monolayer driven across a 2D periodic substrate profile
% of hexagonal symmetry [83].  \cite{Manini_2016}


\subsubsection{Other stuff}


At nanoscales things get a bit more unclear. SFM (explain) experiments have
reported (copy sources 5, 6, 21 from \cite{mo_friction_2009}) where $F_f \propto
F_N$ or even with these quantities being nearly independent of each other.

% \cite{physicsworld_2005}

Physically relevant quantities, including the average friction force, the slider and the lubricant mean velocities, several correlation functions, and the heat flow can be evaluated numerically by carrying out suitable averages over the model dynamics of a sliding interface, as long as it is followed for a sufficiently long time. The modeling of friction must first of all address correctly ordinary equilibrium and near-equilibrium phenomena, where the fluctuation-dissipation theorem (Sec. 2) governs the smooth conversion of mechanical energy into heat, but most importantly it must also deal with inherently nonlinear dissipative phenomena such as instabilities, stick-slip, and all kinds of hysteretic response to external driving forces, characteristic of non-equilibrium dynamics. 
\cite{Manini_2016}




In several works by J. Fineberg’s group [2–4] the transition from sticking to
sliding is characterized by slip fronts propagating along the interface.
\cite{Manini_2017}[p. 2]. 



% As expected, high levels of
% friction were present in the commensurate positions and extremely low friction
% was found when the surfaces were incommensurate.
% (\url{https://physicsworld.com/a/friction-at-the-nano-scale/})


% Superlubricity, now a pervasive concept of
% modern tribology, dates back to the math- ematical framework of the Frenkel
% Kontorova model for incommensurate interfaces [40]. When two contacting
% crystalline workpieces are out of registry, by lattice mismatch or angular
% misalignment, the minimal force required to achieve sliding, i.e. the static
% friction, tends to zero in the thermodynamic limit – that is, it can at most
% grow as a power less than one of the area – provided the two substrates are
% stiff enough. (Current trends in the physics of nanoscale friction)


% Superlubricity is experimentally rare. Until recently, it has been
% demonstrated or im- plied in a relatively small number of cases [29, 42–46].
% There are now more evidences of superlubric behavior in cluster
% nanomanipulation [32, 33, 47], sliding colloidal layers [48–50], and
% inertially driven rare-gas adsorbates [51, 52]. (Current trends in the physics
% of nanoscale friction)


% A breakdown of structural lubricity may occur at the heterogeneous interface
% of graphene and h-BN. Because of lattice mismatch (1.8\%), this interface is
% intrinsically incommen- surate, and superlubricity should persist regardless
% of the flake-substrate orientation, and become more and more evident as the
% flake size increases [57]. However, vertical cor- rugations and planar strains
% may occur at the interface even in the presence of weak van der Waals
% interactions and, since the lattice mismatch is small, the system can de-
% velop locally commensurate and incommensurate domains as a function of the
% misfit angle [58, 59]. Nonetheless, spontaneous rotation of large graphene
% flakes on h-BN is observed after thermal annealing at elevated temperatures,
% indicative of very low friction due to incommensurate sliding [60, 61].
% (Current trends in the physics of nanoscale friction)

% Indeed, we know from theory and simulation [74–76] that even in clean wearless
% friction experiments with perfect atomic structures, superlubricity at large
% scales may, for example, surrender due to the soft elastic strain deformations
% of contacting systems. (Current trends in the physics of nanoscale friction)




\subsection{Multi scale models?}
\cite{Manini_2016} p. 24.

\subsection{Summary of expected frictional properties}


\begin{enumerate}
  \item Smooth kinetic friction generally increase with speed  \cite{Manini_2016}. so-called velocity strengthening.  Logaritmic with speed % Gnecco E, Bennewitz R, Gyalog T, Loppacher C, Bam- merlin M, Meyer E, Güntherodt H-J (2000) Velocity dependence of atomic friction. Phys Rev Lett 84:1172–1175. Maybe crosscheck with macroscale to ensure that this is valid for higher velocities. 
  \item Friction coeffients on the scale 0.01 to 0.03. 
  % Ruan and Bhushan (1994) used an AFM to investigate the effect of surface roughness on the tribological characteristics of graphite using a Si3N4 tip. It was found that friction coefficient varied with respect to roughness of the substrate. The friction coefficient was below 0.01 and 0.03 for RMS roughness of about 10 nm and 140 nm, respectively. This outcome was attributed to the loss of orientation of the substrate with increasing roughness.33 \cite{kim_nano-scale_2009}
  \item Occurrence of stick slip (also in MD) \cite{kim_nano-scale_2009} (p. 146)
\end{enumerate}

% Thus, it is commonly expected that the
% friction of a dry nanocontact should classically decrease with increasing
% temperature provided no other surface or material parameters are altered by
% the temperature changes [77, 80–83]. (Current trends in the physics of
% nanoscale friction)

% Thus far we have used thermal activation to explain the velocity dependence of friction. The same mechanism also predicts that friction should change with temperature. \cite{BHUSHAN20051507}


% Zhao et al. (2007) observed that friction on graphite decreases as 1/T over a wide temperature range (140–750 K), supporting the hypothesis of thermal activation of the stick–slip process. However, it was only recently that group of Schirmeisen reported atomic-scale FFM mea- surements in UHV at different temperatures (Jan- sen et al. 2010). When silicon, SiC, ionic crystals and graphite surfaces were cooled down from room temperature to cryogenic conditions, a good agreement with the thermally activated PT model was found down to a peak or a plateau, appearing between 50 and 200 K. Below these values, the friction was found to decrease with temperature, which the authors attributed to the competition between thermally activated rupture and formation of chemical bonds (Barel et al. 2010). \cite{BHUSHAN20051507}


% For monolayers sliding along atomically uniform substrates, however, there is
% essentially no static friction. Indeed, the friction in these systems can be
% up to 105 times less than that for macroscopic lubricants such as graphite.
% This raises questions about the fundamental dissipation mechanisms that are at
% work in systems at different scales.
% (\url{https://physicsworld.com/a/friction-at-the-nano-scale/})


% Look for source on affect on friction when stretching. Since we control the area through that there might be a stretch effect that is even stronger. 





% Hint for explaining increase with stretch: Both micro-tribotester and AFM were used to investigate the micro/nano frictional behavior. It was reported that contact angle between the groove and the pin affected the frictional characteristics significantly. High contact angle led to a sudden increase in the frictional force due to interlocking mechanism. \cite{kim_nano-scale_2009}
% These works suggest that frictional behavior at micro/nano scale is very much dependent on the surface structure and topography. Furthermore, the contact geometry between the tip and the surface such as area and orientation of the contacting angle affect the frictional force significantly. \cite{kim_nano-scale_2009}



Should period macth the lattice spacing as described in \cite{kim_nano-scale_2009}[p. 144]



Maybe talk about the slip line as shown in figure \ref{fig:slip_line}


\begin{figure}[H]
  \centering
  \includegraphics[width=0.5\linewidth]{figures/theory/slip_line.png}
  \caption{\hl{Temporary} figure from \cite{kim_nano-scale_2009}[p. 144]}
  \label{fig:slip_line}
\end{figure}

% \begin{enumerate}
%   \item Friction should decrease by increasing temperature.
%   \item We expect stick slip motion
%   \item What about dependence on normal force?
%   \item Dependence on contact area?
%   \item Dependense on speed? 
% \end{enumerate}

% \begin{itemize}
%   \item Different friction models on macro-and microscopic scale
% \end{itemize}


% Smooth kinetic friction generally increases with speed (velocity strengthening), but sometimes decreases with increasing speed in certain intervals \cite{Manini_2016}.

% the smallest force needed to set a slider in motion – is also dependent on the simulation time (a longer wait may lead to depinning when a short wait might not), and generally dependent on system size, often increasing with sub-linear scaling with the slider’s contact area. To address this kind of behavior in MD simulations, it is often necessary to resort to scaling arguments in order to extrapolate the large-area static friction from small-size MD simulations [131, 140] \cite{Manini_2016}.


% In AFM and SFA experiments, stickslip and its associated characteristically high friction and mechanical hysteresis tend to transition into smooth sliding when the speed exceeds $\sim$ 1 $\mu$m/s; in contrast, in MD modeling the same transition is observed in the $\sim$ 1 m/s region.  This 6-order-ofmagnitude discrepancy in speed between experiments and simulations is well known and has been largely discussed [141–144] in connection with the effective mass distributions and spring-force constants, that are vastly different, and highly simplified in simulations. \cite{Manini_2016}



% Concerning stick-slip friction, another problem is that, unlike simulations, real experiments contain mesoscale or macroscale component intrinsically involved in the mechanical instabilities of which stick-slip consists. Here the comforting observation is that stick-slip is nearly independent of speed, so that so long as a simulation is long enough to realize a sufficient number of slip events, the results may already be good enough [148]  \cite{Manini_2016}.


% A serious aspect of stick-slip friction which MD simulation is unable to attack is ageing. The slip is a fast event, well described by MD, but sticking is a long waiting time, during which the frictional contact settles very slowly. The longer the sticking time, the larger the static friction force necessary to cause the slip. Typicall experiments show a logarithmic increase of static friction with time [150] \cite{Manini_2016}.

% Rate and state friction approaches, widely used in geophysics [151], describe phenomenologically frictional ageing, but a quantitative microscopic description is still lacking. Mechanisms invoked to account for contact ageing include chemical strengthening at the interface in nanoscale systems [152], and plastic creep phenomena in macroscopic systems [153]. \cite{Manini_2016}.


% See ``Selected Results of MD Simulations'' in \cite{Manini_2016} p. 24.

\subsection{Graphene friction}
Theory of friction experiment involving graphene.


Because of this frictional reduction, many studies indicate graphene as the
thinnest solid-state lubricant and anti-wear coating [104–106]. (Current trends
in the physics of nanoscale friction)


Accurate FFM measurements on few-layer graphene systems show that friction
decreases by increasing graphene thickness from a single layer up to 4-5 layers,
and then it approaches graphite values [97, 99, 101, 107, 108]. (Current trends
in the physics of nanoscale friction)


\subsubsection{Experimental procedures}

\begin{itemize}
  \item Atomic Force Microscope (AFM)
\end{itemize}


% The trouble is that the coefficients of friction measured in nanotribological
% experiments and in macroscopic “tribotests” routinely differ by orders of
% magnitude. (\url{https://physicsworld.com/a/friction-at-the-nano-scale/})



\section{Molecular Dynamics}


Maybe also ´´Computer Simulations 7 of Nanometer-Scale Indentation
and Friction'' from \cite{BHUSHAN20051507}

Read \cite{Manini_2016}[p. 18]

A promising compromise could possibly be provided by the so-called reactive potentials [120–122], capable of describing some chemical reactions, including interface wear with satisfactory computational efficiency in large-scale atomic simulations, compared to semi-empirical and first-principles approaches. \cite{Manini_2016}




\begin{itemize}
  \item MD simulation (classical or ab initio)
  \item Basics of classical MD simulations: Integration and stuff
  \item Ab initio simulation (quantum mechanics, solving schrödinger)
\end{itemize}



% Quantum-mechanical calculations is more accurate but to numerical intensive.

% Despite recent progress in this respect, it is clear that there will always be
% interesting problems beyond the reach of ab initio approaches
% \cite{PhysRevB.37.6991}.

\subsection{Potentials}
% \cite{PhysRevB.37.6991}

The choices of potentials used in the MD simulation is mainly based on the on
\cite{li_evolving_2016} which have a somewhat similar MD friction simulation,
the difference being that they impose a Si-tip on the graphene sheet supported
by a Si-substrate where we impose drag the whole sheet upon the substrate.
Nonetheless this serves as a good anchor for the methodology of the setup. The
covalent bonds of C-C in graphene and Si-Si in the substrate is described by the
Tersoff and Stillinger–Weber potentials, respectively. A typical 12-6
Lennard–Jones potential is used to describe the van der Waals adhesive
interaction between graphene and the substrate. \\

\subsubsection{General formulation of potentials...?}

On a general note we can generalize the n-body potential as the expansion in
orders of participating atoms as 
\begin{align*}
  E = \sum_i V_1(\vec{r}_i) + 
      \sum_{\substack{i, j \\ i < j}} V_2(\vec{r}_i, \vec{r}_j) +  
      \sum_{\substack{i,j,k \\ i < j < k}} V_3(\vec{r}_i, \vec{r}_j, \vec{r}_i) + \cdots.
\end{align*} 
where $\vec{r}_n$ is the position of the $n$th particle and $V_m$ is called an
$m$-body potential  \cite{PhysRevB.37.6991}. The first one-body term corresponds
to an external potential, followed by the two-body term, the three-body term and
so on.The simplest model that includes parrticle interaction is the pair
potential truncating the expansion after the two-body term. A general feature of
the pair potentials is that they favor close-packed structures which is unsuited
to describe covalent bonds that take more open structures. In particular, pair
potentials are completely inapplicable to strongly co- valent systems such as
semiconductors \cite{PhysRevB.37.6991}. In order to accomodate the description
of covalent bonds the natural step is thus to include the next step of the
expansion, the three-body terms, as we will see for the modeling of the graphene
sheet C-C bonds and the Silicon sheet Si-Si bonds. For the interaction between
the sheet and the substrate we can nøjes med a Lennard Jones pair potential
describing the non-bonded van der Waals interaction.


\subsection{Lennard Jones}
% TODO: Add potential curve figure
This sections is based on [\cite{docs_lammps_LJ}, \cite{C9CP05445F},
\cite{chem_libretexts_LJ}].

The Lennard-Jones (LJ) model is probably one of the most famous pair potentials
used in MD simulations. LJ models the potential energy between two non-bonding
atoms based solely on interatomc distance $r$. The model accounts for attractive
forces arising from dipole-dipole, dipole-induced dipole and London
interactions, and repulsive forces that capture the hard core (is this safe to
say?) of overlapping wave functions at small distances. Thus it is assummes
neutrally charged atoms and was orginally proposed for noble gases. The
classical 12-6 version of the model (refering to the power law of the repulsive
and attractive forces respectively) reads
\begin{align}
  E = 4\epsilon \left[\left(\frac{\sigma}{r}\right)^{12} - \left(\frac{\sigma}{r}\right)^6 \right ], \qquad r < r_c,
  \label{eq:LJ}
\end{align}
where $r$ is the interatomic distance with cut-off $r_c$, $\epsilon$ is the
depth of the potential well and $\sigma$ the distance where the potential is
zero. By solving for the potential minimum ($dE/dr = 0$) we find the equilibrium
distance to be $r_0 = \sigma 2^{1/6}$. This makes for an even cleary
interpration of $\sigma$ which effectively sets the equilirbium distance between
atoms, i.e. the dividing line for which the net force is repulsive or
attractive. While the LJ model in many ways is an oversimplified model that is
insufficient in its description of ... (get source and concrete examples) it is
commonly used as a model for intermaterial interactions (between moving object
and substrate) in friction studies [\cite{li_evolving_2016}, \cite{ZHANG201585},
\cite{kim_nano-scale_2009}].


\subsection{Stillinger weber}
% Todo: Add some potential curve figure? or figure of three body angles?
This section is based on [\cite{docs_lammps_sw}, \cite{PhysRevB.31.5262}]

The stillinger weber potential takes the form of a three body potential
\begin{align*}
  E &=\sum_i \sum_{j>i} \phi_2(r_{i j})+\sum_i \sum_{j \neq i} \sum_{k>j} \phi_3(r_{ij}, r_{ik}, \theta_{ijk}),
\end{align*}
where $r_{ij}$ denotes the distance between atom $i$ and $j$ and $\theta_{ijk}$
the angle between bond $ij$ and $jk$. The summations is over all neighbours $j$
and $k$ of atom $i$ within a cut-off distance $r = a\sigma$. \\
The two-body term $\phi_2$ builds from the LJ model with the addition of an
exponetial cutoff term
\begin{align}
  \phi_2(r_{i j}) & =A_{ij} \epsilon_{ij}\left[B_{ij}\left(\frac{\sigma_{ij}}{r_{ij}}\right)^{p_{ij}} - \left(\frac{\sigma_{ij}}{r_{ij}}\right)^{q_{ij}}\right] \exp (\frac{\sigma_{ij}}{r_{ij}-a_{ij} \sigma_{ij}}).
  \label{eq:sw_2}
\end{align}

The model parameters $A$, $\epsilon$, $B$, $\sigma$, $p$, $q$ and $a$ comes with
$i,j$ indices to indicate that theese parameters should be specified for each
unique pair of atom types. However, in our case we will only provide a single
value for each model parameter as we are exclusively dealing with Si-Si bonds.
We see that the first term in eq.~\eqref{eq:sw_2} is reminiscent of the LJ model
in eq.~\eqref{eq:LJ} while the last term effectively drives the potential to
zero at $r=a\sigma$, which is thus the chosen cut-off distance for the potential
evaluation. With the model parameters for the Si-Si modelling (see table
\ref{tab:sw_param}) the cut-off becomes $\sim 3.8$ Å. \\
The three body term includes an angle dependency as
\begin{align}
  \phi_3(r_{ij}, r_{ik}, \theta_{ijk}) &= \lambda_{ijk} \ \epsilon_{ijk} \Big[\cos \theta_{ijk}-\cos \theta_{0,ijk}\Big]^2 \exp (\frac{\gamma_{ij} \sigma_{ij}}{r_{ij} - a_{ij} \sigma_{ij}}) \exp (\frac{\gamma_{ik} \sigma_{ik}}{r_{ik} - a_{ik} \sigma_{ik}}),
  \label{eq:sw_3}
\end{align}
where $\theta_{0,ijk}$ is the equilibrium angle. The first term of
eq.~\eqref{eq:sw_3} includes an angle dependency analog to a harmonic oscillator
based on a cosine angle distance from the equilibrium angle. The final two terms
act again as a cut-off function by driving the potential to zero at $r_{ij} =
a_{ij}\sigma_{ij}$ and $r_{ik} = a_{ik}\sigma_{ik}$ respectively. \\ 
The parameters used for the Si-Si bond modeling is displayed in table
\ref{tab:sw_param} along with an interpretation of each model parameter.



\begin{table}[H]
  \begin{center}
  \caption{Parameters for the stilliner weber potential used for intermolecular interactions in the silicon substrate.}
  \label{tab:sw_param}
  \begin{tabular}{ | c | c | L{9cm} |} \hline
    Parameter & Value & Description \\ \hline 
    $\epsilon$ & 2.1683  & Individual depth of the potential well for each atom
    type pair/tiplets. \\ \hline
    $\sigma$ & 2.0951 & Distance for which the individual pair interactions has
    zero potential (analog to the LJ model). \\ \hline
    $a$ & 1.80 & The individual cut-off distance for each atom type pair. \\
    \hline
    $\lambda$ & 21.0 & The overall depth of the three-body potential well. \\
    \hline
    $\gamma$ & 1.20 & The shape of the three-body cut-off terms. \\ \hline
    $\cos{(\theta_0)}$ & -1/3 & Cosine of equilibrium angle. \\ \hline
    $A$ &  7.049556277 & The overall depth of the two-body potential well. \\
    \hline
    $B$ &  0.6022245584 & Scales the repulsion part of the two-body term. \\
    \hline
    $p$  & 4.0 & The power dependency for the repulsion part of the two-body
    term. \\ \hline
    $q$  & 0.0 & The power dependency for the attraction part of the two-body
    term. \\ \hline
    tol  & 0.0 & LAMMPS: Option to define a different cut-off than the
    theoretical of $r = a\sigma$. $tol = 0$ refers to the theoretical being
    used. \\ \hline
  \end{tabular}
  \end{center}
\end{table}



\subsection{Tersoff}
% Add figure similar to:
% https://en.wikipedia.org/wiki/Bond_order_potential#/media/File:Bond-order_interatomic_potential.png,
% showing bond order curves.


% https://interatomic-potentials.readthedocs.io/en/latest/doc/tersoff.html
% https://chem.libretexts.org/Bookshelves/
% Physical_and_Theoretical_Chemistry_Textbook_Maps/Supplemental_Modules_(Physical_and_Theoretical_Chemistry)/Chemical_Bonding/Fundamentals_of_Chemical_Bonding/Bond_Order_and_Lengths
This section is based on [\cite{docs_lammps_tersoff}, \cite{PhysRevB.37.6991}].


The tersoff potential abandon the idea of a general $n$-body form and attempts
instead to build the model on a more physics informed approach; The more
neighbours an atom has the weaker the bonds will be. Thus it introduces the bond
order (bond strentgh), that is environment specific and decrease with increasing
bond coordination (number of neighbours for a given atom). The potential energy
is taken to have the form

\begin{align*}
  E &= \sum_i E_i = \frac{1}{2}\sum_{i \ne j} V_{ij}, \\
  V_{ij} &= f_C(r_{ij}) \big[f_R(r_{ij}) + b_{ij}f_A(r_{ij})  \big],
\end{align*}

% where the total potential energy is decomposed into an atom site energy $E_i$
% and a bond energy $v_{ij}$. 
where the total potential energy is decomposed into a bond energy $V_{ij}$. The
indices $i$ and $j$ run over the atoms of the system with $r_{ij}$ denoting the
distance between atom $i$ and $j$. Notice that the sum includes all combinations
of $i,j$ where $i\ne j$ meaning that the same bond is double counted which is
the reason for the additional factor $1/2$. The reasoning behind comes from the
asymmetry of the bond order $b_{ij}\ne b_{ji}$ leading to a $V_{ij}\ne V_{ji}$.
The bond energy is composed of a repulsive term $f_R$, arising from overlapping
wave functions, and an attractive term $f_A$ associated with bonding. $f_c$ is
simply a smooth cut-off function to increase computational efficiency. $b_{ij}$
represent the bond order, i.e. the strength of the bonds, which depends
inversely on the number of bonds, the bond angles ($\theta_{ijk}$) and
optionally the relative bonds lengths ($r_{ij}$, $r{jk}$). Notice that an
additional cut-off term $a_{ij}$ was orginally multiplied to $f_R$ as a way of
including terms that limit the range of the interactions to the first neighbour
shell. These kind of limitations is already included in $b_{ij}$ for the
attractive term $f_A$ but is often omitted for the repulsive term $f_R$, and we
do so to by setting $a_{ij} = 1$. \\
The cut-off function $f_C$ goes from 1 to 0 over a small interval range $R \pm
D$ as
\begin{align*}
  f_C(r) =
  \begin{cases}
    1 & r < R - D \\
    \frac{1}{2} - \frac{1}{2} \sin{(\frac{\pi}{2} \frac{r - R}{D})} & R - D < r < R + D\\
    0 & r > R + D
  \end{cases},
\end{align*}
which is continuous and differentiable for all $r$. $R$ is usually chosen to
include only the first neighbour shell. \\
The repulsive and attractive terms $f_R$ and $f_A$ is modelled as an exponetial
function, similar to a morse potential, 
\begin{align*}
 f_R(r) &= A \exp(-\lambda_1 r), \\
 f_A(r) &= -B \exp \big(-\lambda_2 r\big).
\end{align*}

The novel feature of the model lies in modeling of the bond order $b_{ij}$ which
includes three-body interactions by summing over a third atom $k \ne i,j$ within
the cut-off $r_{ik} < R + D$ as shown in the following.

\begin{align}
  b_{i j} & =\big(1+\beta^n \zeta_{i j}^n\big)^{-\frac{1}{2 n}} \\
  \zeta_{i j} & =\sum_{k \ne i,j} f_C(r_{i k}) g\Big(\theta_{i j k}\left(r_{i j}, r_{i k}\right)\Big) \exp \left(\lambda_3{ }^m\big(r_{i j}-r_{i k}\right)^m\big) \\
  g(\theta) & =\gamma_{i j k}\left(1+\frac{c^2}{d^2}-\frac{c^2}{\left[d^2+\left(\cos \theta-\cos \theta_0\right)^2\right]}\right).
  \label{eq:tersoff_bond_order}
\end{align}

In eq.~\eqref{eq:tersoff_bond_order} $\zeta_{i,j}$ is an effective coordination
and $g(\theta)$ captures angle dependency as it is minimized at the equilibrium
angle $\theta = \theta_0$. \\
The parameters used to model the graphene C-C bonds is summarized in table
\ref{tab:tersoff_param}



\begin{table}[H]
  \begin{center}
  \caption{Parameters for the tersoff potential used for intermolecular interations in the graphene sheet}
  \label{tab:tersoff_param}
  \begin{tabular}{ | c | c | L{9cm} |} \hline
    Parameter & Value & Description \\ \hline 
    $m$ & 3.0 & Default (not used since $\lambda_3 = 0$ ) \\ \hline
    $\gamma$ & 1.0 & ... \\ \hline
    $\lambda_3$ & 0.0 Å$^{-1}$ & ... \\ \hline
    $c$ & \num{3.8049e4} & Strength of the angular effect \\ \hline
    $d$ & 4.3484 & Determines the ``sharpness'' of the angular dependency \\
    \hline
    $\cos{(\theta_0)}$ & -0.57058 & Cosine of the equilibrium angle \\ \hline
    $n$ & 0.72751 & Power law exponent for the bond order dependency \\ \hline
    $\beta$ & \num{1.5724e-7} & ... \\ \hline
    $\lambda_2$ & 2.2119 Å$^{-1}$ & Decay of repulsion potential term \\ \hline
    $B$ & 346.74 eV & Attractive potential term minimum at core ($ r_{ij} = 0$).
    \\ \hline
    $R$ & 1.95 Å & Center distance for cut-off \\ \hline
    $D$  & 0.15 Å & Thickness of cut-off layers \\ \hline
    $\lambda_1$ & 3.4879 Å$^{-1}$ & Decay of repulsion potential term \\ \hline
    $A$ & 1393.6 eV & Repulsion potential term at core ($ r_{ij} = 0$) \\ \hline
  \end{tabular}
  \end{center}
\end{table}



\subsection{LAMMPS}
\subsection{Integration}
% https://www.eng.uc.edu/~beaucag/Classes/AdvancedMaterialsThermodynamics/Books/%5BComputational%20science%20(San%20Diego,%20Calif.)%5D%20Daan%20Frenkel_%20Berend%20Smit%20-%20Understanding%20molecular%20simulation%20_%20from%20algorithms%20to%20applications%20(2002,%20Academic%20Press%20)%20-%20libgen.lc.pdf

Having defined a system of particles governed by interartomic potentials we need
to move the system forward in time. By solving Newtons equations of motion we
effectively do so by sampling the microcanonical ensemble characterized by a
constant number of particles $N$, volume $V$ and energy $E$, hence denoted NVE.
Newtons equaitons of motion read
\begin{align}
  m_i \frac{d^2 \vec{r}_i}{dt^2} = \vec{F}_i = -\nabla U_i
  \label{eq:NE}
\end{align}
where $i$ is the particle index and $m_i$ its mass, $\vec{r}_i = (x_i, y_i,
z_i)$ the position, $t$ is time,  $\nabla_i = (\frac{\partial}{\partial x_i},
\frac{\partial}{\partial y_i}, \frac{\partial}{\partial z_i})$ and $U_i$ the
potential energy. In system the potential energy is a function of the particle
positions of nearby particles depending on the specefic potential in use. Since
the forces defined by the potentials is conservative we expect the energy of the
solution to be conserved. We redefine eq.\eqref{eq:NE} in terms of two coupled
first order differential equations 
\begin{align}
  \dot{\vec{v}}_i(t) = \frac{\vec{F}}{m_i}, \qquad \dot{\vec{r}}_i(t) = \vec{v}_i(t),
  \label{eq:NE_2}
\end{align}
where $\dot{x} = dx/dt$ (Newton's notation) and $\vec{v} = (v_x, v_y, v_z)$ is
velocity. Numerically we can solve the coupled equations (eq.\eqref{eq:NE_2}) by
integrating over discrete timnesteps. That is, we discretize the solution into
temporal steps $t_k = t_0 + k\cdot \Delta t$ with time-step $\Delta t$. 

% \begin{align*} \ddot{x}(t) = \frac{F(x)}{m} \quad \rightarrow \quad \dot{x} =
%   v(t), \ \ \dot{v}(t) = \frac{F(x(t))}{m} \end{align*}

% Integration of newtons equations of motion (just like, Euler, Euler cromer and
% so on) and specify the verlet algorithm which is used in Lammps.

% The forces (form the potential) is conservative so the energy should be
% conserved before applying the thermostat.

% However small erros applied by the discrete integraiton algorithm we end up
% having an energy error. This is sensitive to time step. 


\subsubsection{Velocity Verlet}
% http://www.physics.drexel.edu/~valliere/PHYS305/Diff_Eq_Integrators/Verlet_Methods/Diffrntleqn3.pdf
% ttps://www2.ph.ed.ac.uk/~dmarendu/MVP/MVP03.pdf

A common algorithm to integrate Newtons equation of motion (as formulated in
eq.\eqref{eq:NE_2}) is the \textit{velocity verlet}. We can derive the algorithm
by the use of Taylor expansions. We begin by expanding the next-step position
vector $\vec{r}_i(t + \Delta t)$ at time $t$
\begin{align}
  \vec{r}_i(t + \Delta t) &= \vec{r}_i(t) + \dot{\vec{r}}_i(t) \Delta t + \frac{\ddot{\vec{r}}_i(t)}{2} \Delta t^2 + \mathcal{O}(\Delta t^3) \label{eq:vv_comp1},
\end{align}
where $\ddot{\vec{r}} = d^2\vec{r}/dt^2$ and $\Delta t^n$ is simply the relaxed
notation for $(\Delta t)^n$. Similar we take the expansions of the next-step
velocity vector $\vec{v}_i(t+\Delta t)$ at time $t$ 
\begin{align}
  \vec{v}_i(t+\Delta t) = \vec{v}_i(t) + \dot{\vec{v}}_i(t) \Delta t + \frac{\ddot{\vec{v}}_i(t)}{2}\Delta t^2 + \mathcal{O}(\Delta t^3).
  \label{eq:tay_v1}
\end{align}
Finnally, by taking the expansion of $\dot{\vec{v}}_i(t+\Delta t)$ we can
eliminate the $\ddot{\vec{v}}_i$-term in eq.~\eqref{eq:tay_v1} and simplify it
as shown in the following.
\begin{align}
  \dot{\vec{v}}_i(t+\Delta t) &= \dot{\vec{v}}_i(t) + \ddot{\vec{v}}_i(t) \Delta t + \mathcal{O}(\Delta t^2) \nonumber \\
  \frac{\ddot{\vec{v}}_i(t)}{2}\Delta t^2 &= \frac{\Delta t}{2}\Big( \dot{\vec{v}}(t+\Delta t) - \dot{\vec{v}}_i(t)\Big) + \mathcal{O}(\Delta t^3) \nonumber \\
  &\Downarrow \nonumber \\
  \vec{v}_i(t+\Delta t) &= \vec{v}_i(t) + \dot{\vec{v}}_i(t) \Delta t + \frac{\Delta t}{2}\Big( \dot{\vec{v}}_i(t+\Delta t) - \dot{\vec{v}}_i(t)\Big) + \mathcal{O}(\Delta t^3) \nonumber \\
  &=  \vec{v}_i(t) + \frac{\Delta t}{2}\Big( \dot{\vec{v}}_i(t) +  \dot{\vec{v}}_i(t+\Delta t)\Big) + \mathcal{O}(\Delta t^3).
  \label{eq:vv_comp2}
\end{align}
By combining eq.~\eqref{eq:vv_comp1} and eq.~\eqref{eq:vv_comp2} and using
Newton's second equation $\dot{\vec{v}} = \vec{F}_i(t)/m_i = $ and $\vec{v} =
\dot{\vec{r}}$ we arrive at the final scheme
\begin{align*}
  \vec{r}_i(t + \Delta t) &= \vec{r}_i(t) + \vec{v}_i(t) \Delta t + \frac{\vec{F}_i(t)}{2m_i}\Delta t^2 + \mathcal{O}(\Delta t^3), \\
  \vec{v}_i(t+\Delta t)  &= \vec{v}_i(t) + \frac{\vec{F}_i(t) + \vec{F}_i(t+\Delta t)}{2m_i}  \Delta t + \mathcal{O}(\Delta t^3).
\end{align*}
The scheme will give a local error of order $\Delta t^3$ corresponding to a
global error of $\Delta t^2$. One of the most popular ways to implement this
numerically is as stated in the following steps.
\begin{enumerate}
  \centering
  \item Calculate $v_{k+\frac{1}{2}} = v_k + \frac{F_k}{2m} \Delta t$.
  \item Calculate $r_{k+1} = r_k + v_{k+\frac{1}{2}} \Delta t$.
  \item Evaluate the force $F_{k+1} = F(r_{k+1})$.
  \item Calculate $v_{k+1} = v_{k+\frac{1}{2}} + \frac{F_{k+1}}{2m} \Delta t$  
\end{enumerate}





% \begin{align*} \vec{r}_i(t + \Delta t) &= \vec{r}_i(t) + \vec{v}_i(t) \Delta t
%   + \frac{\vec{F}_i(t)}{2m_i}\Delta t^2 + \mathcal{O}(\Delta t^3), \\
%   \vec{v}_i(t+\Delta t)  &= \vec{v}_i(t) + \frac{\vec{F}_i(t) +
% \vec{F}_i(t+\Delta t)}{2m_i}  \Delta t + \mathcal{O}(\Delta t^3). \end{align*}
% This scheme will give a local error of order $\Delta t^3$ corresponding to a
% global error of $\Delta t^2$. 

% Lets descritize the time as $t_k = k * \Delta t$ with position $\vec{r}_k$ and
% velocity $\vec{v}_k$. At time $t_k$ the acceleration is given $\vec{a}_k =
% F(\vec{r}_k)/m$. We get the implementation steps as follows \begin{enumerate}
% \item Calculate $\vec{r}_{k+1} = \vec{r}_k + \vec{v}_k \Delta t +
% \frac{F(\vec{r}_k)}{2m}(\Delta t)^2$ \item Evaluate $F(\vec{r}_{k+1})$ \item
% Calculate $\vec{v}_{k+1} = \vec{v}_k + \frac{F(\vec{r}_k) +
% F(\vec{r}_{k+1})}{2m} \Delta t$ \end{enumerate}


% It is sometimes expressed as the following. This is mathematically the same,
% but is it computaitonally more efficient?


% \begin{align*} \mathbf{r}_i(t+\delta t) & =\mathbf{r}_i(t)+\mathbf{v}_i(t)
%   \delta t+\frac{\mathbf{f}_i(t)}{2 m_i} \delta t^2 \\
%   \mathbf{v}_i(t+\delta t / 2) & =\mathbf{v}_i(t)+\frac{\delta t}{2}
%   \frac{\mathbf{f}_i(t)}{m_i} \\
%   \mathbf{f}_i(t+\delta t) & =\mathbf{f}_i\left(\mathbf{r}_i(t+\delta
%   t)\right) \\
%   \mathbf{v}_i(t+\delta t) & =\mathbf{v}_i(t+\delta t / 2)+\frac{\delta t}{2}
%   \frac{\mathbf{f}_i(t+\delta t)}{m_i} \end{align*}

%   % https://www2.ph.ed.ac.uk/~dmarendu/MVP/MVP03.pdf Further they suggest this
%   to be the most used algorithm

%   \begin{align*} \mathbf{v}_i(t+\delta t / 2) & =\mathbf{v}_i(t)+\frac{\delta
%     t}{2} \frac{\mathbf{f}_i(t)}{m_i} \\
%     \mathbf{r}_i(t+\delta t) & =\mathbf{r}_i(t)+\mathbf{v}_i(t+\delta t / 2)
%     \delta t \\
%     \vec{f}_i(t + \delta t) &= \vec{f}_i(\vec{r}_i(t+ \delta t)) \\
%     \mathbf{v}_i(t+\delta t) & =\mathbf{v}_i(t+\delta t / 2)+\frac{\delta
%   t}{2} \frac{\mathbf{f}_i(t+\delta t)}{m_i} \end{align*} Make sure to check
%   this out. 

\subsection{Thermostats}

As we already mentioned above in Sec. 2, any kind of sliding friction involves mechanical work, some of which is then transformed into heat (the rest going into structural transformations, wear, etc.). The heat is then transported away by phonons (and electrons in the case of metallic sliders) and eventually dissipated to the environment \cite{Manini_2016}.



Likewise all excitations generated in the simulations should be allowed to propagate in the system and disperese in the bulk of both sheet and substrate. Due to small simulation size theese is likely to relfect back and ´´pile up'' unphysically Thus in order to avoid continuous heating and attain a steady state the (Joule) heat must be removed at a steady state. This is very the viscous damping of the langevin equations enter the picture. It can be difficulut to set the value $\gamma$ for the magnitude of this damping. The unphysical introduction of heat sink can be mittigated by some modifictions he mention, which is kind of next level I guess. 


\subsubsection*{Langevin thermostat}

% Check out \cite{Manini_2016} for a good theory section on this that I had
% completely missed when writing this!

% Based on
% https://www.uio.no/studier/emner/matnat/fys/FYS4130/v19/pensumliste/stat-phys_2019.pdf

% http://physics.gu.se/~frtbm/joomla/media/mydocs/LennartSjogren/kap6.pdf


In order to control the temperature of the system we introduce the so-called
Langevin thermostat. This is a stochastic thermostat that modifies Newtons
equation of motion such that solution lies in the canonical ensemble
characterized by a constant number of particles $N$, constant volume $V$ and
constant temperature $T$, hence denoted NVT. The canonical ensemble system is
represented by the finite system being in contact with an infinte heat bath of
temperature $T$. The NVT ensemble is equivalent to sampling a system in
theromodynamic equilibrium where the weight of each microscopic state is given
by the boltzmann factor $\exp[-E/(k_B T)]$.

The Langevin equation is the modified version of Newtons second law for a
Brownian particle. A brownian particle is a small particle suspendend in liquid,
e.g. pollen or dust, named after Robert brown (1773–1858) who was the first to
observe its jittery motion. The Langevin equation describes this motion as the
combination of viscous drag force $ -\gamma \vec{v}$, where $\gamma$ is a
positive friction coefficient and $\vec{v}$ the velocity vector, and a random
fluctuation force $\vec{R}$. The langevin equation reads
\begin{align}
  m \frac{d \vec{v}}{dt} = -\gamma \vec{v} + \vec{R}
  \label{eq:Langevin}
\end{align}
where $m$ is the particle mass. This effectively describes the particle of
interest, the brownian particle, as being suspendend in a sea of smaller
particles. The collision with these smaller particles is modelled by the drag
force and the fluctuation force. We notice that if the fluctuation force is
excluded eq.~\eqref{eq:Langevin} becomes 
\begin{align*}
  m \frac{d \vec{v}}{dt} = -\gamma \vec{v} \quad \Rightarrow \quad 
  \vec{v}_i(t) = v(0)e^{- \frac{\gamma t}{m}},
\end{align*}
where the solution shows that the brownian particle will come to a complete stop
after a long time ${\vec{v}_i(t\to\infty) \to \vec{0}}$. This is in violation
with the equipartion theorem
\begin{align*}
  \frac{1}{2}m\langle v^2 \rangle_{eq} = \frac{k_B T}{2},
\end{align*}
and hence the fluctuation force is nessecary to obtain the correct equilibrium. 

The following calculations are done in one dimension in order to simplify the
notation. We describe the statistical nature of the collisions as a sum of
independent momentum transfers
\begin{align*}
  \Delta P = \sum_i^N \delta p_i
\end{align*}

where $\Delta P$ denotes the change of momentum after $N$ momentum transfers
$\delta p_i$ from the environment to the brownian particle. We assume the first
and second moments $\langle \delta p \rangle = 0$ and  $\langle \delta p \rangle
= \sigma^2$. When $N$ is large the central limit theorem states that the random
variable $\Delta P$ has a gaussian distribution with  $\langle P \rangle = 0$
and $\langle \Delta P^2 \rangle = N\sigma^2$. If we consider the momentum change
$\Delta P$  over a discrete time $\Delta t$, where the number of collisiosn is
proportional to time $N \propto \Delta t$, the corresponding fluctuation force
$R = \Delta P / \Delta t$ will have a variance 


\begin{align*}
  \langle R^2 \rangle = \frac{\langle \Delta P^2 \rangle}{\Delta t^2} = \frac{N \sigma^2}{\Delta t^2}  \propto \frac{1}{\Delta t}.
\end{align*}

In a computer simulation we need to pick a random force $R(t)$ from a Gaussian
distribution every time-step $\Delta t$. These forces will not be correlated as
long as $\Delta t$ is larger than the correlation time of the forces from the
molecules which we will assume for this model (I think there exist corrections
for this to refer to here). With this assumption we can write the correlation
function as 
\begin{align}
  \langle R(t) R(0) \rangle = 
  \begin{cases}
    \frac{a}{\Delta t}, & |\Delta t| < \Delta t/2 \\
    0, & |\Delta t| > \Delta t/2,
    \label{eq:disc_corr}
  \end{cases}
\end{align}

where $a$ is some strength of (...?). In the limit $\Delta t \to 0$ the
correlation function becomes

\begin{align}
  \langle R(t)R(0) \rangle = a \delta(t),
  \label{eq:F_corr}
\end{align}

where $\delta$ denotes the dirac delta function. This is valid for all spatial
coordinates which will all be independent of each other. Since both the drag
force and the fluctuation force originate from the molecular fluid, where the
drag force $-\alpha \vec{v}$ is velocity dependent it is reasonible to assume
that fluctuation force is independent of velocity, i.e. $\langle R_i v_j \rangle
= 0$ for all cartesian indices $i$ and $j$.


% Since $\langle \tilde{F} \rangle = 0$ the random force (fluctuating force)
% will not conribute with a average decay (change) to the velocity. The
% macroscopic decay comes from the friction force (dissipitative force)  $-ma
% =\alpha v$ (give variable explanation). 

In the following we will attempt justify the Langevin equaiton (why it is like
it is) and determine the relationship between the drag coefficient $\gamma$ and
the random force $R$.


From the Langevin equation eq.~\eqref{eq:Langevin} we can compute the velocity
autocorrelation function (Move to appendix?). We do this in one dimension for
simplicity. We begin by multiplying by $(e^{\gamma t /m})/m$

\begin{align*}
  \dot{ v}(t)e^{\gamma t /m} + \frac{\gamma}{m} v(t)e^{\frac{\gamma t}{m}}  = \frac{ F}{m}e^{\frac{\gamma t}{m}},
\end{align*}
and integrate from $t = -\infty$. By the use of integration by parts on the
latter term on the left hand side we calculate the velocity 
\begin{align*}
  \int_{-\infty}^t dt' \ \dot{ v}(t')e^{\frac{\gamma t'}{m}} + \frac{\gamma}{m} v(t)e^{\frac{\gamma t'}{m}} &=  \int_{-\infty}^t dt' \ e^{\frac{\gamma t'}{m}} \frac{ F(t')}{m}  \\
  \int_{-\infty}^t dt' \ \dot{ v}(t')e^{\frac{\gamma t'}{m}} + \left(\Big[ v(t')e^{\frac{\gamma t'}{m}}\Big]_{-\infty}^t - \int_{-\infty}^t dt' \ \dot{ v}(t')e^{\frac{\gamma t'}{m}}\right) &= \int_{-\infty}^t dt' \ e^{\frac{\gamma t'}{m}} \frac{ F(t')}{m}  \\
   v(t) &= \int_{-\infty}^t dt' \ e^{\frac{-\gamma(t - t')}{m}} \frac{ F(t')}{m},
\end{align*}
where $e^{\frac{-\gamma t}{m}}$ plays the role of a response function. We can
then calculate the autocorrelation 
\begin{align*}
  \big\langle  v(t) v(0) \big\rangle &= \int_{-\infty}^t dt_1 \ \int_{-\infty}^0 dt_2 \ e^{\frac{t - t_1 - t_2}{m}} \frac{\langle  F(t_1)  F(t_2) \rangle}{m^2} \\
  &= \int_{-\infty}^t dt_1 \ \int_{-\infty}^0 dt_2 \ e^{\frac{t - t_1 - t_2}{m}} \frac{a \delta(t_1 - t_2)}{m^2} \\
  &= \int_{-\infty}^0 dt_2 \ e^{\frac{t - 2t_2}{m}} \frac{a}{m^2} = \frac{a}{2m\gamma}e^{-\frac{\gamma t}{m}},
\end{align*}
where we used eq.~\eqref{eq:F_corr} and the fact that the integration commutes
with the average (we are allowed to flip the order). By comparing this with the
equipartition theorem we get 
\begin{align*}
  \frac{1}{2}m\langle  v^2 \rangle &= \frac{k_BT}{2} \\
  \frac{1}{2}m\langle  v(0) v(0) \rangle = \frac{a}{4\gamma} &= \frac{k_BT}{2} \\
  a &=  2\gamma k_B T \\
\end{align*}
We notice the appereance of $\gamma$ meaning that the magnitude of the
fluctuations increase both with friction and temperature. Further we can
integrate the velocity over time to get displacement $x(t)$ and show that the
variance (show this? In appendix maybe?) is 
\begin{align*}
  \big\langle x^2(t) \big\rangle = \frac{2 k_B T}{\gamma} \left(t - \frac{m}{\gamma}\left(1 - e^{-\gamma t/m} \right) \right),
\end{align*}
where for $t \gg m/\gamma$ only the $t$-term survies yielding
\begin{align*}
  \langle x^2(t) \rangle = 2 k_BTt/\gamma.
\end{align*}
In 1D, the diffusion constant $D$ is related to the variance as $\langle x^2
\rangle = 2Dt$, meaning that this represents the einstein relation $D = \mu k_B
T$ with the mobility $\mu = 1/\gamma$.

when $t \ll m/\gamma$ we use the Taylor expansion $1 - e^{-x} \approx x - x^2/2$
for $x\ll 1$ to get 
\begin{align*}
  \big\langle x^2(t) \big\rangle = \frac{k_B T}{m} t^2
\end{align*}
which exactly mathces the thermal velocity
\begin{align*}
  v_{\text{th}} \frac{\big\langle x^2(t) \big\rangle}{t^2} = \frac{k_B T}{m}
\end{align*}
which follows from the equipartition theorem. The finite correlation time
$\gamma/m$ hence describe the crossover from the ballistic regime $\sqrt{\langle
x^2(t) \rangle} \propto t$ to the diffusive regime $\sqrt{\langle x^2(t)
\rangle} \propto \sqrt{t}$.

Introduce the fluctuation-dissipation theorem concept earlier since this is a
motivaiton for the Langeivn equation. 


% Of course, this approach is not rigorous, since the relevant particles
% colliding with each given simulated atom are already all included in the
% conservative and deterministic forces explicitly accounted for by the “force
% field”. The Langevin approach is quite accurate to describe small
% perturbations away from equilibrium, but it may fail quite badly in the
% strongly out-of-equilibrium nonlinear phenomena which are the target of the
% present paper.  \cite{Manini_2016}

\subsubsection{Implementing Langevin}
% https://docs.lammps.org/fix_langevin.html

% https://www2.ph.ed.ac.uk/~dmarendu/MVP/MVP03.pdf

% https://chem.libretexts.org/Bookshelves/Physical_and_Theoretical_Chemistry_Textbook_Maps/Non-Equilibrium_Statistical_Mechanics_(Cao)/01%3A_Stochastic_Processes_and_Brownian_Motion/1.04%3A_The_Langevin_Equation

% \cite{Hunenberger2005}(pp. 115, 120-121) \cite{docs_lammps_langevin}

% Make a note: We only introduce the thermostat to the edges of the parts of
% interest as the thermostat might disrupt important properties of the
% simulation (accoridng to Henrik, get a source/example on this).


The implementation of the Langevin equation into LAMMPS follows
\cite{PhysRevB.17.1302} and updates the force vector for each particle as 

\begin{align}
  \vec{F} &= \vec{F_c} + \vec{F}_{f} + \vec{F}_{r} \nonumber \\
  &= -\nabla U - \gamma m \vec{v} + \sqrt{\frac{2 k_B T m \gamma}{\Delta t}}\vec{h}(t)
  \label{eq:Langevin_generalized}
\end{align}
where $\vec{F_c}$ is the conservative force computed via the usual
inter-particle interactions described by the potential $U$, $\vec{F}_f$ is the
drag force and $\vec{F}_r$ is the random fluctuation force where $\vec{h}$ is a
random vector drawn from a normal distribution with zero mean and unit variance.
Notice that this generalized description of the Langevin equation deviates from
the presentation in eq.~\eqref{eq:Langevin} since we have added the conservative
force $\vec{F_c}$, but also by the appearance of the mass in both the drag force
and the fluctuation force due to the introduction of damping. It is beyond out
scope to comprehend this. However, the fact that $\Delta t$ now appears in the
denomiator for the random force variance $2k_B T m \gamma / \Delta t$ is due to
the fact that we have discretized time. This in agreement with the formulation
in eq.~\eqref{eq:disc_corr}. By applying eq.~\eqref{eq:Langevin_generalized} we
get the refined velocity verlet scheme


\begin{align*}
  \vec{v}_i(t + \Delta t/2)  &= \vec{v}_i(t) - \frac{\Delta t}{2}\left(\frac{\nabla_i U(t)}{m_i} + \gamma \vec{v}_i \right) + \sqrt{\frac{k_B T \gamma \Delta t}{2m_i}} \vec{h}_i \\ 
  \vec{r}_i(t + \Delta t) &= \vec{r}_i(t) + \vec{v}_i(t + \Delta t/2) \Delta t \\
  \vec{v}_i(t + \Delta t) &= \vec{v}_i(t+ \Delta t/2) - \frac{\Delta t}{2}\left(\frac{\nabla_i U(t + \Delta t)}{m_i} + \gamma \vec{v}_i(t + \Delta t/2) \right) + \sqrt{\frac{k_B T \gamma \Delta t}{2m_i}} \vec{h}_i
\end{align*}
% A little unsure whether the factor 2 in the denominator in the random force is
% correct.
with new random vector $\vec{h}_i$ for each particle and each update. Notice
however, that LAMMPS only apply this scheme to the particle groups with the
thermostat on. 

% \newpage

% \begin{align} \vec{F}_i &= \vec{f}_{i} + \vec{f}_{f,i} + \vec{f}_{r,i}
%   \nonumber \\
%   m_i \ddot{\vec{r}}_i &= \vec{f}_{i} - \gamma_i \vec{p}_i + \vec{R}_i(t)
%   \label{eq:langevin_old} \end{align}

% where $m_i$ is the mass of particle $i$ $\vec{r}_i$ the position vector. The
% random force $\vec{R}_i(t)$ is described as gaussian white noise and have the
% following properties. 

% \begin{enumerate} \item It is uncorrelated with the velocities
%   $\dot{\vec{r}}(t)$ and deterministic forces $\vec{f}_i(t')$ at previous
%   times $t' < t$. \item The time average is zero: $\langle \vec{R}(t) \rangle
%   = \vec{0}$. \item The mean-square components evaluate to $2m_i\gamma_i k_B
%   T$. \item The force component of particle $i$ $R_{i,\mu}$ along cartesian
%   axis $\mu$ is uncorrelated with any component of particle $j$ $R_{i,\nu}$
%   along cartesian axis $\nu$, unless $i=j$, $\mu=\nu$ and $t=t'$.
%   \end{enumerate}

% The ladder two conditions can be formulated as

% \begin{align*} \langle R_{i,\mu}(t) R_{j,\nu}(t') \rangle = 2m_i \gamma_i k_B
%   T  \delta_{ij} \delta_{\mu\nu}\delta(t'-t),  
% \end{align*}

% where $\delta_{\cdot\cdot}$ is the Kronecker Delta function and
% $\delta(\cdot)$ is the Dirac Delta function. It can be shown that a trajectory
% generated by integrating the Langevin equations of motion
% (eq.~\eqref{eq:langevin_old}) maps a cononical distribution of microstates at
% temperature $T$. \cite{Hunenberger2005}(p.121). (Can I say something more
% about the relationship between the friction froce and the random force - How
% to the balance to give NVT. Is the proof complicated?)\\

% In practice the implementation of the thermostat is implemented discretely by
% updating the force on each particle by the addition of the described forces
% $f_f$ and $f_r$ for each particle. In LAMMPS this is controlled by the user
% defined damping factor ``damp'' = $\gamma^{-1}$ in units of time whih control
% how fast we are going to reach the temperature equlibrium. Thus the
% documentation (\cite{docs_lammps_langevin}) defines the added forces as

% \begin{align*} f_f &= -\frac{m}{\text{damp}}, \qquad f_r \propto \sqrt{\frac{m
%   k_B T}{dt \ \text{damp}}} \end{align*}

% where $v = \dot[r]$ is the velocity. The definition of $f_f$ falls straight
% out of the definition above in eq.~\eqref{eq:Langevin} while the
% proportionality of $f_r$ comes from

% \begin{align*} \sqrt{\langle R(t)^2 \rangle} = \sqrt{2m\gamma k_B T} \propto
%   \sqrt{\frac{m k_B T}{\text{damp}}} \end{align*}

% Thus we have $f_r \propto \sqrt{\langle R(t)^2 \rangle / dt}$ which I do not
% quite understand. It is mentioned in
% \url{https://pubs.acs.org/doi/full/10.1021/ct8002173} when talkikng about
% residual forces, as it is natural when taking a step length of $dt$...
% Probably simply, but check this later. 


% https://www2.ph.ed.ac.uk/~dmarendu/MVP/MVP03.pdf



% \begin{align*} \vec{r}_i(t + \Delta t) &= \vec{r}_i(t) + \vec{v}_i(t) \Delta t
%   + \frac{\vec{F}_i(t)}{2m_i}\Delta t^2 + \mathcal{O}(\Delta t^3), \\
%   \vec{v}_i(t+\Delta t)  &= \vec{v}_i(t) + \frac{\vec{F}_i(t) +
% \vec{F}_i(t+\Delta t)}{2m_i}  \Delta t + \mathcal{O}(\Delta t^3). \end{align*}



% \newpage This section is based on \cite{PhysRevB.17.1302},
% \cite{Hunenberger2005}(pp. 115, 120-121) and \cite{docs_lammps_langevin}



% %
% https://www.uio.no/studier/emner/matnat/fys/FYS4130/v19/pensumliste/stat-phys_2019.pdf

% ``The Langevin equation is Newtons second law for a Brownian particle, where
% the forces include both the viscous drag due to the surrounding fluid and the
% fluctuations caused by the individual collisions with the fluid molecules.''


% % http://physics.gu.se/~frtbm/joomla/media/mydocs/LennartSjogren/kap6.pdf

% % https://www2.ph.ed.ac.uk/~dmarendu/MVP/MVP03.pdf ``Another option to
% simulate a system in the NVT ensemble is to use a stochastic thermostat, as
% opposed to the deterministic thermostat defined through the Nose-Hoover
% equations.''

% ``The equations of mo- tion of a system with a stochastic thermostat are known
% as Brownian dynamic equations''

% In order to simulate the canonical ensemble, that is the ensemble defined by a
% constant amount of particles $N$, constant temperature $T$ and constant volume
% $V$, hence often denoted $NVT$, we apply a so-called thermostat. There exist a
% variety of such including Nosé-Hoover, Gaussian, Berendsen, Langevin
% thermostat and many other (give example of how they modify) We will use the
% ladder. \\
% The connical ensemble is an ensemble of systems by the assumption that the
% system of interests is connected to an infitely large heat bath of temperature
% $T$. The Langevin thermostat assumes that the particles collides with much
% smaller lighter particles representing the heat bath as a sea of small
% particles. The collisions is described by a friction force $f_f = -\gamma
% \vec{p}$, where $\gamma$ is positiv friction coefficient and $\vec{p}$ the
% momentum vector, and a random force $f_r = \vec{R}_i(t)$. By denoting the
% conservative force arising from the usual inter-particle interactions
% $\vec{f}_i$ on particle $i$ we get the Langevin equations of motion describing
% the total force $\vec{F}_i$ on particle $i$ as

% see https://www2.ph.ed.ac.uk/~dmarendu/MVP/MVP03.pdf p. 4


\subsubsection{MD limitations}

% One of the limitations of concern is in regard to the simulation scale. Despite relatively fast processor speed and efficient algorithm the number of atoms that can be handled realistically is still insufficient for complex system simulation \cite{kim_nano-scale_2009}.

\section{Defining the system}

\subsection{Groups: Sheet, pullblocks and substrate}
Include figure of system to point out thermo layers and freeze layers.


\begin{figure}[H]
  \centering
  \begin{subfigure}[b]{0.80\textwidth}
      \centering
      \includegraphics[width=\textwidth]{figures/system_sideview.png}
      \caption{Side view}
      \label{fig:sideview}
  \end{subfigure}
  \hfill
  \begin{subfigure}[b]{0.80\textwidth}
      \centering
      \includegraphics[width=\textwidth]{figures/system_topview.png}
      \caption{Top view}
      \label{fig:topview}
  \end{subfigure}
  \hfill
     \caption{System. 27456 atoms in total: 7272 with thermostat (orange), 7272 is locked (light blue) and the remaing 12912 just with NVE. (Get better colors)}
     \label{fig:system}
\end{figure}



\begin{table}[H]
  \begin{center}
  \caption{System atom count and region division.}
  \label{tab:system_count}
  \begin{tabular}{ |c|| c | c | c | c | c | c |} \hline
    \textbf{Region} & \textbf{Total}  & Sub region & Sub total & \textbf{NVE} &
    \textbf{NVT} & \textbf{Locked} \\ \hline   
    \multirow{2}{*}{Sheet} & \multirow{2}{*}{7800} & Inner sheet & 6360 & 6360 &
    0 & 0 \\ %\hline
    & & Pull blocks & 1440 & 0 & 720 & 720 \\ \hline   
    \multirow{2}{*}{Substrate} & \multirow{2}{*}{19656} & Upper & 6552 & 6552 &
    0 & 0 \\ %\hline
    & & Middle & 6552 & 0 & 6552 & 0 \\ %\hline
    & & Bottom & 6552 & 0 & 0 & 6552 \\ \hline \hline   
    All & 27456 & \multicolumn{2}{r|}{} & 12912 & 7272 & 7272 \\ \hline 
  \end{tabular}
  \end{center}
\end{table}


The sheet dimensions is 
\begin{table}[H]
  \begin{center}
  \caption{Sheet dimensions}
  \label{tab:}
  \begin{tabular}{ | l | r@{}l | r@{}l | c |} \hline
    \textbf{Group} & \multicolumn{2}{c|}{$x,y$-dim} & \multicolumn{2}{c|}{dim
    [Å]} & Area [Å$^2$]\\ \hline
  Full sheet & $x_S \: \times \: $ & $y_S$ &  $130.029 \: \times \:$ & $163.219$
  Å & $\phantom{2\times} 21,223.203$ \\ \hline
  Inner sheet & $x_S \: \times \:$ & $81.40 \ \%_{y_s}$ &  $130.029  \: \times
  \:$ & $132.853$ Å & $\phantom{2\times} 17,274.743$\\ \hline
  Pull blocks & $2 \times x_S \: \times \:$ & $ \phantom{0}9.30 \ \%_{y_s}$ & $2
  \times 130.029  \: \times \: $ & $\phantom{0}15.183$ Å  & $2 \times
  \phantom{0}1,974.230$ \\ \hline  
  \end{tabular}
  \end{center}
\end{table}


\subsubsection{Pressure reference}
\text{Find place to put this.}

% source 1: stiletto heeled shoes with less than 1cm diameter:
% https://www.researchgate.net/publication/342223559_How_the_stiletto_heeled_shoes_which_are_popularly_preferred_by_many_women_affect_balance_and_functional_skills

In order to relate the magntidue of the normal force in our friciton measurement
we will use the pressure as a reference. We will use the pressure underneath a
stiletto shoe as a worst case for human pressure execuation underneath the
shoes. From (source 1) it is reported that the diameter of a stiletto heeled
shoe can be less than 1 cm. Hence a 80 kg man\footnote{Yes, a man can certainly
wear stilleto heals.} standing on one stiletto heels (with all the weight on the
heel) will result in a pressure

corresponding diameter of 
\begin{align*}
  P = \frac{F}{A} = \frac{mg}{r^2\pi} = \frac{\SI{80}{kg} \cdot \SI{9.8}{\frac{m}{s^2}}}{(\frac{\SI{1e-2}{m}}{2})^2 \pi} = \SI{9.98}{MPa} \\
\end{align*} 

% source 1:
% https://www.schoolphysics.co.uk/age16-19/Mechanics/Statics/text/Pressure_/index.html
While this is in itself a spectacular realization that is often used in
introduction physics courses (source 2) to demonstrate the rather extreme
pressure under a stiletto heel (greater than the foot of an elephant) (how many
Atmos) this serves as a reasonible upperbound for human executed pressure. With
a full sheet area of $\sim\SI{21e3}{Å^2}$ we can achieve a similar pressure of
$\sim \SI{10}{MPA}$ with a normal force 
\begin{align*}
  F_N = \SI{10}{MPa} \cdot \SI{21e-17}{m^2} = \SI{2.10}{nN}  
\end{align*}

Of course this pressure might be insufficient for various industrial purposes,
but with no specific procedure in mind this serves as a decent reference point.
Notice that if we consider a human foot with ares $\SI{113}{cm^2}$ the pressure
drops to a mere $\SI{70}{kPa}$ corresponding to $\sim \SI{0.01}{nN}$.

% source 3: foot area ≈ 113 cm^2:
% https://www.footbionics.com/Patients/Foot+Facts.html source 4:
% https://hypertextbook.com/facts/2003/JackGreen.shtml


\subsection{Creating sheets}

We are going to create a 2D sheet graphene sheet. 

\subsubsection{Graphene}
% (https://community.wvu.edu/~miholcomb/graphene.pdf)
% https://www.physics-in-a-nutshell.com/article/4/lattice-basis-and-crystal

Graphene is a single layer of carbon atom, graphite is the bulk, arranged in a
hexagonallattice structure. We can describe the 2D crystal structure in terms of
its primitive lattice vector and a basis. That is we populate each lattice site
by the given basis and translate it to fill the whole plane by any linear
combination of the lattice vectors
\begin{align*}
  \vec{T}_{mn} = m\vec{a_1} + n\vec{a_2}, \qquad m,n \in \mathbb{N}.
\end{align*}
For graphene we have the primitive lattice vectors 
\begin{align*}
  \vec{a_1} = a \left(\frac{\sqrt{3}}{2}, -\frac{1}{2}\right), \qquad \vec{a_2} = a \left(\frac{\sqrt{3}}{2}, \frac{1}{2}\right), \qquad |\vec{a_1}| = |\vec{a_2} = 2.46 \ \text{Å}.
\end{align*}
Notice that we deliberately excluded the third coordinate as we only consider a
single graphene layer on not the bulk graphite consisting of multiple layers
stacked on top of each other. The basis is 
\begin{align*}
  \Big\{\Big(0,0\Big), \frac{a}{2}\Big(\frac{1}{\sqrt{3}}, 1 \Big) \Big\}
\end{align*}
It turns out that the spacing between atoms is equal for all paris with an
interatomic distance 
\begin{align*}
  \left|\frac{a}{2}\Big(\frac{1}{\sqrt{3}}, 1 \Big)\right| \approx 1.42 \ \text{Å}.
\end{align*}


\begin{figure}[H]
  \centering
  \includegraphics[width=0.3\linewidth]{figures/crystal.png}
  \caption{Graphene crystal structure with basis.}
  \label{fig:graphene_crystal}
\end{figure}



\subsubsection{Indexing}

In order to define the cut patterns applied to the graphene sheet we must define
an indexing system. We must ensure that this gives an unique description of the
atoms as we eventually want to pass a binary matrix, containg 0 for removed atom
and 1 for present atom, that uniquely describes the sheet. We do this by letting
the x-coordinate point to zigzag chains and the y-coordinate to the position
along that chain. This is illustrated in figure \ref{fig:atom_indexing}. Other
solutions might naturally invole the lattice vectors, but as these only can be
used to translate to similar basis atoms a unfortunate duality is introduced as
ones need to include the basis atom of choice into the indexing system. With the
current system we notice that locallity is somewhat preserved. That is, atom
$(i, j)$ is in the proximity of $\{(i+1, j), (i-1, j), (i, j+1), (i, j-1)\}$,
but only three of them is categorized as nearest neighbours due to the hexgonal
structure of the lattice. While $(i, j\pm 1)$ is always nearest neighbours the
neighbour in the x-direction flip sides with incrementing y-coordinate. That is
the nearest neighbours (NN) is decided as
\begin{align*}
  j \ \text{is even} &\rightarrow \text{NN} = \{(i+1, j), (i, j+1), (i, j-1)\}, \\
  j \ \text{is odd} &\rightarrow \text{NN} = \{(i-1, j), (i, j+1), (i, j-1)\}.
\end{align*}

\begin{figure}[H]
  \centering
  \includegraphics[width=0.7\linewidth]{figures/atom_indexing.pdf}
  \caption{Graphene atom indexing}
  \label{fig:atom_indexing}
\end{figure}

\subsubsection{Removing atoms}

As a mean to ease the formulation of cut patterns we introduce pseudo center
element in each gap of the hexagonal honeycombs, see figure
\ref{fig:center_indexing}. 

\begin{figure}[H]
  \centering
  \includegraphics[width=0.7\linewidth]{figures/center_indexing.pdf}
  \caption{Graphene center indexing}
  \label{fig:center_indexing}
\end{figure}




Similar to the case of the indexing for the carbon atoms themself the nearest
neighbour center elements alternate with position, this time along the
x-coordinate. Each center element has six nearest neighbours, in clock wise
direction we can denote them: ``up'', ``upper right'', ``lower right'',
``down'', ``lower left'', ``upper left''. The ``up'' and ``down'' is always
accesed as $(i,j\pm 1)$, but for even $i$ the $(i+1,j)$ index corresponds to the
``lower right'' neighbour while for odd $i$ this corresponds to the ``upper
right'' neighbour. This shifting applies for all left or right neighbours and
the full neighbour list is illustrated in figure \ref{fig:center_directions}. 


\begin{figure}[H]
  \centering
  \includegraphics[width=0.7\linewidth]{figures/center_directions.pdf}
  \caption{Graphene center elements directions}
  \label{fig:center_directions}
\end{figure}


We define a cut pattern by connecting center elements into connected paths. As
we walk element to element we remove atoms according to one of two rules 
\begin{enumerate}
  \item Remove intersection atoms: We remove the pair of atoms placed directly
  in the path we are walking. That is, when jumnping to the ``up'' center
  element we remove the two upper atoms located in the local hexagon of atoms.
  This method is sensitive to the order of the center elements in the path. 
  \item Remove all surrounding atoms: We simply remove all atoms in the local
  hexagon surrounding each center element. This method is indepdent of the
  ordering of center elements in the path.
\end{enumerate}

We notice that removing atoms using either of these rules will not garuantee an
unique cut pattern. Rule 1 is the more sensitive to paths but we realize that,
for an even $i$, we will remove the same five atoms following either of the
following paths.
\begin{align*}
  (i, j) &\rightarrow \underbrace{(i+1,j+1)}_{\text{upper right}} \rightarrow \underbrace{(i, j+1)}_{\text{up}} \rightarrow \underbrace{(i+1, j+2)}_{\text{upperright + up}} \rightarrow \underbrace{(i+1, j+1)}_{\text{upper right}} \\
  (i, j) &\rightarrow \underbrace{(i+1,j+1)}_{\text{upper right}} \rightarrow \underbrace{(i+1, j+2)}_{\text{upperright + up}} \rightarrow \underbrace{(i, j+1)}_{\text{up}}
\end{align*}

For rule 2 it is even more abovious that different paths can result in the same
atoms being removed. This is the reason that we needed to define and indexing
system for the atom position itself even though that all cuts generated manually
will use the center element path as reference. \\

Illustrate some delete path?




\subsection{Pull blocks}

\subsection{Kirigami inspired cut out patterns}
\subsubsection{Pop-up pattern}
\subsubsection{Honeycomb}
\subsubsection{Random walk}







\section{Fourier Transform (light)}
% https://www.brown.edu/research/labs/mittleman/sites/brown.edu.research.labs.mittleman/files/uploads/lecture21_0.pdf
% https://mathworld.wolfram.com/FourierTransform.html

% https://lpsa.swarthmore.edu/Fourier/Xforms/FXformIntro.html
\textbf{Find out where to put this if nessecary}. \\

Fourier transform is a technique where we transform a function $f(t)$ of time to
a function $F(k)$ of frequency. The Forward Fourier Transform is done as
\begin{align*}
  F(k) = \int_{-\infty}^\infty f(t) e^{-2\pi ikx} dx
\end{align*}

For any complex function $F(k)$ we can decompose it into magnitude $A(k)$ and
phase $\phi(k)$
\begin{align*}
  F(k) = A(k) e^{i \phi(k)}
\end{align*}

Hence when performing a Forward Fourier transform on a time series we can
determine the amplitude and phase as a function of freqeuncy as 
\begin{align*}
  A(k) = |F(k)|^2, \qquad \phi(k) = \Im{\ln{F(k)}}
\end{align*}





\subsection{Real life experimental procedures}
From Introduction to Tribology, Second Edition, p. 526: \par The surface force
apparatus (SFA), the scanning tunneling microscopes (STM), and atomic force and
friction force microscopes (AFM and FFM) are widely used in nanotribological and
nanomechanics studies.



\begin{itemize}
  \item Real life procedures to mimic in computation, for instance Atomic Force
  Microscoopy (AFM) for friction measurements.
  \item Available technology for test of my findings if successful
  (possibilities for making the nano machine) 
\end{itemize}


\section{Machine Learning (ML)}
\begin{itemize}
  \item Feed forward fully connected
  \item CNN
  \item GAN (encoder + decoder)
  \item Genetic algorithm
  \item Using machine learning for inverse designs partly eliminate the black
  box problem. When a design is produced we can test it, and if it works we not
  rely on machine learning connections to verify it's relevance. 
  \item However, using explanaitons techniques such as maybe t-SNE, Deep dream,
  LRP, Shapley values and linearizations, we can try to understand why the AI
  chose as it did. This can lead to an increased understanding of each design
  feature. Again this is not dependent on the complex network of the network as
  this can be tested and veriied independently of the network. 
\end{itemize}

\subsection{Feed forward network / Neural networks}
\subsection{CNN for image recognition}
\subsection{GAN (encoder + deoder)}
\subsection{Inverse desing using machine learning}
\subsection{Prediction explanation}
\subsubsection{Shapley}
\subsubsection{Lineariations}
\subsubsection{LRP}
\subsubsection{t-SNE}



