\chapter{Dataset study}

\section{Generating data}

Present the configuration and variable choices for the generated dataset. Perhaps include appendix with all the configurations shown in a grid


% \section{Rupture}
% Better word might be detached and nondetached as used by Hanakate in Accelerated search article.

\section{Data analysis}
A summary of the data points is given in table \ref{tab:dataset_summary}

\begin{table}[H]
  \begin{center}
  \caption{Summary of the number of generated data points used for machine learning. Due to random stretch values not all submitted simulations makes it through the framework to become a data point (\hl{explain this somewhere})  Notice that the Tetrahedon (7, 5, 2) and Honeycomb (2, 2, 1, 5) from the pilot study is rerun as a part fo the Tetrahedon and Honeycomb dataset seperately. In the latter the reference point for the pattern is randomized and thus theese configurations is included twice. This is the reason for the total configuration count summing up to two less than otherwise expected. }
  \label{tab:dataset_summary}
  \begin{tabular}{ | c | c | c | c |} \hline
  \textbf{Type} & \textbf{Configurations} & \textbf{Submitted data points} & \textbf{Final data points} \\ \hline
  Pilot study & 3 & 270 & 261    \\ \hline
  Tetrahedon & 68 & 3060 & 3015  \\ \hline
  Honeycomb & 45 & 2025 & 1983  \\ \hline
  Random walk & 100 & 4500 & 4401 \\ \hline \hline
  Total & 214 (216) & 9855 & 9660 \\ \hline
  \end{tabular}
  \end{center}
\end{table}

% Reason for some missing sims is that if restart file is set to be produced close to rupture point
% it might rupture a bit before and not submit the restart file. 

In order to gain insight into the correlations between variables associated to the simulations we calculate the correlations coefficients between all variable combinations. More specific, we are going to calculate the Pearson product-moment correlation coefficient (PPMCC) for which is defined, between data set $X$ and $Y$, as
\begin{align*}
  \mathrm{corr}(X,Y) = \frac{\mathrm{Cov}(X,Y)}{\sigma_X \sigma_Y} = \frac{\langle (X - \mu_X)(Y - \mu_Y)\rangle}{\sigma_X \sigma_Y} \ \in [-1, 1]
\end{align*}
where $\mathrm{Cov}(X,Y)$ is the covariance, $\mu$ the mean value and $\sigma$ the standard deviation. The correlation coefficients ranges from perfect negative correlation $(-1)$ through no correlation $(0)$ to a perfect positive correlation $(1)$. The correlation coefficients is shown in figure \ref{fig:corrcoef_matrix}

\begin{figure}[H]
  \centering
  \includegraphics[width=\linewidth]{figures/ML/corrcoef_matrix.pdf}
  \caption{Pearson product-moment correlation coefficients for the full datset (see table \ref{tab:dataset_summary}).}
  \label{fig:corrcoef_matrix}
\end{figure}

From figure \ref{fig:corrcoef_matrix} we especially notice that the mean
friction force $\langle F_{\parallel} \rangle$ has a signifciant positively
correlation with stretch $(0.77)$ and porosity $(0.60)$ (void fraction).
However, the relative stretch, which is scaled by the rupture stretch, has a
weaker correlation of only 0.25 which indicates that it is the absolute stretch
value that has the most significant impact on the friction force increase during
stretching. This is further supported by the fact that the mean friction and the
rupture stretch is also strongly positively correlated $(0.78)$. From figure
\ref{fig:corrcoef_matrix} we also observe that the contact bond count is
negatively correlated with the mean friction $(-0.67)$ and the stretch value
$(-0.74)$ which is consistent with the trend observed in the pilot study (figure \ref{fig:multi_stretch_contact} and \ref{fig:multi_stretch_mean_fric}) of the
contact decreasing with increasing stretch and mean friction. However, we must
take note that the correlation coefficients is a measure of the strength and slope of a
forced linear fit on the data. We clearly observed a non-linear relationship between stretch and mean friction for the tetrahedron and honeycomb pattern used in the pilot study (figure \ref{fig:multi_stretch_mean_fric}) where the relationship was partwise characterized by a postive correlation for some stretch ranges and partwise negative correlation for other stretch ranges. Hence, interesting strong regime-specific correlations might not be accurately highlighted by the correlation coefficients shown in figure \ref{fig:corrcoef_matrix}.

In figure \ref{fig:corr_vis} we have visualized the data (excluding the pilot study) for chosen pairs of variables on the axes. In addition to a visual confirmation of how the given correlations look in a 2D plot we also get a feeling for the coverage in various areas of the parameter space that we are eventually going to feed the neural network. The honeycomb pattern is spanning a significant larger range of stretch, contact and mean friction makes the data rather biased towards the Honeycomb pattern in thoose areas. 
% Judging form the combinned information of the pilot study and the data distribution shown in figure \ref{fig:corr_vis} it would not be surprising if the machine learning were to learn that the honeycomb pattern is superiour for 

\begin{figure}[H]
  \centering
  \begin{subfigure}[t]{0.49\textwidth}
      \centering
      \includegraphics[width=\textwidth]{figures/ML/corr_stretch_Ff.pdf}
      \caption{}
      % \label{fig:}
  \end{subfigure}
  \hfill
  \begin{subfigure}[t]{0.49\textwidth}
      \centering
      \includegraphics[width=\textwidth]{figures/ML/corr_stretch_contact.pdf}
      \caption{}
      % \label{fig:}
  \end{subfigure}
  \hfill
  \begin{subfigure}[t]{0.49\textwidth}
      \centering
      \includegraphics[width=\textwidth]{figures/ML/corr_contact_Ff.pdf}
      \caption{}
      % \label{fig:}
  \end{subfigure}
  \hfill
  \begin{subfigure}[t]{0.49\textwidth}
      \centering
      \includegraphics[width=\textwidth]{figures/ML/corr_porosity_Ff.pdf}
      \caption{}
      % \label{fig:}
  \end{subfigure}
  \hfill
     \caption{Scatter plot of the data sets Tetrahedron, Honeycomb and Random Walk (excluding the pilot study) for various variable combinations in order to visualize some chosen  correlations of interest and distributions in the data}
     \label{fig:corr_vis}
\end{figure}


\section{Properties of interest / Stretch profiles}
\hl{Define somewhere that we will look at low friction, high friction and the biggest (forward) drop in friction corresponding to a significant negative friction coefficient}.



\begin{table}[H]
  \begin{center}
  \caption{Interesting properties}
  \label{tab:}
  \begin{tabular}{| c | c | c | c|} \hline
  \textbf{Tetrahedron} & Configuration & Stretch & Value [nN]  \\ \hline
  Min $F_{\text{fric}}$ & $(3,9,4)$ &  0.0296 & 0.0067 \\ \hline
  Max & $(5,3,1)$ & 0.1391 & 1.5875 \\ \hline
  Max $\Delta F_{\text{fric}}$  & $(5, 3, 1)$ & $[0.0239, 0.1391]$ & 1.5529 \\ \hline
  Max drop & $(5,3,1)$ & $[0.1391, 0.1999]$ & 0.8841 \\ \hline
  \multicolumn{4}{c}{} \\ \hline
  % \textbf{Tetrahedron} & \multicolumn{3}{c|}{} \\ \hline
  \textbf{Honeycomb} & Configuration & Stretch & Value [nN]  \\ \hline
  Min $F_{\text{fric}}$ & $(2, 5, 1, 1)$ &  0.0267 & 0.0177 \\ \hline
  Max & $(2, 1, 1, 1)$ & 1.0654 & 2.8903 \\ \hline
  Max $\Delta F_{\text{fric}}$  & $(2, 1, 5, 3)$ & $[0.0856, 1.4760]$ & 2.0234 \\ \hline
  Max drop & $(2, 3, 3, 3)$ & $[0.5410, 1.0100]$ & 1.2785 \\ \hline
  \multicolumn{4}{c}{} \\ \hline
  \textbf{Random walk} & Configuration & Stretch & Value [nN]  \\ \hline
  Min $F_{\text{fric}}$ & 12 &  0.0562 & 0.0024\\ \hline
  Max & 96 & 0.2375 & 0.5758 \\ \hline
  Max $\Delta F_{\text{fric}}$  & 96 & $[0.0364, 0.2375]$ & 0.5448 \\ \hline
  Max drop & 01 & $[0.0592, 0.1127]$ & 0.1818 \\ \hline
\end{tabular}
\end{center}
\end{table}


% Popup
% ['(3, 9, 4)', 0.0296407442523106, 0.006738434728040425]
% ['(5, 3, 1)', 0.139120019152679, 1.5874991413277917]
% ['(5, 3, 1)', 0.0238700191526787, 0.139120019152679, 1.5529155085058322]
% ['(5, 3, 1)', 0.139120019152679, 0.199920019152683, 0.8840614643066859]

% Honeycomb
% ['(2, 5, 1, 1)', 0.0267215709876031, 0.01771035812444661]
% ['(2, 1, 1, 1)', 1.06536290170726, 2.8903313732271183]
% ['(2, 1, 5, 3)', 0.085589883539189, 1.47601988353919, 2.023377918411005]
% ['(2, 3, 3, 3)', 0.541004661720013, 1.01001466172002, 1.278541503443495]

% RW
% ['12', 0.0562087686350666, 0.002350782025058632]
% ['96', 0.237523191134115, 0.5757864994802119]
% ['96', 0.0363631911341175, 0.237523191134115, 0.5447910475168634]
% ['01', 0.0591598822685843, 0.112739882268582, 0.18175926264779968]


\begin{figure}[H]
  \centering
  \begin{subfigure}[t]{0.49\textwidth}
      \centering
      \includegraphics[width=\textwidth]{figures/stretch_profiles/PP_pop_27.pdf}
      \caption{}
  \end{subfigure}
  \hfill
  \begin{subfigure}[t]{0.49\textwidth}
      \centering
      \includegraphics[width=\textwidth]{figures/stretch_profiles/PP_pop_31.pdf}
      \caption{}
  \end{subfigure}
  \hfill
  \begin{subfigure}[t]{0.49\textwidth}
      \centering
      \includegraphics[width=\textwidth]{figures/stretch_profiles/PP_hon_6.pdf}
      \caption{}
  \end{subfigure}
  \hfill
  \begin{subfigure}[t]{0.49\textwidth}
      \centering
      \includegraphics[width=\textwidth]{figures/stretch_profiles/PP_hon_12.pdf}
      \caption{}
  \end{subfigure}
  \hfill
  \begin{subfigure}[t]{0.49\textwidth}
      \centering
      \includegraphics[width=\textwidth]{figures/stretch_profiles/PP_hon_28}
      \caption{}
  \end{subfigure}
  \hfill
  \begin{subfigure}[t]{0.49\textwidth}
      \centering
      \includegraphics[width=\textwidth]{figures/stretch_profiles/PP_hon_42.pdf}
      \caption{}
  \end{subfigure}
  \hfill
  \begin{subfigure}[t]{0.49\textwidth}
      \centering
      \includegraphics[width=\textwidth]{figures/stretch_profiles/PP_RW01.pdf}
      \caption{}
  \end{subfigure}
  \hfill
  \begin{subfigure}[t]{0.49\textwidth}
      \centering
      \includegraphics[width=\textwidth]{figures/stretch_profiles/PP_RW12.pdf}
      \caption{}
  \end{subfigure}
  \hfill
  \begin{subfigure}[t]{0.49\textwidth}
      \centering
      \includegraphics[width=\textwidth]{figures/stretch_profiles/PP_RW96.pdf}
      \caption{}
  \end{subfigure}
  \hfill
     \caption{}
     \label{fig:}
\end{figure}






The stretch profiles for all the configurations are shown in appendix \ref{sec:data_stretch_profiles}.


\section{Machine learning}

Staircase architecture tuning. 



 
\section{Accelerated Search}

Having a network model that can predict friction force for a given configuration are able to search for some desired properties. Low and high friction and maximal negative friction coefficeints


Here we pursue two different approaches for finding 
\begin{enumerate}
  \item Generate an enlarged dataset and run it through the ML model 
  \item Genetic algorithm
\end{enumerate}


\subsection{Markov-Chain Accelerated Genetic Algorithms}
% Following the article:
% Accelerated Genetic Algorithms with Markov Chains
% Guan Wang, Chen Chen, and K.Y. Szeto

\subsubsection{Talk about traditional method also?}

\subsubsection{Implementing for 1D chromosone (following article closely)}

We have the binary population matrix $A(t)$ at time (generation) $t$ consisting of $N$ rows denoting chromosones and with $L$ columns denoting the so-called locus (fixed position on a chromosome where a particular gene or genetic marker is located, wiki). We sort the matrix rowwise by the fitness of each chromosono evaluated by a fitness function $f$ such that $f_i(t) \le f_k(t)$ for $i \ge k$. We assume that there are a transistion probability between the current state $A(t)$ and the next state $A(t+1)$. We consider this transistion probability only to take into account mutation process (mutation only updating scheme). During each generation chromosones are sorted from most to least fitted. The chromosone at the $i$-th fitted place is assigned a row mutation probability $a_i(t)$ by some monotonic increasing function. This is taken to be 
\begin{align*}
  a_i(t) = 
  \begin{cases}
    (i-1)/N',& i-1 < N' \\
    1, &\text{else}
  \end{cases}
\end{align*}
for some limit $N'$ (refer to first part of article talking about this). We use $N' = N/2$. We also defines the survival probability $s_i = 1 - a_i$. In thus wau $a_i$ and $s_i$ decide together whether to mutate to the other state (flip binary) or to remain in the current state. We use $s_i$ as the statistical weight for the $i$-th chromosone given it a weight $w_i = s_i$.
\\
Now the column mutation. For each locus $j$ we define the count of 0's and 1's as $C_0(j)$ and $C_1(j)$ resepctively. These are normalized as
\begin{align*}
  n_0(j, t) = \frac{C_0(j)}{C_0(j) + C_1(j)}, \quad n_1(j, t) = \frac{C_1(j)}{C_0(j) + C_1(j)}.
\end{align*}
These are gathered into the vector $\vec{n}(j,t)=(n_0(j, t), n_1(j, t))$ which characterizes the state distribution of $j$-th locus. In order to direct the current population to a preferred state for locus $j$ we look at the highest weight of row $i$ for locus $j$ taking the value 0 and 1 respectively.
\begin{align*}
  C'_0(j) &= \max\{W_i | A_{ij} = 0; \ i = 1, \cdots, N\} \\
  C'_1(j) &= \max\{W_i | A_{ij} = 1; \ i = 1, \cdots, N\}
\end{align*}
which is normalized again
\begin{align*}
  n_0(j, t+1) = \frac{C'_0(j)}{C'_0(j) + C'_1(j)}, \quad n_1(j, t+1) = \frac{C'_1(j)}{C'_0(j) + C'_1(j)}.
\end{align*}
The vector $\vec{n}(j,t+1)=(n_0(j, t+1), n_1(j, t+1))$ then provides a direction for the population to evolve against. This characterizes the target state distribution of the locus $j$ among all the chromosones in the next generation. We have
\begin{align*}
  \begin{bmatrix}
    n_0(j, t+1) \\
    n_1(j, t+1)
  \end{bmatrix}
  = 
  \begin{bmatrix}
    P_{00}(j,t) \ P_{10}(j,t) \\
    P_{01}(j,t) \ P_{11}(j,t)
  \end{bmatrix}
  \begin{bmatrix}
    n_0(j, t) \\
    n_1(j, t)
  \end{bmatrix}
\end{align*}
Since the probability must sum to one for the rows in the P-matrix we have 
\begin{align*}
  P_{00}(j, t) = 1 - P_{01}(j, t), \quad P_{11}(j, t) = 1 - P_{10}(j, t)
\end{align*}
These conditions allow us to solve for the transition probability $P_{10}(j,t)$ in terms of the single variable $P_{00}{j,t}$.
\begin{align*}
  P_{10}(j,t) &= \frac{n_0(j, t+1) - P_{00}(j,t)n_0(j, t)}{n_1(j,t)} \\
  P_{01}(j,t) &= 1 - P_{00}(j,t) \\
  P_{11}(j,t) &= 1 - P_{10}(j,t)
\end{align*}
We just need to know $P_{00}(j,t)$. We start from $P_{00}(j, t = 0) = 0.5$ and then choose $P_{00}(j,t) = n_0(j,t)$




