% TO-DO: Look at friction force graph again and reconsider whether we have stick-slip. The small oscillations is perhaps noise, and since the big oscialltions go by smooth curves this is actually more likely to be smooth sliding. I guess looking at the period in connection with the lattice spacing show whether it is thoose curves we should look at or the smaller ones result from a lot of small slips...

% Note: A good word for the force curves is "Force traces" as used by \cite{Yalin_2011}.


\chapter{Pilot study}
Having defined our system, we carry out an initial study of the numerical approach. This includes an analysis of how to define and measure the frictional properties of interest, and an investigation of the main parameters governing the numerical solutions. From this point of view we decide on a suitable set of parameters for the remaining study. Additionally, we investigate the frictional behaviour under the variation of load and stretch for a selected set of configurations which serves as a baseline for later comparison. 


\section{Friction simulation parameters}
The \acrshort{MD} simulation is governed by a small set of parameters, some
which are related directly to the numerical aspects of the simulation and other
to the physical conditions in the simmulation. Thus, we differentiate between
the two main categories: 1) \textit{Physical}, parameters which alter the
physical conditions of the ``numerical experiment'' and are expected to effect
the frictional behaviour. 2) \textit{Numerical}, parameters which are related
more closely to the numerical procedure itself, expected to influence the
simulation kinetics, which should be chosen to ensure the most stable results.
For the purpose of creating the machine learning dataset most of these
parameters will be kept constant with only a subset of the physical parameters
being varied. The parameters are summarized in \cref{tab:param} with the grey
shaded area denoting the parameters which we will vary for the dataset. Due to
the great number of parameters it is unreasonable to make an exhaustive search
of all parameters, in order to choose the final settings. Instead, we take a
basis in the parameters used in similar studies \hl{SOURCES} and adjust them as
we carry out the analysis of the simulation results. Thus, we start at values
most representative for other similar simulations and adjust according to the
stability of the results and the computation time. Since we are going to
introduce a lot of complexity to the system, through the cut and stretch
deformation, we are less strict about meeting other paramter criterias (does
this make sense). The final parameter choice is shown in \cref{tab:param} which
we be the default values in the following study, i.e.\ when nothing is stated explicitly we will use these parameters. As we analyse the data we will provide further arguments for specific choices made.


% Thus, by adjusting a few selected parameters one by one we arrived at a final choice as shown in \cref{tab:param}. 

% We will present an analysis of the simulation results and its dependency of these selected variables in the following using the final choice as default values. Thus we should stress that the intention behind this choice is mainly to get a working simulation with stable results and not to optimize for any spefic properties yet. 

% \hl{Say a bit more about the articles used to set a starting reference for these parameters.}

% we proposed a set of parameters for which we investgated the dependencies of only selected ones. By adjusting the parameters according to the findings we landed on a final set as already shown in \cref{tab:param}


% In the following we present the results of the friction simulations in parallel to the procedure of investigating the choice of different parameters. 

% We aim to chose the parameters in order to accomodate a balance between generlizable and stable result which is simmutaneously a suting candidate as a proof of concept for the control of friction properties using kirigami inspired cuts. 



% Due to the great number of parameters, and corresponding range of reasonable numerical values they can take, it is ... to parameter search including all of these. Thus, we will to a great extent rely on a reverse engineering in order to establish a set of parameters for the \textit{physical} and \textit{measurement} categories along with numerical ranges for the $\textit{ML input}$ category which gives stable and promising results. By doing so we effectively narrow down the parameter regime for which the investigated frictional properties belong. We aim to chose the parameters in order to accomodate a balance between generlizable and stable result which is simmutaneously a suting candidate as a proof of concept for the control of friction properties using kirigami inspired cuts. 

% In the following we present the results of the friction simulations in parallel to the procedure of investigating the choice of different parameters. 

% In the following subsections (X to Y) we are going to present the friction simulation results in parallel to the presentation of the reasoning behind the parameter choices. For this we will refer to the default parameter choice showcased in \cref{tab:final_param} which is representative of the final parameter choices. 




% The sliding velocity is 20 m/s, which is comparable to the operating conditions in micromechanical systems (MEMS), but which is much larger than the typical velocity in scanning force microscopy (SFM) experiments. \cite{mo_friction_2009} Supplementary materials.


% We should try to set the physcis and measurement parameters in such a way that we reduce computation speed where it is doesn't infer with the frictional properties study.

\begin{table}[H]
  \begin{center}
  \caption{Parameters of the numerical \acrshort{MD} simulation for measuring friction. The values correspond to the final choice used for the dataset. The shaded area denote the parameters varied in the \acrshort{ML} dataset.}
  \label{tab:param}
  \begin{tabular}{ | c | C{2cm} | C{3.5cm} | X{8cm}|} \hline
    Category & Parameter & Value &  Description \\ \hline
    \multirow{13}{*}{Physical} & $T$ & \SI{300}{K} &  Temperature. \\ \hhline{~|-|-|-|}
    & $v_{\text{slide}}$ &\SI{20}{m/s} & Sliding speed for the sheet translation. \\ \hhline{~|-|-|-|}
    & $K$ & $\inf$ & Spring constant for the coupling between the virtuel atom and the sheet pull blocks. \\ \hhline{~|-|-|-|}
    & Scan direction & $(x,y) = (0,1)$ \linebreak (zigzag direction)  & The direction for which we translate the sheet. \\ \hhline{~|-|-|-|}   
    & \cellcolor{black!7} Sheet configuration & \cellcolor{black!7} Contiguous & \cellcolor{black!7} Binary mapping describing which atoms are removed (0) and which is still present (1) in the graphene sheet.  \\ \hhline{~|-|-|-|}
    & \cellcolor{black!7} Stretch amount & \cellcolor{black!7} 0\% - rupture & \cellcolor{black!7} The relative stretch of the sheet. \\ \hhline{~|-|-|-|}
    & \cellcolor{black!7} $F_N$ & \cellcolor{black!7} [0.1, 10] nN & \cellcolor{black!7} Applied normal force to the pull blocks. \\ \hline
    \multirow{8}{*}{Numerical} & $dt$ & \SI{1}{fs} &  Integration timestep. \\ \hhline{~|-|-|-|}
    & $t_R$ &  \SI{15}{ps} & Relaxtion time before strething. \\ \hhline{~|-|-|-|}
    & Pauses & \SI{5}{ps} & Relaxtion pauses after stretch, and during the normal load phase (before translating the sheet). \\ \hhline{~|-|-|-|}
    & Stretch Speed & \SI{0.01}{ps^{-1}} & The rate of stretching for the sheet. \\ \hhline{~|-|-|-|}
    & Slide distance & \SI{400}{Å} & How far to translate the sheet. \\ \hhline{~|-|-|-|}
    & Sheet size & $130.029 \times \SI{163.219}{\text{Å}}$ & Spatial 2D size of the sheet.  \\ \hhline{~|-|-|-|}
    & Pull block size & $2 \times 130.029 \times \SI{15.183}{\text{Å}}$ & Spatial 2D size of the pull blocks. \\ \hline
  \end{tabular}
  \end{center}
\end{table}



% Parameters expected to have a physical effect on the friction properties, which is kept fixed and thus not included in the machine learning input set. 

% Paramters influecing the simulation kinetics and being representative of the experimental procedure that we are mimicking. These parameters is chosen with the aim of getting stable parameters under small perturbations of the given parameter. 

% The remaining paramters serves as input variables for optimization process and is thus given as input variables for the machine learning (ML).




% Say someting about how these parameters is chosen. Reference to articles for which these was mirrored from. 





% We need to define some ranges for the ML input paramters. $F_N$, stretch ranges where it is not prone to ruptures. The configuration it self does not have clear rules but is also being regulated by the no rupture requirement. 

% Retardation effects due to the finiteness of the speed of sound are usually irrelevant in slow-speed experiments (v < 1 mm/s) \cite{Manini_2016}

% In macroscopic tribology experiments, sliding speeds often range in the 0.1 − 10 m/s region \cite{Manini_2016}

% By contrast, in nanoscale AFM experiments the tip usually advances at much lower speeds $\sim$ 1 $\mu$m/s: over a typical run it is possible to simulate a tiny ∼ 1 pm displacement, far too small to explore even a single atomic-scale event, let alone averaging over a steady state.\cite{Manini_2016}


% However, MD simulations can provide so much physical insight that they make sense even if carried out at much higher speeds than in real-life AFM or surface force apparatus (SFA) experiments: in practice, currently the sliding speeds of most atomistic tribology simulations are in the $\sim$ 1 m/s region.\cite{Manini_2016}


% Besides the limitations of system size and simulation times that are obvious and will be discussed later, there is another limitation concerning temperature, that is rarely mentioned. All classical frictional simulations, atomistic or otherwise, are only valid at sufficiently high temperature. They become in principle invalid at low temperatures where the mechanical degrees of freedom of solids progressively undergo ”quantum freezing”, and both mechanics and thermokinetics deviate from classical. \cite{Manini_2016}.




% \newpage
% Single friction simulation analysis
\section{Force traces}\label{sec:single_analysis}
We begin by assessing the force traces for a single friction simulation using the default parameters shown in \cref{tab:final_param} for a non-cut sheet with no stretch applied and a normal force of $\SI{1}{nN}$.


\subsection{Force oscillations}\label{sec:force_oscillations}
We evaluate the friction force as the force acting on the sheet from the
substrate. We consider initially the force componenet $F_{\parallel}$ parallel
to the drag direction as plotted in \cref{fig:drag_Ff}. We use a sample rate of
10 ps$^{-1}$ = 100 timesteps$^{-1}$ for which each sample is the mean value of
the preceding 100 timesteps. We observe immediately that the data carriers
oscillations on different time scales which matches our expectations for sliding
involving periodic surfaces. By applying a savgol filter to the data with a
polyorder of 5 and window length of 150 timesteps, corresponding to a sliding
distance of 3 Å or a time window of 15 ps, we can qualitatively point out at
least two different frequencies of osccilation. During the first 10 Å of sliding
in \cref{fig:drag_Ff_10} we see roughly three waves on the savgol filter
corresponding to a relative high frequency, while for the duraction of 100 Å of
sliding in \cref{fig:drag_Ff_100} the same savgol filter reveals a lower
frequency on top, creating the visual pattern of a wavepacket. The data does not
indicate clear signs of stick-slip behaviour as otherwise found in other
studies, e.g.\ by Zhu and Li \cite{zhu_study_2018} for graphene on gold, who saw
a more typical saw tooth shape in the force trace. Beside the different
substrate material, gold instead of silicon, they used a lower sliding speed of
\SI{10}{m/s} and soft spring of $\SI{10}{N/m}$. When rerunning the simmulation
using these values (\cref{fig:drag_Ff_10_K10_v10}) we found a different force
trace pattern showing signs of stick-sliop behaviour, but not still as evident
as we see lowering the sliding speed further down to \SI{1}{m/s} as shown in
\cref{fig:drag_Ff_10_K10_v1}. This result agrees with ... that the stick-slip
behaviour is suppresed for high sliding speed and stiff springs. However, the low sliding speed comes with a high computational cost which is the reason that we stick to \SI{20}{m/s} and a infinitely stiff spring constant to instead gain stable results in the domain of smooth sliding \hl{the spring constant is further discussed later right}.

% Take a look at the soft spring slow speed result for 100 Å as well
% Alternatively make a 3 x 2 plot showing the 10 and 100 Å figures for (K = 30, v = 1), (k = 10, v = 10), (k = 10, v = 1) or something similar.


\begin{figure}[H]
  \centering
  \begin{subfigure}[t]{0.49\textwidth}
      \centering
      \includegraphics[width=\textwidth]{figures/baseline/drag_Ff_10Å.pdf}
      \caption{$K = \inf$, $v = \SI{20}{m/s}$ (10 Å sliding).}
      \label{fig:drag_Ff_10}
  \end{subfigure}
  \hfill
  \begin{subfigure}[t]{0.49\textwidth}
      \centering
      \includegraphics[width=\textwidth]{figures/baseline/drag_Ff_100Å.pdf}
      \caption{$K = \inf$, $v = \SI{20}{m/s}$ (100 Å sliding).}
      \label{fig:drag_Ff_100}
    \end{subfigure}
    \hfill
    \begin{subfigure}[t]{0.49\textwidth}
      \centering
      \includegraphics[width=\textwidth]{figures/baseline/drag_Ff_10Å_K10_v10.pdf}
      \caption{$K = \SI{10}{N/m}$, $v = \SI{10}{m/s}$ (10 Å sliding).}
      \label{fig:drag_Ff_10_K10_v10}
    \end{subfigure}
    \hfill
    \begin{subfigure}[t]{0.49\textwidth}
      \centering
      \includegraphics[width=\textwidth]{figures/baseline/drag_Ff_10Å_K10_v1.pdf}
      \caption{$K = \SI{10}{N/m}$, $v = \SI{1}{m/s}$ (100 Å sliding).}
      \label{fig:drag_Ff_10_K10_v1}
  \end{subfigure}
  \hfill
     \caption{\hl{Update caption} Friction force $F_\parallel$ with respect to the drag direction between (full) sheet and substrate versus sliding distance. The sliding distance is measured by the constant movement of the virtual atom and not the COM of the sheet. However, we expect these measures to be fairly identical due the fact that the pull blocks is rigidly coupled to the virtual atom. The red line represents a savgol filter with window polyorder 5 and window length of 150 timesteps (corresponding to a sliding distance of 3 Å or a time window of 15 ps).}
     \label{fig:drag_Ff}
\end{figure}

By performing a Fourier Transform on the data (default parameters) we can quantify the leading frequencies observed in figure \cref{fig:drag_Ff_10} and \cref{fig:drag_Ff_100}. The Fourier transform is shown in \cref{fig:ft_a}, and by plotting the two most dominant frequencies $f_1 = 0.0074$ ps$^{-1}$ and $f_2 = 0.0079$ ps$^{-1}$ as $\sin{(2\pi f_1)} + \sin{(2\pi f_2)}$ we find a qualitatively convincing fit to the observed wavepacket shape as seen in \cref{fig:ft_b}. By using the trigonometric identity
\begin{align*}
\sin (a+b) &= \sin (a) \cos (b) + \cos (a) \sin (b), \\
\sin (a-b) &= \sin (a) \cos (b) - \cos (a) \sin (b),
\end{align*}
and decomposing the frequencies as $f_1 = a - b$, $f_2 = a + b$, we can rewrite the sine sum as the sinusoidal product
\begin{align*}
  \sin(2\pi f_1) + \sin(2\pi f_2) &= \sin\big(2\pi (a - b)\big) + \sin\big(2\pi (a + b)\big) \\
  &= \sin(2\pi a)\cos(2\pi b) + \cancel{\cos(2\pi a)\sin(2\pi b)} + \sin(2\pi a)\cos(2\pi b) - \cancel{\cos(2\pi a)\sin(2\pi b)} \\
  &= 2 \sin(2\pi a) \cos(2\pi b),
\end{align*} 

with 
\begin{align*}
  a = \frac{f_1 + f_2}{2} &= 0.0763 \pm \SI{0.0005}{ps^{-1}},& 
  b = \frac{f_2 - f_1}{2} &= 0.0028 \pm \SI{0.0005}{ps^{-1}},& \\
  &= 0.381 \pm \SI{0.003}{{\text{Å}}^{-1}},& 
  &= 0.014 \pm \SI{0.003}{{\text{Å}}^{-1}},& 
\end{align*}
where the latter frequency is denoted with respect to the sliding distance. This
makes us recognize the high osccilation frequency as $a$ and the low frequency
as $b$. The faster one has a period of $T_a = 2.62 \pm 0.02$ Å\footnote{The
uncertainty $\Delta y$ is calculated as $\Delta y = \left|\frac{\partial
y}{\partial x} \Delta x \right|$ for uncertainty $\Delta x$ and $y(x)$}. This
corresponds well with the magnitude of the lattice spacing and especialy that of
graphene at 2.46 Å as expected theoretically. The longer period $T_b = 71 \pm
15$ Å$^{-1}$ is not obviously explained. The build up in friction force is
reminiscent of a friction strengthening, but the periodic oscillation does not
really support this idea. Instead, we might attribute it to some kind of phonon
resosnance which could be a physical phenonama or simply a feature of our
\acrshort{MD} modelling. 
% However, we take note of the longest period $T_b = 71
% \pm 15$ Å$^{-1}$ which will be relevant for the evaluation of measurement
% uncertainty in section \cref{sec:def_dyn_and_stat}.

\begin{figure}[H]
  \centering
  \begin{subfigure}[t]{0.49\textwidth}
    \centering
    \includegraphics[width=\textwidth]{figures/baseline/ft_zoom.pdf}
    \caption{FT result shown for a reduced frequency range.}
    \label{fig:ft_a}
  \end{subfigure}
  \hfill
  \begin{subfigure}[t]{0.49\textwidth}
      \centering
      \includegraphics[width=\textwidth]{figures/baseline/ft_sine.pdf}
      \caption{Two most dominant frequencies applied to the data from \cref{fig:drag_Ff_100}}
      \label{fig:ft_b}
  \end{subfigure}
  \caption{Fourier transform analysis of the full friction force data (all 400 Å sliding distance) shown in \cref{fig:drag_Ff}. (a) shows the two most dominant frequency peaks. Note that no significant peaks was found in a higher frequency than included here. (b) shows a comparison between the raw data and the wavefunction corresponding to the two peaks in figure (a).}
  \label{fig:ft}
\end{figure}


\subsection{Decompositions}
In the previous analysis we have looked only at the friction force for the full
sheet, including the rigid pull blocks, and with
respect to the drag direction. We found this way of measuring the friction force to be the most reliable, but we will present the underlying arguments for this choice in the following.

Due to the fact that we are only applying cuts to the inner sheet, and not the
pull blocks, it might seem more natural to only consider the friction on that
part. If the desired frictional properties can be achieved by altering the inner
sheet one can argue that any opposing effects from the pull blocks can be
mitigated by scaling the relative size between the inner sheet and the pull
blocks. However, when looking at the force traces decomposed with respect to the
inner sheet and pull block regions respectivly (see \cref{fig:decomp_group}), we
observe that the friction force arrising from those parts are seemingly
antisymmetric. That is, the distribution of the fricitonal pull from the
substrate on the sheet is oscillating between the inner sheet and the pull
block. Keeping in mind that normal force is only applied to the pull blocks we
might take this as an integrated feature of the system which does not nessecary
dissapear when changing the spatial ratio between inner sheet and pull block.
Any interesting friciton properties might depend on this internal distribution
of forces. Hence, we hedge our bets and use the full sheet friction force as a
hollistic approach to avoid excluding relevant data in the measurements.

Similar we might question the decision of
only considering the frictional force projected onto the sliding direction as
we are then neglecting the ``side shift'' induced during the slide phase. In \cref{fig:decomp_direc} we see the decomposition in terms of force components parallel $F_{\parallel}$ and perpendicular $F_{\perp}$ to the sliding direction respectively. We notice that the most dominant trend appears for the parallel component. If we want to include the perpendicular component as well we would have to evaluate friction as the length of the force vector instead. This would remove the sign of the force direction and skew the whole friction force above as we clearly see both negative and positive contributions in the force trace. One option to accommodate this issue is by using the vector length for the magnitude but keeping the sign from the parallel component. However, we omit such compromises as this might make the measurement interpretation unessercary complex, and we use only the parallel component going forward. 

\begin{figure}[H]
  \centering
  \begin{subfigure}[t]{0.49\textwidth}
    \centering
    \includegraphics[width=\textwidth]{figures/baseline/decomp_group.pdf}
    \caption{Decomposition into group inner sheet (sheet) and pull blocks (PB).}
    \label{fig:decomp_group}
  \end{subfigure}
  \hfill
  \begin{subfigure}[t]{0.49\textwidth}
      \centering
      \includegraphics[width=\textwidth]{figures/baseline/decomp_direc.pdf}
      \caption{Decomposition into parallel ($F_{\parallel}$) and perpendicular ($F_{\perp})$ to drag sliding direction.}
      \label{fig:decomp_direc}
  \end{subfigure}
  \caption{Friction force decomposition on the data shown in \cref{fig:drag_Ff} with applied savgol filters similar to that of \cref{fig:drag_Ff_100} with window polyorder 5 and window length of 150 timesteps (corresponding to a sliding distance of 3 Å or a time window of 15 ps).}
  \label{fig:decomp}
\end{figure}


\subsection{Center of mass path}
\hl{Go through here again when done with the stick-slip motion analysis.}

From the previous observations of the friction force time series we see evidence
of a stick-slip behvaiour. Specially, we see in \cref{fig:decomp_direc}
that this might be the case both parallel and perpendicular to the sliding
direction. By looking at the $x,y$-position for the sheet center of mass (COM) \hl{change to CM}
we observe the stick-slip motion manifested as a variation in COM speed combinned
with a side to side motion as shown in figure $\cref{fig:COM_path_K0}$. In an attempt to increase the magnitude of the slips we evaluate a similar simulation with spring contant $K = \SI{30}{N/m}$ (see figure
\cref{fig:COM_path_K30}) in contrast to that of an infinte spring constant. While
the maximum slip speed stays within a similar order of magnitude the slip length
in the sliding direction is increased along with the side to side motion. Note
that the axis scale is different between \cref{fig:COM_path_K0} and
\cref{fig:COM_path_K0}. However, in both cases we observe that the side to side
motion is associated with a low speed, meaning that is more reminiscent of a ``slow'' creep alignment with the substrate than a slip. 


\begin{figure}[H]
  \centering
  \begin{subfigure}[t]{0.85\textwidth}
    \centering
    \includegraphics[width=\textwidth]{figures/baseline/COM_path_K0.pdf}
    \caption{$K=\inf$ (Fix move)}
    \label{fig:COM_path_K0}
  \end{subfigure}
  \hfill
  \begin{subfigure}[t]{0.85\textwidth}
      \centering
      \includegraphics[width=\textwidth]{figures/baseline/COM_path_K30.pdf}
      \caption{$\SI{30}{N/m}$}
      \label{fig:COM_path_K30}
  \end{subfigure}
  \caption{Center of mass posotion relative to the start of the sliding phase in terms of the direction parallel to the sliding direction $\Delta COM_{\parallel}$ and the axis perpendicular to the sliding direction $\Delta COM_{\perp}$. The colorbar denotes the absolute speed of the COM.}
  \label{fig:COM_path}
\end{figure}


\section{Defining metrics for kinetic and static friction}\label{sec:def_dyn_and_stat}

In order to evaluate the frictional properties of the sheet we aim to reduce the force trace results adressed in section \cref{sec:single_analysis} into single metrics describing the kinetic and static friciton. 
% The common choice is to use the mean and max values of the time series. 

\subsection{Kinetic friction} 
For the kinetic friction measurement we take the mean value of the latter half
of the dataset to ensure that we are sampling from a stable system. For a full
sliding simulation of 400 Å we thus base our mean value on the latter 200 Å (1000 ps) of
sliding. In \cref{fig:runmean} we have shown the force trace for the
first 10 Å of sliding together with a 50\% running mean window with the value being plotted at the end of the window. This is merely done to illustrate the
samplig procedure, and we see that for such a short sliding period the final mean estimate (marked with a dot) takes a negative value due to the specific cut-off of the few oscialltion captured here. Nonetheless, one approach to quanity the
uncertainy of the final mean estimate is to consider the variation of the
running mean preceeding the final mean value. The more the running mean
fluctuates the more uncertainty associated with the final estimate. However, only the
running mean ``close'' to the ending should be considered, since the first part
will rely on data from the beginning of the simulation. From the Fourier analyse
in section \cref{sec:force_oscillations} we found the longest significant
oscillation period to be $\sim 71$ Å$^{-1}$ corresponding to $\sim 35 \%$ of the
running mean window which have a window length of 200 Å when including all the data. Hence, we use the standard deviation of the final 35\% of the running mean to approximate the
uncertainty of the final mean value. We consider the standard deviation as an estimate of the absolute error and calculate the relative error by a division of the final mean value. In \cref{fig:runstd} we showcase a running relative error based on the standard deviation, with a window of length $35 \%$ the mena window, in a continuation of the illustrative case of a 10 Å sliding from \cref{fig:runmean}. In this case we get a high relative error of $\sim 257\%$ which alligns well with the short sampling period and the fact that this lead to the mean value taking an unphysical negative value. 

\begin{figure}[H]
  \centering
  \begin{subfigure}[t]{0.49\textwidth}
    \centering
    \includegraphics[width=\textwidth]{figures/baseline/Ff_runmean.pdf}
    \caption{Running mean with window length $\SI{5}{\text{Å}}$ (50\% the data length).}
    \label{fig:runmean}
  \end{subfigure}
  \hfill
  \begin{subfigure}[t]{0.49\textwidth}
      \centering
      \includegraphics[width=\textwidth]{figures/baseline/Ff_runstd.pdf}
      \caption{Running std with window length $\SI{1.75}{\text{Å}}$ (35\% the mean window length.)}
      \label{fig:runstd}
  \end{subfigure}
  \caption{Running mean and running standard deviation (std) on the friction force data from a $\SI{10}{{\text{Å}}}$ of sliding simulation. The running mean window is 50\% the data length while the running std window is 35\% the running mean window length.}
  \label{fig:running}
\end{figure}


When including the full dataset of 400 Å of sliding, such that the std window actually matches with the longest period of oscillations expected from the data, we get a final relative error of $\sim 12 \%$ as shown in fig \cref{fig:runstd_long}. This is arguable just at the limit for an acceptable error, but as we shall see later on in \cref{sec:load_and_stretch} this high relative error is mainly associated with the cases of low friction. When investigating different configurations under variation of load and stretch we see a considerable lower relative error as the mean friction evaluates to higher values. One interpration of this finding is simply that the oscialltiosn in the running mean not strongly dependent of the magnitude of the friction. In that case, the relative error will spike for the low friction cases, and the absolute error might be there more reliable measure.


\begin{figure}[H]
  \centering
  \includegraphics[width=0.6\linewidth]{figures/baseline/Ff_runstd_long.pdf}
  \caption{Running standard deviation (std) for a full \SI{400}{{\text{Å}}} sliding simulation. The running std window is 70 Å (35\% the running mean window of 50\% the data length).}
  \label{fig:runstd_long}
\end{figure}


\subsection{Static friction} 
The maximum value is probably the most obvious choice for adressing the static friction, even though the definition of the static friction is a bit vague. When
considering the force traces in \cref{fig:drag_Ff} we observe that the force oscillations increase in magnitude toward a global peak at $\sim \SI{20}{\text{Å}}$. Thus, one could be inclined to identify this peak as the
maximum value aassociated with friction force. However, as we have already clarified, this steady increase in friction is a part of a slower oscillation which repeats by a period of $\sim 71$ Å$^{-1}$. By plotting the top three max values recorded during a full 400 Å simulation, for 30 logaritmicly spaced load values in the range $[0.1, 100]$ nN, we observe that the global max in fact rarely fall within this first oscillation period as shown in \cref{fig:max_dist}. Only 2/30 global values and 4/90 top three values 
can be associated to the start of the sliding by this definition. Thus, this
result suggest that we can not really measure a significant static friction response in the sense of an initial increase in friction due to a depinning of the sheet from the static state. A possible approach to increase the likelihood of seeing a significant static friction response is by extending the relaxtion period, as static friction is theorized to increase logaritmic with time, and to increase the sliding force slowly and through a soft spring. As an attempt to test parts of this hypothesis we ran a series of simulations with varying spring constant, $K\in [?, 200]$ nN (and $\inf$), but keeping the relaxtion time and sliding speed at the default value. The result is shown in \cref{fig:max_vs_K}, but we did not find any signs of the maximum value falling within the first oscillation period for low spring constants. Due to the ambiguousness in the assesment of the static friction we will mainly consern ourselves with the kinetic friction in this study.



% In most numerical studies \cite{bonelli_atomistic_2009, zhu_study_2018} they define static friction as the max peak or top 5\% quantile throughout the simulation and do not consern themselves with a requirement for an initial increasement toward this max. Thus, we interpret this as a measure of the force oscillation magnitude which is highly related to the stick-slip behaviour. 



% that the max value cannot be used as a reliable measure for the
% static friction either due to its lack of presence or due to the simulation
% setup procedure. For a more typycal evaluation of the static friction force one
% would increase force slowly until the first slip significant slip is recorded (a
% series of precursors is expected to precede this). In our simulations we drag
% the sheet relatively fast in a rigid manner which might be the reason for the lacking the static friction. Bonelli et al.\ \cite{bonelli_atomistic_2009} reported that the stick-slip behaviour was only presented when using a relatively soft spring. Thus, by changing the spring constant we investigate possibility to observe a static friction (\hl{I kind of interchanged stick-slip and static friction int his argument, but I still think it can be used to argue for doing the test...}) response within the framework of our simulation procedure as shown in
% \cref{fig:max_vs_K}. However, the results do not indicate any implications
% that a recognizable domain exist for which the static friction response would be
% reliable. Hence, we will base the final assesment on frictional properties purely on the kinetic friction force. 


\begin{figure}[H]
  \centering
  \begin{minipage}{.47\textwidth}
    \centering
    \includegraphics[width=\linewidth]{figures/baseline/max_dist.pdf}
    \captionof{figure}{Distribution of top three max friction force peaks for 30 uniformly sampled normal forces $F_N \in [0.1, 10]$ nN. The dotted line and the grey area marks the slowest significant oscialltion period found in the data and thus marking a dividing line for whether a peak falls within the ``beginning'' of the sliding simulation.}
    \label{fig:max_dist}
  \end{minipage}%
  \hspace{0.2cm}
  \begin{minipage}{.48\textwidth}
    \centering
    \includegraphics[width=\linewidth]{figures/baseline/max_vs_K}
    \captionof{figure}{Sliding displacement for the max friction peak to appear as a function of spring constant. \hl{Fixmove is tmp mapped to K = 200 here without any discontinuous lines.}}
    \label{fig:max_vs_K}
  \end{minipage}
  \end{figure}




% We investigate the placement of the max values, i.e. the sliding distance length for which we measure the max friction force. We show the placement of the top three max values for different simulatiosn with varying normal force in \cref{fig:max_dist}. We observe immediately that only a few top three max values is measured within a full slow period of $\sim$ 71 Å. In fact many max values is measured just before the end of the simulation. This indicates that the naive approach of using the overall max value to describe the static friction coefficient might be a to naive approach. Another approach is to use the max value within a single period, but we do not really know if this period will be similar for alle cut patterns and thus this might be limiting. 



% Look into static friction when having a spring connected to the drag force with
% rather low spring constant. Maybe compare to critical sitffness in FK model.
% Some rough calculations follow here (make a note about this being a very naive
% approach to determine a suitable stiffness for static friction scenarious. In
% reality one should increase force slowly to observe this probably). When
% dragging the sheet in the y-direction we effectively have a lattice spacing
% \begin{align*}
%   a_c = a_{2,x} + B_x = a_G\frac{\sqrt{3}}{2} + \frac{a_G}{2\sqrt{3}} = \frac{2a_G}{\sqrt{3}}
% \end{align*}
% for graphene lattice constant $a_G = 2.46$ Å. For the diamond silicon structure
% this is essentially equal to the lattice constant $a_D = 5.4210$ Å. This gives 
% \begin{align*}
%   \theta = \frac{a_c}{a_b} = \frac{2}{\sqrt{3}}\frac{a_G}{a_D} \approx 0.5230.
% \end{align*}
% Since we have the factor $2/\sqrt{3}$ it is safe to assume that this is a
% irrational number leadning to incommensurability. The worst case scnario of
% incommensurability (where $\theta$ equals the golden-mean, Can we get the exact
% number?) gives the minimal critical stiffness $K_c \sim 2U_0
% (\frac{\pi}{a_b})^2$, where $U_0$ is the substrate potentiual magnitude and
% $a_b$ the lattice spacing of the substrate. The potential barrier $U_0$ can be
% approximated by the work done when resisting the normal force as $\sim F_N
% a_D/2$ such that the critical stiffness can be approximated to 
% \begin{align*}
%   K_c \sim 2 F_N \frac{a_D}{2} \left(\frac{\pi}{a_D}\right)^2 = \frac{F_N}{a_D}\pi^2
% \end{align*}
% With a normal force of 1 nN we get $K_c \sim 18$ N/m. Hence, we should try a
% spring constant lower than that as qualified way of determining if this is the
% reason why we do not really see static friciton in the simulation. By plotting the max position (in terms of drag length) as a function of spring constant as seen in \cref{fig:max_vs_K} we can investigate if the concept of a critical spring constant is governing this simulation. However, as I'm writing this I'm realizing that the spring constant in the model applies to the interatomic forces and not the one dragging the system.....



\subsection{Pressure reference for normal load domain}
\text{Find place to put this.}

% source 1: stiletto heeled shoes with less than 1cm diameter:
% https://www.researchgate.net/publication/342223559_How_the_stiletto_heeled_shoes_which_are_popularly_preferred_by_many_women_affect_balance_and_functional_skills

In order to relate the magntidue of the normal force in our friciton measurement
we will use the pressure as a reference. We will use the pressure underneath a
stiletto shoe as a worst case for human pressure execuation underneath the
shoes. From (source 1) it is reported that the diameter of a stiletto heeled
shoe can be less than 1 cm. Hence a 80 kg man\footnote{Yes, a man can certainly
wear stilleto heels.} standing on one stiletto heel (with all the weight on the
heel) will result in a pressure
\begin{align*}
  P = \frac{F}{A} = \frac{mg}{r^2\pi} = \frac{\SI{80}{kg} \cdot \SI{9.8}{\frac{m}{s^2}}}{(\frac{\SI{1e-2}{m}}{2})^2 \pi} = \SI{9.98}{MPa} \\
\end{align*} 

% source 1:
% https://www.schoolphysics.co.uk/age16-19/Mechanics/Statics/text/Pressure_/index.html
While this is in itself a spectacular realization that is often used in
introductory physics courses (source 2) to demonstrate the rather extreme
pressure under a stiletto heel (greater than the foot of an elephant) (how many
Atmos?) this serves as a reasonible upperbound for human executed pressure. With
a full sheet area of $\sim\SI{21e3}{{\text{Å}}^2}$ we can achieve a similar pressure of
$\sim \SI{10}{MPA}$ with a normal force of
\begin{align*}
  F_N = \SI{10}{MPa} \cdot \SI{21e-17}{m^2} = \SI{2.10}{nN}  
\end{align*}

Of course this pressure might be insufficient for various industrial purposes,
but with no specific procedure in mind this serves as a decent reference point.
Notice that if we consider a human foot with ares $\SI{113}{cm^2}$ the pressure
drops to a mere $\SI{70}{kPa}$ corresponding to $\sim \SI{0.01}{nN}$.

% source 3: foot area ≈ 113 cm^2:
% https://www.footbionics.com/Patients/Foot+Facts.html source 4:
% https://hypertextbook.com/facts/2003/JackGreen.shtml



\section{Out of plane buckling}

The out of plane buckling is the main motivation for applying the kirigami
inspired cuts to the sheet. Thus, we perform a stretch simulation in a low
temperature $T = \SI{5}{K}$ vacuum in order to verify that the chosen cut
configurations do in fact contribute to a significant out of plane buckling when
stretched. For the non-cut, popup and honeycomb configuration we assess the
movement in the z-direction (perpendicular to the plane) during the stretch,
which we visualize by the min and max z-value along with the atom count
quartiles 1\%, 10\%, 25\%, 50\% (median), 75\%, 90\% and 99\% as shown in figure
\cref{fig:buckling_quartiles}. We observe that the popup and honeycomb pattern
buckles considerable out of plane during the stretch in comparison to the non-cut sheet which only exhibit minor buckling of $\sim 2$ Å which is on the same order as the
atomic spacing in the sheet. We also notice that the popup pattern
buckles more in consideration to the min and max peaks while the 1\%, 99\%
quartiles is on the same magnitude as the honeycomb. By looking at the simulation visualization
(\hl{include OVITO figures for vacuum stretch as well?}) we can conclude that this is mainly due to the fringes
of the sheet ``flapping'' around. 


\begin{figure}[H]
  \centering
  \includegraphics[width=\linewidth]{figures/baseline/vacuum_normal_buckling}
  \caption{Out of plane buckling during stretch of sheets in vacuum at $T = 5$ K. Reading from left to right the vacuum rupture stretch are 0.38, 0.22 and 1.37. \hl{perhaps use a color scale instead of the standard color cycles here.}}
  \label{fig:buckling_quartiles}
\end{figure}


The next step is to verify that the buckling will lead to a significant altering
of the contact area when the sheet is in put in contact with the substrate. We
investigate this by simulating the stretch at the default temperature $T =
\SI{300}{K}$ with the presence of contact forces between the sheet and
substrate. Note that no normal load is applied as the sheet and substrate is
sufficiently attracted by the LJ potential. Selected frames from the simulation is shown in appendix \cref{sec:sheet_stretch}. We assess the contact area by the
relative amount of atoms in the sheet within chemical range of the substrate.
The cut-off for this interaction is 4 Å corresponding to $\sim 120$\% the LJ
equilibrium distance. Since the contact area is usually calculated as the amount
of atoms in contact multiplied with an associated area for each contact this
feature is taken to be proportional to the contact area. The relative amount of
bonds as a function of stretch for the various configurations is shown in figure
\cref{fig:contact_vs_stretch} which clearly indicates a drop in contact area as
the cutted sheets are stretched. 

\begin{figure}[H]
  \centering
  \includegraphics[width=0.6\linewidth]{figures/baseline/contact_vs_stretch.png}
  \caption{Contact vs. stretching of the sheet, where the contact is measured by the relative amount atoms in the sheet within chemical interaction range to the substrate. The cut-off for this interaction range is 4 Å corresponding to $\sim 120 \%$ the LJ equilibrium distance. $T = 300$ K }
  \label{fig:contact_vs_stretch}
\end{figure}

\hl{Compare figure} \cref{fig:contact_vs_stretch} \hl{to that of figure} \cref{fig:multi_stretch_contact} \hl{where multiple simulations constitute the stretch-contact curve.}


\section{Investigating selected parmeters}

We investigate the importance of the physical variables $T$, $v_{\text{slide}}$ and $K$ (\hl{make plots for scan angle as well?}) and the choice of timestep $dt$. This is done partly understand how the dependencies relate to theoretical, numerical and experimerimental results, and partly to understand how these parameter choices defines the regime for our multi configurational search. We use the default parameters in \cref{tab:final_param} with exception of the single parameter of interest which is varied in a reasonable range of the default choice. In \cref{fig:var_temp}-\cref{fig:var_dt} the kinetic friction estimate and the max friction force is shown as a function of $T$, $v_{\text{slide}}$, $K$ and $dt$ respectively. For the kinetic friction estimate the absolute error is denoted by a shaded error which linearly connects the points.


\begin{figure}[H]
  \centering
  \begin{subfigure}[t]{0.49\textwidth}
      \centering
      \includegraphics[width=\textwidth]{figures/baseline/variables_temp_mean_fixmove_v20.pdf}
      \caption{kinetic friction force estimate.}
      \label{fig:var_temp_mean}
  \end{subfigure}
  \hfill
  \begin{subfigure}[t]{0.49\textwidth}
      \centering
      \includegraphics[width=\textwidth]{figures/baseline/variables_temp_max_fixmove_v20.pdf}
      \caption{Max friction}
      \label{fig:var_temp_max}
  \end{subfigure}
  \hfill
     \caption{Temperature.}
     \label{fig:var_temp}
\end{figure}



\begin{figure}[H]
  \centering
  \begin{subfigure}[t]{0.49\textwidth}
      \centering
      \includegraphics[width=\textwidth]{figures/baseline/variables_vel_mean_fixmove.pdf}
      \caption{kinetic friction force estimate.}
      \label{fig:var_vel_mean}
  \end{subfigure}
  \hfill
  \begin{subfigure}[t]{0.49\textwidth}
      \centering
      \includegraphics[width=\textwidth]{figures/baseline/variables_vel_max_fixmove.pdf}
      \caption{Max friction}
      \label{fig:var_vel_max}
  \end{subfigure}
  \hfill
     \caption{Sliding speed}
     \label{fig:var_vel}
\end{figure}



\begin{figure}[H]
  \centering
  \begin{subfigure}[t]{0.49\textwidth}
      \centering
      \includegraphics[width=\textwidth]{figures/baseline/variables_spring_mean_fixmove.pdf}
      \caption{kinetic friction force estimate.}
      \label{fig:var_K_mean}
  \end{subfigure}
  \hfill
  \begin{subfigure}[t]{0.49\textwidth}
      \centering
      \includegraphics[width=\textwidth]{figures/baseline/variables_spring_max_fixmove.pdf}
      \caption{Max friction}
      \label{fig:var_K_max}
  \end{subfigure}
  \hfill
     \caption{Spring constant}
     \label{fig:var_K}
\end{figure}



\begin{figure}[H]
  \centering
  \begin{subfigure}[t]{0.49\textwidth}
      \centering
      \includegraphics[width=\textwidth]{figures/baseline/variables_dt_mean_fixmove.pdf}
      \caption{kinetic friction force estimate.}
      \label{fig:var_dt_mean}
  \end{subfigure}
  \hfill
  \begin{subfigure}[t]{0.49\textwidth}
      \centering
      \includegraphics[width=\textwidth]{figures/baseline/variables_dt_max_fixmove.pdf}
      \caption{Max friction}
      \label{fig:var_dt_max}
  \end{subfigure}
  \hfill
     \caption{Timestep}
     \label{fig:var_dt}
\end{figure}


Quick thoughts:
\begin{itemize}
  \item Temperature: We do clearly not see the $1/T$ temperature decrease. The non-cut sheet seems to showcase a lienar relationship which is also somewaht present for the honeycomb which matches some of the findings in other MD simulations. For the popup we do see a local decrease at low temperatures which flip at around the default $T = \SI{300}{K}$ temperature. The max friction peaks seem to increase with temperatur as well indicating that the peaks might be associated with thermal fluctuations rather than actual stick-slip behaviour. This supports the finding that the static friction response is not significantly present in these simulations. 
  \item Velcotiy: Considering the non-cut sheet first the velocity dependency is seemingly linear which deviates from the expected logaritmic trend. For the cutted configurations we find some peaks which might indicate the presence of resonance frequencies. The cutted sheet might be closer to a logaritmic trend, but this is not spot on either. The max friction seems to decrease slightly with small velcoties and then stay rather constant. This can probably be explained by the reduced time to stick between stick slip. 
  \item Spring constant: On all three configurations the kinetic friction decreases with an increasing spring constant. The best explanations might be due to the lack of freedom to ``get stuck'' in incommensurable configurations. We also notice that the friction varies a lot at lower spring constants supporting the choice of having a stiff spring for stability reasons. Especially the non-cut sheet peaks at $K = \SI{40}{N/m}$. The max friction seem to be constant with $K$.
  \item $dt$: The kinetic friction is relatively stable around the default choice of $dt = \SI{1}{fs}$. However, the fluctuations with respect to $dt$ is more significant for popup pattern and even more for the honeycomb pattern. This indicates that the more complex kinetics of the simulation is more sensitive to the timestep. We might interpret this information as an additional measure of uncertainty. The maximum friction decreases with increasing timestep which can be asserted a statistical interpretation: Higher peaks will be captured by the high resolution of a low $dt$ and vice versa. The high max values towards the point of $dt = \SI{2}{fs}$ is most likely due to the approach of unstability in the simulation as seen more clearly for the kinetic friction evaluation. 
\end{itemize}


\section{Normal force and stretch dependencies}\label{sec:load_and_stretch}
Till this point we have only changed variables one by one to investigate single dependencies. We now advance the sutdy to a simultaneous variation of stretch and normal force.

\hl{Explain how the stretch is uniformly sampled within equally divided intervals and the normal force is actually uniformly sampled in a given range. Argue that the first might be approximately uniformly distributed for large numbers.}

\hl{Talk about rupture test also. Maybe in the theory/method section under numerical procedure: Before simulating a rupture test is perform to determine under what stretch the sheet ruptures. This is a slightly higher threshold than when applied normal load and sliding along the substrate.}

\subsection{Contact area}\cref{sec:contact_area}

We reproduce the contact area investigation of \cref{fig:contact_vs_stretch} with the modificaiton that the contact count is measured as an average of the latter 50\% of the sliding simulation at a non-zero applied normal load. The results are shown in \cref{fig:multi_stretch_contact} with 30 attempted (some rupture) stretch (pseudo) uniformly distributed stretch between 0 and the rupture point and 3 uniform distributed normal loads in the interval $[0.1, 10]$ nN. 


\begin{figure}[H]
  \centering
  \includegraphics[width=\linewidth]{figures/baseline/multi_stretch_area_compare.pdf}
  \caption{Average relative amount of bonds beetwen the sheet and the substrate defined by the cut-off distance of 4 Å. The average is taken over the latter half of the sliding phase. The red shade denotes the stretch range where ruptures accour at certain normal loads under sliding while the black-dotted line represent the rupture point due to stretching (rupture test)}
  \label{fig:multi_stretch_contact}
\end{figure}

From \cref{fig:contact_vs_stretch} we observe a significant decrease in the contact due to stretching of the cut configurations in contrast to the non-cut which stays roughly constant. This is reminiscent of the non-sliding stretch vs. contact curve shown in \cref{fig:contact_vs_stretch}. Given these results, theoretically one would expect the kinetic friction to decrease with stretch for the cut configurations.


\subsection{Stretch} 

We make a similar analysis as done in the previous section \cref{sec:contact_area} with the substitution of friction force instead of contact (The data is taken from the same simulaitons runs). The kinetic friction force (\hl{put uncertainty here even though that it is quite low?}) and the max friction is shown in \cref{fig:multi_stretch_mean_fric} and \cref{fig:multi_stretch_max_fric} respectively.



\begin{figure}[H]
  \centering
  \begin{subfigure}[t]{\textwidth}
      \centering
      \includegraphics[width=\textwidth]{figures/baseline/multi_stretch_mean_compare.pdf}
      \caption{kinetic friction estimate. }
      \label{fig:multi_stretch_mean_fric}
  \end{subfigure}
  \hfill
  \begin{subfigure}[t]{\textwidth}
      \centering
      \includegraphics[width=\textwidth]{figures/baseline/multi_stretch_max_compare.pdf}
      \caption{Max friction}
      \label{fig:multi_stretch_max_fric}
  \end{subfigure}
  \hfill
     \caption{\hl{CAPTION}}
     \label{fig:multi_stretch_max_fric}
\end{figure}


From \cref{fig:multi_stretch_mean_fric} we find to our surprise that the
kinetic friction increase with stretch for the cut configurations despite a
simmutaneous decrease in contact area as shown in figure
\cref{fig:multi_stretch_contact}. This suggests that the amount of chemical
bonding atoms is not the dominant mechanism for the friction of this system.
Instead, we might point to a mechanism more mechanical of nature associated to
phonon exications. When the cut sheet is strecthed the stress (\hl{show stress
maps somewhere or not nessecary?}) might induce a certain distribution and
magnitude of point pressures to favor energy dissipation. Nonetheless, the
results showcase a strong coupling between stretch and friction force, also for
the max friction force, which is beyond the expectations at this stage of the
study. The non-cut configuration does not show significant dependency on the stretch which reveal that this effect is only present when combining cut and stretch and not purely by strecthing the sheet. 

By considering the increase in kinetic friction towards the first peak we get a relative friction increase and increase vs. stretch ratios as described in \cref{tab:first_peak_stretch}. While the honeycomb force increase towards the first peak is approximately linear the popup exhibits seemingly exponential growth which yield a slope on the order $\sim \SI{30}{nN}$. 

\begin{table}[H]
  \begin{center}
  \caption{(stretch, kinetic friction) coordinates from \cref{fig:multi_stretch_mean_fric} at start and the first peak respectively used to approximate the relative increase in friction force and the ratio for friction increaese vs. stretch for sait range. In practice  the latter ratio denotes the slope of a forced linear trend. }
  \label{tab:first_peak_stretch}
  \begin{tabular}{ | c | c | c | c | c |} \hline
  Configuration & Start & First peak & Relative increase & Friction force vs. stretch ratio [nN]  \\ \hline
  Popup & $\sim (0, 0.03)$ & $\sim(0.082, 0.83)$ & 27.7 & 9.76  \\ \hline
  Honeycomb & $\sim (0, 0.07)$ &  $\sim (0.728, 1.57)$ & 22.4 & 2.06 \\ \hline
  \end{tabular}
  \end{center}
\end{table}

Additionally, we notice that booth the popup and honeycomb also exhibits stretch ranges where the kinetic friction force decrease with increasing stretch. Qualitatively we assign the slope to be on the same order of magnitude as those towrds the first peak. This is useful for the prospect of taking advantage of this phenonama as we can essentially achieve booth higher and lower friction for increasing stretch for different starting points. 


\subsection{Normal force}

\hl{Main take away from this section should be that the normal force does not really change the friction much; The friciton coefficient is extremely low, but I'm not sure how well the linear fits are (whether they are linear or sublinear). Not sure if I should do a linearly increasing normal force for better linear plots?}

\begin{figure}[H]
  \centering
  \includegraphics[width=\linewidth]{figures/baseline/multi_FN_mean_compare.pdf}
  \caption{...}
  \label{fig:}
\end{figure}


\begin{figure}[H]
  \centering
  \includegraphics[width=\linewidth]{figures/baseline/multi_FN_max_compare.pdf}
  \caption{Colorbar is only fitted for the right plot (honeycomb)... this should be fixed. Should I have run a linear distribtion of FN so I could plot it linear here also...?}
  \label{fig:}
\end{figure}



\begin{table}[H]
  \begin{center}
  \caption{Mean friction coeff}
  \label{tab:fric_coeff}
  \begin{tabular}{| c | c | c | c | c | c |} \hline
    nocut & 0.00009 $\pm\num{1e-05}$ & 0.00005 $\pm\num{1e-05}$ & 0.00004 $\pm\num{1e-05}$ & 0.00005 $\pm\num{2e-05}$ & \\ \hline
    popup & 0.00005 $\pm\num{3e-05}$ & 0.00024 $\pm\num{5e-05}$ & 0.0002 $\pm\num{2e-04}$ & 0.0005 $\pm\num{1e-04}$ & 0.0003 $\pm\num{2e-04}$ \\ \hline
    honeycomb & 0.00013 $\pm\num{6e-05}$ & 0.0006 $\pm\num{3e-04}$ & 0.0004 $\pm\num{6e-04}$ & 0.0007 $\pm\num{6e-04}$ & 0.0009 $\pm\num{3e-04}$ \\ \hline
  \end{tabular}
  \end{center}
\end{table}

\begin{table}[H]
  \begin{center}
  \caption{Max friciton coeff}
  \label{tab:fric_coeff}
  \begin{tabular}{| c | c | c | c | c | c |} \hline
    nocut & $0.0139 \pm \num{9e-04}$& $0.0083 \pm \num{7e-04}$& $0.010 \pm \num{1e-03}$& $0.0105 \pm \num{9e-04}$ &  \\ \hline
    popup & $0.007 \pm \num{2e-03}$& $0.010 \pm \num{2e-03}$& $0.007 \pm \num{2e-03}$& $0.009 \pm \num{3e-03}$& $0.006 \pm \num{2e-03}$ \\ \hline
    honeycomb & $0.010 \pm \num{1e-03}$& $0.007 \pm \num{2e-03}$& $0.007 \pm \num{3e-03}$& $0.000 \pm \num{3e-03}$& $0.004 \pm \num{3e-03}$ \\ \hline
  \end{tabular}
  \end{center}
\end{table}


\hl{One theory for the low friction coefficient might dependent on the fact that the normal force is only applied on the pull blocks. Especially with the cutted sheet the tension drops such that the effecive normal force on the inner sheet is not changing very much. By this theory the friction force vs. normal force on the pull blocks should look a bit more like expected and we might make some plots of thoose to check}

\hl{When looking at the graphs for the PB the max friction is visually textbook linear, while the mean friction is a bit more linear but also with negativ coefficients...}



\section{Computational cost}

Talk about the computatonal cost of different choices. How does computation time scale with drag speed, $dt$ and maybe $T$ and $K$ as well. One could also mention scaling with system size.

Show how the number of cores per simulation scale to argue that running on just one core (maybe 4) is smart for the next step of many simulations. 

Mention the trouble with GPU to show that this was considered, and in fact this was the reason for choosing the Tersoff potential over the AIREBO which is perhaps more common these days...
