% Small introtext to motivate this chapter. What am I going to go over here.

% The friction force is not an independent external force that acts on a body but an internal force that opposes the externally applied force. Thus, it may be thought of as a reaction force rather than an action force. In this sense, it is similar to the adhesion force between two bodies, which appears only when one tries to separate the bodies from contact. \cite{gao_frictional_2004}





\chapter{Friction} % Tribology - fiction
Friction plays a central role for the topic of this thesis being the key concept that we want to explore through the design of nanostructures. In this chapter we review the relevant theoretical understanding and highlight the derived expectations for our study.

Friction is a part of the wider field tribology which includes the study of
friction, wear and lubrication between two surfaces in relative motion \cite[p.
1]{gnecco_meyer_2015}. In this thesis we will only concern ourselves with so-called wearless dry friction. That is, without any use of lubrication and without any resulting wear of the contacting surfaces. 

\section{Friction across scales}
Tribological systems take place across a broad
range of time and length scales, ranging from geological stratum layers involved
in earthquakes \cite{kim_nano-scale_2009} to atomistic processes, as
in the gliding motion of a nanocluster or a nanomotor \cite{Manini_2016}. This
vast difference in scale gives rises to different frictional mechanism being
dominating. On a macro scale the system is usually subject
to a relatively high load and sliding speed leading to a high contact stress and
wear. On the other hand, the micro-/nanoscale regime occupies the opposite domain operating under relatviely small load and sliding speed with negligible wear \cite{kim_nano-scale_2009} \cite[p. 5]{bhushan_2013}. While macroscale friction is often reduced into a few variables such as load, material type, sliding speed and surface roughness, it is clear that the micro-/nanoscale friction cannot be generalized under such a simple representation. On the micro-/nanoscale the tribological propteries are dominated by surface properties which will yield a more complex behaviour of said variables and introduce an additional sensitivity to variables such as temperature, humidity and even sliding history. The works of Bhushan and Kulkarni \cite{BHUSHAN199649} showed that the friction coefficient decreased with scale even though the materials used was unchanged. This reveals an intrinsic relationship between friction and scale as the contact condition is altered.

The phenomenological descriptions of macroscale friction cannot yet be derived from the fundamental atomic principles, and bridging the gap between different length scales in tribological systems remains an open challenge \cite{Manini_2016}. Hence, the following sections will be organized into macroscale (\cref{sec:macroscale}), microscale (\cref{sec:microscale}) and nanoscale (\cref{sec:nanoscale}) representing the theoretical understanding governing each scale regime. Realising that the field of tribology across all scales is a vastly broad and intricate topic, we will aim to introduce only the essential findings for each scale, while keeping a main focus on nessecary theoretical background associated to the understanding of our system of interest which takes place at the lowest scale.


% While our study of the graphene sheet is based on a nanoscale perspective the hypothesizing about application possibilities will eventually draw upon a macroscale perspective as well. Thus, we argue that a brief theoretical introduction to all three major scales is suitable for a more complete interpreation of the findings in this thesis. 



% Tribological systems pose a wide range of dimensional scale. At the largest
% scale, geological stratum layers that are involved in earthquakes may be
% considered as a tribological system. The movement of the stratum occurs when
% the frictional forces between the layers are overcome by internal pressure
% inside the earth. At the smallest scale, relative motion of atoms at the
% interface of two materials would be a good example that involves frictional
% interaction. Owing to the vast difference in scale, the dominant mechanisms of
% friction and wear in macro-scale systems may be different from those of
% micro/nano-scale systems. In macro-scale, the tribological systems experience
% relatively large contact stresses and speeds. On the other hand,
% micro/nano-scale systems operate under relatively low loads and speeds.
% Particularly, the inertial effects that are prominent in macro-scale may be
% insignificant at the micro/nano-scale. Rather, surface forces often dictate
% the tribological interactions at small scales. \cite{kim_nano-scale_2009}.




% The differences between the conventional or macrotribology and
% micro/nanotribology are contrasted in Figure 1.3.1. In macrotribology, tests
% are conducted on components with relatively large mass under heavily loaded
% conditions. In these tests, wear is inevitable and the bulk prop- erties of
% mating components dominate the tribological performance. In
% micro/nanotribology, measurements are made on components, at least one of the
% mating components, with relatively small mass under lightly loaded conditions.
% In this situation, negligible wear occurs and the surface properties dominate
% the tribological performance. \cite{bhushan_2013}[p. 5]



% Quotes: Sliding friction that takes place between two surfaces in the absence
% of lubricant is termed "dry" friction even if the process occurs in an ambient
% environment. (Nanotribology and Nanomechanics, p. 329)




% We were astonished to discover that molecules that could flex or slide even
% just a little in response to the oscillatory motion of the microbalance were
% linked to low friction levels at the macro-scale. Put another way,
% exceptionally low friction at the atomic scale was not a prerequisite for the
% substantial reduction in macroscopic friction.
% (\url{https://physicsworld.com/a/friction-at-the-nano-scale/})



% Sliding friction that takes place between two surfaces in the absence of
% lubricant is termed ``dry''  friction even if the process occurs in an ambient
% environment. (Nanotriology and Nanomechanics, p. 329)



% It is generally accepted that friction is caused by more than one mechanism in
% a given sliding system. Generally, frictional force arises due to two
% fundamentally different causes, namely one that is mechanical in nature and
% the other being chemical in its origin. In the case of mechanical cause of
% friction, plowing of the surface by hard particles or asperities is mainly
% responsible for generating the frictional force.2,4,5-7 As for the chemical
% mechanism of friction, adhesion between surfaces of the two solids in contact
% is the cause of friction.2,4,5,8 Another point to note is that tribological
% phenomena are heavily dependent on system parameters of the operating machine
% such as speed, temperature, load, and environment. As such, the dominating.
% \cite{kim_nano-scale_2009}.







\section{Macroscale}\label{sec:macroscale}
Our working definition of the \textit{macroscale} is everything on the scale of visible objects. This is usually denoted to the size of milimeters \SI{e-3}{\metre} and above. Most importantly, we want to make a distinction to the microscale, where the prefix indicates the size of micrometers $m^{-6}$. Hence, we essentially consider everything larger than \textit{micro} to belong to the macroscale\footnote{The width of a human hair is often used as a reference for the limit of human perception. Since the width of a human hair is on the length scale $10^{-5}$ to \SI{e-4}{m} we find this limit alligns rather well with the defined transistion from macro- to microscale.}.

\subsection{Amontons’ law}
% Based on \cite{gnecco_meyer_2015}
% and \cite{gao_frictional_2004}
 
 In order to start and keep a solid block moving against a solid surface we must
 overcome certain frictional forces $F_{\text{fric}}$ \cite{gnecco_meyer_2015}.
 The static friction force $F_s$ corresponds to the minimum tangential force
 required to initiate the sliding while the kintec friciton force $F_k$
 corresponds to the tangential force needed to sustain such a sliding at steady
 speed. The work of Leonardo da Vinci (1452–1519), Guillaume Amontons (1663-705)
 and Charles de Coulomb (1736-1806) all contributed to the empirical law,
 commonly known as \textit{Amontons’ law}, which serves as a common base for macroscale
 friction. Amontons’ law states that the fricitonal forces is entirely
 independent of contact area and sliding velocity. Instead, it relies only on
 the normal force $F_N$, acting perpendicular to the surface, and the material specific friction coefficient $\mu$ as
\begin{align}
  F_{\text{fric}} = \mu F_N.
  \label{eq:amonton}
\end{align}
Notice that the term \textit{Normal force} is often used interchangeably
with \textit{load} and \textit{normal load} allthough the latter two terms refer to the applied force, ``pushing'' the object into the surface, and the first is the reaction force acting from the surface on the object. In equilibrium, these forces are equal in magnitude and hence we will not make a distinction between these terms. On the same note, we point out that the frictional force is different from a conventional force which in the Newtonian definition acts on a body from the outside and make it accelerate \cite{gao_frictional_2004}. Rather than being an independent external force the friction force is an internal \textit{reaction} force opposing the externally applied ``sliding'' force. 

The friction coeffcient $\mu$ is typically different for the cases of static
($\mu_s$) and kinetic ($\mu_k$) friction, usually both with values lower than
one and $\mu_s \ge \mu_k$ in all cases \cite[p. 6]{gnecco_meyer_2015}. The
friction coefficient is taken to be a constant defined by either
\cite{gao_frictional_2004} \\
\vspace{0.1cm}
\begin{subequations}
\noindent\begin{minipage}{.2\linewidth}
  \hfill
\end{minipage}
\begin{minipage}[b]{0.2\linewidth}
  \begin{align}
    \mu_1 = \frac{F_{\text{fric}}}{F_N},
    \label{eq:mu_def1}
  \end{align}
\end{minipage}
\begin{minipage}[b]{0.2\linewidth}
  \begin{align*}
    \text{or}
  \end{align*}
\end{minipage}
\begin{minipage}[b]{0.2\linewidth}
  \begin{align}
    \mu_2 = \frac{dF_{\text{fric}}}{dF_N}.
    \label{eq:mu_def2}
  \end{align}
\end{minipage}
\begin{minipage}{.2\linewidth}
\end{minipage}
\label{eq:mu_def}
\end{subequations}
\vspace{0.1cm}
\\
\noindent The first definition \cref{eq:mu_def1} requires zero friction at zero
load, i.e.\ $F_{\text{fric}} = 0$ at $F_N = 0$, while the second definition
\cref{eq:mu_def2} allows for a finite friction force at zero load as the
coefficient is defined by the slope of the $F_{\text{fric}}$-$F_N$-curve. The
consequences of these definitions are illustrated in
\cref{fig:fric_coef_example}, for selected $F_{\text{fric}}$-$F_N$-curves in
\cref{fig:fric_coef_example_a} and corresponding friction coefficients in
\cref{fig:fric_coef_example_b} and \cref{fig:fric_coef_example_c}. For adhesive
contacts the friction force will not be zero under zero load (red curve: Linear
+ shift) which can be mittigated by adding an extra constant to
\cref{eq:amonton} \cite{gao_frictional_2004}. Using \cref{eq:mu_def1} for
adhesive contacts would make the friction coefficient diverge for decreasing
load as illustrated in \cref{fig:fric_coef_example_b}. Thus, we find the second
definition \cref{eq:mu_def2} more robust and versitile. This also allows for a
better interpretation of the friciton coefficient in the hypothetical case where
friction depends non-linearly on load (Purple curve in
\cref{fig:fric_coef_example}). 


% In reality the friction coeffcient is not truly a
% material specific constant as it is often found to vary under different
% conditions such as humidity or smooth and rough morphologies of the sliding
% surfaces \cite{gao_frictional_2004}.


\begin{figure}[H]
  \centering
  \begin{subfigure}[t]{0.32\textwidth}
      \centering
      \includegraphics[width=\textwidth]{figures/theory/fric_coef_example_a.pdf}
      \caption{}
      \label{fig:fric_coef_example_a}
    \end{subfigure}
    \hfill
    \begin{subfigure}[t]{0.32\textwidth}
      \centering
      \includegraphics[width=\textwidth]{figures/theory/fric_coef_example_b.pdf}
      \caption{}
      \label{fig:fric_coef_example_b}
    \end{subfigure}
    \hfill
    \begin{subfigure}[t]{0.32\textwidth}
      \centering
      \includegraphics[width=\textwidth]{figures/theory/fric_coef_example_c.pdf}
      \caption{}
      \label{fig:fric_coef_example_c}
  \end{subfigure}
  \hfill
  \caption{\hl{CAPTION}}
  \label{fig:fric_coef_example}
\end{figure}


Although Amontons' law has been successful in is description of the majority of
rubbing surfaces, involving both dry and lubricated, ductile and brittle and
rough and smooth (as long as they are not adhesive) surfaces
\cite{gao_frictional_2004}, it has its limitations. It is now known that
\cref{eq:amonton} is not valid over a large range of loads and sliding
velocities and that it completely breaks down for atomically smooth surfaces in
strongly adhesive contact \cite{gao_frictional_2004}. For instance, the
independency of sliding velocity dissapears at low velocities as thermal effects
becomes important and for high velocities due to intertial effetcs \cite[pp.
5-6]{gnecco_meyer_2015}. For the case of static friction, it was discovered to
be dependent on the so-called contact history with increasing friction as the
logarithm of time of stationary contact \cite{dieterich_1972}.

In cases where Amontons' law breaks down we might still use the conceptual
definition of the friction coefficient as defined by (\cref{eq:mu_def2}).
Especially, in the context of achieving negative friction coefficients (in
certain load ranges) we would refer to this definition, since \cref{eq:mu_def1}
would imply a truly unphysical situation of the frictional force acting in the
same direction as the sliding motion. This would accelerate the object
indefinitelly\footnote{You would most likely have a good shot at the Nobel Prize
with that paper.}.


Due to the emperical foundation of Amontons’ law it does not provide any
physical insight into the underlying mechanisms of friction. However, as we will
later discuss in more detail, we can understand the overall phenomena of
friction through statistical mechanics by the concept of \textit{equipartition
of energy} \cite{Manini_2016}. A system in equilibrium has its kinetic energy
uniformly distributed among all its degrees of freedom. When a macroscale object
is sliding in a given direction it is clearly not in equilibrium since one of
its degrees of freedom carriers considerable more kinetic energy. Thus, the
system will have a tendency to transfer kinetic energy to the remaining
degrees of freedom as heat which dissipate to the sourroundings and making the object slow down if not continously driven forward by an extern energy source.  Hence, we can understand the overall concept of friction simply
as the tendency of going toward equilibrium energy equipartitioning among many
interacting degrees of freedom \cite{Manini_2016}. From this point of view it is
clear that friction is an inevitable part of contact physics, but even though
friction cannot be removed altogether, we are still capable of manipulating it
in usefull ways. \\
\\
The attentive reader might point out that we have already moved the discussion
into the microscopic regime as \textit{statistical mechanics} generally
aim to explain macroscale behaviour by microscopic interactions. In fact, this 
highlight the nessecity to consider smaller scales in order to achieve a more fundamental understadning of friction.


% All the terms in Amontons’ law refer to macroscopic, i.e., space- and time-averaged or “mean-field”, values. Thus, the contact area is the “apparent” or projected geometric area rather than the “real” contact area at the molecular level. And V is the mean relative velocity of the sliding bodies even though the shearing microjunctions may be moving with large fluctuations or in a stick-slip fashion.10 \cite{gao_frictional_2004}




% This includes taking the microsopic roughness
% into account together with surface chemistry. This more complex perspective
% introduces new (...() as real contact area, contact stresses, surface adhesion
% which makes frictional properties dependent on sliding speed, temperature and
% environment in general \cite{kim_nano-scale_2009}. 


% The conclusion is that the friction coefficient is not an intrinsic physical
% property \cite{Szlufarska_2008}.

% The basic difficulty of friction is intrinsic, involving the dissipative
% dynamics of large systems, often across ill-characterized interfaces, and
% generally violent and nonlinear \cite{Manini_2016}


% The severity of the task is also related to the experimental difficulty to
% probe systems with many degrees of freedom under a forced spatial confinement,
% that leaves very limited access to probing the buried sliding interface.
% Thanks to remarkable developments in nanotechnology, new inroads are being
% pursued and new discoveries are being made. \cite{Manini_2016}


\section{Microscopic scale}\label{sec:microscale}
Going from a macro- to a microscale perspective, at a length scale on the order
\SI{e-6}{m}, it was realised that most surfaces is in fact rough
\cite{mo_friction_2009}. The contact between two surfaces consist of numerous
smaller contact points, so-called \textit{asperities}, which form junctions due to contact pressure and adhesion as visualized in \cref{fig:asperity_contact} \cite{kim_nano-scale_2009}. In the macroscale perspective of Amonton's law we refer to time- and space-averaged values, i.e.\ the apparent contact area and the average
sliding speed \cite{gao_frictional_2004}. However, microscopically we find the
real contact area to be much smaller than the apparent area \cite{kim_nano-scale_2009}, and the shearing motion of local microjunctions to happen at large fluctuations rather than as one syncronized movement throughout the surface. 

It is generally accepted that friction is caused by two mechanism: Mechanical
friction and chemical friction \cite{kim_nano-scale_2009}. The mechanical
friction is the ``plowing'' of the surface by hard particles or said asperities
with an energy loss attributed to deformations of the asperity. While plastic
deformations, corresponding to wear, gives rise to an obvious attribution for
the energy loss, elastic deformations is also sufficient in explaining energy
loss due to phonon excitations. In fact the assumption of plastic deformations
has been critizised as this is theorized only to be present in the beginning of
a surface contact while it is neglible for prolonged or repeated contacts
\cite{CARBONE20082555}. That is, when machine parts slide against each other for
millions of cycles, the plastic deformation would only take place in the beginning for which the system then reaches a steady state with only elastic deformations.
The chemical friction arrises from adhesion between microscopic contacting
surfaces, with an energy loss attributed to the breaking and forming of bonds. 



\subsection{Asperity theories} % Surface roughness --- Asperity theories
% Sources in general: \cite{mo_friction_2009}, \cite{kim_nano-scale_2009} \\

Asperity theories have its foundations in the adhesion model proposed by Bowden and Tabor \cite{bowden2001friction} which is based on the fundamental reasoning that friction is governed by the adhesion between two surfaces \cite{Kim_2012}. Adhesion is proportional to the real contact area defined by asperity junctions and interfacial shear strength $\vec{\tau}$ between such contacting junctions. For an asperity contact area $A_{\text{asp}}$ we get a true contact area $\sum A_{\text{asp}}$ leading to 
\begin{align*}
  F_\text{fric} = \vec{\tau} \sum A_{\text{asp}}.
\end{align*}
Note that this is still compatible with Amontons’ law in \cref{eq:amonton} by having a linear relatioship between the real contact area and the
applied load. In fact, this is exactly how the theoretical model explains the friction dependency of load. By increasing the normal load it is hypothesized that the real contact area will increase as the asperity tips are deformed (plastically or elastically) into broader contact points as visualized qualitatively in \cref{fig:asperity_contact}.

\begin{figure}[H]
  \centering
  \begin{subfigure}[b]{0.49\textwidth}
      \centering
      \includegraphics[width=\textwidth]{figures/theory/asperities_top.png}
      \caption{Low load.}
      \label{fig:asp_left}
  \end{subfigure}
  \hfill
  \begin{subfigure}[b]{0.49\textwidth}
      \centering
      \includegraphics[width=\textwidth]{figures/theory/asperities_bottom.png}
      \caption{High load.}
      \label{fig:asp_right}
  \end{subfigure}
  \hfill
     \caption{Qualitatively illustration of the microscopic asperity deformation
     under increasing load from frame (a) to (b) \cite{wiki:asperities}. While this figure seemingly portrays plastic deformation the concept of increased contact area under increased load applies for elastic deformation as well.}
     \label{fig:asperity_contact}
\end{figure}

Many studies have focused on single asperity contacts to reveal the relationship
between the contact area and load
\cite{Szlufarska_2008,PhysRevLett.56.930,perry_scanning_2004}. By
assuming perfectly smooth asperities, with radii of curvature from micrometers
all the way down to nanometers, continuum mechanics can be used to predict the
deformation of asperities as load is applied. A model for non-adhesive contact
between homogenous, isotropic, linear elastic spheres was first developed by
Hertz \cite{HertzOnTC}, which predicted $A_{\text{asp}} \propto F_N^{2/3}$.
Later adhesion effects were included in a number of subsequent models, including
Maugis-Dugdale theory \cite{MAUGIS1992243}, which also predicts a sublinear
relationship between $A_{\text{asp}}$ and $F_N$. Thus, the common feature of all
single-asperity theories is that $A_{\text{asp}}$ is a sublinear function of
$F_N$, leading to a similar sublinear relationship for $F_\text{fric}(F_N)$,
which fails to align with the macroscale observations modelled by Amontons’ law
(eq. \eqref{eq:amonton}).

% Concurrently with single-asperity studies, roughness contact theories are being developed8–10,16 to bridge the gap between the mechanics of single asperities and that of macroscopic contacts.\cite{mo_friction_2009}

Concurrently with single-asperity studies, roughness contact theories are being developed \cite{PhysRevLett.100.055504,Persson,GW,BUSH197587} to bridge the gap between single asperities and macroscopic contacts \cite{mo_friction_2009}. A variety of multi-asperity theories has attempted to combine single asperity
mechanics by statistical modelling of the asperity height and spatial
distributions \cite{CARBONE20082555}. This has led to a partially success in the establishment of a linear relationship between $A_{\text{asp}}$ and $F_N$. Unfortunately, these results are restricted
in terms of the magnitude of the load and contact area, where multi-asperity
contact models based on the original ideas of Greenwood and Williamson \cite{GW}
only predicts linearity at vanhising low loads, or Persson \cite{Persson} which predicts linearity for more reasonable loads but only up to 10-15 \% of the macroscale contact area. However, as the load is further increased all multi-asperity models
predict the contact area to fall into the sublinear dependency of normal force
as seen for single aperity theories as well \cite{CARBONE20082555}.


% Da Vinci-Amontons law – friction independent of area – is not confirmed at the
% microscopic scale. In most nanoscale investigations the friction of a single
% con- tact is found to increase linearly with the contact area [27–29]. In
% contrast, structurally mismatched atomically flat and hard crystalline or
% amorphous surfaces are expected to produce a sublinear increase of friction
% with contact area. The frequent finding of friction proportional to area even
% in some of these cases can be understood as a consequence of softness, either
% if the interface, or of surface contaminants leading to effectively pseudo-
% commensurate interfaces [30, 31] (Current trends in the physics of nanoscale
% friction)


% Other authors proposed an empirical model in which mechanics of a nanoscale non-adhesive contact is controlled by load, that is, $F_f = \mu L$ and the contact area is undefined and unnecessary5,29 \cite{mo_friction_2009}

% \cite{mo_friction_2009} agues that the break-down of single-asperity theories of friction is due to the asperity (circumfereance defined) area is not proportional to the real one. By obtaining the real area (contacting bond) he arrives at the macroscale relationship... Quote: As shown in Table 1, friction force is now proportional to contact area at all length scales as long as the contact area is correctly defined at each length scale. When adhesion is added they arrive at the sublinear trend again. 


% \hl{What about multi asperity theory?}

% Our model predicts that as the adhesion between the contacting surfaces is reduced, a transition takes place from nonlinear to linear dependence of friction force on load. \cite{mo_friction_2009}




% This approach enables the bottom-up derivation of the linear scaling laws of macroscopic friction with size, and their transition to the sublinear ones for incommensurate nanosized contacts. We can now understand that such transition takes place when the contact roughness becomes large compared to the range of interfacial interactions [162] \cite{Manini_2016}.


% However, practical single- and multiplecontact conditions are characterized by
% complex interaction profiles plus nontrivial internal dynamics. As a result,
% the interplay of thermal drifts, contact ageing, contact-contact interactions,
% and macroscopic elastic deformations introduce significant complications, and
% make the depinning transition from static to kinetic friction an active field
% of research. \cite{Manini_2017}[p. 2]. 



% However, even though the successes of continuum mechanics there is no reasion to
% believe that it will be capable of reproducing tribological behaviour at the
% nanometre length scale where the discreteness of atoms often has a direct effect
% on physical properties \cite{Szlufarska_2008}.


\section{Nanoscale --- Atomic scale}\label{sec:nanoscale}
Going from a micro- to a nanoscale, on the order of \SI{e-9}{m}, it has been
predicted that continnum mechanics will start to break down \cite{luan_breakdown_2005} due to the discreteness of individual atoms. In a numerical \acrshort{MD} study by Mo et
al.\ \cite{mo_friction_2009}, considering asperity radii of 5-30 nm, it has been
shown that the asperity area $A_{\text{asp}}$, defined by the circumfereance of
the contact zone, is sublinear with $F_N$. This is accommodated by the
observation that not all atoms within the circumference make chemical contact
with the substrate. By modelling the real contact area $A_{\text{real}} =
NA_{\text{atom}}$, where $N$ is the amount of atoms within the range of chemical
interaction and $A_{\text{atom}}$ the associated surface area for a contacting atom, they
found a consistent linear relationship between friction and the real contact
area. Without adhesive forces this lead to a similar linear relationship
$F_{\text{fric}} \propto F_N$, while adding van der Waals adhesion to the
simulation gave a sublinear relationship matching microscale single asperity theory, even though the $F_{\text{fric}}
\propto A_{\text{real}}$ was maintained. This result emphasizes that the predictions of continuum mechanchs might still apply at the nanoscale and that the contact area can still be expected to be play an important role for nanoscale asperity contact. It is simply the definition of the contact area that undergoes a change when transistioning from micro- to nanoscale. 


% Although both numerical \cite{zhu_study_2018}\cite{ma12091425}\cite{bonelli_atomistic_2009} and experimental \cite[2005]{DIENWIEBEL2005197}\cite{feng_superlubric_2013} studies have been done for so-called nanoflakes
% sliding on a substrate, the dependence of friction force on contact area is not investigated.  One reasonable explanation is that the contact area is already at its maximum for atomically smooth contacting surfaces and hence does not play an important role. In a numerical study of atomic-scale frictional behavior of corrugated nano-structured surfaces \cite{C2NR30691C} they reported that the contact area only affected the friction significantly for big corrugations as opposed to small. Since an increasing friction is still reported under increasing load in most nanoflake studies (see
% \cref{sec:expected_prop} for a more detailed dicussion), this suggests
% that some other mechanisms are governing friction at this level. 
% we are going to study a nanoscale system with no
% initial asperities present.

% make it unfounded to rely on asperity theories.

% Note that atom spacing lies in the
% domain of a few ångströms Å (\SI{e-10}{m}) and thus we take the so-called
% atomic-scale to be a part of the nanoscale regime. 

% ``Load-independent friction has also been observed in FFM experiments on thermally oxidized MoS2, and it was proposed that MoO3 nanocrystals, that grew during the oxidation process on the MoS2 surface [35], acted as a spacer between the tip and the sample, such that the contact area remained unchanged upon loading.In the present case, the contact area would be completely determined by the flake size, which would be independent of the loading force.Hence, the friction would only increase slightly with normal load as the result of the increase in contact pressure.'' \cite{DIENWIEBEL2005197}

% Before diving into alternative theoretical approaches to adress this issue we point out that exactly this transistion, between nanoscale asperities and atomically smooth surfaces, is of outermost importance for the objective of the thesis. By introducing kirigami cuts and stretching the sheet we expect to see an out of plane buckling which induce an ensemble of asperities on the sheet. Hence, we might hypothesize that such a transition will contribute to a significant change in the goverening mechanism of friction bridgering the two domains of nanoscale asperity and smooth surface theory. 

While the study by Mo et al.\ \cite{mo_friction_2009} considers a single
asperity on a nanoscale, some models take this even further to what we will
denote as the atomic-scale. This final leap is motivated by the fact that our
system of interest, an atomically flat graphene sheet imposed on a flat silicon
substrate, lacks the presence of nanoscale asperities in its inital uncut
underformed state. In the lack of noteworthy structural asperities, friction can
instead be modelled as a consequencse of the ``rough'' potential layed out by
the atomic landscape. A series of so-called reduced-order models builds on a
simplified system of atomic-scale contacts based on three essential parts: 1) A
periodic potential modelling the substrate as rigid crystaline surface. 2) An
interacting particle, or collection of particles, placed in the potential. 3) A
moving body, moving at steady speed, and connected to the particles through a
harmonic coupling. In figure \cref{fig:PT_FK_FKT} three of the most common 1D
models is displayed which we will adress in the following sections. The
time-honered Prandtl-Tomlinson (PT) model describes a point-like tip sliding
over a space-periodic fixed crystalline surface with a harmonic coupling to the
moving body. This is analog to that of an experimental cantilever used for
Atomic Force Microscopy which we will introduce in more details in
\cref{sec:SPM}. Further extensions was added in the Frenkel-Kontorova
(\acrshort{FK}) model by substituting the tip with a chain of harmonic coupled
particles dragged from the end, and finally combinned in the
Frenkel-Kontorova-Tomlinson (\acrshort{FKT}) with the addition of a more
rigorous harmonic coupling between the moving body and each of the atoms in the
chain. While these models cannot provide the same level of details as atomistic
simulations such as \acrshort{MD} it enables investigation of atomic friction
under most conditions, some of which are inaccessible to \acrshort{MD}
\cite{Yalin_2011}. This makes these models an appropiate tool for investigating
inidvidual parameters and mechanisms governing affecting friction.


\begin{figure}[H]
  \centering
  \includegraphics[width=0.4\linewidth]{figures/theory/PT_FK_FKT.png}
  \caption{\hl{Temporary} figure from \cite{Yalin_2011}. Be careful to aling notation on the figures with the text later on.}
  \label{fig:PT_FK_FKT}
\end{figure}


% Analytical Models for Atomic Friction \cite{Yalin_2011}
% \textbf{Analytical Models for Atomic Friction}
\subsection{Prandtl–Tomlinson} % Sources for PT model \cite[2,3]{Yalin_2011} 
We consider the Prandtl–Tomlinson model (\acrshort{PT}) with added thermal activation as proposed by Gnecco et al. \cite{PhysRevLett.84.1172}. For the theoretical foundation of this section we generally refer to \cite{Yalin_2011}.



% An alternative to experimental and atomistic simulation such as \acrshort{MD} is
% an analytical or reduced-order model that simplifies the system
% \cite{Yalin_2011}. While it cannot provide the same level of detail as atomistic
% simulations it enables investigations of atomic friction under almost all
% experimental conditions, some of which are in inaccessible to \acrshort{MD}.
% Here we consider the reduced-order models that is the Prandtl–Tomlinson (PT) and
% extensions. This simple model consist of a point mass coupled via a harmonic
% spring to a driven support which captures the combination of stiffness of the
% tip, cantilever torsion (first eigenmode) and effective mass of the... These
% models simply single asperity friction into one or more point-masses (contact
% area atoms) pulled via an elastic tether (corresponding to a cantilever and a
% tip) along a period potential energy profile (the substrate). There are many
% extensions: adding thermal activation proposed by Gnecco et al.
% \cite{PhysRevLett.84.1172}. Other related are Frenkel–Kontorova (\acrshort{FK})
% \cite{Frenkel_1938} and Frenkel–Kontorova–Tomlinson (\acrshort{FKT})
% \cite{weiss_dry_1997} illustrated in \cref{fig:PT_FK_FKT}. An advantage is that
% they can be easily implemented and investigated with respect to isloating the
% effects from individual parameters. 

% Mathematical formulation 
The 1D \acrshort{PT} model assumes a single ball-tip coupled harmonically to a support moving at constant speed which makes the tip slide along the rigid substrate. The interaction between tip and substrate is modelled as a sinusoidal corrugation potential mimicking the periodicity found in a crystaline substrate. The total potential energy is given as
\begin{align}
  V(x,t) = \frac{1}{2}K(vt - x)^2 - \frac{1}{2}U_0 \cos \left(\frac{2\pi x}{a} \right).
  \label{eq:V_PT}
\end{align}
The first term describes the harmonic coupling at time $t$, with spring constant $K$, between the tip at position $x$ and the moving body at position $vt$, given by its constant speed $v$. The second term describs the corrugation potential with amplitude $U_0$ and period $a$ representing the lattice spacing of the substrate. The dynamics of the tip can be described by the Langevin equations 
\begin{align}
  m \ddot{x}+m \mu \dot{x}=-\frac{\partial V(x, t)}{\partial x}+\xi(t),
  \label{eq:Langevin_PT}
\end{align}
\hl{Match notation with later use}. \\
where $m$ is the mass of the tip, $\mu$ the viscous friction and $\xi(t)$ the thermal activation term. The equation is solved for tip position $x$ and the friction force is retrieved as the force acting on the moving body
\begin{align*}
  F_{\text{fric}} = K(vt - x).
\end{align*}
The governing equation \cref{eq:Langevin_PT} belongs to a family of stochastic differential equations composed of deterministic dynamics and stochastic processes. In this case the deterministic term is the viscous friction, $m\mu\dot{x}$, to resist the movement of the tip and the force acting from the corrugation potential. The stochastic term is a random force field modelling thermal noise according to the fluctuation–dissipation relation. Thus, there is no single path but rather multiple paths the tip can take. While the Langevin equations is one of the most common ways to handle thermal activaiton other methods exist to solve this problem such as Monte Carlo sampling methods. We ommit the numerical scheme for solving this and refer to a more in depth discussion of the Langevin equations with respect to the use in \acrshort{MD} simulations in \cref{sec:langevin}. 


\subsubsection{Thermal activation}
The solving of the Langevin equations, as opposed to Newtons equation of motion, introduces thermal effects to the system. Generally, when the energy barrier comes close to $k_B T$ (0.026 eV at room temperature) thermal effects can not be neglected. In the case of a single asperity contact the energy barrier is on the order 1 eV which make thermal activation significant \cite{Yalin_2011}. Due to the moving body travelling at constant speed the potential energy will increase steadily. Without any temperature, $T = 0$, the slip will only accour when the energy barrier between the current potential well (i) and the adjacent (j) is zero $\Delta V_{i\to j} = 0$. However, in the presence of temperature we get thermal activation, meaning that the tip can slip to the next potential well sooner $\Delta V_{i\to j} > 0$. Provided that the sliding speed is slow enough (\hl{Elaborate}) the transistion rate $\kappa$ for a slip from the current to the next well is given by
\begin{align}
  \kappa = f_0 e^{-\Delta V / k_B T},
  \label{eq:PT_kappa}
\end{align}
with $\Delta V$ being the energy barrier and $f_0$ the attempt rate. The attempt rate following Kramer’s rate theory \cite{RevModPhys.62.251} is related to the mass and damping of the system and can be thought of as the frequency which the tip ``attempts'' to overcome the barrier. Notice that \cref{eq:PT_kappa} resembles a microstate probability in the canonical ensemble with $f_0$ in place of the inverse partion function $Z^{-1}$ which can be used as another intepration of $f_0$. The probability $p_i$ that the tip occupies the current well $i$ relative to the adjacent well $j$, as illustrated in \cref{fig:PT_slip} is governed by 
\begin{align*}
  \frac{dp_i}{dt} = -\kappa_{i\to j}p_i + \kappa_{j\to i}p_j.
\end{align*}
This probability is related to temperature, speed and mass.

\begin{figure}[H]
  \centering
  \includegraphics[width=0.6\linewidth]{figures/theory/PT_slip.png}
  \caption{\hl{Temporary} figure from \cite{Yalin_2011}}
  \label{fig:PT_slip}
\end{figure}


\begin{figure}[H]
  \centering
  \begin{subfigure}[t]{0.49\textwidth}
      \centering
      \includegraphics[width=\textwidth]{figures/theory/PT_temp.png}
      \caption{}
      \label{fig:PT_temp_a}
  \end{subfigure}
  \hfill
  \begin{subfigure}[t]{0.49\textwidth}
      \centering
      \includegraphics[width=\textwidth]{figures/theory/PT_temp_force.png}
      \caption{}
      \label{fig:PT_temp_b}
  \end{subfigure}
  \hfill
  \hfill
     \caption{\hl{Temporary} figures from \cite{Yalin_2011}}
     \label{fig:PT_temp}
\end{figure}

Generally, there exist two temperature regimes in the model: Thermal activation at low temperature and thermal drift at high temperature as shown in \cref{fig:PT_temp}. At lower temperature the system is subject to standard thermal activation with $\Delta V_{i \to j} \gg \Delta V_{j \to i}$ resulting in  $\kappa V_{i \to j} \gg \kappa V_{j \to i}$. Effectively, this inhibits any backwards slip and we get  
\begin{align*}
  \frac{dp_i}{dt} = -\kappa_{i\to j}p_i,
\end{align*}
which make the relationship between friction, temperature and speed follow Sang et al.’s prediction \cite{Sang_2001}
\begin{align}
  F=F_c-\left|\beta k_B T \ln \left(\frac{v_c}{v}\right)\right|^{2 / 3}, \qquad v_c = \frac{2f_0\beta k_B T}{3 C_{\text{eff}} \sqrt{F_c}},
  \label{eq:F_thermal_ac}
\end{align}
where $F_c$ is the maximum friction at $T = 0$, $v_c$ a critical velocity, $f_0$
is the attempt rate, $c_{eff}$ the effective stifness, and $\beta$ a
parameter determined by the shape of the corrugation well.
\cref{eq:F_thermal_ac} characterizes the decrease in friction with temperature
in the thermal activation regime, shown in \cref{fig:PT_temp_a} at low temperature, with the assumption of only forward slips, as seen in the force trace shown in \cref{fig:PT_temp_a}. When the temperatuere is high enough, for the system to be consistenly close to thermal equilibrium, it enters the regime of thermal drift \cite{PhysRevE.71.065101}. This regime transistion can be understood through a comparison of two time scales: The time it takes for the moving body to travel one lattice spacing
$t_v = a/v$ and the average time for a slip to accour due to thermal activation
$\tau = 1/\kappa = f^{-1}\exp(\Delta V / k_BT)$. If $t_v \gg \tau$ the system falls within the thermal drift regime, with slips both backwards and forwards as shown in \cref{fig:PT_temp_b}, and the friction follows the prediction of Krylov et
al.\ \cite{Krylow_2007, PhysRevE.71.065101, Jinesh_2008}
\begin{align}
  F \propto \frac{v}{T}e^{1/T}.
  \ref{eq:PT_thermal_drift}
\end{align}

% \hl{Should we comment about the velocity dependence being different here also?}




% shows the effect of temperature on mean friction and illustrates two regimes: thermal activation and thermal drift. Force traces for these regimes are shown in \cref{fig:PT_temp_force}. 
  
% \begin{figure}[H]
%   \centering
%   \includegraphics[width=0.6\linewidth]{figures/theory/PT_temp.png}
%   \caption{\hl{Temporary} figure from \cite{Yalin_2011}}
%   \label{fig:PT_temp}
% \end{figure}

% \begin{figure}[H]
%   \centering
%   \includegraphics[width=0.6\linewidth]{figures/theory/PT_temp_force.png}
%   \caption{\hl{Temporary} figure from \cite{Yalin_2011}}
%   \label{fig:PT_temp_force}
% \end{figure}


\subsubsection{Sliding speed}
In the thermal activation regime (low temperature) and at low speeds the friction relation follows \cref{eq:F_thermal_ac} making friction scale logarithmically with speed. For higher speed, $v > v_c$, if only thermal effects are considered, \cref{eq:F_thermal_ac} predicts that friction will eventually saturate and come to a plateau at $F_{\text{fric}} = F_C$. This is illustrated in \cref{fig:PT_speed} with this prediction being represented by the dotted line. However, as given away by the figure, for higher speed the model will enter an athermal regime where the thermal effects are negligible compared to other contributions \cite{Yalin_2011}(32). In the athermal regime the damping term $m\mu \dot{x}$ will dominate yielding $F_{\text{fric}}
\propto v$. The athermal regime is often observedd in reduced models if the system is overdamped or at high speeds. This concept is also interesting in connection to \acrshort{MD} simulations where the accesible speeds often fall into the athermal regime \cite{Li_2011}. It is unclear how this effects real physical systems for which there exist more dissipation channels than just a single viscous term \cite{Dong_2013}. For the thermal drift regime at higher temperatures the linear relation $F_{\text{fric}} \propto v$ is predicted for low speed as well by \cref{eq:PT_thermal_drift}.

\begin{figure}[H]
  \centering
  \includegraphics[width=0.6\linewidth]{figures/theory/PT_speed.png}
  \caption{\hl{Temporary} figure from \cite{Yalin_2011}}
  \label{fig:PT_speed}
\end{figure}


\subsubsection{Tip mass}
The mass of the tip affects the dynamics due to a change of intertia, which changes the attempt rate $f_0$. A smaller intertia leads to a larger attempt rate and vice verse. Effectively, this will affect the transistion point for the temperature and speed regimes described in the previous. A smaller interia, giving a larger attempt rate, will cause an earlier transsition (i.e.\ at lower temperature) to the thermal drift regime, and result in a later speed saturation such that it transistions to the athermal regime at hihger speed. 


\subsubsection{Friction Regimes: Smooth Sliding, Single Slip, and Multiple Slip}
Stick-slip motion is a crucial instability mechanism associated with high energy dissipation and high friction. Thus, controlling the transistion between smooth sliding and stick-slip is considered key to control friction. We can divide the frictional stick-slip behaviour into three regimes: 1) Smooth sliding, where the tip slides smoothly on the substrate. 2) Single slip, where the tip stick at one potential well before jumping one lattice spacing to the next. 3) Multiple slip, where the tip jumps more than one lattice spacing for a slip event. The underlying mechanisms behind these regimes can be understood through a static and a dynamic contribution. 

To understand the static mechanism we consider a quasistatic process for which temperature, speed and damping can be neglected and where we must have $\partial(V)/\partial x = 0$. This simiplifies \cref{eq:V_PT} to 
\begin{align}
  \frac{\pi U_0}{a} \sin\left(\frac{2\pi x}{a}\right) \frac{2 \pi}{a} = K(vt - x).
  \label{eq:static_V}
\end{align}
The friction regime is determined by the number of solutions $x$ to \cref{eq:static_V}. Only one solution corresponds to
smooth sliding, two solutions to a single slip and so on. It turns out that the
regimes can be defined by the parameter $\eta = 2\pi^2U_0/a^2K$ \cite{Johnson_1998, Medyanik_2006} yielding transistions at $\eta = 1, 4.6, 7.79, 10.95, \hdots$, such that $\eta \le 1$
corresponds too smooth sliding, $1<\eta \le 4.6$ to a single slip and so on. These static derivation lays out the foundamental probabilities for being in one of the regimes stick-slip regimes. 

Considering the dynamics on top, one find that damping, speed and temperature will affect this probability. A high damping, equivalenet of a high transfer
of kinetic energy to heat, will result in less energy available for the slip events. This will make multiple slip less likely. By a similar argument, we find that increasing the speed will contribute to more kinetic energy which will increase the likelihood of multiple slips. Finally, temperature will contribute to earlier slips, due to thermal activation, such that
less potential energy can be accumulated and it will result in fewer multiple slips. The effects of damping, speed and temperature is illustrated for the force traces in \cref{fig:PT_slip_var}

\begin{figure}[H]
  \centering
  \begin{subfigure}[t]{0.32\textwidth}
      \centering
      \includegraphics[width=\textwidth]{figures/theory/PT_slip_damping.png}
      \caption{}
  \end{subfigure}
  \hfill
  \begin{subfigure}[t]{0.32\textwidth}
      \centering
      \includegraphics[width=\textwidth]{figures/theory/PT_slip_speed.png}
      \caption{}
  \end{subfigure}
  \hfill
  \begin{subfigure}[t]{0.32\textwidth}
      \centering
      \includegraphics[width=\textwidth]{figures/theory/PT_slip_temp.png}
      \caption{}
  \end{subfigure}
  \hfill
     \caption{\hl{Temporary} figure from \cite{Yalin_2011}}
     \label{fig:PT_slip_var}
\end{figure}



% A direct result of thermal
% activation, contribution of thermal energy overcoming energy barriers, is that
% the increase in temperature will decrease friction \cite[7, 21]{Yalin_2011}.
% Similar friction will increase with speed \cite[4, 22–24]{Yalin_2011} due to the
% decrease in time between slips making it less likely to get a thermal
% activation. 




% \subsubsection{2D nature}
%%%%%%%%%%%%%%%%%%%%%%%%%%%%%%%%%%%



% Our results confirm the conclusions of other authors that single-asperity theories break down at the nanoscale1,5. \cite{mo_friction_2009}


% Experimental research to examine the frictional characteristics at the
% atomic-scale has been conducted for the past two decades. It is well known
% that frictional behavior cannot be generalized by a few factors such as normal
% load, surface roughness, speed, and material type of the tribological system.
% Other conditions such as temperature, humidity, and even sliding history can
% affect the tribological phenomena significantly. Particularly at nano-scale,
% the tribological behavior tends to be more sensitive to the state of outermost
% layer of the surface region. Thus, contamination layer, adsorbed gas,
% capillary junctions, and oxide layer become more important at small scales.
% This is because at nano-scale the contact forces are often too low for the
% asperities to penetrate the surface layers and the magnitude of the surface
% force may be comparable to the frictional force. {kim_nano-scale_2009}.


% Together with the current experimental possibility to perform well-defined
% measurements on well-characterized materials at the fundamental microscopic
% level of investigation of the sliding contacts, advances in the computer
% modeling of interatomic interactions in materials science and complex systems
% encompass molecular-dynamics (MD) simulations of medium to large scale for the
% exploration of the tribo-dynamics with atomic resolution [4, 5].




% \subsection{Tomlinson model}
% % \cite{kim_nano-scale_2009}

% One of the first atomic scale models. Here we have no asperities but it is based on the assumptions that the atomic surface is not completely smooth. Since atoms was modelled as spheres the surface topography would not be completely flat. The idea is shown in \cref{fig:tomlinson_model} where the moving body is connected the atom with springs.
% \begin{figure}[H]
%   \centering
%   \includegraphics[width=0.5\linewidth]{figures/theory/tomlinson_model.png}
%   \caption{\hl{Temporary} figure from \cite{kim_nano-scale_2009}}
%   \label{fig:tomlinson_model}
% \end{figure}


% This model gives an explanation to the stick-slip behaviour. This layed the foundation for the Frenkel-Kontorova model so maybe go straight to that?


% Double check that I got it right regarding connection to the moving body and how the model is driven. 
\subsection{Frenkel-Kontorova}
% Based on \cite{Manini_2016} and \cite{FK2D}.

The Frenkel-Kontorova (\acrshort{FK}) model \cite{Frenkel_1938} extends the \acrshort{PT} model by considering a chain of atoms in constrast to just a single particle (tip). This extension is usefull for understanding the importance of the alignment between the atoms and the substrate, the so-called \textit{commensurability}.

The standard (\acrshort{FK}) model consists of a 1D chain of $N$ classical particles of equal mass, representing atoms, interacting via hamornic forces and moving in a sinusoidal potential as sketched in \cref{fig:FK_model} \cite{Manini_2016}. The hamiltonian is 
\begin{align}
  H = \sum_{i=1}^N \left[\frac{p_i^2}{2m} + \frac{1}{2}K(x_{i+1} - x_i - a_c)^2 + \frac{1}{2}U_0 \cos{\left(\frac{2\pi x_i}{a_b}\right)}\right],
  \label{eq:H_FK}
\end{align}
where the atoms are labelled sequently $i = 1, \hdots, N$. The first term $p_i^2/2m$ represents the kinetic energy with momentum $p_i$
and mass $m$. Often the effetcs of inertia are neglected, reffered to as the static \acrshort{FK} model, while the inclusion in \cref{eq:H_FK} is known as the dynamic \acrshort{FK} model \cite{FK2D}. The next term describes the harmonic interaction with elastic
constant $K$, nearest neighbour distance $\Delta x = x_{i+1} - x_i$ and 
corresponding nearest neighbour equilibrium distance $a_c$. The final term represents the periodic corrugation potential, with amplitude $U_0$ and period $a_b$. By comparison to the potential used in the \acrshort{PT} model \cref{eq:V_PT}, the only difference is the introduction of a harmonic coupling between particles in the chain as opposed to the moving body, and that we have not yet specified any force incentivizing sliding. Different boundary choices can be made where both free ends and periodic conditions gives similar results. The choice of fixed ends however makes the chain incapable of sliding.

\begin{figure}[H]
  \centering
  \includegraphics[width=0.8\linewidth]{figures/theory/FK_model.png}
  \caption{\hl{Temporary} figure from \cite{Manini_2016}}
  \label{fig:FK_model}
\end{figure}

To probe static friction one can apply an external adiabatically increasing force until sliding accours. This corresponds to the static \acrshort{FK} model, and it turns out that the sliding properties are entirely governed by its topological excitations referred to as so-called \textit{kinks} and \textit{antikinks}

\subsubsection{Commensurability} We can subdivide the frictional behaviour in terms of commensurability, that is, how well the spacing of the atoms match the periodic substrate potential. We describe this by the length ratio $\theta = a_b / a_c = N / M$ where $M$ denotes the number of minemas in the potential (within the length of the chain). A rational number for $\theta$ means that we can achieve a perfect alignment between the atoms in the chain and the potential minemas, without stretching the chain, corresponding to a \textit{commensurate} case. If $\theta$ is irrational the chain and substrate cannot fully align without some stretching of the chain, and we denote this as being \textit{incommensurate}.

We begin with the simplest commensurate case of $\theta = 1$ where the spacing
of the atoms matches perfectly with the substrate potential periodicity, i.e.\
$a_c = a_b$, $N = M$. The ground state (\acrshort{GS}) is the configuration
where each atom is aligned with one of the substrate minema. By adding an extra
atom to the chain we would effectively shift over some of the atoms, out of this
ideal state, giving rise to a kink excitation. This leads to the case where two
atoms will have to ``share'' the same potential corrugation as sketched in
\cref{fig:incommensurable_example}.  On the other hand, removing an atom from
the chain results in an antikink excitation where one potential corrugation will
be left ``atomless''. In order to reach a local minimum the kink (antikink) will
expand in space over a finite length such that the chain undertakes a local
compression (expansion). Notice that for low ratios of $\theta$, fewer atoms than minema, the chain will not be able to fill each corrugation well in any case, meaning that commensurability can instead be thought of as whether the atoms are forced to deviate, by a lattice spacing, from the spacing otherwise dictated by the spring forces inbetween. When applying a tangential force to the chain it is much
easier for an excitation to move along the chain than it is for the non-exicted
atoms since the activation energy $\epsilon_{PN}$ for a kink/antikink
displacement is systematically smaller (often much smaller) than the potential
barrier $U_0$. Thus, the motion of kinks (antikinks), i.e.\ the displacement of
extra atoms (atom vacancies), is represententing the fundamental mechanism for
mass transport. These displacements are responsible for the mobility,
diffusivity and conductivity within this model. 

In the zero temperature commensurable case with an adiabatical increase in force, all atoms would be put into an accelerating motion as soon as the potential barrier energy is present. However, just as discussed for the \acrshort{PT} model, thermal activations will excite the system at an earlier stage resulting in kink-antikink pairs traveling down the chain. For a chain of finite length these often accour at the end of the chain running in opposite direction. This cascade of kink-antikink exications is shown in \cref{fig:kink_antikink}. Notice, that for the 2D case, where an island (or flake) is deposited on a surface, we generally also expect the sliding to be initated by kink-antikink pairs at the boundaries. 
% As a kink travels down the chain the atoms are advanced by one lattice spacing $a_b$ along the substrate potential. 

\begin{figure}[H]
  \centering
  \includegraphics[width=0.8\linewidth]{figures/theory/kink_antikink.png}
  \caption{\hl{Temporary} figure from \cite{Manini_2016}}
  \label{fig:kink_antikink}
\end{figure}


For the case of incommensurability, i.e.\ $\theta = a_b/a_c$ is irrational, the
\acrshort{GS} is characterized by a sort of ``staircase''  deformation. That is, the chain will exhibit regular periods of regions where the chain is slightly compressed (expanded) to match the substrate potential, seperated by kinks (antikinks), where the increased stress is eventually released.
% as illustrated in \cref{fig:incommensurable_example} 
% through a localized expansion (compression) 

% \hl{Go through this last part  again. Even though this is what the source says I'm not quite sure I understand why it is not opposite ``...released through a localized compression (expansion)''?}.

\begin{figure}[H]
  \centering
  \includegraphics[width=0.5\linewidth]{figures/theory/incommensurable_example.png}
  \caption{\hl{Temporary} figure from
  url{http://www.iop.kiev.ua/~obraun/myreprints/surveyfk.pdf} p. 14.
  Incommensurable case ($\theta = ?$) where atoms sits slightly closer than
  otherwise dictated by the substrate potential for which this regularly result
  in a kink here seen as the presence of two atoms closely together in on of the
  potential corrugations.}
  \label{fig:incommensurable_example}
\end{figure}

% Maybe use the strength $\lambda = U_0 / (K a_b^2)$ with critical strength $\lambda_c$ instead of critical $K$?

The incommensurable \acrshort{FK} model contains a critical elastic constant $K_c$, such that for $K > K_c$ the static friction $F_s$ drops to zero, making the chain able to initiate a slide at no energy cost, while the low-velocity kinetic friction is dramatically reduced. This can be explained by the
fact that the displacement accouring in the incommensurable case will yield just
as many atoms climbing up a corrugation as there are atoms climbing down. For a big (infinite) chain this will exactly balance the forces making it
non-resistant to sliding. Generally, incommensurability guarantees that the
total energy (at $T=0$) is independent of the relative position to the
potential. However, when sliding freely, a single atom will eventually occupy a
maximum of the potential, and thus when increasing the potential magnitude $U_0$ or
softning the chain stiffness, lowering $K$, the possibility to occupy such a
maximum dissapears. This marks the so-called \text{Aubry transition},
at the critical elastic constant $K = K_c(U_0, \theta)$, where the chain goes
from a free sliding to a \textit{pinned} state with a nonzero static friction.
$K_c$ is a discontinuous function of the ratio $\theta$, due to the reliance on
irrational numbers for incommensurability. The minimal
value $K_c \simeq 1.0291926 $ in units $[2 U_0 (\pi / a_b)^2]$ is achieved for
the golden-mean ratio $\theta = (1+\sqrt{5}/2)$. Notice that the pinning is
provided despite translational invariance due to the inaccessibility to move
past the energy barrier which act as a dynamical constraint. The Aubry transistion can be invistigated as a first-order phase transistion for which power laws can be defined for the order parameter, but this is beyond the scope of this thesis.

The phenonema of non-pinned configurations is named \textit{superlubricity} in
tribological context. Despite the misleadning name this referes to the case
where the static friction is zero while the kinetic friction is nonzero but
reduced. For the case of a 2D sheet it is possible to alter the
commensurability, not only by changing the lattice spacing through material
choices, but also by changing the orientation of the sheet relative to the
substrate. Dienwiebel et al.\ \cite{DIENWIEBEL2005197} have shown that the
kinetic friction, for a graphene flake sliding over a graphite surface (multiple
layers of graphene), exhibits extremely low friction at certain orientations as
shown in \cref{fig:graphene_rot}. Here we clearly see that friction changes as a
function of orientation angles with only two spikes of considerable friction
force. This relates back to the concept of frictional regimes introduced through
the simpler \acrshort{PT} model, where the change in orientation affects the
effective substrate potential. Merely from the static consideration, we found that
lowering the potential amplitude $U_0$ will decrease the parameter $\eta =
2\pi^2U_0/a^2K$ shifting away from the regimeme of multiple slips towards smooth
sliding associated with low friction. Such transistions will also be affected by the shape of the
potential and corresponding 2D effects of the sliding path \cite{Yalin_2011}.
% Say more here

\begin{figure}[H]
  \centering
  \includegraphics[width=0.5\linewidth]{figures/theory/graphene_rot.png}
  \caption{\hl{Temporary} figure from \cite{DIENWIEBEL2005197} showing superlubricity for incommensurable orientations between graphene and graphite. \hl{temporary}}
  \label{fig:graphene_rot}
\end{figure}




% The reason that friction lowers for incommensurability is 
% probably that we avoid stick-slip behaviour which is highly 
% ``strongly dissipative stick-slip motion'' \cite{bonelli_atomistic_2009}


% Maybe check out for more info on the 1D model: Y. S. Kivshar O. M. Braun. The Frenkel-Kontorova Model. Springer, 1st edition, 2004.
% The Frenkel-Kontorova Model.pdf p. 38 3.2 Dynamics of Kinks <---------

%Based on \cite{FK2D}


% The \acrshort{FK} model can also describe phonons and heat in the lattices, which absorb the kinetic energy of the sliding[4]. \cite{FK2D}

% Consequently, the \acrshort{FK} model is the simplest model in which dynamic friction is emergent, while in other models some form of heuristic damping must be included.\cite{FK2D}
\subsubsection{Velocity resosnance} % Velocity resosnance 
While many of the same arguments used for the \acrshort{PT} model regarding velocity dependence for frictiona can be made for the \acrshort{FK} model, the addition of multiple atoms introduces the possibility of resonance. In the \acrshort{FK} model the kinetic friction is primarily attributed to resonance between the sliding induced vibrations and phonon modes in the chain \cite{FK2D}. The specific dynamics is found to be highly model and dimension specific, and even for the 1D case this is rather complex. However, we make a simplified analysis of the 1D rigid chain case to showcase the reasoning behind the phenomena.

When all atoms are sliding rigidly with center of mass velocity $v_{{\text{CM}}}$ the atoms will pass the potential maxima with the so-called \textit{washboard frequency} $\Omega = 2\pi v_{{\text{CM}}} / a_b$. For a weak coupling between the chain and the potential we can use the zero potential case as an approximation for which the known dispersion relation for the 1D harmonic chain is given \cite[p. 92]{Kittel2004}
\begin{align*}
  \omega_k = \sqrt{\frac{4 K}{m}} \left|\sin{\left(\frac{k}{2}\right)}\right|,
\end{align*}
where $\omega_k$ is the phonon frequency and $k = 2\pi i / N$ the wavenumber with $i\in [N/2, N/2)$. Resonance will accour when the washboard frequency $\Omega$ is close to the frequency of the phonon modes $\omega_q$ in the chain with wavenumber $q = 2\pi a_c / a_b = 2\pi \theta^{-1}$ or its harmonics $nq$ for $n = 1, 2, 3, \hdots$ \cite{van_den_Ende_2012}. Thus, we can approximate the resonance center of mass speed as
\begin{align*}
    n \Omega &\sim \omega_{nq} \\
    n \frac{2\pi v_{\text{CM}}}{a_b} &\sim \sqrt{\frac{4K}{m}} \left| \sin{\left(\frac{2n \pi \theta^{-1}}{2}\right)}\right| \\
    v_{\text{CM}} &\sim \frac{\sin{(n\pi \theta^{-1})}}{n \pi} \sqrt{\frac{Ka_b^2}{m}}.
\end{align*}
When the chain slides with a velocity around resonance speed, the washboard
frequency can excite acoustic phonons which will dissipate to other phonon modes
as well. At zero temperature, the energy will transform back and forth between
internal degrees of freedom and center of mass movement of the chain. Without any dissipation mechanism this is actually theorized to speed up the translational decay \cite{FK2D}. However, as soon as we add a dissipation channel through the substrate, energy will dissipate from the chain to the substrate degrees of freedom. This suggets that certain sliding speeds will exhibit relatively high kinetic friction while
others will be subject to relative low kinetic friction. Simulations of
concentric nanotubes in relative motion (telescopic sliding) supports this idea as it has revealed the
occurence of certain velocities at which the friction is enhanced, corresponding
to the washboard frequency of the system \cite{Manini_2016}, where the friction
response was observed to be highly non-linear as the resonance velocities were
approached. 

The analysis of the phonon dynamics is highly simplfied here, and a numerical study of the \acrshort{FK} by Norell et al.\ \cite{FK2D} showed that the behaviour was highly dependent on model parameter choices, but that the friction generally increased with velocity and temperature. Here the latter observation differs qualitative from that of the \acrshort{PT} model.



% This is strongly connected to the superlubricity term, although the phonon
% dynamics in this analysis is overly simplified, and additionaly we expect more
% complex resonance dynamics for higher dimensions. 


% A common way to model the non-zero temperature case is by the use of a Langevin
% thermostat, which models the dissipation of heat by adding a viscous damping
% force and thermal fluctuations by the addition of Gaussian random forces with
% variance proportional to the temperature (see \cref{sec:langevin} for more details). In combination, this gives rise to a kinetic
% friction that is both velocity and temperature dependent. By extending the \acrshort{FK} model into 2D \cite{FK2D} it can be shown numerically that
% the friction coefficient generally increases with increasing velocity and
% temperature resepectively, although the specific of the trend is highly
% sensitive to model parameters. 




% As the system is Hamiltonian (no heuristic damping) the total energy is conserved. Nevertheless, energy can be transferred from the centre of mass to the internal degrees of freedom, leading to the arrest of the chain in time. This effect can be interpreted as an effective friction. \cite{van_den_Ende_2012}


% The majority [6–12] examines the steady state of the dynamical \acrshort{FK} model in the presence of dissipation, rep- resenting the coupling of phonons to other, undescribed degrees of freedom. \cite{PhysRevLett.85.302}


% Static friciton behvaiour is robust and remains similar in 2D but dynamical friction is strongly influenced. 




% When considering the nonzero temperature thermal fluctuations can then overcome pinning effects even in fully commensurate cases.



% By applying a finite driving force it is known that a pinned configuration will go through several first-order dynamical pahse transistions as the system transfers from a pinned to a sliding state. 



% At face value, the transition from a static strained configuration to full
% sliding is conceptually as simple as overcoming an energy barrier. However,
% practical single- and multiple- contact conditions are characterized by
% complex interaction profiles plus nontrivial internal dynamics. As a result,
% the interplay of thermal drifts, contact ageing, contact-contact in-
% teractions, and macroscopic elastic deformations introduce significant
% complications, and make the depinning transition from static to kinetic
% friction an active field of research. The depinning dynamics affects in
% particular the transition between stick-slip and smooth slid- ing for sliding
% friction. (Current trends in the physics of nanoscale friction)


% In Atomic Force Microscopy (AFM) experiments, when the tip scans over the
% monolayers at low speeds, friction force is reported to increase with the
% logarithm of the velocity, similar to that observed when the tip scans across
% crystalline surfaces. This velocity dependence is interpreted in terms of
% thermally activated depinning of interlocking barriers involving interfacial
% atoms. (Current trends in the physics of nanoscale friction)






% "However, it is well known that continuum mechanics is valid only when the dimensions of the studied object are much larger than the length scale of the atomic disconti- nuity. Therefore, when the scale of the single asperity is on the order of nanometers, the effect of the discontinuity of the atoms within the tip and substrate can no longer be neglected [47]. Recent AFM experiments on metals revealed that friction varies little with increase of normal load in the low load regime [24, 74]. In addition, a newly invented AFM technique referred to as tip-on-top mode that enables researchers to manipulate nanoparticles of different size showed that in some cases friction increased linearly with interface area while in other cases near fric- tionless sliding was observed [75]. These conflicting results suggest that there may be a non-linear relationship between real contact area and friction." \cite{Yalin_2011}


% The essential difference between the FK and FKT models is that only the end atom is attached to the support in FK while all atoms are attached to the support in FKT. \cite{Yalin_2011}


% The friction comes from the viscous force which is proportional to sliding speed. Physically speaking, the viscous term is due to phonon excitation. So the term ‘‘superlubricity’’ which suggests cancelation of friction to zero may not be appropriate; structural lubricity proposed by Mu ̈ser is a better term to describe the phenomenon [77]. \cite{Yalin_2011}



\subsection{Frenkel-Kontorova-Tomlinson}
A final extension of the atomic models worth mentioning here is the
Frenkel-Kontorova-Tomlinson (\acrshort{FKT}) model \cite{weiss_dry_1997}, which
introduces a harmonic coupling of the sliding atom chain to the driving moving
body, effectively combining \acrshort{PT} and \acrshort{FK} (see \cref{fig:PT_FK_FKT}). This introduces
more degrees of freedom to the model based on the intention of getting a more realistic connection between the moving body and the chain. 
modelling of a broad contact point. Dong et al.\ carried out a numerical
analysis using the 1D \acrshort{FKT} model investigating the effect of chain
length. They observed that the friction increased linearly with number of atoms
in the chain on a long range, but certain lattice mismatch resulted in local non-linear relationship as shown in \cref{fig:FKT_contact}. Similar, taking the
\acrshort{FKT} model to 2D they were able to achieve a similar sensitivity to
commensurability as observed experimentally by \cite{DIENWIEBEL2005197} (shown in
\cref{fig:graphene_rot}) with the numerical result shown in \cref{fig:FKT_2D_rot}. Besides a recreation of the commensurability effect they also observed increasing friction with an increasing flake size. Combinned, the 1D and 2D results supports the idea of an increasing friction with contact size although it might showcase non-linear behvaiour depending on commensurability.


\begin{figure}[H]
  \centering
  \begin{subfigure}[t]{0.49\textwidth}
      \centering
      \includegraphics[width=\textwidth]{figures/theory/FKT_contact}
      \label{fig:FKT_contact}
      \caption{\hl{Temporary} figure from \cite{Yalin_2011} }
  \end{subfigure}
  \hfill
  \begin{subfigure}[t]{0.49\textwidth}
      \centering
      \includegraphics[width=\textwidth]{figures/theory/FKT_2D_rot.png}
      \label{fig:FKT_2D_rot}
      \caption{\hl{Temporary} figure from \cite{Yalin_2011} }
  \end{subfigure}
  \hfill
    %  \caption{}
    %  \label{fig:FKT_size}
\end{figure}




% \begin{figure}[H]
%   \centering
%   \includegraphics[width=0.5\linewidth]{figures/theory/FKT_contact}
%   \caption{\hl{Temporary} figure from \cite{Yalin_2011} }
%   \label{fig:FKT_contact}
% \end{figure}

% \begin{figure}[H]
%   \centering
%   \includegraphics[width=0.5\linewidth]{figures/theory/FKT_2D_rot.png}
%   \caption{\hl{Temporary} figure from \cite{Yalin_2011} }
%   \label{fig:FKT_2D_rot}
% \end{figure}

% thus makig it possible to investigate stick-slip features ??. The $\acrshort{FKT}$ made it possible to study the combinned interface incommensurability, finite-size effects, mechanical stiffness of the contacting materials and normal load variations \cite{Manini_2016}. 

% Due to the lack of connection between the chain and the moving body the \acrshort{PT} and \acrshort{FK} model was merged by including a harmonic coupling between each of the atoms to the moving body \cite{kim_nano-scale_2009}. 


\subsection{Shortcomings of atomic models}
\hl{To-DO: Shortcomings of PT-based reduced-models} % \cite{Yalin_2011}. 
\begin{itemize}
  \item Assummes a rigid substrate with a simplified potential shape. 
  \item Energy dissipation is added through a viscous term $-m\mu \dot{x}$ being the only dissipation channel availble. Does not capture a more complex real life electron and phonon dissipation. Taking phonon dissipation as an example there are many virbaiton modes (3N). This will effect the thermal activationm derivation. 
  \item The moving body is simplified as constant moving rigid body, while in fact this will also be subject to a more complex dynamic behavior.
\end{itemize}


% However, Weiss and Elmer (1995) proposed that the model had a deficiency. They suggested that in the FK model, there was no connection between the atoms and the sliding body. Therefore, Frenkel-Kontorova-Tomlinson (FKT) model that combines the FK model with the Tomlinson model was proposed. \cite{kim_nano-scale_2009}


% The Frenkel-Kontorova-Tomlinson (FKT) model [61, 62] introduces an harmonic
% coupling of the sliding atomic chain to a driving moving body, thus making it
% possible to investigate stick-slip features in a 1D extended simplified
% contact. The FKT framework provided the ideal platform to investigate the
% tribological consequences of combined interface incommensurability,
% finite-size effects, mechanical stiffness of the contacting materials, and
% normal-load variations \cite{Manini_2016}.


% Important generalizations involving increased dimensionality compared to the
% regular FK model bear significant implications for tribological properties
% such as critical exponents, size-scaling of the friction force, depinning
% mechanisms, and others. \cite{Manini_2016}

% Maybe check this out: An interesting example of such a transient is the
% depinning of an atomic monolayer driven across a 2D periodic substrate profile
% of hexagonal symmetry [83].  \cite{Manini_2016}


% \subsection{Other stuff}


% At nanoscales things get a bit more unclear. SFM (explain) experiments have
% reported (copy sources 5, 6, 21 from \cite{mo_friction_2009}) where $F_f \propto
% F_N$ or even with these quantities being nearly independent of each other.

% \cite{physicsworld_2005}

% Physically relevant quantities, including the average friction force, the slider and the lubricant mean velocities, several correlation functions, and the heat flow can be evaluated numerically by carrying out suitable averages over the model dynamics of a sliding interface, as long as it is followed for a sufficiently long time. The modeling of friction must first of all address correctly ordinary equilibrium and near-equilibrium phenomena, where the fluctuation-dissipation theorem (Sec. 2) governs the smooth conversion of mechanical energy into heat, but most importantly it must also deal with inherently nonlinear dissipative phenomena such as instabilities, stick-slip, and all kinds of hysteretic response to external driving forces, characteristic of non-equilibrium dynamics. 
% \cite{Manini_2016}


% In several works by J. Fineberg’s group [2–4] the transition from sticking to
% sliding is characterized by slip fronts propagating along the interface.
% \cite{Manini_2017}[p. 2]. 



% As expected, high levels of
% friction were present in the commensurate positions and extremely low friction
% was found when the surfaces were incommensurate.
% (\url{https://physicsworld.com/a/friction-at-the-nano-scale/})


% Superlubricity, now a pervasive concept of
% modern tribology, dates back to the math- ematical framework of the Frenkel
% Kontorova model for incommensurate interfaces [40]. When two contacting
% crystalline workpieces are out of registry, by lattice mismatch or angular
% misalignment, the minimal force required to achieve sliding, i.e. the static
% friction, tends to zero in the thermodynamic limit – that is, it can at most
% grow as a power less than one of the area – provided the two substrates are
% stiff enough. (Current trends in the physics of nanoscale friction)


% Superlubricity is experimentally rare. Until recently, it has been
% demonstrated or im- plied in a relatively small number of cases [29, 42–46].
% There are now more evidences of superlubric behavior in cluster
% nanomanipulation [32, 33, 47], sliding colloidal layers [48–50], and
% inertially driven rare-gas adsorbates [51, 52]. (Current trends in the physics
% of nanoscale friction)


% A breakdown of structural lubricity may occur at the heterogeneous interface
% of graphene and h-BN. Because of lattice mismatch (1.8\%), this interface is
% intrinsically incommen- surate, and superlubricity should persist regardless
% of the flake-substrate orientation, and become more and more evident as the
% flake size increases [57]. However, vertical cor- rugations and planar strains
% may occur at the interface even in the presence of weak van der Waals
% interactions and, since the lattice mismatch is small, the system can de-
% velop locally commensurate and incommensurate domains as a function of the
% misfit angle [58, 59]. Nonetheless, spontaneous rotation of large graphene
% flakes on h-BN is observed after thermal annealing at elevated temperatures,
% indicative of very low friction due to incommensurate sliding [60, 61].
% (Current trends in the physics of nanoscale friction)

% Indeed, we know from theory and simulation [74–76] that even in clean wearless
% friction experiments with perfect atomic structures, superlubricity at large
% scales may, for example, surrender due to the soft elastic strain deformations
% of contacting systems. (Current trends in the physics of nanoscale friction)



% \section{Multi scale models?}
% \cite{Manini_2016} p. 24.


% Might find something interesting here \cite{zhao_thermally_2007} or \cite{PhysRevE.71.065101}.

% Find a suitable place to introduce smooth sliding. Above certain velocities the stick-slip motion dissapear. \cite[p. 142-ish]{gnecco_meyer_2015}

\subsection{Experimental procedures}
% \cite{gnecco_meyer_2015}
% Check out this \cite{Dong_2013} parameter choices for MD



Experimentally, the study of nanoscale friction is challenging due to the low
forces on the scale of nano-newtons along with difficulties of mapping the
nano-scale topography of the sample. In opposition to numerical simulations, which provides full
transparency regarding atomic-scale structures, sampling of forces, velocities
and temperature, the experimental results are limited by the state-of-the-art
experimental methods. In order to compare numerical and experimental results it is useful to adress the most common experimental.

\subsubsection{Scanning Probe Microscopy}\label{sec:SPM} Scanning probe
microscopy (\acrshort{SPM}) includes a variety of experimental methods which is used to
examine surfaces with atomic resolution \cite[pp. 6-27]{BHUSHAN20051507}. This was
orginally developed for surface topography imaging, but today it plays a crucial
role in nanoscale science as it is used for probe-sampling regarding
tribological, electronic, magnetic, biological and chemical character. The
famility of methods involving the measurement of forces is generally referred to
as \textit{scanning force microscopies} (\acrshort{SFM}) or for friction purposes
\textit{friction force microscopes} (\acrshort{FFM}).

One such method arose from the \textit{atomic force microscope} \acrshort{AFM}, which consist
of a sharp micro-fabricated tip attacthed to a contilever force sensor, usually
with a sensitivy below 1 nN all the way down to pN. The force is measured by recording the bending of
the cantilever, either as a change in electrical conduction or more commonly, by
a light beam reflected from the back of the cantilever into a photodetector
\cite[p. 183]{gnecco_meyer_2015}. By adjusting the tip-sample height to keep a constant
normal force while scanning accross the surface this can be used to produce a
surface topography map. By tapping the material (dynamic force microscopy) with
sinusoidally vibrated tip the effects from friction and other disturbing forces
can be minimized in order to produce an even clearer image (\hl{include example,
preferable showing the surface structure of graphene}). However, when scanning
perpendicularly to the cantilever axis, one is also able to measure the
frictional force as torsion of the cantilever. By having four quadrants in the
photodetector (as shown in figure \cref{fig:AFM}), one can simultaneously
measure the normal force and friction force as the probes scans accross the
surface. 

\begin{figure}[H]
  \centering
  \includegraphics[width=0.6\linewidth]{figures/theory/AFM.png}
  \caption{\hl{Temporary} figure from \cite[p. 184]{gnecco_meyer_2015}}
  \label{fig:AFM}
\end{figure}


\acrshort{AFM} can also be used to drag a nanoflake accross the substrate as done by
Dienwiebel et al.\ \cite{DIENWIEBEL2005197}, where a graphene flake was attahced
to a \acrshort{FFM} tip and dragged accross graphite. Notice that this makes the normal
loading concentrated to a single point on the flake rather than achieving an evenly distributed load. 



\subsubsection{Surface Force Apparatus}
The Surface force apparatus \acrshort{SFA} is based on two curved molecuraly smoooth surfaces brought into contact \cite[p. 188]{gnecco_meyer_2015}. The sample is placed in between the two surfaces as surfaces as lubricant film for which the friction properties can be studied by applying a tangential force to the surfaces. 



% The trouble is that the coefficients of friction measured in nanotribological
% experiments and in macroscopic “tribotests” routinely differ by orders of
% magnitude. (\url{https://physicsworld.com/a/friction-at-the-nano-scale/})





% \subsection{Graphene friction}
% Theory of friction experiment involving graphene.


% Read Rare-gas islands and metal clusters \cite{Manini_2016} for theory of effects on surface area scaling

% Because of this frictional reduction, many studies indicate graphene as the
% thinnest solid-state lubricant and anti-wear coating [104–106]. (Current trends
% in the physics of nanoscale friction)


% Accurate FFM measurements on few-layer graphene systems show that friction
% decreases by increasing graphene thickness from a single layer up to 4-5 layers,
% and then it approaches graphite values [97, 99, 101, 107, 108]. (Current trends
% in the physics of nanoscale friction)




\section{Expected frictional properties of graphene}\label{sec:expected_prop}
% Quick thought: All the three methods in the following is essentialy the same, but the differ in area of contact and force distribution. Even the sharpest AFM tip consist of a few atoms which is just like a small sheet. The SFA takes on a bigger area with even load while the flake is in AFM experiments loaded at a small area while our is loaded at to areas making it less free to rotate. But all in all, they are not so different in the end. The asperity aspect is however a bit different with the AFM tip being able to deform elastically. 

Several studies have investigated the frictional behaviour of graphene by
varying different parameters such as normal force, sliding velocity,
temperature, commensurability and graphene thickness
\cite{penkov_tribology_2014}. In general, we find three types of relevant
systems being studied: 1) An \acrshort{FFM} type setup where the graphene,
either resting on a substrate or suspended, is probed by an \acrshort{AFM} tip
scanning across the surface. 2) \acrshort{SFA} approach with the graphene
``sandwicheded'' in between two substrate layers moving relative to each other
using the graphene as a solid lubricant. 3) A graphene flake sliding on a
substrate, either being dragged by an \acrshort{AFM} tip or by more complex
arrangements in numerical simulations. Considering that even the sharpest
\acrshort{AFM} tip will effectively put multiple atoms in contact with the test
sample, all methods relates to a nanoscale contact involving graphene but differs on contact area. However, the \acrshort{FFM} type suggest a basis in asperity theory as we expect it to deform with increasing load, while the latter two is more alligned with the \acrshort{PT} type models and our system our interest which is an atomic flat sheet on a flat substrate. Having said that, we consider all
three types with the purpose of gaining a more comprehensive insight. The relevant studies considered in the following are are listed in \cref{tab:friction_ref}
for convenience. 



\begin{table}[H]
  % \begin{center}
  \centering
  \caption{\hl{Update multirow line span after completing the table...}}
  \label{tab:friction_ref}
  \begin{tabular}{ |M{1cm}|M{1cm}|M{1.5cm}|X{2.5cm}|X{4cm}|X{4cm}| } \hline
  System & Type & Year & Researcher & Materials & Key words \\ \hline
  \parbox[t]{2mm}{\multirow{12}{*}{\rotatebox[origin=c]{90}{FFM}}} & \multirow{3}{*}{Exp.} & 2007 \cite{zhao_thermally_2007} & Zhao et al.\ & Si\textsubscript{3}N\textsubscript{4} tip on graphite. & Temperature dependence \\ \cline{3-6} 
  & & 2015 \cite{Paolicelli_2015} & G. Paolicelli et al.\ & Si tip, graphene on SiO2 and Ni(111) substrate  & Layers, load, shear strength \\ \cline{2-6} 
  & \multirow{4}{*}{Both}& 2019 \cite{zhang_tuning_2019} & Zhang et al.\ & Monolayer graphene  & Straining sheet \\ \cline{3-6} 
  &  & 2019 \cite{Vazirisereshk_2019} & Vazirisereshk et al.\ & Graphene,  MoS\textsubscript{2} and Graphene/MoS\textsubscript{2} heterostructure & Low friction? \\ \cline{2-6} 
  & \multirow{4}{*}{Num.} & 2015 \cite{Yoon2015MolecularDS} & Yoon et al.\ & Si tip, graphene on SiO\textsubscript{2} & Stick-slip: tip size, scan angle, layer thickness, substrate flexibility \\ \cline{3-6} 
  & & 2016 \cite{li_evolving_2016} & Li et al.\ & Si tip, graphene on a-Si substrate & Increasing layers \\ \cline{1-6} 
  \parbox[t]{2mm}{\multirow{3}{*}{\rotatebox[origin=c]{90}{SFA}}} & \multirow{4}{*}{Num.} & 2011 \cite{Wijn_2011} & Wijn et al.\ & Graphene flakes between graphite  & Rotational dynamics, superlubricity, temperature  \\ \cline{3-6} 
  & & 2012 \cite{Kim_2012} & H.\ J.\ Kim and D.\ E.\ Kim. & Carbon sheet  & Corrugated nano-structured surfaces  \\ \cline{1-6} 
  \parbox[t]{2mm}{\multirow{12}{*}{\rotatebox[origin=c]{90}{Flake}}} & \multirow{3}{*}{Exp.} & 2005 \cite{DIENWIEBEL2005197} & Dienwiebel et al.\ & Graphene on graphite & Commensurability, superlubricity  \\ \cline{3-6} 
   &  & 2013 \cite{feng_superlubric_2013}  & Feng et al.\ & Graphene on graphite &  Free sliding (relevant?)  \\ \cline{2-6} 
   & \multirow{9}{*}{Num.} & 2009 \cite{bonelli_atomistic_2009} & Bonelli et al.\ & Graphene on graphite  & Tight-binding, commensurability, load, flake size \\ \cline{3-6} 
   &  & 2012 \cite{Reguzzoni_2012} & Reguzzoni et al.\ & Graphene on graphite & Graphite thickness  \\ \cline{3-6} 
   &  & 2014 \cite{liu_high-speed_2014} & Liu et al.\ & Graphene on graphite & Thickness, deformations, high speed \\ \cline{3-6} 
   &  & 2018 \cite{zhu_study_2018} & P. Zhu and Li & Graphene on gold & Flake size, commensurability  \\ \cline{3-6} 
   &  & 2019 \cite{ma12091425} & Zhang et al.\  & Graphene on diamond & Temperature, sliding angle, friction coeffcient  \\ \cline{1-6} 
  \end{tabular}
  % \end{center}
\end{table}

% Recent AFM experiments on metals revealed that friction varies little with increase of normal load in the low load regime [24, 74]. In addition, a newly invented AFM technique referred to as tip-on-top mode that enables researchers to manipulate nanoparticles of different size showed that in some cases friction increased linearly with interface area while in other cases near fric- tionless sliding was observed [75]. These conflicting results suggest that there may be a non-linear relationship between real contact area and friction. \cite{Yalin_2011}

One of the earlist tribological simulations of graphene was carried out by
Bonelli et al. \cite{bonelli_atomistic_2009} in 2009 using a tight-binding
method (excluding thermal excitations) to simulate a graphene flake on an
infinite graphene sheet \cite{penkov_tribology_2014}. They implemented a
\acrshort{FKT}-like setup where each atom in the flake is coupled horizontally to a rigid
support by elastic springs. They recovered the stick-slip behaviour, which is
also observed in \acrshort{FFM} setups both experimentally
\cite{zhao_thermally_2007, zhang_tuning_2019} and numerically
\cite{li_evolving_2016, zhu_study_2018}, and they found an agreement with
the qualitative observation that soft springs allows for a clean stick-slip
motion while hard springs inhibit it ($\lesssim$ 40 N/m). In \acrshort{AFM} and
\acrshort{SFA} experiemnts, the stick-slip motion tend to transistion into
smooth sliding when the speed exceeds $\sim \SI{1}{\mu/s}$ while in \acrshort{MD} modelling the same transistion is observed in the $\sim \SI{1}{m/s}$ region
\cite{Manini_2016}. This 6 order of magnitude discrepancy has been largely
discussed in connection to simplifying assumptions in \acrshort{MD} simulations. 

Bonelli et al.\ \cite{bonelli_atomistic_2009} also found that commensurability,
through orientation of the flake and the direction of sliding, had a great impact on the frictional behaviour which generally alligns with the
predictions of the \acrshort{FK} and \acrshort{FKT} models. They confirmed qualitatively the experimental observation of superlubricity for certain incommensurable
orientations as shown experimentally by Dienwiebel et
al.\cite{DIENWIEBEL2005197} and further supported by experimental measurements
of interaction energies by Feng et al.\ \cite{feng_superlubric_2013}. This commensurability importance is also reported for \acrshort{MD} simulations \cite{ma12091425, zhu_study_2018, Wijn_2011}. Bonelli et al.\ found the friction force and coefficient to be one order of magnitude higher than that of the experimental results which they attribute to the details of the numerical modelling. Generally the experimental coefficients between graphite and most materials lies in the range 0.08-0.18 \cite{DIENWIEBEL2005197} and while Dienwiebel et al.\ \cite{DIENWIEBEL2005197}
reported a wide range of frictional forces all the way from $28 \pm 16$ pN to $453 \pm 16$ pN in a load range $\sim [-10, 20] nN$ the change in friction with applied load was as low as 0.05-0.4 \% (coef. of 0.0005-0.004). This indicates an almost independent relationship between friction and load which they attributed to the lack of change in contact area. 

Furthermore, Bonelli et al.\ \cite{bonelli_atomistic_2009} found friction to decrease with flake size which is mainly attributed to the idea that boundary
atoms are dominant in increasing friction, but also due to the fact that the coupling to the support made for a decreased rotational freedom as size increases, which could then be attributed to a forced path decreasing the tendency of stick-slip behvaiour. This disagrees with the \acrshort{FK} and \acrshort{FKT} model which predicts the reverse, an increase in friciton with increasing size, but this might be a shortage to the simplicity of the reduced-models. However, the decreasing friction with increasing flake size is also found for graphene on gold \acrshort{MD} simulation \cite{zhu_study_2018}. This can however be attributed to commensurability as a numerical \acrshort{MD} study of monolayer islands of krypton on copper by Reguzzoni and Righi \cite{PhysRevB.85.201412} reports that the effective commensurability increase drastically below a critical flake radius on the order of $10$ Å. In a numerical study
by Varini et al.\ \cite{Varini_2015}, based on Kr islands adsorbed on Pb(111), this is further elaborated as they found that finite size effects are especally important for static friction as a pinning barrier rise from
the edge (preventing otherwise superlubricity due to incommensurability). They reported a relationship $F_s \sim A^{\gamma_s}$ not only sublinear $\gamma_s <
1$ but also sublinear with respect to the island perimeter $P \propto A^{1/2}$ by having $\gamma_s = 0.25$ for a hexagonal edge and $\gamma_s = 0.37$ when circular, indicating that only a subset of
the edge is responsible for the pinning effect. This alligns with the general change in friction found by \cite{zhu_study_2018} for different flake geometries (square, triangle, circle). Additionally, Varini et al.\ also found the edge pinning effect to decrease with increasing temperature as the edge energy barriers are reduced. Bringing this all together, the main picture forming is that flake size, which we can consider as contact area, is affecting friction through a commensurability mechanism. If the flake is constrained in some way we might not observe the same dependence. While flake size nor contact area is easily measured in experimental \acrshort{FFM}  Mo et al.\ \cite{mo_friction_2009} found that $F_{\text{fric}} \propto A$ where $A$ is the real contact area defined by atoms within chemical range.

Evolution effects, or so-called friction strengthening, are also observed, meaning that the friction force increases during the initial stick-slip cycles. This is observed experimentally by Zhang et al.\ \cite{zhang_tuning_2019} and numcerically by Li et al.\ \cite{li_evolving_2016}. However, this is only found when having the graphene sheet resting on a substrate
\cite{zhang_tuning_2019} opposed to a suspended sheet, and it was found to
dimish when increasing number of graphene layers stacked (graphite)
\cite{li_evolving_2016}. In general, the friction was found to decrease with
increasing number of layers which is also supported by the findings in
\cite{Yoon2015MolecularDS} but dissagree with \cite{Reguzzoni_2012}?? Zhang et
al. \cite{zhang_tuning_2019} additionally found that straining a suspended
stretch, modulating the flexibility which consequently change the local pinning
capability of the contact interface and lowers the dynamic friction. Another surface manipulating study was performed by H.\ J.\ Kim and D.\ E.\ Kim. \cite{Kim_2012} where the investigated the effects of corrugated
nano-structured surfaces which altered the contact area and structural stiffness resulting in both increased and slightly decreased friction under certain load ranges. These studies highlight the importance of surface structure and mechanical conditions. 

% The high-speed sliding friction of graphene and novel routes to persistent superlubricity talks about achieving superlubricity by stretching the graphite substrate.

\hl{TO-DO: Negative friction coefficients} 


% Normal load
The dependency on friction of normal load turns out to be a complex matter and
has proven to be highly system dependent. As already mentioned, asperity theory
mainly point to a sublinear relationship between friction and load, while the
\acrshort{PT} models points to a dependence through the change of the effective
substrate potential leading to a commensurability effect. Experimentally rather
different trends have been observed, although the majority agree on an
increasing friction with increasing load \cite[p. 200]{gnecco_meyer_2015}. For
the graphene flake Dienwiebel et al.\ \cite{DIENWIEBEL2005197} found a seemingly
non-dependent relationship while \acrshort{FFM} study by G. Paolicelli et al.\
\cite{Paolicelli_2015} found a great fit with the sublinear predictions of
Maugis-Dugdale theory $(F_f \propto (F_N - F_{N,0})^{2/3})$. Here the
discrepancy might lie in the difference between a spherical tip indenting the
graphene sheet (matching asperity theory) as opposed to atomic flatness of the
graphene/graphite sheets in contact making for a constant contact area. However,
numerical studies with graphene in contact with graphite still find both
sublinear \cite{bonelli_atomistic_2009} and linear \cite{ma12091425} load
dependencies.
% Look at Vazirisereshk 
% maybe Zhang tuning

The dependency of velocity is generally found to increase logarithmically with
velocity in experimental \acrshort{AFM} studies \cite[p. 201]{gnecco_meyer_2015}
which match the low velocity regime of the \acrshort{PT} type models. At higher
velocities thermally activated processes are less important and friction becomes
independent of velocity according to the continuation of the
\cref{eq:F_thermal_ac} without entering the athermal regime related to the
\acrshort{PT} model which is attributed to a numerical damping effect. The
saturation of the velocity dependency has been observed numerically for Si tips
and diamond, graphite and amorphous carbon surfaces with scan velocities above
\SI{1}{\mu/s} \cite{zworner1998velocity}. Guerra et al.\ \cite{Guerra_2010}
studying gold clusters on graphite using \acrshort{MD} simulations  found a
viscous friction response, friction proportional to sliding velocity,
in both low and high speed domains. However, thermal effects reversed: at low
speed (diffusive) friction decreased with increased temperature while at high
speed (ballistic) speed friction increases with temperature. In the
\acrshort{MD} simlations the crossover from ballistic to diffusive occoured
between 10 and 1 m/s. 

% Temperature
For the temperature the general experimental trend is an decreasing friction with increasing temperature as found by  Zhao et al.\ \cite{zhao_thermally_2007} in a series of \acrshort{AFM} graphene on graphite experiments with $F_{\text{fric}} \propto \exp{(1/T)}$. This agrees with the thermal drift regime of the \acrshort{PT} type models even though the temperature range used in the study does not match the range of this regime according to the \acrshort{PT} model. Wijn et al.\ \cite{Wijn_2011} find that friction commensurability can be lost at higher temperature (above 200K) were the found a power law behaviour $F_k \propto T^{-1.13 \pm0.04}$. Numerically, Zhang et al.\ \cite{ma12091425} found that friction increased with temperature, using a velocity of 10 m/s. Considering the findings of \cite{Guerra_2010} related to \acrshort{MD} this qualitative different dependence might be due the to low speed diffisuve behvaiour as opposed to high speed ballistic behaviour in \acrshort{MD} simulations. 

A summary of the expectations is given in \cref{tab:exp_summary}.

% Textwidth = 484,20988 pt*0,035146 pt/cm = 17,02 % \the\textwidth
\begin{table}[H]
  \begin{center}
  \caption{Summary}
  \label{tab:exp_summary}
  \begin{tabular}{  M{5cm}  X{12cm} } \hline
  \textbf{Stick slip} & Generally we expect to observe periodic stick-slip motion with a period mathing the lattice constant(s) involved \cite{mo_friction_2009}. This is however inhibited for high stiffness of the spring coupling to a driving support and at large sliding velocity. \\ \\
  % Flake size and temperature? 
  \textbf{Static friction} & The static friciton is highly related to the presence of stick-slip motion. The static friction is most pronounced for commensurable configurations and will decrease drastically for incommensurability. However, further reducton of static friction is expected for an increasing flake size and increasing temperature. \\ \\
  \textbf{Commensurability} & Both static and dynamic friction is expected to be highly sensitive to commensurability, through lattice spacing, orientation of the flake relative to the substrate and by the path of sliding along the substrate. By changing the spring stiffness of the coupled driving support we expect to get a response in commensurability due to a change in translational freedom. \\ \\
  \textbf{Friction evolution} \linebreak \textbf{(Friction strengthening)} & Friction evolution is found to be present in mono layer graphene resting on a substrate, and thus we expect this to be present in our simulaiton setup as well.  \\ \\
  \textbf{Negative coef} & \hl{TO-DO} \\ \\
  \textbf{Normal load} & Generally an increasing friction force is expected with increasing load. Both non-dependent, sublinear and linear relationship can be expecteded here. \\ \\
  \textbf{Velocity} & Generally an increasing friction force is expected with increased sliding velocity. Experimental results suggest that kinetic friction goes as $F_k\propto \ln{(v)}$, with the expectation that friction become independent of velocity at ``high'' velocities above \SI{1}{\mu/s}. Numerically a viscous $F_k \propto v$ is expected for all velocity ranges. \\ \\
  \textbf{Temperature} & Experimentally and numerically friction is expected to decrease with friction in a power law or expontial manner. However, for high velocity ranges, according to a ballistic regime, which might coincide with the capabilities of \acrshort{MD}, the friction is predicted to increase with temperature. \\ \\
  \textbf{Contact area} & For our system we do not expect any contact area changes during load, however an increasing number of atoms in contact is expected to increase friction.  \\
  \hline
  \end{tabular}
  \end{center}
\end{table}



% Non rigid flake which could deform and rotate,
% but(?) the graphene flake was fixed at a certain angle after applying normal
% force to reproduce \acrshort{AFM} experiments. They found stick-slip behvaiour.
% The flake size and ``stacking angle'' affected the frictional behaviour, such
% that a large area resulted in experienced less friciton due the idea that the
% reactive atoms at the boundary were dominant in increasing friction. The
% influence of the stacking angle was attributed to the commensurability effect
% highlighted by the \acrshort{FK} model. Similar bahviour in various structures
% such as carbon nano tubes CNT and mica (what is mica) have previously been
% reported \cite[41-42]{penkov_tribology_2014}. It has also been found that
% graphene (the ideal nano-structure) exhibits superlubricity in cases of lattice
% mismatch.

% % Thickness of graphene substrate
% Reguzzoni et al. \cite[33]{penkov_tribology_2014} carried out a MD simulation of a graphene flake sliding sliding over a multi-layer (1-4) graphite substrate. They found that out-of-plane deformation of the graphene sheet (which one) increased with an increasing number of graphene layers. Meaning, that the vertical contatc stifness decreased with graphene thickness. Because frictional force and stick-slip motion were found to increase with decreasing stifness if was concluded that the single layer graphene were the best candidate for a solid nano-lubricant film. Moreover, the authors reported that the out-of-plane deformation nd the shear deformation induced by the stick-slip motion were the dominant factors incluencing friction.  


% %CNT probing suspended graphene sheet
% Smolyanitsky et al. \cite[34]{penkov_tribology_2014} performed a \acrshort{MD} simulation of a CNT probing a suspended graphene sheet. They found that for a positive normal force the friction generally increased with normal force, while for a negative normal force (as the tip pulled the suspended graphene due to adhesion) the friction increased for a decreasing normal force. This is in practive signs of a negative friction coefficient the negative range of normal load.  


% Various simulation methods such as Molecular Dynamics (MD), Monte Carlo and ab initio calculations have been applied to investigate the diverse characteristics of graphene \cite{penkov_tribology_2014}. Especially \acrshort{MD} simulations for tribological properties have been used in the recent years.


% --- Lowering friction with strain --- %
% Tuning friction to a superlubric state via in-plane straining


% Also \cite{gao_frictional_2004} might be usefull to go through again

% Maybe rememeber to talk about the properties of asperity systems as the sheet creates asperities as it buckles. 

% Include "Molecular dynamics simulation of atomic-scale frictional behavior of corrugated nano-structured surfaces" somewhere.

% The structural setup of our simulation is most reminiscent a graphene flake sliding on a substrate. This has been studied numerically in molecular dynamic simulations by Zhu and Li \cite[2018]{zhu_study_2018} for a graphene flake on a gold substrate and by Zhang et al.\ \cite{ma12091425}(2019) on a diamond substrate, and in a tight-binding simulation by Bonelli et al.\ \cite{bonelli_atomistic_2009}(2009) for graphene on graphite. Experimental studies of a graphene flake attatched to an AFM is done by Dienwiebel et al.\ \cite[2005]{DIENWIEBEL2005197} and Feng et al.\ \cite[2013]{feng_superlubric_2013} sliding on graphite, but these are mainly concerned with superlubricity due to flake oreientation commensurability. 

% In our study we simmulate a graphene flake on a silicon substrate which deviates slightly from the above-mentioned references by having a different material combination. Additionally, the normal force is only applied to the ends of the sheet which might have an important effect. Obviously stretching and cutting the sheet will seperate our study dramatically from the references, but we aim to compare the frictional properties to the references before applying stretch or cuts. 



% ---- Quantitative --- %

% See comparison on friction coef in the one with diamond study 



% \subsubsection{Qualitatively}
% \begin{enumerate}
%   \item \textbf{Stick slip}: Generally we expect to observe periodic stick-slip motion with a period mathing the lattice constant(s) involved \cite{mo_friction_2009}. This was both present in the MD simulations \cite{zhu_study_2018}, \cite{ma12091425} and in the experiment by \cite{DIENWIEBEL2005197}. In AFM and SFA experiemnts, the stick-slip motion tend to transistion into smooth sliding when the speed exceeds $\sim \SI{1}{\mu/s}$ while in MD modelling the same transistion is observed in the $\sim \SI{1}{m/s}$ region \cite{Manini_2016}. Since we use a sliding speed of \SI{20}{m/s} we might transistion into smooth sliding. This 6 order of magnitude discrepancy has been largely discussed in connection to simplifying assumptions in MD simulations. Bonelli et al.\ \cite{bonelli_atomistic_2009} found that the stick-slip behaviour was present when the cantilever-tip-flake coupling was done with a relatively soft springs in contrast to hard springs which inhibited it.   

%   \item \textbf{Static friction}: As highlighted in the FK model static friction
%   will be sensitive to commensurability, which will additionally be affected by
%   flake size. Reguzzoni and Righi \cite{PhysRevB.85.201412} have shown that the
%   effective commensurability will increase drastically below a critical flake
%   radius on the order of $10$ Å. Macroscopically we expect to see a logarithmic
%   increase in static friction with contact time before sliding
%   \cite{dieterich_1972}, and hence due to the short time-span of the static
%   contact before dragging, it is not obvious to determine whether a significant
%   static friction peak will be found. Also, the static friction best asserted by
%   increasing the tangential sliding force slowly which is not the case in our
%   simulation, especially considering that we are going to move the sheet rigidly
%   (infinte spring constant) without any slack of a soft spring. Edge related
%   origin of the pinning effects suggest that static friction can increase with
%   sheet size up to a factor $A\propto A^{1/2}$, but this is also reported to be specific of the contact shape \cite{Manini_2016}.
  
%   % Talk about size effects?

%   % Concerning stick-slip friction, another problem is that, unlike simulations, real experiments contain mesoscale or macroscale component intrinsically involved in the mechanical instabilities of which stick-slip consists. Here the comforting observation is that stick-slip is nearly independent of speed, so that so long as a simulation is long enough to realize a sufficient number of slip events, the results may already be good enough [148]  \cite{Manini_2016}.


%   % A serious aspect of stick-slip friction which MD simulation is unable to attack is ageing. The slip is a fast event, well described by MD, but sticking is a long waiting time, during which the frictional contact settles very slowly. The longer the sticking time, the larger the static friction force necessary to cause the slip. Typicall experiments show a logarithmic increase of static friction with time [150] \cite{Manini_2016}.
  
%   % For monolayers sliding along atomically uniform substrates, however, there
%   is % essentially no static friction. Indeed, the friction in these systems
%   can be % up to 105 times less than that for macroscopic lubricants such as
%   graphite. % This raises questions about the fundamental dissipation
%   mechanisms that are at % work in systems at different scales. %
%   (\url{https://physicsworld.com/a/friction-at-the-nano-scale/})

%   \item \textbf{Oritentation (friction anisotropy)}: As predicted by the FK model and confirmed both numerically \cite{zhu_study_2018}, \cite{ma12091425} and experimentally \cite{DIENWIEBEL2005197}, \cite{feng_superlubric_2013} we expect a dependence of friction force on orientation due to changing commensurability. Zhu and Li \cite{zhu_study_2018} (gold substrate) reported the highest friction when sliding along the armchair direction, while Zhang et al.\ \cite{ma12091425} (diamond substrate) found the zigzag-direction to give the highest friction force (also the most evident stick-slip behvaiour in this direction). 
% \end{enumerate}


% By consulting with the most related numerical and experimental studies, while also considering the theoretical gap between smooth atomic contact and asperity theory, we have summarized an evaluation of the most important quantative trends expected from our numerical results in \cref{var_dep}. 






% \begin{table}[H]
%   \begin{center}
%   \caption{Qualitative trends}
%   \label{tab:qual_exp}
%   \begin{tabular}{  M{5cm}  X{12cm} } \hline
%   \textbf{Normal load} & Generally an increasing friction force is expected with increasing load. Both non-dependent, sublinear and linear relationship can be expecteded here. \\ \\
%   \textbf{Velocity} & Generally an increasing friction force is expected with increased sliding velocity. Experimental results suggest that kinetic friction goes as $F_k\propto \ln{(v)}$, with the expectation that friction become independent of velocity at ``high'' velocities above \SI{1}{\mu/s}. Numerically a viscous $F_k \propto v$ is expected for all velocity ranges. \\ \\
%   \textbf{Temperature} & . \\
%   \hline
%   \end{tabular}
%   \end{center}
% \end{table}

% \begin{table}[H]
%   \begin{center}
%   \caption{Quantitative nano friction dependence on various variables. \hl{work in progress.}}
%   \label{tab:var_dep}
%   \begin{tabular}{ | c | C{3cm}| m{5cm}| m{5cm}|} \hline
%   \textbf{Variable} & \textbf{Dependency} & \textbf{Numerical studies} & \textbf{Experimental} \\ \hline 
%   Normal force $F_N$ 
%   & {\begin{align*}
%     F_{\text{fric}} &\propto F_N^{\alpha} \\
%     \alpha &\le 1
%   \end{align*}} 
%   & Zhang et al.\ \cite{ma12091425} finds a seemingly linear relationship $F_{\text{fric}} \propto F_N$ while Bonelli et al.\ \cite{bonelli_atomistic_2009} reports a sublinear relationship. The latter corresponds with that of nanosasperity simulations where Mo et al.\ \cite{mo_friction_2009}, using an amorphous carbon tip on a diamond sample, also found a sublinear relationship when including adhesion and linear without adhesion.
%   & Experimentally rather different trends have been observed, although the majority agree on increasing friction with increasing load \cite[p. 200]{gnecco_meyer_2015}. For the graphebne flake Dienwiebel et al.\ \cite{DIENWIEBEL2005197} found a seemingly non-dependent relationship while Feng et al.\ \cite{feng_superlubric_2013} did not report on this. FFM analog to the single asperity setup have yielded both linear relationship \cite{gao_frictional_2004} (silicon tip on gold) while Schwarz et al.\ \cite{PhysRevB.56.6987} found that FFM with well-defined spherical tips mathed with theoretical results (DMT, elastic spheres pressed together \cite[p. 200]{gnecco_meyer_2015}), yielding a power law $F_{\text{fric}} = F_N^{2/3}$. 
%   \\ \hline
%   Velocity $v$ & 
%   {\begin{align*}
%     F_{\text{fric}} &\propto \ln{v} \ (\text{exp.})\\
%     &\text{or} \\
%     F_{\text{fric}} &\propto v \ (\text{num.})\\
%   \end{align*}}
%   % $F_{\text{fric}} \propto \ln{v}$ 
%   &  Studies of gold clusters on graphite suggest that friction is viscous, i.e. proportional to velocity \cite{Manini_2016}.

%   & Logaritmic velocity dependence of friction has been measured for nanotip friction \cite[p. 201]{gnecco_meyer_2015} associated to thermal activation and possibly the time availble to form bond between the tip and the substrate. At higher velocities thermally activated processes are less important and friction becomes independent of velocity. This has been observed for Si tips and diamond, graphite and amorphous carbon surfaces with scan velocities above \SI{1}{\mu/s}.
%   \\ \hline
%   Temperature $T$
%   & Either increase (MD) or decrease as $F_{\text{fric}} \propto \exp{(1/T)}$  (experimental)
%   &  Zhang et al.\ \cite{ma12091425} found simply that friction increased with temperature. The \cite{Manini_2016} gold cluster on graphite study (\hl{get reference}) found that the temperature dependence was dependend on velocity regime. Low speed (diffusive) friction decreases upon heating while high speed (ballistic) friction rises with temperature. 
%   & Zhao et al.\ \cite{zhao_thermally_2007} found $F_{\text{fric}} \propto \exp({1/T})$.
%   \\ \hline
%   Real contact area $A$ 
%   & $F_{\text{fric}} \propto A$ 
%   & Mo et al.\ \cite{mo_friction_2009} found that $F_{\text{fric}} \propto A$ where $A$ is the real contact area defined by atoms within chemical range. This is not studied for the case of a nanoflake where the contact area is presumingly rather constant.
%   & \\ \hline
%   \end{tabular}
%   \end{center}
% \end{table}



% \begin{table}[H]
%   \begin{center}
%   \caption{Quantitative nano friction dependence on various variables.}
%   \label{tab:var_dep}
%   \begin{tabular}{ | c | m{3cm}| m{5cm}| m{5cm}|} \hline
%   \textbf{Variable} & \textbf{Dependency} & \textbf{Numerical studies} & \textbf{Experimental} \\ \hline 
%   Normal force $F_N$ 
%   &  $F_f$ increasing $F_N$
%   & MD simulations of amorphous carbon asperity on diamond substrate suggest a linear relatioship to $F_f\propto F_N$ for nonadhesive contact and sublinear with van der waals adhesive forces \cite{mo_friction_2009}. Graphene flake?
%   & The general trend observed in AFM nanoscale friction is that friction force increase with normal load \cite[p.200]{gnecco_meyer_2015}. Various trends have been observed from linear to power trends. 
%   \\ \hline
%   Velocity $v$& $F_f \propto \ln{v}$ & & \\ \hline
%   Temperature $T$& $F_f \propto 1/T$ & & \\ \hline
%   Real contact area $A$ & $F_f \propto A$ & & \\ \hline
%   \end{tabular}
%   \end{center}
% \end{table}




% the stick-slip motion was more evident when changing the sliding direction from armchair to zigzag direction \cite{ma12091425}
  
%   \subsubsection{Variable dependence}

%   \begin{enumerate}
%   \item Normal force: The general trend observed in AFM nanoscale friction is that friction force increase with normal load \cite{gnecco_meyer_2015}. 
%   \item Velocity: Smooth kinetic friction generally increase with speed (velocity strengthening) \cite{Manini_2016}. E. Gnecco et al.\ \cite{PhysRevLett.84.1172} showed a logaritmic increase in mean friction with velocity (the tip of a friction force microscope and NaCl(100) at low velocity ($10^-9$ - $10^-6$ m/s)).  If the scan velocity increases thermally activated processes becomes less important and beyond a critical value the friction forces becomes independent of velocity \cite[p. 202]{gnecco_meyer_2015}
%   \item Temperature: A decrease in friction as $1/T$ was observed by Zhao et al.\ in a series of AFM measurements on graphite in a wide temperature range (140-750 K) \cite[source 351]{gnecco_meyer_2015}
%   % Zhao et al.\ (2007) observed that friction on graphite decreases as 1/T over a wide temperature range (140–750 K), supporting the hypothesis of thermal activation of the stick–slip process. However, it was only recently that group of Schirmeisen reported atomic-scale FFM mea- surements in UHV at different temperatures (Jan- sen et al.\ 2010). When silicon, SiC, ionic crystals and graphite surfaces were cooled down from room temperature to cryogenic conditions, a good agreement with the thermally activated PT model was found down to a peak or a plateau, appearing between 50 and 200 K. Below these values, the friction was found to decrease with temperature, which the authors attributed to the competition between thermally activated rupture and formation of chemical bonds (Barel et al.\ 2010). \cite{BHUSHAN20051507}
%   \item Ruan and Bhushan (1994) (source) found in an AFM study on graphite that the friction coefficient was around $\sim 0.01 - 0.03$
%   % Ruan and Bhushan (1994) used an AFM to investigate the effect of surface roughness on the tribological characteristics of graphite using a Si3N4 tip. It was found that friction coefficient varied with respect to roughness of the substrate. The friction coefficient was below 0.01 and 0.03 for RMS roughness of about 10 nm and 140 nm, respectively. This outcome was attributed to the loss of orientation of the substrate with increasing roughness.33 \cite{kim_nano-scale_2009}
%   \item Contact area
% \end{enumerate}






% \begin{enumerate}
%   \item Smooth kinetic friction generally increase with speed  \cite{Manini_2016}. so-called velocity strengthening.  Logaritmic with speed % Gnecco E, Bennewitz R, Gyalog T, Loppacher C, Bam- merlin M, Meyer E, Güntherodt H-J (2000) Velocity dependence of atomic friction. Phys Rev Lett 84:1172–1175. Maybe crosscheck with macroscale to ensure that this is valid for higher velocities. 
 
%   \item Occurrence of stick slip (also in MD) \cite{kim_nano-scale_2009} (p. 146)
% \end{enumerate}

% Thus, it is commonly expected that the
% friction of a dry nanocontact should classically decrease with increasing
% temperature provided no other surface or material parameters are altered by
% the temperature changes [77, 80–83]. (Current trends in the physics of
% nanoscale friction)

% Thus far we have used thermal activation to explain the velocity dependence of friction. The same mechanism also predicts that friction should change with temperature. \cite{BHUSHAN20051507}








% Look for source on affect on friction when stretching. Since we control the area through that there might be a stretch effect that is even stronger. 





% 

% Hint for explaining increase with stretch: Both micro-tribotester and AFM were used to investigate the micro/nano frictional behavior. It was reported that contact angle between the groove and the pin affected the frictional characteristics significantly. High contact angle led to a sudden increase in the frictional force due to interlocking mechanism. \cite{kim_nano-scale_2009}
% These works suggest that frictional behavior at micro/nano scale is very much dependent on the surface structure and topography. Furthermore, the contact geometry between the tip and the surface such as area and orientation of the contacting angle affect the frictional force significantly. \cite{kim_nano-scale_2009}



% Should period macth the lattice spacing as described in \cite{kim_nano-scale_2009}[p. 144]



% Maybe talk about the slip line as shown in \cref{fig:slip_line}


% \begin{figure}[H]
%   \centering
%   \includegraphics[width=0.5\linewidth]{figures/theory/slip_line.png}
%   \caption{\hl{Temporary} figure from \cite{kim_nano-scale_2009}[p. 144]}
%   \label{fig:slip_line}
% \end{figure}

% \begin{enumerate}
%   \item Friction should decrease by increasing temperature.
%   \item We expect stick slip motion
%   \item What about dependence on normal force?
%   \item Dependence on contact area?
%   \item Dependense on speed? 
% \end{enumerate}

% \begin{itemize}
%   \item Different friction models on macro-and microscopic scale
% \end{itemize}


% Smooth kinetic friction generally increases with speed (velocity strengthening), but sometimes decreases with increasing speed in certain intervals \cite{Manini_2016}.

% the smallest force needed to set a slider in motion – is also dependent on the simulation time (a longer wait may lead to depinning when a short wait might not), and generally dependent on system size, often increasing with sub-linear scaling with the slider’s contact area. To address this kind of behavior in MD simulations, it is often necessary to resort to scaling arguments in order to extrapolate the large-area static friction from small-size MD simulations [131, 140] \cite{Manini_2016}.


% Section 15.3. As shown in Section 15.1, the maximum value of the static friction and the slope of the turning points of the F(x) curves can be used to determine the corrugation U0 of the tip–surface interaction potential and the effective lat- eral stiffness k of the system. From Fig. 18.1(b) we estimate U0 ≈ 0.22 eV and k ≈1N/m. \cite{gnecco_meyer_2015} p. 197






% Concerning stick-slip friction, another problem is that, unlike simulations, real experiments contain mesoscale or macroscale component intrinsically involved in the mechanical instabilities of which stick-slip consists. Here the comforting observation is that stick-slip is nearly independent of speed, so that so long as a simulation is long enough to realize a sufficient number of slip events, the results may already be good enough [148]  \cite{Manini_2016}.


% A serious aspect of stick-slip friction which MD simulation is unable to attack is ageing. The slip is a fast event, well described by MD, but sticking is a long waiting time, during which the frictional contact settles very slowly. The longer the sticking time, the larger the static friction force necessary to cause the slip. Typicall experiments show a logarithmic increase of static friction with time [150] \cite{Manini_2016}.

% Rate and state friction approaches, widely used in geophysics [151], describe phenomenologically frictional ageing, but a quantitative microscopic description is still lacking. Mechanisms invoked to account for contact ageing include chemical strengthening at the interface in nanoscale systems [152], and plastic creep phenomena in macroscopic systems [153]. \cite{Manini_2016}.


% See ``Selected Results of MD Simulations'' in \cite{Manini_2016} p. 24.

% \section{Real life experimental procedures}
% From Introduction to Tribology, Second Edition, p. 526: \par The surface force
% apparatus (SFA), the scanning tunneling microscopes (STM), and atomic force and
% friction force microscopes (AFM and FFM) are widely used in nanotribological and
% nanomechanics studies.



\section{Research questions}

% Formulate the research questions guiding the reading of the thesis moving on. 