\section{Fourier Transform (light)}
% https://www.brown.edu/research/labs/mittleman/sites/brown.edu.research.labs.mittleman/files/uploads/lecture21_0.pdf
% https://mathworld.wolfram.com/FourierTransform.html

% https://lpsa.swarthmore.edu/Fourier/Xforms/FXformIntro.html
\textbf{Find out where to put this if nessecary}. \\

Fourier transform is a technique where we transform a function $f(t)$ of time to
a function $F(k)$ of frequency. The Forward Fourier Transform is done as
\begin{align*}
  F(k) = \int_{-\infty}^\infty f(t) e^{-2\pi ikx} dx
\end{align*}

For any complex function $F(k)$ we can decompose it into magnitude $A(k)$ and
phase $\phi(k)$
\begin{align*}
  F(k) = A(k) e^{i \phi(k)}
\end{align*}

Hence when performing a Forward Fourier transform on a time series we can
determine the amplitude and phase as a function of freqeuncy as 
\begin{align*}
  A(k) = |F(k)|^2, \qquad \phi(k) = \Im{\ln{F(k)}}
\end{align*}




\begin{itemize}
  \item Real life procedures to mimic in computation, for instance Atomic Force
  Microscoopy (AFM) for friction measurements.
  \item Available technology for test of my findings if successful
  (possibilities for making the nano machine) 
\end{itemize}


\section{Machine Learning (ML)}
\begin{itemize}
  \item Feed forward fully connected
  \item CNN
  \item GAN (encoder + decoder)
  \item Genetic algorithm
  \item Using machine learning for inverse designs partly eliminate the black
  box problem. When a design is produced we can test it, and if it works we not
  rely on machine learning connections to verify it's relevance. 
  \item However, using explanaitons techniques such as maybe t-SNE, Deep dream,
  LRP, Shapley values and linearizations, we can try to understand why the AI
  chose as it did. This can lead to an increased understanding of each design
  feature. Again this is not dependent on the complex network of the network as
  this can be tested and veriied independently of the network. 
\end{itemize}

\subsection{Feed forward network / Neural networks}
\subsection{CNN for image recognition}
\subsection{GAN (encoder + deoder)}
\subsection{Inverse desing using machine learning}
\subsection{Prediction explanation}
\subsubsection{Shapley}
\subsubsection{Lineariations}
\subsubsection{LRP}
\subsubsection{t-SNE}



