\section{Fourier Transform (light)}
% https://www.brown.edu/research/labs/mittleman/sites/brown.edu.research.labs.mittleman/files/uploads/lecture21_0.pdf
% https://mathworld.wolfram.com/FourierTransform.html

% https://lpsa.swarthmore.edu/Fourier/Xforms/FXformIntro.html
\textbf{Find out where to put this if nessecary}. \\

Fourier transform is a technique where we transform a function $f(t)$ of time to
a function $F(k)$ of frequency. The Forward Fourier Transform is done as
\begin{align*}
  F(k) = \int_{-\infty}^\infty f(t) e^{-2\pi ikx} dx
\end{align*}

For any complex function $F(k)$ we can decompose it into magnitude $A(k)$ and
phase $\phi(k)$
\begin{align*}
  F(k) = A(k) e^{i \phi(k)}
\end{align*}

Hence when performing a Forward Fourier transform on a time series we can
determine the amplitude and phase as a function of freqeuncy as 
\begin{align*}
  A(k) = |F(k)|^2, \qquad \phi(k) = \Im{\ln{F(k)}}
\end{align*}




% \begin{itemize}
%   \item Real life procedures to mimic in computation, for instance Atomic Force
%   Microscoopy (AFM) for friction measurements.
%   \item Available technology for test of my findings if successful
%   (possibilities for making the nano machine) 
% \end{itemize}


\chapter{Machine Learning (ML)}
\begin{itemize}
  \item Feed forward fully connected
  \item CNN
  \item GAN (encoder + decoder)
  \item Genetic algorithm
  \item Using machine learning for inverse designs partly eliminate the black
  box problem. When a design is produced we can test it, and if it works we not
  rely on machine learning connections to verify it's relevance. 
  \item However, using explanaitons techniques such as maybe t-SNE, Deep dream,
  LRP, Shapley values and linearizations, we can try to understand why the AI
  chose as it did. This can lead to an increased understanding of each design
  feature. Again this is not dependent on the complex network of the network as
  this can be tested and veriied independently of the network. 
\end{itemize}

\section{Feed forward network / Neural networks}
\section{CNN for image recognition}
\section{GAN (encoder + deoder)}
\section{Inverse desing using machine learning}
\section{Prediction explanation}
\subsection{Shapley}
\subsection{Lineariations}
\subsection{LRP}
\subsection{t-SNE}

\chapter{Accelerated Search}
\section{Markov-Chain Accelerated Genetic Algorithms}
% Following the article:
% Accelerated Genetic Algorithms with Markov Chains
% Guan Wang, Chen Chen, and K.Y. Szeto

\subsection{Talk about traditional method also?}

\subsection{Implementing for 1D chromosone (following article closely)}

We have the binary population matrix $A(t)$ at time (generation) $t$ consisting of $N$ rows denoting chromosones and with $L$ columns denoting the so-called locus (fixed position on a chromosome where a particular gene or genetic marker is located, wiki). We sort the matrix rowwise by the fitness of each chromosono evaluated by a fitness function $f$ such that $f_i(t) \le f_k(t)$ for $i \ge k$. We assume that there are a transistion probability between the current state $A(t)$ and the next state $A(t+1)$. We consider this transistion probability only to take into account mutation process (mutation only updating scheme). During each generation chromosones are sorted from most to least fitted. The chromosone at the $i$-th fitted place is assigned a row mutation probability $a_i(t)$ by some monotonic increasing function. This is taken to be 
\begin{align*}
  a_i(t) = 
  \begin{cases}
    (i-1)/N',& i-1 < N' \\
    1, &\text{else}
  \end{cases}
\end{align*}
for some limit $N'$ (refer to first part of article talking about this). We use $N' = N/2$. We also defines the survival probability $s_i = 1 - a_i$. In thus wau $a_i$ and $s_i$ decide together whether to mutate to the other state (flip binary) or to remain in the current state. We use $s_i$ as the statistical weight for the $i$-th chromosone given it a weight $w_i = s_i$.
\\
Now the column mutation. For each locus $j$ we define the count of 0's and 1's as $C_0(j)$ and $C_1(j)$ resepctively. These are normalized as
\begin{align*}
  n_0(j, t) = \frac{C_0(j)}{C_0(j) + C_1(j)}, \quad n_1(j, t) = \frac{C_1(j)}{C_0(j) + C_1(j)}.
\end{align*}
These are gathered into the vector $\vec{n}(j,t)=(n_0(j, t), n_1(j, t))$ which characterizes the state distribution of $j$-th locus. In order to direct the current population to a preferred state for locus $j$ we look at the highest weight of row $i$ for locus $j$ taking the value 0 and 1 respectively.
\begin{align*}
  C'_0(j) &= \max\{W_i | A_{ij} = 0; \ i = 1, \cdots, N\} \\
  C'_1(j) &= \max\{W_i | A_{ij} = 1; \ i = 1, \cdots, N\}
\end{align*}
which is normalized again
\begin{align*}
  n_0(j, t+1) = \frac{C'_0(j)}{C'_0(j) + C'_1(j)}, \quad n_1(j, t+1) = \frac{C'_1(j)}{C'_0(j) + C'_1(j)}.
\end{align*}
The vector $\vec{n}(j,t+1)=(n_0(j, t+1), n_1(j, t+1))$ then provides a direction for the population to evolve against. This characterizes the target state distribution of the locus $j$ among all the chromosones in the next generation. We have
\begin{align*}
  \begin{bmatrix}
    n_0(j, t+1) \\
    n_1(j, t+1)
  \end{bmatrix}
  = 
  \begin{bmatrix}
    P_{00}(j,t) \ P_{10}(j,t) \\
    P_{01}(j,t) \ P_{11}(j,t)
  \end{bmatrix}
  \begin{bmatrix}
    n_0(j, t) \\
    n_1(j, t)
  \end{bmatrix}
\end{align*}
Since the probability must sum to one for the rows in the P-matrix we have 
\begin{align*}
  P_{00}(j, t) = 1 - P_{01}(j, t), \quad P_{11}(j, t) = 1 - P_{10}(j, t)
\end{align*}
These conditions allow us to solve for the transition probability $P_{10}(j,t)$ in terms of the single variable $P_{00}{j,t}$.
\begin{align*}
  P_{10}(j,t) &= \frac{n_0(j, t+1) - P_{00}(j,t)n_0(j, t)}{n_1(j,t)} \\
  P_{01}(j,t) &= 1 - P_{00}(j,t) \\
  P_{11}(j,t) &= 1 - P_{10}(j,t)
\end{align*}
We just need to know $P_{00}(j,t)$. We start from $P_{00}(j, t = 0) = 0.5$ and then choose $P_{00}(j,t) = n_0(j,t)$