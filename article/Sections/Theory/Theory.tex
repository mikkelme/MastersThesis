\newpage
\chapter{Background Theory and Method}
% \addcontentsline{toc}{chapter}{Background Theory and Method} 

Small introtext to motivate this chapter. What am I going to go over here.


\section{Tribology - friction}

% (from wikipedia so far: https://en.wikipedia.org/wiki/Friction)

% the elementary processes of dry, wearless surface sliding 

% Terms:
% mesoscale: Of medium size or extent; between microscale and macroscale

% Quotes:
% Sliding friction that takes place between two surfaces in the absence of lubricant is termed "dry" friction even if the process occurs in an ambient environment. (Nanotribology and Nanomechanics, p. 329)


% For monolayers sliding along atomically uniform substrates, however, there is essentially no static friction. Indeed, the friction in these systems can be up to 105 times less than that for macroscopic lubricants such as graphite. This raises questions about the fundamental dissipation mechanisms that are at work in systems at different scales. (\url{https://physicsworld.com/a/friction-at-the-nano-scale/})

% The trouble is that the coefficients of friction measured in nanotribological experiments and in macroscopic “tribotests” routinely differ by orders of magnitude. (\url{https://physicsworld.com/a/friction-at-the-nano-scale/})

% We were astonished to discover that molecules that could flex or slide even just a little in response to the oscillatory motion of the microbalance were linked to low friction levels at the macro-scale. Put another way, exceptionally low friction at the atomic scale was not a prerequisite for the substantial reduction in macroscopic friction. (\url{https://physicsworld.com/a/friction-at-the-nano-scale/})


% At face value, the transition from a static strained configuration to full sliding is conceptually as simple as overcoming an energy barrier. However, practical single- and multiple- contact conditions are characterized by complex interaction profiles plus nontrivial internal dynamics. As a result, the interplay of thermal drifts, contact ageing, contact-contact in- teractions, and macroscopic elastic deformations introduce significant complications, and make the depinning transition from static to kinetic friction an active field of research. The depinning dynamics affects in particular the transition between stick-slip and smooth slid- ing for sliding friction. (Current trends in the physics of nanoscale friction)

% In Atomic Force Microscopy (AFM) experiments, when the tip scans over the monolayers at low speeds, friction force is reported to increase with the logarithm of the velocity, similar to that observed when the tip scans across crystalline surfaces. This velocity dependence is interpreted in terms of thermally activated depinning of interlocking barriers involving interfacial atoms. (Current trends in the physics of nanoscale friction)

% Da Vinci-Amontons law – friction independent of area – is not confirmed at the microscopic scale. In most nanoscale investigations the friction of a single con- tact is found to increase linearly with the contact area [27–29]. In contrast, structurally mismatched atomically flat and hard crystalline or amorphous surfaces are expected to produce a sublinear increase of friction with contact area. The frequent finding of friction proportional to area even in some of these cases can be understood as a consequence of softness, either if the interface, or of surface contaminants leading to effectively pseudo- commensurate interfaces [30, 31] (Current trends in the physics of nanoscale friction)


% \begin{itemize}
%     \item Amontons' First Law: The force of friction is directly proportional to the applied load.
%     \item Amontons' Second Law: The force of friction is independent of the apparent area of contact.
%     \item Coulomb's Law of Friction: Kinetic friction is independent of the sliding velocity.
% \end{itemize}
% Columbd friction 
% \begin{align*}
%     F_f \le \mu F_n
% \end{align*}


% The study of friction, wear and lubrication between two surfaces in relative motion is called tribology. \cite{gnecco_meyer_2015}.

% We will limit ourself to the elementary processes of dry, wearless surface sliding. In general we can divide friction in to two scales: Macroscopic and microscopic, for which the properties turns out to differ quite substantially.  

% Make an explicit distinction between macroscale and microscale. Get some numbers on it!

% Sliding friction that takes place between two surfaces in the absence of lubricant is termed ``dry''  friction even if the process occurs in an ambient environment. (Nanotriology and Nanomechanics, p. 329)



% The differences between the conventional or macrotribology and micro/nanotribology are contrasted in Figure 1.3.1. In macrotribology, tests are conducted on components with relatively large mass under heavily loaded conditions. In these tests, wear is inevitable and the bulk prop- erties of mating components dominate the tribological performance. In micro/nanotribology, measurements are made on components, at least one of the mating components, with relatively small mass under lightly loaded conditions. In this situation, negligible wear occurs and the surface properties dominate the tribological performance.
% \cite{bhushan_2013}[p. 5]


% It is generally accepted that friction is caused by more than one mechanism in a given sliding system. Generally, frictional force arises due to two fundamentally different causes, namely one that is mechanical in nature and the other being chemical in its origin. In the case of mechanical cause of friction, plowing of the surface by hard particles or asperities is mainly responsible for generating the frictional force.2,4,5-7 As for the chemical mechanism of friction, adhesion between surfaces of the two solids in contact is the cause of friction.2,4,5,8 Another point to note is that tribological phenomena are heavily dependent on system parameters of the operating machine such as speed, temperature, load, and environment. As such, the dominating. \cite{kim_nano-scale_2009}.


% Tribological systems pose a wide range of dimensional scale. At the largest scale, geological stratum layers that are involved in earthquakes may be considered as a tribological system. The movement of the stratum occurs when the frictional forces between the layers are overcome by internal pressure inside the earth. At the smallest scale, relative motion of atoms at the interface of two materials would be a good example that involves frictional interaction. Owing to the vast difference in scale, the dominant mechanisms of friction and wear in macro-scale systems may be different from those of micro/nano-scale systems. In macro-scale, the tribological systems experience relatively large contact stresses and speeds. On the other hand, micro/nano-scale systems operate under relatively low loads and speeds. Particularly, the inertial effects that are prominent in macro-scale may be insignificant at the micro/nano-scale. Rather, surface forces often dictate the tribological interactions at small scales. \cite{kim_nano-scale_2009}.


% Experimental research to examine the frictional characteristics at the atomic-scale has been conducted for the past two decades. It is well known that frictional behavior cannot be generalized by a few factors such as normal load, surface roughness, speed, and material type of the tribological system. Other conditions such as temperature, humidity, and even sliding history can affect the tribological phenomena significantly. Particularly at nano-scale, the tribological behavior tends to be more sensitive to the state of outermost layer of the surface region. Thus, contamination layer, adsorbed gas, capillary junctions, and oxide layer become more important at small scales. This is because at nano-scale the contact forces are often too low for the asperities to penetrate the surface layers and the magnitude of the surface force may be comparable to the frictional force. {kim_nano-scale_2009}.

% The first well known studies of friction concerns itself is based on a macroscopic level. That is, the scale of visible everyday objects.
\subsection{Friction on a macroscopic scale - macroscale theories}
\subsubsection {Amontons’ law. }
 The work of Leonardo da Vinci (1452–1519), Guillaume Amontons (1663-705) and Charles de Coulomb (1736-1806) all contributed to what is commonly known as  Amontons’ law describing the frictional force accuring when starting and keeping a solid block sliding against a solid surface. This emperical law states that the frictional force tangential to the sliding direction is entirely independent of contact area and sliding velocity (at ordinary sliding velocities). Instead it relies only on the normal force $F_N$ acting from the surface on the block and the material specific friction coefficient $\mu$ as
\begin{align*}
  F_f = \mu F_N.
\end{align*}

Further it distinguish between \textit{static} and \textit{kinetic} friction as the cases of stationary and sliding contact resepectively. Each type of friction comes with its own friction coefficient, $\mu_s$ for static and $\mu_k$ for kinetic friction, usually with values lower than one and $\mu_s \ge \mu_k$ in all cases. \cite{gnecco_meyer_2015}[p. 6]. \\

This simple law is a natural starting point for the

Allthough this model is a common base for understanding friction on a macroscopic level is has its limitations. It turns out that static friction is not constant, but depends on the so-called contact history with increasing friction as the logarithm of time of stationary contact \cite{dieterich_1972}. For the kinetic friction the independency of sliding velocity dissapears at low velocities as thermal effects becomes important and for high velocities due to intertial effetcs. \cite{gnecco_meyer_2015}[pp. 5-6]. \\

It fails to explain the mechanisms behind fritction.
 

In order to understand what is causing friction between moving objects and how this might result in a linear relationship between friction and normal force we must take the study to a smaller scale...
Having an emperical law that seems to predict the friction in many cases leads to the next natural desire for deriving these from fundamental atomic or molecular principles.

\subsection{Friction on a microscopic scale - Nanotribology}
It is generally accepted that friction is caused by two mechanism: mechanical friction and chemical friction. The mechanical friction is the plowing of the surface by hard particles or asperities. The chemical mechanism is adhesion between contacting surfaces. \cite{kim_nano-scale_2009}.


Sources in general: \cite{mo_friction_2009}, \cite{kim_nano-scale_2009} \\

\subsubsection{Surface roughness - Asperity theories}

Going beyond a macroscopic perspective we realise that most surfaces is in fact rough. The contact between two surfaces consist of numerous smaller contact point, so-called asperities, each with a contact area of $A_{\text{asp}}$. The true contact area $\sum A_{\text{asp}}$ is found to be much smaller than the apperent macroscopic area $A_{\text{macro}}$. The friction force is shown to be proportional (extra source on this) to this true contact area as 

\begin{align*}
  F_f = \vec{\tau} \sum A_{\text{asp}},
\end{align*}

where $\vec{\tau}$ is an effective shear strength of the contacting bodies. This is still compatible with Amontons’ law as long as we differenciate between the macroscopic macroscopic and true area and by having the true contact area dependt linearly on applied normal force. \\

Thus many studies have focused individual asperities to reveal the relationship between the contact area and normal force  (13-15 from \cite{mo_friction_2009}). By assuming perfectly smooth asperities with radii of curvature from nanometers to micrometres in size continuum mechanics can be used to predict the deformation of asperities as normal force is applied. A model for non-adhesive contact between homogenous, isotropic, linear elastic spheres was first developed by Hertz (17 \cite{mo_friction_2009}), which predicted $A_{\text{asp}} \propto F_N^{2/3}$. Later adhesion effects were included in a number of subsequent models, including Maugis-Dugdale theory (18 from \cite{mo_friction_2009}), which also predicts a sublinear relatinship between $A_{\text{asp}}$ and $f_N$ leading to a similar sublinear relationship for $F_f$ and $F_N$.

% Picture of asperities from wiki: https://en.wikipedia.org/wiki/Asperity_%28materials_science%29#/media/File:Asperities.svg
\cite{mo_friction_2009}.

% However, practical single- and multiplecontact conditions are characterized by complex interaction profiles plus nontrivial internal dynamics. As a result, the interplay of thermal drifts, contact ageing, contact-contact interactions, and macroscopic elastic deformations introduce significant complications, and make the depinning transition from static to kinetic friction an active field of research. \cite{Manini_2017}[p. 2]. 



\subsubsection{Atomic level friction }
On the smallest possible scale, atomic scale, the surfaces does not have structural asperities. Instead atomic level friction is being model as a consequencse of the rough potential of the atomic landscape. 

\subsubsection{Frenkel-Kontorova-Tomlinson (FKT)}
Describes atomic scale friction (not fully accurately though) and gives insight in stick slip motion. 



\subsubsection{Commensurate and incommensurate}
\subsubsection{Stick slip}



At nanoscales things get a bit more unclear. SFM (explain) experiments have reported (copy sources 5, 6, 21 from \cite{mo_friction_2009}) where $F_f \propto F_N$ or even with these quantities being nearly independent of each other.



% \cite{physicsworld_2005}





In several works by J. Fineberg’s group [2–4] the transition from sticking to sliding is characterized by slip fronts propagating along the interface. \cite{Manini_2017}[p. 2]. 

\subsubsection{Commensurate and incommensurate}
As expected, high levels of friction were present in the commensurate positions and extremely low friction was found when the surfaces were incommensurate. (\url{https://physicsworld.com/a/friction-at-the-nano-scale/})


\subsubsection{Superlubricity?}
Superlubricity, now a pervasive concept of modern tribology, dates back to the math- ematical framework of the Frenkel Kontorova model for incommensurate interfaces [40]. When two contacting crystalline workpieces are out of registry, by lattice mismatch or angular misalignment, the minimal force required to achieve sliding, i.e. the static friction, tends to zero in the thermodynamic limit – that is, it can at most grow as a power less than one of the area – provided the two substrates are stiff enough. (Current trends in the physics of nanoscale friction)


Superlubricity is experimentally rare. Until recently, it has been demonstrated or im- plied in a relatively small number of cases [29, 42–46]. There are now more evidences of superlubric behavior in cluster nanomanipulation [32, 33, 47], sliding colloidal layers [48–50], and inertially driven rare-gas adsorbates [51, 52]. (Current trends in the physics of nanoscale friction)


A breakdown of structural lubricity may occur at the heterogeneous interface of graphene and h-BN. Because of lattice mismatch (1.8\%), this interface is intrinsically incommen- surate, and superlubricity should persist regardless of the flake-substrate orientation, and become more and more evident as the flake size increases [57]. However, vertical cor- rugations and planar strains may occur at the interface even in the presence of weak van der Waals interactions and, since the lattice mismatch is small, the system can de- velop locally commensurate and incommensurate domains as a function of the misfit angle [58, 59]. Nonetheless, spontaneous rotation of large graphene flakes on h-BN is observed after thermal annealing at elevated temperatures, indicative of very low friction due to incommensurate sliding [60, 61]. (Current trends in the physics of nanoscale friction)

Indeed, we know from theory and simulation [74–76] that even in clean wearless friction experiments with perfect atomic structures, superlubricity at large scales may, for example, surrender due to the soft elastic strain deformations of contacting systems. (Current trends in the physics of nanoscale friction)




\subsection{Temperature dependence}
Thus, it is commonly expected that the friction of a dry nanocontact should classically decrease with increasing temperature provided no other surface or material parameters are altered by the temperature changes [77, 80–83]. (Current trends in the physics of nanoscale friction)


\subsection{Summary of expected frictional properties}
\begin{enumerate}
  \item Friction should decrease by increasing temperature.
  \item We expect stick slip motion
  \item What about dependence on normal force?
  \item Dependence on contact area?
  \item Dependense on speed? 
\end{enumerate}

\begin{itemize}
  \item Different friction models on macro-and microscopic scale
\end{itemize}
\section{Graphene}
Because of this frictional reduction, many studies indicate graphene as the thinnest solid-state lubricant and anti-wear coating [104–106]. (Current trends in the physics of nanoscale friction)


Accurate FFM measurements on few-layer graphene systems show that friction decreases by increasing graphene thickness from a single layer up to 4-5 layers, and then it approaches graphite values [97, 99, 101, 107, 108]. (Current trends in the physics of nanoscale friction)



\begin{itemize}
  \item General properties and crystal structure
\end{itemize}
\section{Molecular Dynamics}
\subsection{Potentials}
\subsection{LAMMPS}
\subsection{Intregration}
\subsection{Thermostats}
\subsection{Graphene}



\begin{itemize}
  \item MD simulation (classical or ab initio)
  \item Basics of classical MD simulations: Integration and stuff
  \item Ab initio simulation (quantum mechanics, solving schrödinger)
\end{itemize}
\section{Real life experimental procedures}
From Introduction to Tribology, Second Edition, p. 526: \par
The surface force apparatus (SFA), the scanning tunneling microscopes (STM), and atomic force and friction force microscopes (AFM and FFM) are widely used in nanotribological and nanomechanics studies.



\begin{itemize}
  \item Real life procedures to mimic in computation, for instance Atomic Force Microscoopy (AFM) for friction measurements.
  \item Available technology for test of my findings if successful (possibilities for making the nano machine) 
\end{itemize}
\section{Machine Learning (ML)}
\begin{itemize}
  \item Feed forward fully connected
  \item CNN
  \item GAN (encoder + decoder)
  \item Genetic algorithm
  \item Using machine learning for inverse designs partly eliminate the black box problem. When a design is produced we can test it, and if it works we not rely on machine learning connections to verify it's relevance. 
  \item However, using explanaitons techniques such as maybe t-SNE, Deep dream, LRP, Shapley values and linearizations, we can try to understand why the AI chose as it did. This can lead to an increased understanding of each design feature. Again this is not dependent on the complex network of the network as this can be tested and veriied independently of the network. 
\end{itemize}

\subsection{Feed forward network / Neural networks}
\subsection{CNN for image recognition}
\subsection{GAN (encoder + deoder)}
\subsection{Inverse desing using machine learning}
\subsection{Prediction explanation}
\subsubsection{Shapley}
\subsubsection{Lineariations}
\subsubsection{LRP}
\subsubsection{t-SNE}



\section{Generating cuts in the graphene sheet}
\subsection{Defining the sheet configuration}
\subsubsection{Indexing}
\subsubsection{Removing atoms}
\subsubsection{Pull blocks}
\subsection{Kirigami inspired cut out patterns}
\subsubsection{Pop-up pattern}
\subsubsection{Honeycomb}
\subsubsection{Random walk}

