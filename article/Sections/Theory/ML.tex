\chapter{Machine Learning}\label{chap:ML}
We will use machine learning to predict the friction resulting from the stretching
and loading of a given Kirigami pattern. To this end, we will generate data through \acrshort{MD} simulations that will serve as the ground truth for training the machine learning model. The advantage of using machine learning is that it can significantly speed up the exploration of new configurations compared to full \acrshort{MD} simulations. However, there is no guarantee that the machine learning model can accurately capture the physical mechanisms governing our system. Hence, a key objective is to assess
the viability of this approach for further studies of Kirigami friction, It is not guaranteed that the machine learning model can accurately
capture the physical mechanisms of our system. Hence, one of our objectives is to evaluate the applicability of this approach in the study of Kirigami friction, which we will pursue using a rather traditional machine-learning approach. In this
chapter, we introduce the key concept behind machine learning and some of the concepts and techniques relevant to our implementation. For the numerical implementation, we will use the machine learning framework PyTorch~\cite{NEURIPS2019_9015}


\section{Neural network}\label{sec:NN}
The neural network, or more precisely the \textit{feed forward dense neural
network}, is one of the original concepts in machine learning arising from the attempt of mimicking the way neurons work in the brain~\cite{lederer2021activation, Shankar_2022} brain. The neural network can be considered in terms of three major parts: The input layer, the
so-called \textit{hidden layers} and finally the output layer as shown in
\cref{fig:ffnn}. The input is described as a vector $\vec{x} = x_0, x_1, \ldots,
x_{n_x}$ where each input $x_i$ is usually denoted as a \textit{feature}. The
input features are densely connected to each of the \textit{nodes} in the first
hidden layer as indicated by the straight lines in~\cref{fig:ffnn}. Each line
represents a weighted connection that can be adjusted to configure the
importance of that feature. Similar dense connections are present throughout the
hidden layers to the final output layer. For a given note $a_j^{[l]}$ in layer $l$ the input from all the nodes in the previous layer $l-1$ are processed as
\begin{align*}
  a_j^{[l]} = f\left(\sum_i w^{[l]}_{ij}a_i^{[l-1]} + b_j^{[l]}\right),
\end{align*}
where $w^{[l]}_{ij}$ is the weight connection node $a_i^{[l-1]}$ of the previous layer to the node $a_j^{[l]}$ in the current layer. Note that having the weight belong to layer $l$ as opposed to $l-1$ is simply a notation choice. $b_j^{[l]}$ denotes a bias and $f(\cdot)$ the \textit{activation function}. The activation function provides a non-linear mapping of the input to each node. Without this, the network will only be capable of approximate linear functions~\cite{lederer2021activation}. Two common activation functions are the \textit{sigmoid}, mapping the input to the range $(0,1)$, and the \textit{ReLU} which cuts off negative contributions
\begin{align*}
  \text{Sigmoid:} \quad f(z) = \frac{1}{1 + e^{-z}}, \qquad \qquad
  \text{ReLU:} \quad 
  f(z)= \begin{cases}
    z & z > 0   \\
    0 & z \leq 0.
    \end{cases}
\end{align*}
Often the same activation function is used throughout the network, except for the output layer where the activation function is usually omitted or the sigmoid is used for classification tasks. The whole process of sending data through the model is called \textit{forward propagation} and constitutes the mechanism for mapping an input $\vec{x}$ to the model output $\hat{\vec{y}}$. In order to get useful predictions we must \textit{train} the model which involves tuning the model parameters, i.e.\ the weight and biasses.
\begin{figure}[H]
  \centering
  \includegraphics[width=0.9\linewidth]{figures/theory/ffnn.png}
  \caption{\hl{From overleaf IN5400}}
  \label{fig:ffnn}
\end{figure}
The model training relies on two core concepts: \textit{backpropagation} and \textit{gradient descent} optimization. First, we define the error associated with a model prediction, otherwise known as the \textit{loss}, through the \textit{loss function} $L$ that evaluates the model output $\hat{\vec{y}}$ against the ground truth $\vec{y}$. For a continuous scalar output, we might simply use the mean squared error (\acrshort{MSE})
\begin{align*}
  L_{\text{\acrshort{MSE}}} = \frac{1}{N_y} \sum_{i = 1}^N (y_i - \hat{y}_i)^2.
\end{align*}
For a binary classification problem, meaning that the true output is True or False (1 or 0), a common choice is binary cross entropy (BSE)
\begin{align*}
  L_{\text{BSE}} =  -\sum_{i=1}^n \ \Big[y_i\log(\hat{y_i}) + (1-y_i)\log(1 - \hat{y_i}) \Big] =  \sum_{i=1}^n   \begin{cases}
    - \log{(\hat{y_i})},& \quad y_i = 1 \\
    -\log{(1-\hat{y_i})},& \quad y_i = 0.
\end{cases}
\end{align*}
The cross-entropy loss can be derived from a maximum likelihood estimation \hl{SOURCE}. Without going into details with the derivation we can convince ourselves that the error is minimized for the correct prediction and maximized for the worst prediction. When $y_i = 1$ we get the negative term $-\log(\hat{y_i})$ where a correct prediction $\hat{y}_i \to 1$ yields a loss contribution $L_i \to 0$. For a wrong prediction $\hat{y}_i \to 0$ the loss contribution will diverge $L_i \to \infty$. Similar applies to the case of $y_i = 0$ with opposite directions. 

Given a loss function, we can calculate the loss gradient $\nabla_\theta L$ with respect to each of the weights and biases in the model. This is called \textit{backpropagation} since we follow the propagation of the errors as we go backward back through the model layers calculating the gradient using the chain rule. These gradients express how each parameter is connected to the loss and the overall idea is then to ``nudge'' each parameter in the right direction for reducing the loss. We usually denote a full cycle of forward-, backpropagation and an update of all model parameters as an \textit{epoch}. We calculate the updated parameter $\theta_t$ for epoch $t$ using the \textit{gradient descent} method
\begin{align}
  \theta_{t} = \theta_{t-1} - \eta \nabla_\theta L(\theta_t).
  \label{eq:grad_descent}
\end{align}
Gradient descent is analog to taking a step in parameter space in the direction
that yields the biggest decrease in the loss. If we imagine a simplified case
with only two parameters $\theta_1$ and $\theta_2$ we can think of these as directions on a map and the loss being the terrain height. The gradient descent steps in the direction perpendicular to the contour lines shaped by loss function terrain as shown in~\cref{fig:gd}. Notice, however, that state-of-the-art models in general contain on the order of $\num{e6}-\num{e9}$ parameters \cite{thompson2022computational} which poses some challenges for the visualization.  The length of each step is proportional to the gradient norm and the learning rate $\eta$. There are three main flavors to the gradient descent: Batch,
stochastic and mini-batch gradient descent. In \textit{batch gradient descent} we simply
calculate the gradient based on the entire dataset by averaging the contribution
from each data point before updating the parameters. This gives the most robust estimate of the gradient and thus the most direct path through parameter space in terms of minimizing the loss function as
indicated in~\cref{fig:gd}. However, for big datasets, this calculation can be
computationally heavy as it must carry the entire dataset in memory at once. A
solution to this issue is provided by \textit{stochastic gradient descent}
(\acrshort{SGD}) which considers only one data point at a time. Each data point is chosen randomly and the parameters are updated based on the corresponding gradient. This leads to more frequent updates of the parameters and a more ``noisy'' path through parameter space as shown in~\cref{fig:sgd}.  Under some circumstances, this might compromise the precision. However, the presence of noise can actually increase the chances of avoiding local minima in parameter space The \textit{mini-batch gradient descent} serves as a middle ground between the above methods by dividing the full dataset into a subset of mini-batches. Each parameter update is then based on the gradient within a mini-batch. By choosing a suitable batch size we get the robustness of the (full) batch gradient descent and the computational efficiency and resistance to local minima of the \acrshort{SGD} method. 


\begin{figure}[H]
  \centering
  \begin{subfigure}[t]{0.49\textwidth}
    \centering
    \includegraphics[width=\textwidth]{figures/theory/gd.png}
    \caption{}
    \label{fig:gd}
  \end{subfigure}
  \hfill
  \begin{subfigure}[t]{0.49\textwidth}
    \centering
    \includegraphics[width=\textwidth]{figures/theory/sgd.png}
    \caption{}
    \label{fig:sgd}
  \end{subfigure}
  \hfill
  % https://www.samlau.me/test-textbook/ch/11/gradient_stochastic.html
  \caption{\hl{TMP}}
  \label{fig:gradient_descent}
\end{figure}



\subsection{Optimizers}
The name \textit{optimizers} covers a variety of gradient descent methods. In our study, we will use the ADAM (adaptive moment estimation)~\cite{kingma2017adam}. ADAM combines several ``tricks in the book'' which we will introduce in the following.

% Momentum 
One considerable extension of the gradient descent scheme is by the introduction of a momentum term $m_t$ such that we get
\begin{align}
  \theta_t = \theta_{t-1} - m_t, \qquad m_t = \alpha m_{t-1} + \eta \nabla_\theta L(\theta_t)
  \label{eq:mom}
\end{align}
with $m_0 = 0$. If we introduce the shorthand $g_t = \nabla_\theta L(\theta_t)$ we find
\begin{align}
  m_1 &= \alpha m_0 + \eta g_1 = \eta g_1 \nonumber \\
  m_2 &= \alpha m_1 + \eta g_2 = \alpha^1 \eta g_1 + \eta g_2 \nonumber \\
  m_3 &= \alpha m_2 + \eta g_3 = \alpha^2 \eta g_1 + \alpha\eta g_2 + \eta g_3 \nonumber \\
  &\vdots \nonumber \\
  m_t &= \eta \left(\sum_{k=1}^{t} \alpha^{t-k}g_k\right).
  \label{eq:mom_rec}
\end{align}
Hence $m_t$ is a weighted average of the gradients with an exponentially decreasing weight. This act as a memory of the previous gradients and aid to pass local minima and to some degree plateaus in the parameter space. It also provides a general steadiness to the descent which counteracts the transition from batch to mini-batch gradient descent. A variation of momentum can be achieved with the introduction of the exponential moving average (EMA) which builds on the recursion
\begin{align*}
    \text{EMA}(g_1) &= \alpha \overbrace{\text{EMA}(g_0)}^{\equiv \ 0} + (1-\alpha)g_1 \\
    \text{EMA}(g_2) &= \alpha \text{EMA}(g_1) + (1-\alpha)g_2 \\
    &\vdots \\
    \text{EMA}(g_t) &= \alpha \text{EMA}(g_{t-1}) + (1-\alpha)g_t  = \sum_{k=0}^t \alpha^{t-k}(1-\alpha)g_t,
\end{align*}
which is similar to that of momentum~\cref{eq:mom_rec}, but with the explicit weighting by $(1-\alpha)$. The second moment of the exponential moving average is utilized in the root mean square propagation method (\acrshort{RMSProp}) which is motivated by the issue of passing long loss plateaus in the parameter space. Since the size of the updates are proportional to the norm of the gradient
\begin{align*}
  \theta_{t+1} = \theta_t - \eta g_t \ \Longrightarrow \ ||\theta_{t+1}-\theta_{t}|| = \eta ||g_t||,
\end{align*}
we might get the idea of normalizing the gradient step by division by the norm $|||g_t|$. However, this does not immediately solve the problem of long plateaus as we need to consider multiple past gradients as can be done with the use of the \acrshort{EMA}. When reentering a steep region again we need to ``quickly'' downscale the gradient steps which can be achieved more efficiently by using the squared norm $||g_t||^2$ for the \acrshort{EMA} which makes it more sensitive to outliers. The \acrshort{RMSProp} update scheme is given
\begin{align}
  \theta_t = \theta_{t-1} - \eta \frac{g_t}{\sqrt{\text{EMA}(||g_t||^2)}+ \epsilon},
  \label{eq:RMSProp}
\end{align}
where $\epsilon$ is simply a small number to avoid division by zero issues. 

ADAM merges the idea of first order \acrshort{EMA} for the momentum $m_t$, and the second order \acrshort{EMA} $v_t$, as used in the root mean square propagation technique in~\cref{eq:RMSProp}
\begin{align*}
  m_t &= \beta_1 m_{t-1} + (1-\beta_1)g_t, \\
  v_t &= \beta_2 v_{t-1} + (1-\beta_2)g_t^2. 
\end{align*}
Since these are initially set to zero ADAM introduces the scaling terms  $(1-\beta^t_1)$ and $(1-\beta^t_2)$ to correct for a bias towards zero. The ADAM scheme is given~\cite{kingma2017adam}
\begin{align}
  \theta_{t+1} = \theta_t - \eta \frac{\hat{m}_t}{\sqrt{\hat{v}_t} + \epsilon}, \qquad \qquad \hat{m}_t = \frac{m_t}{1-\beta^t_1}, \qquad \hat{v}_t = \frac{v_t}{1-\beta^t_2}.
  \label{eq:ADAM}
\end{align}


\subsection{Weight decay}
By adding a so-called \textit{regularization} to the loss function we can penalize high magnitudes of the model parameters, usually intended for the model weights primarily. This is motivated by the idea of preventing overfitting during training. The most common way to do this is by the use of L2 regularization, adding the squared $l^2$ norm $||\theta||_2^2$, where $||\theta||_2 = \sqrt{\theta_1^2 + \theta_2^2 + \ldots}$, to the model. The loss and gradient then become
\begin{align}
  L_{l^2}(\theta) &= L(\theta) + \frac{1}{2}\lambda ||\theta||^2_2 \label{eq:weight_decay_L} \\
  \nabla_\theta L_{l2}(\theta) &=  \nabla_\theta L(\theta) + \lambda \theta,
  \label{eq:l2_grad}
\end{align}
where $\lambda\in [0,1]$ is the weight decay parameter. The name \textit{weight decay} relates to the fact that some practitioners only apply this penalty to the weights in the model, but we will include the biases as well (standard in PyTorch). Following the original gradient descent scheme~\cref{eq:l2_grad} we get
\begin{align}
  \theta_{t+1} = \theta_t - \eta g_t - \eta\lambda \theta_t = \theta_t\underbrace{(1-\eta\lambda)}_{\text{weight decay}} - \eta g_t. \label{eq:weight_decay_descent}
\end{align}
Thus we notice that choosing a high weight decay ($\lambda \to 1$) will downscale the model parameters while choosing a low weight decay ($\lambda \to 0$) yields the original gradient descent scheme. Note that we will use the weight decay principle in combination with ADAM. In~\cref{eq:weight_decay_descent} we have simply used the original gradient descent scheme~\cref{eq:grad_descent} since this makes it easier to demonstrate the consequences of introducing the L2 regularization into the loss function~\cref{eq:weight_decay_L}.


\subsection{Parameter distributions}
In order to get optimal training conditions it has been found that the initial
state of the weight and biases are important \hl{SOURCE}. First of all, we must
initialize the weight by sampling from some distribution. If the weights are set
to equal values the gradient across a layer would be the same. This results in a
complexity reduction as the model can only encode the same values across the
layer \hl{SOURCE?}. Further, we want to consider the gradient flow during training. Especially for deep networks, networks with many layers, we must pay attention to the problem of \textit{vanishing} or \textit{exploding} gradients. If we for instance consider the sigmoid activation function and its derivative 
\begin{align*}
  f(z) = \frac{1}{1 + e^{-z}}, \qquad f'(z) = \frac{df(z)}{dz} = \frac{e^{-z}}{(1+e^{-z})^2} = \frac{e^{z}}{(1+e^{z})^2},
\end{align*}
we notice that for large and small input values $z$ we get $f(z\to \pm\infty)
\to 0$. However, even a small finite gradient can vanish throughout a deep
network as the calculation of the gradient involves the chain rule. This gives
rise to a gradient that potentially gets smaller and smaller for each layer it
passes in the backpropagation. A similar problem can be found with the ReLU
activation function which contributes toward a gradient of zero for inputs
$z<0$. This can be mitigated by the so-called leaky RelU which maps the $z<0$
to a small negative slope $a<0$ as $f(z) = az$. On the other hand, we have
exploding gradients, which are simply a result of the chain rule gradient
calculation. For a sufficiently deep network, the gradient can grow
exponentially large and sometimes result in a numerical overflow. While there
exist techniques to accommodate the problem of vanishing or exploding gradients,
like for instance the leaky ReLU for the vanishing gradients and so-called
gradient clipping, cutting off the gradient at a maximum, they both benefit from
a properly initialized set of weights \hl{SOURCE?}. That is, we want the
gradients across a given layer to have a zero mean while the variance is similar
between layers in the model. This balanced gradient flow is more likely to
happen if we initialize the weight by the same set of criteria \hl{SOURCE?}. The
specific actions to achieve this depend on the model architecture, including the
choice of activation functions. For instance, using the ReLU activation
functions it was found that the node standard deviation will depend on the
number of input nodes from the previous layer $N^{[l-1]}$ as $\sim
\sqrt{N^{[l-1]}}/\sqrt{2}$. Thus we can simply generate the weight from a zero mean uniform distribution scaled by this value. This is part of the Kaiming initialization scheme which is standard in Pytorch \hl{SOURCe}. The bias is initialized from a similar consideration.

\textit{Batch normalization} is another technique that can also help reduce the issue of poor gradient flow. Furthermore, it can benefit by speeding up convergence and making the training process more stable \hl{SOURCE}. In general, model parameters are modified throughout training meaning that the range of values coming from a previous layer will shift, even though the same training data is fed through the network repeatedly. By scaling the input for a given layer, for each mini-batch, we can mitigate this problem and make for a more standardized input range. This often results in a faster training convergence. For layer $l$ we calculate the mean $\mu^{[l]}$ and variance $\sigma^{2[l]}$ across the layer with nodes $x_1^{[l]}, x_2^{[l]}, \ldots, x_d^{[l]}$ for each mini-batch of size $m$ as
\begin{align*}
  \mu^{[l]} = \frac{1}{m} \sum_i z^i, \qquad \sigma^{2[l]} = \frac{1}{m} \sum_i^d (x^i-\mu)^2.
\end{align*}
We then perform a normal scaling of the inputs within the batch
\begin{align*}
  \hat{x}_i^{[l]} = \frac{x_i^{[l]} - \mu^{[l]}}{\sqrt{\sigma^{2[l]} + \epsilon}},
\end{align*}
where $\epsilon$ is a small number to ensure numerical stability (similar to what we used for \acrshort{RMSProp} gradient descent). In the final step, the input values are rescaled as
\begin{align*}
  \tilde{x_i} = \gamma^{[l]} \hat{x}_i^{[l]} + \beta^{[l]}
\end{align*}
with trainable parameters $\gamma$ and $\beta$. \hl{Comment about the reason for the final step}.


\subsection{Learning rate decay strategies}
Until now we have assumed a constant learning rate, but many training variations use a changing learning rate beyond the adaptiveness included in the optimizers covered so far. Under some circumstances, it can be beneficial to start with a higher learning rate to speed up the initial part of training and then lower the learning rate for the final gradient descent~\cite{smith2018disciplined}. One straightforward strategy is a step-wise learning rate decay where the learning rate is reduced by a factor $\gamma \in (0,1)$ every $K$ steps. A more smooth change can be achieved by for instance a polynomial decay $\eta_t = \eta_0/t^{\alpha}$ for $\alpha > 0$. More advanced approaches use multiple cycles of increasing and decreasing cycles. We will mainly concern ourselves with a one-cycle policy for which we start at an
intermediate value, increase toward a maximum bound and then decrease toward a
final lower learning rate bound. We do this by following a cosine function that is shifted and stretched to increase towards the first 30\% of the training length and decrease toward the lower bound learning rate for the remaining epochs. 



\section{Convolutional Neural Network}\label{sec:CNN}

% For
% our task we will consider the Kirigami configuration, load and stretch of the system as input features on which we want the model to base its prediction.

% . Thus, it does not matter how the input is arranged
% initially, but it cannot be changed at a later stage. This is
% impractical if 
Convolutional Neural Networks (\acrshort{CNN}s) build upon many of the same
concepts as introduced with the feed-forward neural network in~\cref{sec:NN}.
The difference lies in its specialization for a spatially correlated input, such
as pixels in an image. In a dense neural network, every node is connected to
each of the nodes from the previous layers which is not ideal for image
recognition. For instance, if we want the model to recognize images of animals
the dense network will be very sensitive to where that animal is placed within
the frame. The \acrshort{CNN} is motivated by the idea of capturing spatial
relations in the input, but without being sensitive to the relative placement
within the input, i.e.\ being translational invariant. This is achieved by
having a so-called \textit{kernel} or \textit{filter} which slides over the
images\footnote{Note, that we will be using the word ``image'' as a reference
for a spatially dependent input, but in reality, it does not have to be an
actual image in the classical sense.} as it processes the input. The overall flow of data for a typical convolutional network is illustrated in \cref{fig:CNN}. A convolutional layer contains multiple kernels, each consisting of a set of trainable weights
and a bias. Each kernel will produce a separate output channel to the resulting
\textit{feature map} layer. The kernel has a 2D spatial size, specific to the
model architecture, and a depth that matches the number of input channels to the
layer. For instance, a typical RGB will have three channels, while the number of
channels usually increases for each layer in the model. The kernel lines up with
the image and calculates the feature map output as a dot product between the
weights in the kernel and the aligning subset of the input. This is done for
each input channel and summed up with the addition of a bias as illustrated
in~\cref{fig:conv_calculation}. The kernel then slides over by a step size given
by the \textit{stride} model parameter and repeats the calculation. Choosing a
stride of 2 or higher results in a reduction of the output spatial size. If we want to
preserve the spatial size we must keep a stride of one and additionally apply
\textit{padding} to the input images, such that we can achieve one kernel
position for each input ``pixel''. The spatial size of the feature map is given
as
\begin{align}
  N_d^{[l+1]} = \left\lfloor \frac{N_d^{[l]} - F_d + 2P}{S} + 1 \right\rfloor,
  \label{eq:down_scaling}
\end{align}
for padding $P$, stride $S$, spatial size of the kernel filter $F_d$, spatial size
of input $N_d^{[l]}$, for dimension $d = {x, y}$ and layer $l$. The \textit{down-sampling} is often done through a pooling layer. A pooling layer is reminiscent of a kernel, but instead of calculating the output as a dot product, it utilizes the mean (mean pooling) or the max value (max pooling) of the values within its scope. For instance, by using a max pooling of size $2 \times 2$ and stride two we essentially half the dimensions of the image as dictated by~\cref{eq:down_scaling}. \acrshort{CNN}s will often use repeating series of convolution (applying a kernel), pooling and then an activation function. Most architectures aim to slowly down-sample the spatial input while increasing the number of channels throughout the model layers.


\begin{figure}[H]
  \centering
  \includegraphics[width=0.9\linewidth]{figures/theory/CNN.png}
  \caption{\hl{TMP}}
  % https://www.researchgate.net/figure/Representation-of-a-Convolutional-Neural-Network-The-CNN-performs-automatic-spatial_fig3_359896672
  \label{fig:CNN}
\end{figure}

\begin{figure}[H]
  \centering
  \begin{subfigure}[t]{0.26\textwidth}
    \centering
    \includegraphics[width=\textwidth]{figures/theory/conv_kernel_movement.png}
    % \includegraphics[width=\textwidth]{figures/theory/conv_input_channels.png}
    \caption{}
    % \label{fig:}
  \end{subfigure}
  \hfill
  \begin{subfigure}[t]{0.70\textwidth}
    \centering
    \includegraphics[width=\textwidth]{figures/theory/conv_calculation.png}
    \caption{}
    \label{fig:conv_calculation}
  \end{subfigure}
  \hfill
  % https://towardsdatascience.com/a-comprehensive-guide-to-convolutional-neural-networks-the-eli5-way-3bd2b1164a53
  \caption{\hl{TMP}}
  \label{fig:conv_example}
\end{figure}

For a \acrshort{CNN}, we often consider the \textit{receptive field}. The
receptive field relates to the spatial size of the input that affects a given
node in the feature map at a given layer of the model. Often this term is used in consideration of the output nodes. In \cref{fig:receptive_field} the receptive field is illustrated for a 1D
representation of a \acrshort{CNN} with repetitive use of a kernel of width 1 and  stride 1. Going from the output and backward, we see that the output layers are connected to two nodes in the previous layer. Each of these nodes is connected to two nodes in the layer before that, however with one of them being the same due to the stride of 1. By back-tracking to the input we see that this
corresponds to a receptive field of $D = 5$. By increasing the filter size and
the stride the 2D receptive field will grow a lot faster than shown in this
simple 1D example. For a receptive field $D_d$, with respect to the spatial dimension $d$, a
spatial size of the filter $F_d$, stride $S_l$ (from layer $l-1$ to $l$) we have
\begin{align*}
    D_l = D_{l-1} + \left[(F_l - 1) \cdot \prod_{i=1}^{l-1}S_i \right],
\end{align*}
with $D_0 = 1$ and $l=0$ as the input layer. Note that by convention, the
product of zero elements is 1, such that for the first layer, the product is 1. The receptive field is important in understanding the connectivity in the model. The model output will be completely independent of the inputs and feature maps outside the receptive field. Furthermore, we differentiate between the theoretical
receptive field and the effective receptive field. The effective receptive field
will have a Gaussian distribution within the theoretical receptive field because the nodes in the center of the receptive field will have more connections leading to the output, as seen in~\cref{fig:receptive_field}. Thus, in practice, the effective receptive field will be smaller than the theoretical. Implementations like dilated convolutions, which make the filter expand in circumference and skipping positions within the filter, can be used to further increase the effective field. \hl{The receptive field is perhaps not that relevant... Should I remove it?}

\begin{figure}[H]
  \centering
  \includegraphics[width=0.7\linewidth]{figures/theory/receptive_field.png}
  \caption{\hl{TMP}}
  % https://lukeguerdan.com/blog/2019/intro-to-tcns/
  \label{fig:receptive_field}
\end{figure}


On a final note regarding the \acrshort{CNN} we point out that convolution is often used in combination with a dense network, or \textit{fully connected}, at the end. We can then think of the convolution part to handle the translation from a spatial input to some internal features. For the animal detection network, we would perhaps think of features such as the number of legs, size, color and so on. In practice, the network will not create easily interpreted features for the processing of the fully connected layer to see. We discuss one approach for the interpreting of the model internals in~\cref{sec:explanation}


\subsection{Training, validation and test data}
So far, we have simply considered the concept of \textit{training data} as a means to update the model parameters. Yet, we want to evaluate the model performance as
it improves. The problem arises immediately from the fact that a complex model can fit about any function. More precisely, it has been proven that a deep convolutional neural network is universal (follows the universal approximation theorem), meaning that can approximate any continuous function to an arbitrary accuracy when the depth of the network is large enough \cite{cybenko_approximation_1989}. Thus for a complex model, it is just a matter of time before the model eventually finds a good approximation for the training data. However, we want the model to learn general trends and not to ``memorize'' all the data points which are known as \textit{overfitting}. While the predictions for the training data can grow arbitrarily good in most cases, the performance on unseen data within the domain will yield poor performance in the case of overfitting. The common way to address this issue is by putting aside a subset of the
data, the so-called \textit{validation} data, which we use to validate the model performance during and after training. By keeping this \textit{validation} set separate from the training data we can get a more reliable performance estimate for the model. Random partitioning is crucial for ensuring an equal distribution of data across both sets. To strike a balance between the quality of training and validation, a commonly used partitioning ratio is usually around 20:80 in favor of the training set. Other techniques exist which aim to optimize the
data used for sparse data situations, like cross-validation and bootstrap \hl{right?}, but we will not consider such methods for this thesis. A third
data set that is often forgotten is the \textit{test} set. While the
validation set should be kept unseen from the model training, the test set
should be kept unseen from the model developer. As we choose the model
architecture and hyper-parameters. We define a hyper-parameter as a variable to be set prior to the actual application of the learning algorithm, one that is not selected by the learning algorithm itself~\cite{Bengio2012}. This includes parameters such as learning rate, momentum and weight decay, but not the weights and biases as these are updated by the learning algorithm. When adjusting the hyper-parameters we will use the performance on the validation set as a guiding metric. Hence, our choices can eventually lead to a higher level of overfitting through the hyper-parameter choices. Hence, we should denote a test set for the final evaluation of our model which has not been considered before the end. Formally, this is the only reliable performance metric for the model. 


\section{Overfitting and underfitting}
% THE UNDERFITTING AND OVERFITTING TRADE-OFF~\cite{smith2018disciplined}. 

Underfitting and overfitting represent a crucial balance going on when training
a model. This concept is highly related to the model complexity and the chosen
hyper-parameters. The textbook visualization of underfitting and overfitting is
shown in \cref{fig:fitting_vs_time}. As we begin to train or model both the
training and validation loss is decreasing. At some point, the model will start
to pick up, not only the general data trends but also specific trends in the
training data. This marks the transition into the overfitting regime where the
validation loss will increase again, even though the training loss is steadily
declining.  \textit{Early stopping} can be utilized to detect this transition
and stop the training in an attempt to hit the sweet spot between under- and
overfitting. We will use a variation of this which is to store the best model
based on validation performance. For this approach, we let the training finish
but only keep the model corresponding to the best validation score. In
principle, we can ``get lucky'' and find the model settings at a state that is
specially overfitted for the validation set, but we consider this highly
unlikely when having a reasonable amount of data and a complex model with many
model parameters. The underfitting and overfitting phenomena can also be thought
of as a function of the complexity and not just training time. For a certain
amount of epochs a simple model will yield underfitting and an overly complex
model will yield overfitting, and this can be expected to follow a similar
qualitative trend as in~\cref{fig:fitting_vs_time} with the substitution for
\textit{model complexity} on the x-axis. \cref{fig:fitting_quality} visualizes
the concept of underfitting and overfitting in terms of the complexity regarding
the fitting of a second-order polynomial. We see how a simple linear function
will make a crude approximation for the true curve. An overly complex model will
pick up the noise in the data and miss the general trend. However, the problem
is that we do not know the true curve. If we did, we would not need machine
learning to approximate it in the first place. Without having additional insight
into the governing source of the data the overfitting case seems to produce the
most confident fit for all we know. 


% practice, the diagnosticating of underfitting and overfitting is not as simple
% as the figures in~\cref{fig:over_under_fitting} imply. 


\begin{figure}[H]
  \centering
  \begin{subfigure}[t]{0.42\textwidth}
    \centering
    \includegraphics[width=\textwidth]{figures/theory/fitting_vs_time.png}
    \caption{}
    \label{fig:fitting_vs_time}
  \end{subfigure}
  \hfill
  \begin{subfigure}[t]{0.57\textwidth}
    \centering
    \includegraphics[width=\textwidth]{figures/theory/fitting_quality.png}
    \caption{}
    \label{fig:fitting_quality}
  \end{subfigure}
  \hfill
  \caption{\hl{TMP}}
  \label{fig:over_under_fitting}
\end{figure}



\section{Hypertuning}\label{sec:hypertuning}
The training of a machine learning model revolves around tuning the model parameters such as weights and biases. However, as mentioned already, a handful of \textit{hyper-parameters} remains for us to decide. First of all, we need to choose an architecture for the model. This includes high-level considerations, for instance, whether to use a neural network or a convolutional network, but also lower-level considerations, such as the depth and the width of the model, i.e.\ how many layers and how many nodes/channels. In addition, we have to define and consider the loss function and the optimizer which come with hyper-parameters such as learning rate, momentum and weight decay. This extensive list of choices makes the designing of a functional machine learning procedure more complicated than simply hitting ``run'' for the learning algorithm. As N. Smith~\cite{smith2018disciplined} puts it: ``Setting the
hyper-parameters remains a black art that requires years of experience to
acquire''. In the following, we will review a general approach for choosing the learning rate, momentum and weight decay hyper-parameters based on the findings of~\cite{smith2018disciplined}. The traditional approach is to
perform a \textit{grid search}, trying out different combinations of hyper-parameters different training sessions, but this might rather quickly become computationally expensive and ineffective. In addition, hyper-parameters will depend on the training data, the model architecture and not at least each other, which make it difficult to narrow down the choice one by one. N. Smith points to the fact that validation loss can be examined early on for
clues of either underfitting or overfitting. 

The learning rate is often regarded as the most important hyper-parameter to
tune~\cite{Bengio2012}. Typical values are in the range $[\num{e-6}, 1]$.
Instead of simply running a grid search, we can perform a so-called \textit{learning rate range test} (LR range test). One then specifies the minimum and maximum learning rate
boundaries and a learning rate step size. A minimum and maximum bound of
$\num{e-7}$ to 10 will most likely cover an appropriate range, but the test will
reveal this immediately. The idea is then to vary the learning rate throughout
the given range in small steps during a short pre-training. We will vary the learning
rate for each iteration, i.e.\ each parameter update following a mini-batch, and
thus we can run this test for a few epochs, or even a single one, depending on
the number of mini-batches. The learning rate can be varied in a linear
increasing or decreasing manner which is found to produce similar results
\cite{smith2017cyclical}. We chose the linear increasing version for
simplicity. For small learning rates, the model will converge slowly. As the
learning rate approaches an appropriate value the convergence will accelerate
which we see as a drop in the validation loss. Eventually, the convergence will
stop and the validation loss will pass a minimum for which it will begin to
diverge. This general behavior can be understood for the simplified 1D example
of finding the minima of a second-order polynomial as shown in
\cref{fig:lr_descents}. Small learning rates will step in the right direction,
but for very small values this will result in a slow convergence. If
the learning rate becomes too large, we will effectively step past the minimum. Each following step will overshoot the minimum more and more (the step is proportional to the gradient os the loss) leading to a diverging
trend. The point of divergence can be used as an upper bound for the learning
rates when considering a cyclic learning rate scheme. The steepest
decline of the validation loss can be used as an estimate for the best constant
learning rate choice~\cite{smith2018disciplined}.


\begin{figure}[H]
  \centering
  \includegraphics[width=0.9\linewidth]{figures/theory/lr_descents.png}
  \caption{\hl{TMP}  }
  % https://www.javatpoint.com/gradient-descent-in-machine-learning
  \label{fig:lr_descents}
\end{figure}


Next, we consider the choice of momentum. Momentum and learning rates are found to affect each other considerably. From the gradient descent scheme with momentum~\cref{eq:mom} we see that the momentum parameter $\alpha$ and the learning rate $\eta$ have a similar effect on the parameter update
\begin{align*}
  \theta_t = \theta_{t-1} - \eta g_t - \alpha m_{t-1},
\end{align*}
since $m_t$ is a moving average of the gradient $g_t$ as well. Like the
learning rate, we want to set the momentum value as high as possible without
causing instabilities in the training. However, it is found that these values
are somewhat inversely related. Choosing a high learning rate should be coupled
with a lower momentum and vice versa. N. Smith~\cite{smith2018disciplined}
reports that a momentum range test is not useful to find the right momentum.
Instead, he suggests doing a few short runs with different values of momentum,
such as 0.99, 0.97, 0.95, and 0.9, to determine a suitable choice. By including
momentum in the LR range test we can balance the learning rate accordingly for such test. Moreover, for a cyclic learning rate scheme he suggests using a cycling momentum scheme reversed with respect to the learning rate. When the
learning rate increase toward the upper bound the learning rate should
decrease toward the lower bound and vice versa. Choosing a lower momentum of
0.80--0.85 often gives similar stabile results \cite{smith2018disciplined}.

Finally, we address weight decay. N. Smith \cite{smith2018disciplined} reports that weight decay is different from learning rate and momentum by the fact that weight decay is better chosen as a constant value as opposed to a cyclic scheme. However, the weight decay is dependent on the model complexity, learning rate and momentum choice and this can often be dialed in after setting those. We can estimate a suitable choice by doing a rough grid search for values such as 0, $\num{e-6}$, $\num{e-5}$ and $\num{e-4}$ for complex architectures and $\num{e-4}$, $\num{e-3}$ and $\num{e-2}$ for more shallow architectures. Choosing the weight decay on the scale of exponential exponents will often provide good enough precision in practice. 



\section{Prediction explanation}\label{sec:explanation}
On a final note, we present a simple method for providing some insight into the prediction from a convolutional neural network. The high complexity of deep learning models limits our ability to gain insight into the decision-making process behind a prediction beyond the input data. This is known as the \textit{black box} problem. A lot of effort is currently being developed for making more transparent models, like decision trees with interpretable rules, and numerical tools for unpacking the inner workings of the model. We will consider a gradient based method called \textit{Grad-CAM}~\cite{Selvaraju_2019} which aim to highlight some of the important features from the input image. The algorithm is based on the idea use the gradients for a certain feature map with respect to the loss. 

First, we forward propagate the input through the model and decide on a feature
map of interest. We then calculate the gradients for the feature map with
respect to the loss of a certain target output. For a classification task, one
would often choose the predicted class, the class with the highest score, as the
target output. The gradients can then be used as an estimate of which part of
the feature maps is most important for the prediction. a ReLU activation layer
is then applied to keep only the positive contributions. Since the convolutional
layers preserve spatial information we can rescale the heatmap provided by the
feature map gradient to make an input-sized heatmap allowing for an overlaid
visualization of the on the input image. This provides a visual clue of which
part of the image the prediction is most strongly based on. We can do this for
different depths of the model and even combine the results for multiple layers.
\cref{fig:grad_cam_example} show an exemplary use, where the Grad-CAM analysis reveals the difference between a biased and unbiased prediction model for the task of predicting professions. The biased model shows to be considering the person more than the actual objective clues given by relevant equipment and work-related clothing


\begin{figure}[H]
  \centering
  \includegraphics[width=0.7\linewidth]{figures/theory/grad_cam_example.png}
  \caption{\hl{TMP}~\cite{Selvaraju_2019} }
  \label{fig:grad_cam_example}
\end{figure}


% This has spiked interest in developing methods to unravel the inner parts of
% the network. This effort lies partly in the development of better analytic tools
% but also in the aim of creating simpler and more transparent models. We will be
% using a deep-learning convolutional network that is not immediately
% transparent. Since the feature maps in the network preserve the relative spatial
% relations of the input one approach is to consult the feature maps for
% information on which parts of the input it considers the most. However, this is
% often not readily interpreted. 


\section{Accelerated search using genetic algorithm}\label{sec:GA}
For the scope of finding new Kirigami designs which exhibit certain frictional properties, we are interested in utilizing a trained machine-learning model for further exploration. This reverses the design process as one has to find the right input to achieve a certain output. A possible strategy is to explore a range of inputs and use the model predictions as a guiding metric. One approach to this is the genetic algorithm (\acrshort{GA}) which is inspired by biological evolution and mimics the Darwin theory of the survival of the fittest. \cite{katoch_review_2021}. \acrshort{GA} is a population-based algorithm for
which the basic elements are chromosome representation, fitness selection and
biological-inspired operators. The chromosomes represent the genes for each
individual in the population and typically take the form of a binary string.
Each position within the chromosome is called a \textit{locus} and has two
possible values (0 or 1). A fitness function is defined to assign a score for
all chromosomes based on some optimization objective. This plays a role for the
biologically inspired operators for which the main ones are selection, mutation
and crossover. Selection is the process of selecting chromosomes based on their
fitness score for further processing. In mutation, some of the loci within a
chromosome are flipped and in crossover, chromosomes are merged to create
offspring. \acrshort{GA} has been implemented in many areas such as the traveling salesman problem~\cite{jiang2000distributed}, function optimization~\cite{szeto1998effects}, adaptive agents in stock markets~\cite{szeto2000adaptive} and airport scheduling~\cite{shiu2008self}. Wang et al.~\cite{Wang2010} note that a general drawback is a need for expertise when choosing parameters that match specific applications. They propose an accelerated genetic algorithm based on a Markov chain transition probability matrix to perform a guided search that reduces the number of parameter choices one has to make. The following introduction of this method is thus based on~\cite{Wang2010}.

We define the binary population matrix $A_{ij}(t)$ at generation $t$, consisting of
$N$ rows denoting chromosomes $i \in \{0, 1, \ldots, N\}$ and $L$ columns denoting the loci $j \in \{0, 1, \ldots L\}$. For our application, we let the locus represent an atom in the Kirigami pattern matrix which is flattened to fit the format of the population matrix. We carry forward the binary values with 0 meaning a removed atom and 1 a present atom. By the use of a fitness function $f(t)$, we sort the population matrix row-wise in descending order by fitness score, i.e.\ $f_i(t) \le f_k(t)$ for $i \ge k$. In the spirit of Markov chains, we assume that some transitions probability exists for the transition between the current state
$A(t)$ and the next state $A(t+1)$. We assume that this transition probability
only takes into account the mutation process, and thus we omit operators like
crossover. For each generation, the chromosomes are sorted according to the
fitness function and the chromosome at the $i^{\text{th}}$ fittest place is assigned a ranking score $r_i(t)$ by some monotonic increasing ranking scheme. We take this to be
\begin{align*}
  r_i(t) = 
  \begin{cases}
    (i-1)/N',& i-1 < N' \\
    1, &\text{else}
  \end{cases}
\end{align*}
with $N' = N/2$ from~\cite{Wang2010}. We assign a row mutation probability $a_i(t)$ meaning that the probability for a mutation will increase towards the lower fitness scores. For the considerations of mutation with respect to each locus in the columns of $A_{ij}(t)$, we define the count of 0's and 1's as $C_0(j)$ and $C_1(j)$ respectively. These are normalized as
\begin{align*}
  n_0(j, t) = \frac{C_0(j)}{C_0(j) + C_1(j)}, \quad n_1(j, t) = \frac{C_1(j)}{C_0(j) + C_1(j)}.
\end{align*}
We can thus describe the state of the $j^{\text{th}}$ locus column as the state vector $\vec{n}(j,t)=(n_0(j, t), n_1(j, t))$. In order to direct the current population to a preferred state for locus $j$ we consider the highest weight $W_i = 1 - r_i$ among the chromosomes for the case of the locus being 0 or 1 respectively. This corresponds to the targets
\begin{align*}
  C'_0(j) &= \max\{W_i | A_{ij} = 0; \ i = 1, \ldots, N\} \\
  C'_1(j) &= \max\{W_i | A_{ij} = 1; \ i = 1, \ldots, N\}.
\end{align*}
These are normalized
\begin{align}
  n_0(j, t+1) = \frac{C'_0(j)}{C'_0(j) + C'_1(j)}, \quad n_1(j, t+1) = \frac{C'_1(j)}{C'_0(j) + C'_1(j)}.
  \label{eq:target_states}
\end{align}
to produce the target state vector $\vec{n}(j,t+1)=(n_0(j, t+1), n_1(j, t+1))$.  This will serve as a direction for each locus to evolve in and thus we can formulate the Markov chain as
\begin{align*}
  \begin{bmatrix}
    n_0(j, t+1) \\
    n_1(j, t+1)
  \end{bmatrix}
  = 
  \begin{bmatrix}
    P_{00}(j,t) \ P_{10}(j,t) \\
    P_{01}(j,t) \ P_{11}(j,t)
  \end{bmatrix}
  \begin{bmatrix}
    n_0(j, t) \\
    n_1(j, t)
  \end{bmatrix},
\end{align*}
where the matrix represents the transition matrix. Since the probability must sum to one for the rows in the transition matrix we get 
\begin{align*}
  P_{00}(j, t) = 1 - P_{01}(j, t), \quad P_{11}(j, t) = 1 - P_{10}(j, t).
\end{align*}
These conditions allow us to solve for the transition probability $P_{10}(j,t)$ in terms of the single variable $P_{00}(j,t)$
\begin{align}
  P_{10}(j,t) &= \frac{n_0(j, t+1) - P_{00}(j,t)n_0(j, t)}{n_1(j,t)}  \label{eq:trans_prop_p10}\\
  P_{01}(j,t) &= 1 - P_{00}(j,t) \label{eq:trans_prop_p01} \\
  P_{11}(j,t) &= 1 - P_{10}(j,t) \label{eq:trans_prop_p11}
\end{align}
The remaining part is to define $P_{00}(j,t)$. We adopt the choice from~\cite{Wang2010} and start from $P_{00}(j, t = 0) = 0.5$ and choose $P_{00}(j,t) = n_0(j,t)$ for the following generations. Thus for a locus $A_{ij}(t)$ we mutate it, changing the binary value, by the probability
\begin{align}
  p_{ij}(t) = 
  \begin{cases}
    a_i(t)P_{01}(t), &A_{ij}(t) = 0 \\
    a_i(t)P_{10}(t), &A_{ij}(t) = 1
  \end{cases}
  \label{eq:p_flip}
\end{align}
In summary, each generation update involves the following steps.
\begin{enumerate}
  \item For generation $t$ calculate the fitness score $f_i(t)$ of each chromosome $i$ and sort the population matrix $A_{ij}(t)$ row-wise according to a descending score. 
  \item From a defined ranking scheme $r_i(t)$ set the chromosome mutation probability to $a_i(t) = r_i(t)$ and the weighting of each row $W_i(t) = 1 - r_i(t)$.
  \item Calculate the target states~\cref{eq:target_states} and the transition probabilities using~\crefrange{eq:trans_prop_p10}{eq:trans_prop_p11} and $P_{00}(j, t = 0) = 0.5$, $P_{00}(j,t>0) = n_0(j,t>0)$.
  \item Mutate (flip) each locus $A_{ij}(t)$ by the $p_{ij}$ given by~\cref{eq:p_flip}.
\end{enumerate}
Notice that this algorithm treats every locus as an independent gene. Thus, we do not incorporate any effects from spatial dependencies in the Kirigami pattern matrix.



\subsubsection{Repair function}
\hl{A numerical scheme for repairing the Kirigami matrix to correspond to a non-detached sheet. This was implemented in order to get candidates that are not immediately marked as a rupture. But it might be better placed in the Random walk section even though we did not make this algorithm before creating the random walk configurations.}
.
