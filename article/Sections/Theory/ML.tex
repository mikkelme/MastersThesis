\chapter{Machine Learning}\label{chap:ML}
We will use machine learning to predict the friction resulting from the straining and loading of a given Kirigami sheet. To this end, we will generate data through \acrshort{MD} simulations that will serve as the ground truth for training a machine learning model. The advantage of using machine learning for this purpose is that it can significantly speed up the exploration of new configurations compared to that using full \acrshort{MD} simulations. However, there is no guarantee that the machine learning model can accurately capture the physical mechanisms governing our system. Hence, a key objective is to assess
the viability of this approach in the study of Kirigami friction, which we will pursue using traditional machine-learning methods. In this
chapter, we introduce the key concept behind machine learning and some of the concepts and techniques relevant to our implementation. For the numerical implementation, we will use the machine learning framework PyTorch~\cite{NEURIPS2019_9015}.


\section{Neural network}\label{sec:NN}
The neural network, or more precisely the \textit{feed-forward dense neural
network}, is one of the original concepts in machine learning arising from the attempt of mimicking the way neurons work in the
brain~\cite{lederer2021activation, Shankar_2022}. The neural network can
be considered in terms of three major parts: The input layer, the so-called
\textit{hidden layers} and the output layer as shown in~\cref{fig:ffnn}.
The input is described as a vector $\vec{x} = x_0, x_1, \ldots, x_{n_x}$ where
each input $x_i$ is usually denoted as a \textit{feature}. The input features
are densely connected to each of the \textit{nodes} in the first hidden layer as
indicated by the straight lines in~\cref{fig:ffnn}. Each line represents a weighted connection that can be adjusted to configure the importance of that
feature. Similar dense connections are present throughout the hidden layers to
the final output layer. For a given note $a_j^{[l]}$ in layer $l$ the input from
all nodes in the previous layer $l-1$ are processed as
\begin{align*}
  a_j^{[l]} = f\left(\sum_i w^{[l]}_{ij}a_i^{[l-1]} + b_j^{[l]}\right),
\end{align*}
where $w^{[l]}_{ij}$ is the weight connection node $a_i^{[l-1]}$ of the previous layer to the node $a_j^{[l]}$ in the current layer. Note that having the weight belong to layer $l$ as opposed to $l-1$ is simply a notation choice. $b_j^{[l]}$ denotes a bias and $f(\cdot)$ is the so-called \textit{activation function}. The activation function provides a non-linear mapping of the input to each node. Without this, the network will only be capable of approximate linear functions~\cite{lederer2021activation}. Two common activation functions are the \textit{sigmoid}, mapping the input to the range $(0,1)$, and the \textit{ReLU} which cuts off negative contributions
\begin{align*}
  \text{Sigmoid:} \quad f(z) = \frac{1}{1 + e^{-z}}, \qquad \qquad
  \text{ReLU:} \quad 
  f(z)= \begin{cases}
    z & z > 0   \\
    0 & z \leq 0
    \end{cases}.
\end{align*}
Often the same activation function is used throughout a network, except for the output layer where the activation function is usually omitted or the sigmoid is used for classification tasks. The whole process of sending data through the model is called \textit{forward propagation} and constitutes the mechanism for mapping an input $\vec{x}$ to the model output $\hat{\vec{y}}$. In order to get useful predictions we must \textit{train} the model which involves tuning the model parameters, i.e.\ the weights and biases.
\begin{figure}[!htb]
  \centering
  \includegraphics[width=0.9\linewidth]{figures/theory/ffnn.png}
  \caption{Illustration of a general feed-forward dense neural network with $n_x$ input features and $n_y$ outputs. Reproduced from~\cite{IN5400_slides}.}
  \label{fig:ffnn}
\end{figure}

The model training relies on two core concepts: \textit{backpropagation} and \textit{gradient descent} optimization. First, we define the error associated with a model prediction, otherwise known as the \textit{loss}, through the \textit{loss function} $L(\hat{\vec{y}}, \vec{y})$ that evaluates the model output $\hat{\vec{y}}$ against the ground truth $\vec{y}$. For a continuous scalar output, we might simply use the mean squared error (\acrshort{MSE})
\begin{align}
  L_{\text{\acrshort{MSE}}} = \frac{1}{n_y} \sum_{i = 1}^{n_y} (y_i - \hat{y}_i)^2.
  \label{eq:MSE}
\end{align}
For a binary classification problem, meaning that the output is either True or False (1 or 0), a common choice is binary cross entropy (BSE)
\begin{align}
  L_{\text{BSE}} =  -\sum_{i=1}^{n_y} \ \Big[y_i\log(\hat{y_i}) + (1-y_i)\log(1 - \hat{y_i}) \Big] =  \sum_{i=1}^{n_y}   \begin{cases}
    - \log{(\hat{y_i})},& \quad y_i = 1 \\
    -\log{(1-\hat{y_i})},& \quad y_i = 0.
  \label{eq:BSE}
\end{cases}
\end{align}
Without going into details with the derivation we can convince ourselves that the error is minimized for the correct prediction and maximized for the worst prediction. When the ground truth is $y_i = 1$ we get the negative loss contribution $-\log(\hat{y_i})$ where a correct prediction $\hat{y}_i \to 1$ yields $L_i \to 0$. For a wrong prediction $\hat{y}_i \to 0$ the loss contribution will diverge $L_i \to \infty$. Similar applies to the case of $y_i = 0$ with opposite directions. 

Given a loss function, we can calculate the loss gradient $\nabla_\theta L$ with respect to each of the model parameters $\theta$, being the weights and biases in the model. This is called \textit{backpropagation} since we follow the propagation of the errors as we go back through the model layers. We calculate the gradients using the derivative chain rule. These gradients express how each parameter is connected to the loss and the overall idea is then to ``nudge'' each parameter in the right direction to reduce the loss. We usually denote a full cycle of forward propagation, backpropagation and an update of all model parameters as one \textit{epoch}. We calculate the updated parameter $\theta_t$ for epoch $t$ using the \textit{gradient descent} method
\begin{align}
  \theta_{t} = \theta_{t-1} - \eta \nabla_\theta L(\theta_t).
  \label{eq:grad_descent}
\end{align}
Gradient descent is analog to taking a step in parameter space in the direction
that yields the biggest decrease in the loss. If we imagine a simplified case
with only two parameters $\theta_1$ and $\theta_2$ we can think of these as longitude and latitude coordinates on a map and the loss being the terrain height. The negative gradient $-\nabla_\theta L(\theta_t)$ represents the direction of the steepest loss decline perpendicular to the contour lines shaped by the loss function terrain as shown in~\cref{fig:gradient_descent}. Notice, however, that state-of-the-art models in general contain on the order of \num{e6}--\num{e9} parameters~\cite{thompson2022computational} which poses some challenges for the visualization.  The length of each step is proportional to the gradient norm $||\nabla_\theta L(\theta_t)||$ and the learning rate $\eta$. There are three main flavors to the gradient descent: Batch,
stochastic and mini-batch gradient descent. In \textit{batch gradient descent} we simply
calculate the gradient based on the entire dataset by averaging the contribution
from each data point before updating the parameters. This gives the most robust estimate of the gradient and thus the most direct path through parameter space in terms of minimizing the loss function as
indicated in~\cref{fig:gd}. However, for big datasets, this calculation can be
computationally heavy as it must carry the entire dataset in memory at once. A
solution to this issue is provided by \textit{stochastic gradient descent}
(\acrshort{SGD}) which considers only one data point at a time. Each data point is chosen randomly with replacement and the parameters are updated based on the corresponding gradient. This leads to more frequent updates of the parameters and a more ``noisy'' path through parameter space as shown in~\cref{fig:sgd}. Under some circumstances, this might compromise the precision. However, the presence of noise can increase the likelihood of avoiding local minima in parameter space. The \textit{mini-batch gradient descent} serves as a middle ground between the above-mentioned methods by dividing the full dataset into a subset of mini-batches. Each parameter update is then based on the gradient within a mini-batch. By choosing a suitable batch size we get the robustness of the (full) batch gradient descent and the computational efficiency and resistance to local minima of the \acrshort{SGD} method. We will use the mini-batch method for our implementation.


\begin{figure}[!htb]
  \centering
  \begin{subfigure}[t]{0.49\textwidth}
    \centering
    \includegraphics[width=\textwidth]{figures/theory/gd.png}
    \caption{}
    \label{fig:gd}
  \end{subfigure}
  \hfill
  \begin{subfigure}[t]{0.49\textwidth}
    \centering
    \includegraphics[width=\textwidth]{figures/theory/sgd.png}
    \caption{}
    \label{fig:sgd}
  \end{subfigure}
  \hfill
  \caption{Qualitative illustration of the gradient descent method for a simplified problem with only two parameters $\theta_1$ and $\theta_2$. The blue shade and lines indicate the contour map of the loss function with the darker shade denoting a lower loss. (a) The batch gradient descent method resulting in a relatively straight path toward the optimal parameters. (b) The stochastic gradient descent method resulting in a more noisy path toward the optimal parameters. Reproduced from~\cite{grad_descent_analog}.}
  \label{fig:gradient_descent}
\end{figure}


\subsection{Optimizers}
The name \textit{optimizers} generally refers to a variety of gradient descent methods. In our study, we will use the ADAM (adaptive moment estimation) optimizer~\cite{kingma2017adam}. ADAM combines several ``tricks in the book'' which we will introduce in the following.

% Momentum 
One considerable extension of the gradient descent scheme is the introduction of a momentum term $m_t$ such that we get
\begin{align}
  \theta_t = \theta_{t-1} - m_t, \qquad m_t = \alpha m_{t-1} + \eta \nabla_\theta L(\theta_t),
  \label{eq:mom}
\end{align}
with $m_0 = 0$. If we introduce the shorthand $g_t = \nabla_\theta L(\theta_t)$ we find
\begin{align}
  m_1 &= \alpha m_0 + \eta g_1 = \eta g_1 \nonumber \\
  m_2 &= \alpha m_1 + \eta g_2 = \alpha^1 \eta g_1 + \eta g_2 \nonumber \\
  m_3 &= \alpha m_2 + \eta g_3 = \alpha^2 \eta g_1 + \alpha\eta g_2 + \eta g_3 \nonumber \\
  &\vdots \nonumber \\
  m_t &= \eta \left(\sum_{k=1}^{t} \alpha^{t-k}g_k\right).
  \label{eq:mom_rec}
\end{align}
Hence $m_t$ is a weighted average of the gradients with an exponentially decreasing weight. This act as a memory of the previous gradients and aid to pass local minima and to some degree plateaus in the parameter space. It also provides a general steadiness to the gradient descent which counteracts the transition from batch to mini-batch gradient descent. A variation of momentum can be achieved with the introduction of the exponential moving average (EMA) which builds on the recursion
\begin{align*}
    \text{EMA}(g_1) &= \alpha \overbrace{\text{EMA}(g_0)}^{\equiv \ 0} + (1-\alpha)g_1 \\
    \text{EMA}(g_2) &= \alpha \text{EMA}(g_1) + (1-\alpha)g_2 \\
    &\vdots \\
    \text{EMA}(g_t) &= \alpha \text{EMA}(g_{t-1}) + (1-\alpha)g_t  = \sum_{k=0}^t \alpha^{t-k}(1-\alpha)g_t,
\end{align*}
which is similar to that of momentum~\cref{eq:mom_rec}, but with the explicit weighting by $(1-\alpha)$. The second moment of the exponential moving average is utilized in the root mean square propagation method (\acrshort{RMSProp}) which is motivated by the issue of passing long loss plateaus in the parameter space. Since the size of the updates are otherwise proportional to the norm of the gradient
\begin{align*}
  \theta_{t+1} = \theta_t - \eta g_t \ \Longrightarrow \ ||\theta_{t+1}-\theta_{t}|| = \eta ||g_t||,
\end{align*}
we might get the idea of normalizing the gradient step by the norm $||g_t||$. However, this does not immediately solve the problem of long plateaus as we need to consider multiple past gradients. Hence, this can be done with the use of the \acrshort{EMA}. When reentering a steep region again we need to ``quickly'' downscale the gradient steps which can be achieved more efficiently by using the squared norm $||g_t||^2$ for the \acrshort{EMA} which makes it more sensitive to outliers. From this motivation, the \acrshort{RMSProp} update scheme is given
\begin{align}
  \theta_t = \theta_{t-1} - \eta \frac{g_t}{\sqrt{\text{EMA}(||g_t||^2)}+ \epsilon},
  \label{eq:RMSProp}
\end{align}
where $\epsilon$ is simply a small number to avoid division by zero issues and thus ensure numerical stability.

ADAM merges the idea of a first-order \acrshort{EMA} for the momentum $m_t$, and a second-order \acrshort{EMA} $v_t$ for gradient normalization similar to the root mean square propagation technique in~\cref{eq:RMSProp}
\begin{align*}
  m_t &= \beta_1 m_{t-1} + (1-\beta_1)g_t, \\
  v_t &= \beta_2 v_{t-1} + (1-\beta_2)g_t^2. 
\end{align*}
Since $m_t$ and $v_t$ are initially set to zero ADAM introduces the scaling terms  $(1-\beta^t_1)$ and $(1-\beta^t_2)$ to correct for a bias towards zero. The ADAM scheme is given~\cite{kingma2017adam}
\begin{align}
  \theta_{t+1} = \theta_t - \eta \frac{\hat{m}_t}{\sqrt{\hat{v}_t} + \epsilon}, \qquad \qquad \hat{m}_t = \frac{m_t}{1-\beta^t_1}, \qquad \hat{v}_t = \frac{v_t}{1-\beta^t_2}.
  \label{eq:ADAM}
\end{align}


\subsection{Weight decay}
By adding a so-called \textit{regularization} to the loss function we can penalize high magnitudes of the model parameters, usually intended for the model weights. This is motivated by the idea of preventing overfitting during training, which we will address in more detail in~\cref{sec:under_over_fit}.  The most common way to regularize the loss function is by the use of L2 regularization, adding the squared $l^2$ norm $||\theta||_2^2$, where $||\theta||_2 = \sqrt{\theta_1^2 + \theta_2^2 + \ldots}$, to the model. The loss and gradient then become
\begin{align}
  L_{l^2}(\theta) &= L(\theta) + \frac{1}{2}\lambda ||\theta||^2_2 \label{eq:weight_decay_L} \\
  \nabla_\theta L_{l2}(\theta) &=  \nabla_\theta L(\theta) + \lambda \theta,
  \label{eq:l2_grad}
\end{align}
where $\lambda\in [0,1]$ is the weight decay parameter. The name \textit{weight decay} relates to the fact that some practitioners only apply this penalty to the weights in the model, but we will include the biases as well (standard in PyTorch). Following the original gradient descent scheme~\cref{eq:l2_grad} we get
\begin{align}
  \theta_{t+1} = \theta_t - \eta g_t - \eta\lambda \theta_t = \theta_t\underbrace{(1-\eta\lambda)}_{\text{weight decay}} - \eta g_t. \label{eq:weight_decay_descent}
\end{align}
We notice that choosing a large weight decay ($\lambda \to 1$) will downscale the model parameters while choosing a low weight decay ($\lambda \to 0$) yields the original gradient descent scheme. Note that we will use the weight decay principle in combination with ADAM. In~\cref{eq:weight_decay_descent} we have simply used the original gradient descent scheme~\cref{eq:grad_descent} since this makes it easier to demonstrate the consequences of introducing the L2 regularization into the loss function~\cref{eq:weight_decay_L}.


\subsection{Parameter distributions}
In order to get optimal training conditions it has been found that the initial
state of the weights and biases are important~\cite{salimans2016weight}. First of all, we must initialize the weights by sampling from some distribution. If the weights are set
to equal values the gradient across a layer would be the same. This results in a
complexity reduction as the model can only encode the same values across the
layer. Further, we want to consider the gradient flow during training. Especially for deep networks, networks with many layers, we must pay attention to the problem of \textit{vanishing} or \textit{exploding} gradients. If we for instance consider the sigmoid activation function and its derivative 
\begin{align*}
  f(z) = \frac{1}{1 + e^{-z}}, \qquad f'(z) = \frac{df(z)}{dz} = \frac{e^{-z}}{(1+e^{-z})^2} = \frac{e^{z}}{(1+e^{z})^2},
\end{align*}
we notice that for large and small input values $z$ we get $f(z\to \pm\infty)
\to 0$. However, even a small finite gradient can vanish throughout a deep
network as the calculation of the gradient involves the chain rule. This gives
rise to a gradient that potentially gets smaller and smaller for each layer it
passes in the backpropagation. A similar problem can be found with the ReLU
activation function which contributes toward a gradient of zero for inputs
$z<0$. This can be mitigated by the so-called leaky RelU which maps the $z<0$ to
a small negative slope $a<0$ as $f(z) = az$. On the other hand, we have
exploding gradients, which are simply a result of the chain rule gradient
calculation. For a sufficiently deep network, the gradient can grow
exponentially large and sometimes result in a numerical overflow. One approach
to mitigate this issue is with the use of gradient clipping, where all gradients above
a certain value are manually set to a predefined maximum number. While there
exist techniques to accommodate the problem of vanishing or exploding gradients
as mentioned above, they both benefit from a properly initialized set of
weights. That is, we want the gradients across a given layer to have a zero mean
while the variance is similar between layers in the model. This balanced
gradient flow is more likely to happen if we initialize the weight by the same
criteria~\cite{salimans2016weight}. The specific actions to achieve this depends
on the model architecture, including the choice of activation functions. For
instance, using the ReLU activation functions it was found that the node
standard deviation will depend on the number of input nodes from the previous
layer $N^{[l-1]}$ as $\sqrt{N^{[l-1]}/2}$~\cite{he2015delving}. Thus we can sample the weights from a zero mean normal distribution $N \sim (0,
2/N^{[l-1]})$ with a standard deviation $\sqrt{2/N^{[l-1]}}$ to ensure a balanced weight initialization. This particular case (using the ReLU activation function) is part of the Kaiming initialization scheme~\cite{he2015delving} which is standard in Pytorch.
The bias is initialized from a similar consideration.

\textit{Batch normalization} is another technique that can help reduce the issue of poor gradient flow. Furthermore, it can benefit by speeding up convergence and making the training process more stable~\cite{ioffe2015batch}. In general, model parameters are modified throughout training meaning that the range of values coming from a previous layer will shift (internal covariate shift), even though the same training data is fed through the network repeatedly. By scaling the inputs for a given layer, for each mini-batch, we can mitigate this problem and make for a more standardized input range. For layer $l$ we calculate the mean $\mu^{[l]}$ and variance $(\sigma^{[l]})^{2}$ across the layer with nodes $x_1^{[l]}, x_2^{[l]}, \ldots, x_m^{[l]}$ for each mini-batch of size $m$ as
\begin{align*}
  \mu^{[l]} = \frac{1}{m} \sum_i^m x_i^{[l]}, \qquad (\sigma^{[l]})^{2} = \frac{1}{m} \sum_i^m (x_i^{[l]}-\mu^{[l]})^2.
\end{align*}
We then perform a normal scaling of the inputs within the batch
\begin{align*}
  \hat{x}_i^{[l]} = \frac{x_i^{[l]} - \mu^{[l]}}{\sqrt{(\sigma^{[l]})^{2} + \epsilon}},
\end{align*}
where $\epsilon$ is a small number to ensure numerical stability (similar to what we used for \acrshort{RMSProp} gradient descent). In the final step, the input values are rescaled as
\begin{align*}
  \tilde{x_i}^{[l]} = \gamma^{[l]} \hat{x}_i^{[l]} + \beta^{[l]}
\end{align*}
with trainable parameters $\gamma$ and $\beta$~\cite{ioffe2015batch}. 

\subsection{Learning rate decay strategies}
Until now we have assumed a constant learning rate, but many training schemes use a changing learning rate beyond the adaptiveness included in the optimizers covered so far. Under some circumstances, it can be beneficial to start with a higher learning rate to speed up the initial part of training and then lower the learning rate for the final part~\cite{smith2018disciplined}. One straightforward strategy is a step-wise learning rate decay where the learning rate is reduced by a factor $\gamma \in (0,1)$ every $K$ steps. A more smooth decrease can be achieved with a polynomial decay $\eta_t = \eta_0/t^{\alpha}$ for $\alpha > 0$. More advanced approaches use multiple cycles of increasing and decreasing cycles. We will mainly concern ourselves with a one-cycle policy for which we start at an
intermediate value, increase toward a maximum bound and then decrease toward a
final lower learning rate bound. We do this by following a cosine function that is shifted and scaled to increase towards the maximum bound for the first 30\% of the training length and decrease toward the lower learning rate bound for the remaining epochs. 


\section{Convolutional Neural Network}\label{sec:CNN}
Convolutional Neural Networks (\acrshort{CNN}s) build upon many of the same
concepts as introduced with the feed-forward neural network in~\cref{sec:NN}.
The difference lies in its specialization for a spatially correlated input, such
as pixels in an image. In a dense neural network, every node is connected to
each of the nodes from the previous layers which is not ideal for image
recognition. For instance, if we want the model to recognize images of animals
the dense network will be very sensitive to where that animal is placed within
the frame. The \acrshort{CNN} is motivated by the idea of capturing spatial
relations in the input, but without being sensitive to the relative placement
within the input, i.e.\ being translational invariant. This is achieved by
having a so-called \textit{kernel} or \textit{filter} which slides over the
images\footnote{Note, that we will be using the word ``image'' as a reference
for a spatially dependent input, but in reality, it does not have to be an
actual image in the classical sense.} as it processes the input. The overall flow of data for a typical convolutional network including a final fully connected neural network is illustrated in~\cref{fig:CNN}. 

\begin{figure}[!htb]
  \centering
  \includegraphics[width=0.8\linewidth]{figures/theory/CNN.png}
  \caption{Representation of a Convolutional Neural Network (\acrshort{CNN}). The \acrshort{CNN} performs automatic spatial feature extraction from images by successively applying feature filters that create feature maps (Convolutional layer) and compressing these maps (Pooling layer). Based on the final feature maps, a fully-connected neural network does a prediction, which can be a classification or regression. Figure and caption reproduced from~\cite{cunha2022review}.}  
  \label{fig:CNN}
\end{figure}

A convolutional layer contains multiple kernels, each consisting of a set of
trainable weights and a bias. Each kernel will produce a separate output channel
to the resulting \textit{feature map} layer. The kernel has a 2D spatial size,
specific to the model architecture, and a depth that matches the number of input
channels to the layer. For instance, a typical RGB image will have three
channels, while the number of channels usually increases for each layer in the
model. The kernel lines up with the image and calculates the feature map output
as a dot product between the weights in the kernel and the aligning subset of
the input. This is done for each input channel and summed up with the addition
of a bias as illustrated in~\cref{fig:conv_calculation}. The kernel then slides over by a
step size given by the \textit{stride} parameter and repeats the
calculation. Choosing a stride of 2 or higher results in a reduction of the
output spatial size. If we want to preserve the spatial size we must keep a
stride of one and additionally apply \textit{padding} to the input images, such
that we can achieve one kernel position for each input ``pixel''. The spatial
size of the feature map is given as
\begin{align}
  N_d^{[l]} = \left\lfloor \frac{N_d^{[l-1]} - F_d + 2P}{S} + 1 \right\rfloor,
  \label{eq:down_scaling}
\end{align}
for padding $P$, stride $S$, spatial size of the kernel filter $F_d$, spatial
size of the input $N_d^{[l-1]}$, for dimension $d = \{x, y\}$ and layer $l$. The
\textit{down-sampling} is often done through a pooling layer. A pooling layer is
reminiscent of a kernel, but instead of calculating the output as a dot product,
it calculates the mean (mean pooling) or the max value (max pooling) of the
values within its scope. For instance, by using a max pooling of size $2 \times
2$ and stride 2 we essentially half the dimensions of the image as dictated
by~\cref{eq:down_scaling}. \acrshort{CNN}s will often use repeating series of
convolution (applying a kernel), pooling and then an activation function. Most
architectures aim to down-sample the spatial input while increasing the number
of channels throughout the model layers. This results in a smaller set of
features extracted from the input which then can then be fed into a dense
network, or \textit{fully connected} connected neural network, as also
illustrated in~\cref{fig:CNN}. The convolution part aims to
handle the transition from a spatial input to some internal features. For a model which recognizes animals, we would perhaps think of features such as the number of legs, size, color and so on. In practice, however, the network will not give readily interpreted features for the processing in the fully connected layer, but the concept is still the same. 


\begin{figure}[!htb]
  \centering
  \begin{subfigure}[t]{0.26\textwidth}
    \centering
    \raggedleft
    \raisebox{0.20\height}{\includegraphics[width=0.8\textwidth]{figures/theory/conv_kernel_movement.png}}
    \caption{}
  \end{subfigure}
  \hfill
  \begin{subfigure}[t]{0.70\textwidth}
    \centering
    \includegraphics[width=\textwidth]{figures/theory/conv_calculation.png}
    \caption{}
    \label{fig:conv_calculation}
  \end{subfigure}
  \hfill

  \caption{Illustration of the convolution procedure for a $3 \times 3$ kernel with a depth of 3 matching an RGB input with 3 channels (a) The kernel goes through the various positions aligning with the input as sketched with the arrow. The precise number of positions along this path depends on the stride parameter and the choice of padding. (b) A specific example of the calculations involved in the convolution process. Here a rim of zero padding is included. For each kernel position, a dot product is calculated between the aligning input and the kernel weights for each corresponding channel. This results in an output value for each channel which is then summed up and added with a bias to constitute the final output in the feature map. Reproduced from~\cite{CNN_calc}.}
  \label{fig:conv_example}
\end{figure}

For a \acrshort{CNN}, we often consider the \textit{receptive field}. The receptive field relates to the spatial size of the input that affects a given
node in the feature map of a given layer in the model. This term is often used in reference to the output nodes. \cref{fig:receptive_field} illustrates the
receptive field for a 1D representation of a \acrshort{CNN} with repetitive use
of a kernel of width 2 and stride 1. Going from the output and backward, we see that the output layers are connected to two nodes in the previous layer. Each of these nodes is connected to two nodes in the layer before that, however with one of them overlapping due to the stride of 1. By back-tracking all the way to the input layer we see that this corresponds to a receptive field of $D = 5$. This means that a single output node is only affected by the 5 input nodes within its receptive field. By increasing the filter size and
the stride the receptive field will grow a lot faster than shown in this example. If we assume a constant filter size throughout the network $F$, a stride $S_l$ from layer $l-1$ to $l$ we get that the receptive field $D_l$ with respect to a certain layer in a given spatial dimension is
\begin{align}
    D_l = D_l + \left[(F_l - 1) \cdot \prod_{i=1}^{l-1}S_i \right],
    \label{eq:receptive_field}
\end{align}
with $D_0 = 1$ and $l=0$ as the input layer. Note that by convention, the
product of zero elements is 1, such that for the first layer, the product is 1. \cref{eq:receptive_field} apply for both spatial dimensions individually. 

The receptive field is important for understanding the connectivity in the model since the output will be completely independent of the inputs and feature maps outside the receptive field. Furthermore, we differentiate between the \textit{theoretical}
receptive field and the \textit{effective} receptive field. The effective receptive field
will have a Gaussian distribution within the theoretical receptive field because the nodes in the center of the receptive field will have more connections leading to the output, as seen in~\cref{fig:receptive_field}. Thus, in practice, the effective receptive field will be smaller than the theoretical. Implementations like dilated convolutions, which make the filter expand in circumference and skip positions within the filter, can be used to further increase the effective field.

\begin{figure}[!htb]
  \centering
  \includegraphics[width=0.7\linewidth]{figures/theory/receptive_field.png}
  \caption{An illustration of the receptive field $D$ with respect to an output node in a 1D convolutional network. This example uses a width of 2 for the kernel and a stride of 1. Reproduced from~\cite{receptive_field_1D}.}
  \label{fig:receptive_field}
\end{figure}


\subsection{Training, validation and test data}
So far, we have simply considered the concept of \textit{training data} as a
means to update the model parameters. Yet, we want to evaluate the model
performance as it improves. The problem arises immediately from the fact that a
complex model can fit about any function. More precisely, it has been proven
that a deep convolutional neural network is universal (it follows the universal
approximation theorem), meaning that it can approximate any continuous function
to an arbitrary accuracy when the depth of the network is large enough~\cite{cybenko_approximation_1989}. Thus for a complex model, it is just a matter
of time (epochs) before the model eventually finds a good approximation for the
training data. However, we want the model to learn general trends and not to
``memorize'' all the data points which are known as \textit{overfitting}. While
the
predictions for the training data can grow arbitrarily good in most cases, the
performance on unseen data within the same domain will yield poor performance in the
case of overfitting. The common way to address this issue is by putting aside a
subset of the data, the so-called \textit{validation} data, which we use to
validate the model performance during and after training. By keeping this
validation set separate from the training data we can get a more
reliable performance estimate for the model. It is crucial to use a random partitioning in order to ensure an equal distribution of data across both sets. To strike a balance
between the quality of training and validation, a commonly used partitioning
ratio is usually around 20:80 in favor of the training set which we will adapt as well. A third data set often forgotten is the
\textit{test} set. While the validation set should be kept unseen from the model
training, the test set should be kept unseen from the model developer when choosing the model architecture and hyper-parameters. We define a hyper-parameter
as a variable to be set prior to the actual application of the learning
algorithm, one that is not selected by the learning algorithm
itself~\cite{Bengio2012}. This includes parameters such as learning rate,
momentum and weight decay, but not the weights and biases as these are updated
by the learning algorithm. When adjusting the hyper-parameters we will use the
performance on the validation set as a guiding metric. Hence, our choices can
eventually lead to high-level overfitting through the hyper-parameter
choices. Hence, we should denote a test set for the final evaluation of our
model which has not been considered before the end. Formally, this is the only
reliable performance metric for the model. However, it is important that this set also possesses the same data distribution to ensure a reliable performance estimate.


\section{Overfitting and underfitting}\label{sec:under_over_fit}
The balance between underfitting and overfitting is essential in model training
and it is closely related to the complexity of the model and the selected
hyper-parameters. A typical textbook visualization of underfitting and
overfitting is given in~\cref{fig:fitting_vs_time}. 


\begin{figure}[!htb]
  \centering
  \begin{subfigure}[t]{0.42\textwidth}
    \centering
    \includegraphics[width=\textwidth]{figures/theory/fitting_vs_time.png}
    \caption{}
    \label{fig:fitting_vs_time}
  \end{subfigure}
  \hfill
  \begin{subfigure}[t]{0.57\textwidth}
    \centering
    \includegraphics[width=\textwidth]{figures/theory/fitting_quality.png}
    \caption{}
    \label{fig:fitting_quality}
  \end{subfigure}
  \hfill
  \caption{Illustrations regarding the concept of underfitting and overfitting. (a) A typical textbook representation of the transition from underfitting to overfitting during prolonged training. Initially, both the training and validation loss decreases for each epoch, but at some point, the validation will begin to increase again marking the beginning of overfitting. This motivates the idea of an early stopping point at the minimum of the validation loss curve. (b) A 2D example of fitting a curve (solid line) to a series of data points (red dots) generated from a second-order polynomial (dotted line) with additional noise. In the underfitting case, we fail to capture the general trend due to a low complexity in our fit, while in the overfitting case, our fit captures the noise in the data as well. Reproduced from~\cite{kaggle}.}
  \label{fig:over_under_fitting}
\end{figure}

As we begin to train or model both the training and validation loss is
decreasing. At a certain point, the model may begin to not only capture general
trends but also specific trends in the training data, which may be related to
noise. This marks the transition from underfitting to overfitting where the
validation loss will increase due to a loss of generalization. However, the
training loss will continue its steady decline as the training procedure
optimize exclusively for the training data loss. \textit{Early stopping} can be
utilized to detect this transition and stop the training early in an attempt to
hit the sweet spot between under- and overfitting. We will use a variation of
this technique which is to store the best model based on the validation
performance history. For this approach, we let the training finish and then
store the model corresponding to the best validation score found during the
entirety of the training. In principle, we can ``get lucky'' and find the model
settings at a state that is overfitted for the validation set. However, we
consider this to be highly unlikely for a reasonably big amount of data and a
complex model with many model parameters to tune.


The underfitting and overfitting phenomena can also be thought of as a
consequence of model complexity and not just training time. For a certain amount
of epochs a simple model will yield underfitting and an overly complex model
will yield overfitting. This can be expected to follow a similar qualitative
trend as shown in~\cref{fig:fitting_vs_time} with the substitution of
\textit{model complexity} on the x-axis. \cref{fig:fitting_quality} visualizes
the concept of underfitting and overfitting in terms of the complexity regarding
the fitting of a second-order polynomial. The figure shows how a simple linear
function will make a crude approximation for the true curve, while an overly
complex model will pick up the noise in the data and miss the general trend as
well. However, the problem is that we do not know the true curve in practice. If
we did, we would not need machine learning to approximate it in the first place.
Without having additional insight into the governing source of the data the
overfitting case from~\cref{fig:fitting_quality} seems to produce the most confident fit for all we know. Thus, it is important to utilize the validation loss in order to mitigate this problem. 




\section{Hypertuning}\label{sec:hypertuning}
The training of a machine learning model revolves around tuning the model parameters such as weights and biases. However, as mentioned already, a handful of \textit{hyper-parameters} remains for us to decide. First of all, we need to choose an architecture for the model. This includes high-level considerations such as whether to use a neural network or a convolutional network, but also lower-level considerations, such as the depth and the width of the model, i.e.\ the number of layers and the number of nodes/channels in a layer. In addition, we have to choose the loss function and the optimizer which come with hyper-parameters such as learning rate, momentum and weight decay. This extensive list of choices makes the designing of a functional machine learning procedure more complicated than simply hitting ``run'' for the learning algorithm. As N. Smith~\cite{smith2018disciplined} puts it ``Setting the
hyper-parameters remains a black art that requires years of experience to
acquire''. In the following, we will review a general approach for choosing the learning rate, momentum and weight decay hyper-parameters based on the findings of~\cite{smith2018disciplined}. The traditional approach is to
perform a \textit{grid search}, trying out different combinations of hyper-parameters in different training sessions, but this might rather quickly become computationally expensive and ineffective. In addition, hyper-parameters will depend on the training data, the model architecture and not at least each other. This makes it difficult to narrow down the choice one by one. N.\ Smith points to the fact that the validation loss can be examined early on for
clues of either underfitting or overfitting. 

The learning rate is often regarded as the most important hyper-parameter to
tune~\cite{Bengio2012}. Typical values are in the range $[\num{e-6}, 1]$.
Instead of simply running a grid search, we can perform a so-called
\textit{learning rate range test}. One then specifies the
minimum and maximum learning rate boundaries for the range test and a learning rate step size. A range of $\num{e-7}$ to 10 will most likely be appropriate, but the test will reveal this immediately. The idea is to
vary the learning rate throughout the given range in small steps during a short
pre-training. We will increase the learning rate for each iteration, i.e.\ each
parameter update following a mini-batch, and thus we can run this test for a few
epochs, or even a single one, depending on the number of mini-batches. For small
learning rates, the model will converge slowly. As the learning rate approaches
an appropriate value, the convergence speed will increase, which can be observed
as a decrease in the validation loss. Eventually, the convergence will stop and
the validation loss will pass a minimum for which it will begin to diverge. This
general behavior can be understood for the simplified 1D example of finding the
minimum of a second-order polynomial as shown in~\cref{fig:lr_descents}. For
small learning rates, the gradient descent update will provide a small step in
the parameter space toward the minimum. However, if the learning rate is too
small this will yield a slow convergence. On the other hand, if the learning
rate becomes too large, we will effectively step past the minimum in one step.
Each following step will overshoot the minimum further since the step
length is proportional to the gradient of the loss, which leads to a diverging
trend. From the learning rate range test, we can use the point of divergence as an upper bound for the learning rates when considering a cyclic learning rate scheme. For a constant learning rate scheme we can use the point of steppest decline of the validation loss as an estimate for the best learning rate choice~\cite{smith2018disciplined}.


\begin{figure}[!htb]
  \centering
  \includegraphics[width=0.9\linewidth]{figures/theory/lr_descents.png}
  \caption{Qualitative illustration of the effect of learning rate choice for gradient descent on a simplified 1D problem. The left frame shows the gradient descent steps using a small learning rate where several steps are needed to approach the minimum of the loss. The right frame shows the corresponding steps with the choice of a too large learning rate which yields a diverging behavior. Reproduced from~\cite{JavaTpoint}.}
  \label{fig:lr_descents}
\end{figure}


Next, we consider the choice of momentum. From the gradient descent scheme with
momentum~\cref{eq:mom} we see that the momentum parameter $\alpha$ and the
learning rate $\eta$ have a similar effect on the parameter update
\begin{align*}
  \theta_t = \theta_{t-1} - \eta g_t - \alpha m_{t-1},
\end{align*}
since $m_t$ is a moving average of the gradient $g_t$ as well. Like the learning
rate, we want to set the momentum value as high as possible without causing
instabilities in the training. However, since the two are related to each other,
they must also be considered together. N. Smith~\cite{smith2018disciplined} reports that a momentum range test is not a useful approach to determine the
optimal momentum value. Instead, he suggests doing a few short runs with different values of momentum,
such as 0.99, 0.97, 0.95, and 0.9, to determine a suitable choice. By including
momentum in the learning range test we can balance the learning rate accordingly. Moreover, for a cyclic learning rate scheme he reports that a cycling momentum scheme, reversed with respect to the learning rate, is beneficial. When the
learning rate increase toward the upper bound the learning rate should
decrease toward a lower bound and vice versa. Choosing a lower momentum bound of
0.80--0.85 is reported to give similar stable results~\cite{smith2018disciplined}. 

Finally, we address weight decay. N.\ Smith~\cite{smith2018disciplined} reports that weight decay is different from learning rate and momentum by the fact that weight decay is better chosen as a constant value as opposed to a cyclic scheme. However, the weight decay is dependent on the model complexity, learning rate and momentum choice. Hence, this can often be chosen after setting those. We can estimate a suitable choice by doing a rough grid search for values such as 0, $\num{e-6}$, $\num{e-5}$ and $\num{e-4}$ for complex architectures and $\num{e-4}$, $\num{e-3}$ and $\num{e-2}$ for more shallow architectures. Choosing the weight decay on the scale of exponents will often provide good enough precision in practice. 


\section{Prediction explanation}\label{sec:explanation}
On a final note, we present a simple method for providing some insight into the prediction from a convolutional neural network. The high complexity of deep learning models limits our ability to understand the decision-making going into the model predictions beyond the input data. This is known as the \textit{black box} problem. A lot of effort is currently being developed for making more transparent models, like decision trees with interpretable rules, and numerical tools for unpacking the inner workings of the model. We will consider a gradient-based method called \textit{Grad-CAM}~\cite{Selvaraju_2019} which aims to highlight some of the important features of a convolutional network model prediction. The method utilizes the gradients for a certain feature map with respect to the loss. 


To begin, we forward propagate the input through the model. Next, we decide on a
feature map of interest and calculate the gradients with respect to the loss of
a certain target output. In classification tasks, one would often choose the
predicted class with the highest score. By doing so, the gradients can be
utilized to determine which parts of the feature map are more important for the
prediction. We apply a ReLU activation layer to keep only the
positive contributions. Since the convolutional layers preserve spatial
information we can rescale the feature map gradients to make an input-sized
heatmap allowing for an overlaid visualization on top of the input image. By using this method, we can obtain a visual indication of the specific regions in the image that plays a key role in the prediction. We can apply this technique to multiple layers of the model and even merge the outcomes as well for an analysis of the contribution from multiple feature maps.

\cref{fig:grad_cam_example} show an exemplary use, where the Grad-CAM analysis reveals the difference between a biased and unbiased model for the task of predicting professions. The biased model appears to be prioritizing the person's facial features over the objective cues provided by relevant equipment and work-related clothing.


\begin{figure}[!htb]
  \centering
  \includegraphics[width=0.7\linewidth]{figures/theory/grad_cam_example.png}
  \caption{An example of the Grad-CAM method in use: In the first row, we can see that even though both models made the right decision, the biased model was looking at the face of the person to decide if the person was a nurse, whereas the unbiased model was looking at the short sleeves to make the decision. For the example image in the second row, the biased model made the wrong prediction (misclassifying a doctor as a nurse) by looking at the face and the hairstyle, whereas the unbiased model made the right prediction looking at the white coat, and the stethoscope. Figure and caption reproduced from~\cite{Selvaraju_2019}. }
  \label{fig:grad_cam_example}
\end{figure}



\section{Accelerated search using genetic algorithm}\label{sec:GA}
For the scope of finding new Kirigami designs which exhibit certain frictional properties, we are interested in utilizing a trained machine-learning model for further exploration. This reverses the design process as one has to find the right input to achieve a certain output. A possible strategy is to explore a range of inputs and use the model predictions as a guiding metric. One approach to this strategy is the genetic algorithm (\acrshort{GA}) which is inspired by biological evolution mimicking the Darwin theory of the survival of the fittest~\cite{katoch_review_2021}. \acrshort{GA} is a population-based algorithm for
which the basic elements are chromosome representation, fitness selection and
biological-inspired operators. The chromosomes represent the genes for each
individual in the population and typically take the form of a binary string.
Each position within the chromosome is called a \textit{locus} and has two
possible values (0 or 1). A fitness function is defined to assign a score for
all chromosomes based on some optimization objective. This plays a role for the
biologically inspired operators for which the main ones are selection, mutation
and crossover. Selection is the process of selecting chromosomes based on their
fitness score for further processing. In mutation, some of the loci within a
chromosome are changed and in crossover, chromosomes are merged to create
offspring. \acrshort{GA} has been implemented in many areas such as the traveling salesman problem~\cite{jiang2000distributed}, function optimization~\cite{szeto1998effects}, adaptive agents in stock markets~\cite{szeto2000adaptive} and airport scheduling~\cite{shiu2008self}. Wang et al.~\cite{Wang2010} note that a general drawback is a need for expertise when choosing parameters that match specific applications. They propose an accelerated genetic algorithm based on a Markov chain transition probability matrix to perform a guided search that reduces the number of parameter choices one has to make. The following introduction of this method is based on their work~\cite{Wang2010}.

We define the binary population matrix $A_{ij}(t)$ at generation $t$, consisting of
$N$ rows denoting chromosomes $i \in \{0, 1, \ldots, N\}$ and $L$ columns denoting the loci $j \in \{0, 1, \ldots L\}$. For our application, we let the locus represent an atom in the Kirigami pattern matrix which is flattened to fit the format of the population matrix. We carry forward the binary values with 0 meaning a removed atom and 1 a present atom. By the use of a fitness function $f(t)$, we sort the population matrix row-wise in descending order by fitness score, i.e.\ $f_i(t) \le f_k(t)$ for $i \ge k$. In the spirit of Markov chains, we assume that some transition probability exists for the transition between the current state
$A(t)$ and the next state $A(t+1)$. We assume that this transition probability
only takes into account the mutation process, and thus we omit operators like
crossover. For each generation, the chromosomes are sorted according to the
fitness function and the chromosome at the $i^{\text{th}}$ fittest place is assigned a ranking score $r_i(t)$ by some monotonic increasing ranking scheme. We take this to be
\begin{align*}
  r_i(t) = 
  \begin{cases}
    (i-1)/N',& i-1 < N' \\
    1, &\text{else}
  \end{cases},
\end{align*}
with $N' = N/2$ from~\cite{Wang2010}. We assign a row mutation probability $a_i(t) = r_i(t)$ meaning that the probability for a mutation will increase towards the lower fitness scores. For the considerations of the mutation probability for the loci within each chromosome, we define the count of 0 and 1 states as $C_0(j)$ and $C_1(j)$ respectively. These are normalized as
\begin{align*}
  n_0(j, t) = \frac{C_0(j)}{C_0(j) + C_1(j)}, \quad n_1(j, t) = \frac{C_1(j)}{C_0(j) + C_1(j)}.
\end{align*}
We can thus describe the state of the $j^{\text{th}}$ locus column as the state vector $\vec{n}(j,t)=(n_0(j, t), n_1(j, t))$. In order to direct the current population to a preferred state for locus $j$ we consider the highest weight $W_i = 1 - r_i$ among the chromosomes for the case of the locus being 0 or 1 respectively. This corresponds to the targets
\begin{align*}
  C'_0(j) &= \max\{W_i | A_{ij} = 0; \ i = 1, \ldots, N\} \\
  C'_1(j) &= \max\{W_i | A_{ij} = 1; \ i = 1, \ldots, N\}.
\end{align*}
These are normalized
\begin{align}
  n_0(j, t+1) = \frac{C'_0(j)}{C'_0(j) + C'_1(j)}, \quad n_1(j, t+1) = \frac{C'_1(j)}{C'_0(j) + C'_1(j)},
  \label{eq:target_states}
\end{align}
to produce the target state vector $\vec{n}(j,t+1)=(n_0(j, t+1), n_1(j, t+1))$.  This will serve as a direction for each locus to evolve in and thus we can formulate the Markov chain as
\begin{align*}
  \begin{bmatrix}
    n_0(j, t+1) \\
    n_1(j, t+1)
  \end{bmatrix}
  = 
  \begin{bmatrix}
    P_{00}(j,t) \ P_{10}(j,t) \\
    P_{01}(j,t) \ P_{11}(j,t)
  \end{bmatrix}
  \begin{bmatrix}
    n_0(j, t) \\
    n_1(j, t)
  \end{bmatrix},
\end{align*}
where the matrix represents the transition matrix. Since the probability must sum to one for the columns in the transition matrix we get 
\begin{align*}
  P_{01}(j,t) &= 1 - P_{00}(j,t), \quad P_{11}(j,t) = 1 - P_{10}(j,t).
\end{align*}
These conditions allow us to solve for the transition probability $P_{10}(j,t)$ in terms of the single variable $P_{00}(j,t)$
\begin{align}
  P_{10}(j,t) &= \frac{n_0(j, t+1) - P_{00}(j,t)n_0(j, t)}{n_1(j,t)}  \label{eq:trans_prop_p10}\\
  P_{01}(j,t) &= 1 - P_{00}(j,t) \label{eq:trans_prop_p01} \\
  P_{11}(j,t) &= 1 - P_{10}(j,t) \label{eq:trans_prop_p11}
\end{align}
The remaining part is to define $P_{00}(j,t)$. We adopt the choice from~\cite{Wang2010} and start from $P_{00}(j, t = 0) = 0.5$ and choose $P_{00}(j,t) = n_0(j,t)$ for the following generations $t>0$. Thus for a locus $A_{ij}(t)$ we mutate it, changing the binary value, by the probability
\begin{align}
  p_{ij}(t) = 
  \begin{cases}
    a_i(t)P_{01}(t), &A_{ij}(t) = 0 \\
    a_i(t)P_{10}(t), &A_{ij}(t) = 1
  \end{cases}
  \label{eq:p_flip}
\end{align}
In summary, each generation update involves the following steps.
\begin{enumerate}
  \item For generation $t$ calculate the fitness score $f_i(t)$ of each chromosome $i$ and sort the population matrix $A_{ij}(t)$ row-wise with descending score. 
  \item From the chosen ranking scheme $r_i(t)$ set the chromosome mutation probability to $a_i(t) = r_i(t)$ and the weighting of each row $W_i(t) = 1 - r_i(t)$.
  \item Calculate the target states in~\cref{eq:target_states} and the transition probabilities using~\crefrange{eq:trans_prop_p10}{eq:trans_prop_p11} and $P_{00}(j, t = 0) = 0.5$, $P_{00}(j,t>0) = n_0(j,t>0)$.
  \item Mutate ($0\to1$ or $1\to0$) each locus $A_{ij}(t)$ by the probability $p_{ij}$ given by~\cref{eq:p_flip}.
\end{enumerate}
Notice that this algorithm treats every locus as an independent gene, which means that we do not take the spatial dependencies in the Kirigami pattern into account.


