\section{Standard Model of Particle Physics}


\subsection{Classical Lagrangians of The Standard Model}
The Lagrangian can be considered as the fundamental basis of field theories. By considering a theory involving several types of fields on spacetime, scalar fields, spinor fields, gauge fields etc, then the Lagrangian of the field theory is a real valued function that contains the dynamics and all interactions between these fields. In classical field theory, once the Lagrangian is established, the evolution of the fields are given by differentiating the Lagrangian giving the Euler-Lagrange equations. In quantum field theory, the Lagrangian enters through the path integral, which are used to calculate correlation functions and scattering amplitudes for elementary particles. Depending on which types of fields and interactions of interest, the Lagrangian can be specified in each case. There are Lagrangians for non-interacting fields (free fields), Lagrangians for single interacting fields and several interacting fields. In general, Lagrangians containing only quadratic terms in the fields are free theories, while higher order terms leads to interactions. Interacting field theories are in general very complicated, and in case of weakly interacting theories these interactions are described using what is called Feynman diagrams, which are perturbative expansions in the coupling constants. For a given field, there are in general infinitely possible Lagrangians that one could consider, but in physics there are guiding principles that tells which one describes the reality that is observed. First, there is the restriction of symmetries, secondly the theory should be renormalizable and lastly the theory should be free of gauge anomalies. These principles restrict the possible Lagrangians that appear in the Standard Model, which contains: Yang-Mills Lagrangian, Higgs Lagrangian, Dirac Lagrangian and Yukawa terms. All these Lagrangians are Lorentz invariant, meaning invariant under local Lorentz transformation of the spacetime manifold, acting on each tangent space.

\subsection{Existence of symmetries}

\subsection{Renormalization criteria}
For a quantum field theory to predict finite results it must be renormalizable, meaning that the parameters in the Lagrangian must be renormalized in such a way that finite results are obtained and may be compared with experiments. This leads to the concept of running coupling, which we will study in more detail in (ref to section of renormalization). The action $S$ is given by the four dimensional space-time integral of the Lagrangian density, and it is a dimensionless quantity. From that observation it follows that the mass dimension of the Lagrangian is four. It can then be shown that the only renormalizable Lagrangians are the ones fulfilling mass dimension four, which greatly restricts the possible Lagrangians in the Standard Model.

\subsection{Gauge anomalies}
The fields in the Lagrangian describes classical fields, and does not describe quantum fields until the theory is quantized. This quantization procedure will be explained in more detail later using the path integral formalism. The symmetries of the classical Lagrangian does not in general hold after the quantization procedure, as the measure involved in the definition of the path integral may not be invariant under the symmetry. When the symmetry of the classical Lagrangian is not inherited in the quantum theory, it is said that the symmetry is anomolous. In the case of gauge symmetry there is an important result that says: In a four dimensional Minkowski spacetime, anomalies of gauge symmetry imply that the quantum theory violates unitarity. If unitrity does not hold, then there exists states of both negative and positive norm which mean that they can not have a probability interpretation, which violats one of the fundamental principals in quantum theory. Therefore it follows that the quantum theory must be free of gauge anomalies, and this actually restricts the representations and charges of fermions. The Standard model is free of gauge anomalies. For example, anomaly cancellation in the electroweak sector requires that leptons and quarks appear in complete multiplets of the form $(E_{L},e_{R},Q_{L},u_{R},d_{R})$, where the $L$ denotes left chiral and are doublets, while $R$ denotes right chiral and are singlets under the gauge group.


\subsection{Spontaneous Symmetry Breaking and the Higgs Mechanism}
There are several ways of breaking a symmetry, one type of symmetry breaking is to add non-invariant terms to a given Lagrangian which is invariant under a certain symmetry group. Another type which is the most relevant for describing the Standard Model is the concept of a spontaneous broken symmetry. In this case the defining Lagrangian is still exactly invariant under the symmetry $G$, however there exists a ground state, called the vacuum state $\phi_{0}$, which is not invariant under the symmetry $G$. According to Goldstone's theorem a spontaneous breaking of a continous symmetry introduces massless Goldstone bosons. Gauge fields arising from Yang-Mills theories describe massless particles, as there is no way to insert a mass term without explicitly breaking gauge invariance. By spontaneously breaking a gauge symmetry, previously massless gauge bosons acquire a mass, and this is called the Higgs mechanism. The customary approach to introducing this mechansim is by the introduction of a mexican hat potential, vevs and about degrees of freedom getting eaten. This approach is of course useful in studying Standard Model phenomonology, but talks very little about the underlying mathematical structure. Therefore a more geometric approach will be taken, and thereby motivating the mechanism in a more general sense and from there look at the specific case of the Standard Model.

\subsection{Higgs Lagrangian}
Geometrically the structure is that of a principal $G$-bundle $P(M,G)$, where $M$ is Minkowski space and $G$ is a compact Lie group. Additionally we equip the structure with a complex representation $\rho:G\rightarrow GL(V)$ with associated vector bundle $E=P\times_{\rho} V\rightarrow M$, where $V=\mathbb{C}^{n}$ as we want unitary representations of $G$ on $E$. The associated vector bundle $E$ is called the Higgs bundle, and a section $\bold{\phi}:M\rightarrow E$ are the Higgs field. The Higgs field is a scalar under Lorentz transformations, and we assume there exists a potential $V:E\rightarrow \mathbb{R}$.
\begin{align*}
    \mathcal{L}_{H}=(D_{\mu}\bold{\phi})^{\dagger}(D^{\mu}\bold{\phi})-V(\bold{\phi})
\end{align*}

%%%%%%%%%%%%%%%%%%%%%%%%%%%%%%%%%%%%%%%%%%%%%%%%%%%%%%%
or more generally, $\phi$ could describe a map
\begin{align}
    \phi:\,M\rightarrow N
\end{align}
from our manifold $M$ to some other manifold $N$, known as the \emph{target} space. For example, in a gauge theory, the fundamental field is a connection $\nabla$ on a principal $G$-bundle $P\rightarrow M$. Including matter fields, these are mathematically formulated as sections of vector bundles $E\rightarrow M$ associated to the principal $G$-bundle $P\rightarrow M$ by a choice of representation. Take scalar Quantum Electrodynamics, a theory involving a scalar field $\phi$ and a gauge field $A_{\mu}$, defined up to gauge transformations as
\begin{align}
    \phi\rightarrow \phi'&=e^{i\alpha}\phi
    \\
    A_{\mu}\rightarrow A'_{\mu}&=A_{\mu}+\partial_{\mu}\alpha
\end{align}
This is the local description of a connection $A_{\mu}$ on a principal $U(1)$-bundle, together with a section $\phi$ of a rank one complex vector-bundle $E\rightarrow M$, whose fibres are equipped with a hermitian metric. Another example is in General Relativity, where one naturally uses a Riemannian manifold, which naturally is equipped with various bundles, such as the tangent, cotangent and frame bundle. The fields we talk about in physics are sections on these bundles, which transform non-trivially under Lorentz transformations. Before we dive into technical details about bundles, and why they are the mathematical description of gauge theories, we will first connect special relativity with quantum mechanics in the classic approach of canonical quantization.